\mysection{Mémoire}

Il y a 3 grands types de mémoire:

\begin{itemize}
\item
Mémoire globale \ac{AKA} \q{allocation statique de mémoire}.
Pas besoin de l'allouer explicitement, l'allocation est effectuée juste en déclarant
des variables/tableaux globalement.
Ce sont des variables globales, se trouvant dans le segment de données ou de constantes.
Elles sont accessibles globalement (ce qui est considéré comme un \gls{anti-pattern}).
Ce n'est pas pratique pour les buffers/tableaux, car ils doivent avoir une taille
fixée.
Les débordements de tampons se produisant ici le sont en général en récrivant les
variables ou les buffers se trouvant à côté d'eux en mémoire.
Il y a un exemple dans ce livre: \myref{scanf_global_variable}.

\item
Stack (pile) \ac{AKA} \q{allocation sur la pile}.
L'allocation est effectuée simplement en déclarant des variables/ tableaux localement
dans la fonction.
Ce sont en général des variables locales de la fonction.
Parfois ces variables locales sont aussi visibles depuis les fonctions \glslink{callee}{appelées},
si l'appelant passe un pointeur sur une variable à la fonction \glslink{callee}{appelée}
qui va être exécutée).
L'allocation et la dé-allocation sont très rapide, il suffit de décaler \ac{SP}.
\myindex{\CStandardLibrary!alloca()}

Mais elles ne conviennent pas non plus pour les tampons/tableaux, car la taille du
tampon doit être fixée, à moins qu'\TT{alloca()} (\myref{alloca}) (ou un tableau de
longueur variable) ne soit utilisé.
Les débordements de tampons écrasent en général les structures de pile importantes:
\myref{subsec:bufferoverflow}.

\myindex{\CStandardLibrary!malloc()}
\myindex{\CStandardLibrary!free()}
\item
Heap (tas) \ac{AKA} \q{allocation dynamique de mémoire}.
L'allocation/dé-allocation est effectuée en appelant \TT{malloc()/free()} ou
\TT{new/delete} en \Cpp.
Ceci est la méthode la plus pratique: la taille du bloc peut être définie lors de
l'exécution.
\myindex{\CStandardLibrary!realloc()}

Il est possible de redimensionner (en utilisant \TT{realloc()}), mais ça peut
être long.
Ceci est le moyen le plus lent d'allouer de la mémoire:
L'allocation de mémoire doit gérer et mettre à jour toutes les structures de
contrôle pendant l'allocation et la dé-allocation.
Les débordements de tampons écrasent en général ces structures.
L'allocation sur le tas est aussi la source des problème de fuite de mémoire: chaque
bloc de mémoire doit être dé-alloué explicitement, mais on peut oublier de le faire,
ou le faire de manière incorrecte.
\myindex{\CStandardLibrary!free()}

Un autre problème est l'\q{utilisation après la libération}----utiliser un bloc de
mémoire après que  \TT{free()} ait été appelé, ce qui est très dangereux.

Exemple dans ce livre: \myref{struct_malloc_example}.

\end{itemize}
