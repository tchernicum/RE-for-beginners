\mysection{CPU}

\subsection{Predictores del saltos}
\label{branch_predictors}

Algunos compiladores modernos intentan deshacerse de las instrucciones de saltos condicionales.
Ejemplos en este libro son: \myref{subsec:jcc_ARM}, \myref{chap:cond}, \myref{subsec:popcnt}.

Esto se debe a que el predictor de saltos no siempre es perfecto, por lo tanto los compiladores
tratan de evitar los saltos condicionales, de ser posible.

\myindex{x86!\Instructions!CMOVcc}
\myindex{ARM!\Instructions!ADRcc}
Las instrucciones condicionales en ARM (como ADRcc) son una forma de hacerlo, otra es el conjunto de instrucciones x86 CMOVcc.

\subsection{Dependencias de datos}

Los CPUs modernos son capaces de ejecutar instrucciones de manera simultanea (\ac{OOE}), pero para
poder lograrlo, los resultados de una instrucci\'on en un grupo no debe influenciar la ejecuci\'on de otras.
Como consecuencia, el compilador se esfuerza en hacer uso de instrucciones que tengan una influencia m\'inima en el estado del CPU.

\myindex{ARM!\Instructions!LEA}
Por eso la instrucci\'on \LEA es tan popular, porque no modifica las banderas del CPU, mientras que otras instrucciones aritm\'eticas s\'i lo hacen.
