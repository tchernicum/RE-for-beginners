\mysection{AND}

\subsection{Tester si une valeur est alignée sur une limite de $2^n$}

Si vous devez vérifier si votre valeur est divisible par un nombre $2^n$ (comme 1024,
4096, etc.) sans reste, vous pouvez utiliser l'opérateur \TT{\%} en \CCpp, mais il
y a un moyen plus simple.
4096 est 0x1000, donc il a toujours les $4*3=12$ bits inférieurs à zéro.

Ce dont vous avez besoin est simplement:

\lstinputlisting[style=customc]{fundamentals/AND_4096_FR.c}

Autrement dit, ce code vérifie si il y a un bit mis parmi les 12 bits inférieurs.
Un effet de bord, les 12 bits inférieurs sont toujours le reste de la division d'une
valeur par 4096 (car une division par $2^n$ est simplement un décalage à droite, et
les bits décalés (et perdus) sont les bits du reste).

Même principe si vous voulez tester si un nombre est pair ou impair:

\begin{lstlisting}[style=customc]
if (value&1)
	// odd
else
	// even
\end{lstlisting}

Ceci est la même chose que de diviser par 2 et de prendre le reste de 1-bit.

\subsection{Encodage cyrillique KOI-8R}

Il fût un temps où la table \ac{ASCII} 8-bit n'était pas supportée par certains services
Internet, incluant l'email.
Certains supportaient, d'autres---non.

Il fût aussi un temps, où les systèmes d'écriture non-latin utilisaient la seconde
moitié de la table ASCII pour stocker les caractères non-latin.
Il y avait plusieurs encodages cyrillique populaires, mais KOI-8R (conçu par Andrey
``ache'' Chernov) est plutôt unique en comparaison avec les autres.

% TODO invert arrow 
% TODO text latex form instead of png!
\begin{figure}[H]
\centering
\includegraphics[width=0.5\textwidth]{fundamentals/koi8r.png}
\caption{KOI8-R table}
\end{figure}

On peut remarquer que les caractères cyrilliques sont alloués presque dans la même
séquence que les caractères Latin.
Ceci conduit à une propriété importante: si tout les 8ème bits d'un texte encodé en
Cyrillique sont mis à zéro, le texte est transformé en un texte translittéré avec
des caractères latin à la place de cyrillique.
Par exemple, une phrase en Russe:

\begin{framed}
\begin{quotation}
Мой дядя самых честных правил, Когда не в шутку занемог, Он уважать себя заставил, И лучше выдумать не мог.
\end{quotation}
\end{framed}

\dots s'il est encodé en KOI-8R et que le 8ème bit est supprimé, il est transformé en:

\begin{framed}
\begin{quotation}
mOJ DQDQ SAMYH \^{}ESTNYH PRAWIL, kOGDA NE W [UTKU ZANEMOG, oN UWAVATX SEBQ ZASTAWIL, i LU\^{}[E WYDUMATX NE MOG.
\end{quotation}
\end{framed}

\dots ceci n'est peut-être très esthétiquement attrayant, mais ce texte est toujours
lisible pour les gens de langue maternelle russe.

De ce fait, un texte cyrillique encodé en KOI-8R, passé à travers un vieux service 7-bit
survivra à la translittération, et sera toujours un texte lisible.

Supprimer le 8ème bit transpose automatiquement un caractère de la seconde moitié
de n'importe quelle table \ac{ASCII} 8-bit dans la première, à la même place (regardez
la flèche rouge à droite de la table).
Si le caractère était déjà dans la première moitié (i.e., il était déjà dans la table
\ac{ASCII} 7-bit standard), il n'est pas transposé.

Peut-être qu'un texte translittéré est toujours récupérable, si vous ajoutez le
8ème bit aux caractères qui ont l'air d'avoir été translittérés.

L'inconvénient est évident: les caractères cyrilliques alloués dans la table KOI-8R
ne sont pas dans le même ordre dans l'alphabet Russe/Bulgare/Ukrainien/etc., et ce
n'est pas utilisable pour le tri, par exemple.
