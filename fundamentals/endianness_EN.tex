\mysection{Endianness}
\label{sec:endianness}

The endianness is a way of representing values in memory.

\subsection{Big-endian}

The \TT{0x12345678} value is represented in memory as:

\begin{center}
\begin{tabular}{ | l | l | }
\hline
\HeaderColor address in memory & \HeaderColor byte value \\
\hline
+0 & 0x12 \\
\hline
+1 & 0x34 \\
\hline
+2 & 0x56 \\
\hline
+3 & 0x78 \\
\hline
\end{tabular}
\end{center}

Big-endian CPUs include Motorola 68k, IBM POWER.

\subsection{Little-endian}

The \TT{0x12345678} value is represented in memory as:

\begin{center}
\begin{tabular}{ | l | l | }
\hline
\HeaderColor address in memory & \HeaderColor byte value \\
\hline
+0 & 0x78 \\
\hline
+1 & 0x56 \\
\hline
+2 & 0x34 \\
\hline
+3 & 0x12 \\
\hline
\end{tabular}
\end{center}

Little-endian CPUs include Intel x86.

\subsection{\Example}

Let's take big-endian MIPS Linux installed and ready in QEMU
\footnote{Available for download here: \url{http://go.yurichev.com/17008}}.

And let's compile this simple example:

\begin{lstlisting}[style=customc]
#include <stdio.h>

int main()
{
	int v;

	v=123;

	printf ("%02X %02X %02X %02X\n", 
		*(char*)&v,
		*(((char*)&v)+1),
		*(((char*)&v)+2),
		*(((char*)&v)+3));
};
\end{lstlisting}

After running it we get:

\begin{lstlisting}
root@debian-mips:~# ./a.out 
00 00 00 7B
\end{lstlisting}

That is it.
0x7B is 123 in decimal.
In little-endian architectures, 7B is the first byte (you can check on x86 or x86-64), 
but here it is the last one, because the highest byte goes first.

That's why there are separate Linux distributions for MIPS
(\q{mips} (big-endian) and \q{mipsel} (little-endian)).
It is impossible for a binary compiled for one endianness to work on an \ac{OS} with different endianness. 

There is another example of MIPS big-endiannes in this book: \myref{MIPS_structure_big_endian}.

\subsection{Bi-endian}

CPUs that may switch between endianness are ARM, PowerPC, SPARC, MIPS, \ac{IA64}, etc.

\subsection{Converting data}

\myindex{x86!\Instructions!BSWAP}
The \TT{BSWAP} instruction can be used for conversion.

\myindex{TCP/IP}
TCP/IP network data packets use the big-endian conventions, so that is why a program working on a little-endian architecture
has to convert the values.
The \TT{htonl()} and \TT{htons()} functions are usually used.

In TCP/IP, big-endian is also called \q{network byte order}, while byte order on the computer \q{host byte order}.
\q{host byte order} is little-endian on Intel x86 and other little-endian architectures,
but it is big-endian on IBM POWER, so \TT{htonl()} and \TT{htons()} don't shuffle any bytes on the latter.

