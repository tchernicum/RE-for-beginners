\subsubsection{Héritage de classe}
\label{cpp_inheritance}


Les classes héritées sont similaires aux simples structures dont nous avons déjà
discuté, mais étendues aux classes héritables.

Prenons ce simple exemple:

\lstinputlisting[style=customc]{\CURPATH/classes/classes1_inheritance.cpp}


Investiguons le code généré de la fonction/méthode \TT{dump()} et aussi \TT{object::print\_color()},
et regardons la disposition de la mémoire pour les structures-objets (pour du code
32-bit).

\myindex{Inline code}

Donc, voici les méthodes \TT{dump()} pour quelques classes, générées par
MSVC 2008 avec les options \Ox et \Obzero
\footnote{L'option \Obzero signifie la désactivation de l'expension inline, puisque
la mise en ligne de fonctions peut rendre notre expérience plus difficile.}.

\lstinputlisting[caption=MSVC 2008 \Optimizing\ /Ob0,style=customasmx86]{\CURPATH/classes/classes1_1.asm}

\lstinputlisting[caption=MSVC 2008 \Optimizing\ /Ob0,style=customasmx86]{\CURPATH/classes/classes1_2.asm}

\lstinputlisting[caption=MSVC 2008 \Optimizing\ /Ob0,style=customasmx86]{\CURPATH/classes/classes1_3.asm}

Donc, voici la disposition de la mémoire:

(classe de base \emph{object})

\begin{center}
\begin{tabular}{ | l | l | }
\hline
  \tableheader{} \\
\hline
  +0x0 & int color \\
\hline
\end{tabular}
\end{center}

(classes héritées)

\emph{box}:

\begin{center}
\begin{tabular}{ | l | l | }
\hline
  \tableheader{} \\
\hline
  +0x0 & int color \\
\hline
  +0x4 & int width \\
\hline
  +0x8 & int height \\
\hline
  +0xC & int depth \\
\hline
\end{tabular}
\end{center}

\emph{sphere}:

\begin{center}
\begin{tabular}{ | l | l | }
\hline
  \tableheader{} \\
\hline
  +0x0 & int color \\
\hline
  +0x4 & int radius \\
\hline
\end{tabular}
\end{center}

Regardons le corps de la fonction \main:

\lstinputlisting[caption=MSVC 2008\Optimizing\ /Ob0,style=customasmx86]{\CURPATH/classes/classes1_4.asm}


Les classes héritées doivent toujours ajouter leurs champs après les champs de la
classe de base, afin que les méthodes de la classe de base puissent travailler avec
ses propres champs.


Lorsque la méthode \TT{object::print\_color()} est appelée, un pointeur sur les deux
objets \emph{box} et \emph{sphere} est passé en \TT{this}, et il peut travailler facilement
avec ces objets puisque le champ \emph{color} dans ces objets est toujours à l'adresse
épinglée (à l'offset \emph{+0x0}).
%TODO better translation of "pinned"

On peut dire que la méthode \TT{object::print\_color()} est agnostique en relation
avec le type d'objet en entrée tant que les champs sont \emph{épinglés} à la même adresse
et cette condition est toujours vraie.


Et si vous créez une classe héritée de la classe \emph{box}, le compilateur ajoutera
les nouveaux champs après le champ \emph{depth}, laissant les champs de la classe
\emph{box} à l'adresse épinglée.


Ainsi, la méthode \TT{box::dump()} fonctionnera correctement pour accéder aux champs
\emph{color}, \emph{width}, \emph{height} et \emph{depth}, qui sont toujours positionnés
à l'adresse connue.

Le code généré par GCC est presque le même, avec la seule exception du passage du
pointeur \TT{this} (comme il a déjà été expliqué plus haut, il est passé en premier
argument au lieu d'utiliser le registre \ECX).
