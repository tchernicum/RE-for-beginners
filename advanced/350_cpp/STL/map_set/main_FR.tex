\subsubsection{std::map and std::set}
\myindex{\Cpp!STL!std::map}
\myindex{\Cpp!STL!std::set}
\myindex{Arbre binaire}

L'arbre binaire est une autre structure de données fondamentale.

Comme son nom l'indique, c'est un arbre où chaque n\oe{}ud a au plus 2 liens sur un autre n\oe{}ud.
Chaque n\oe{}ud a une clef et/ou un valeur:
\TT{std::set} fourni seulement une clef dans chaque n\oe{}ud,
\TT{std::map} fourni à la fois une clef et une valeur dans chaque n\oe{}ud

Les arbres binaire sont générallement la structure utilisée dans l'implémentation
des dictionnaires de clef-valeurs ((\ac{AKA} \q{tableaux associatifs}).

Il y a au moins ces trois propriétés importante qu'un arbre binaire possède:
\begin{itemize}
\item Toutes les clefs sont toujours stockées sous une forme triée.
\item Les clefs de tout type peuvent être stockées facilement.
Les algorithmes de l'arbre binaire ne sont pas conscients du type de clef, seule
une fonction de comparaison de clef est nécessaire.
\item 
Trouver une clef particulière est relativement rapide comparé aux listes chaînées
et tableaux.
\end{itemize}

Voici un exemple très simple: stockons ces nombres dans un arbre binaire:
0, 1, 2, 3, 5, 6, 9, 10, 11, 12, 20, 99, 100, 101, 107, 1001, 1010.

\input{\CURPATH/STL/map_set/example_tikz}

Toutes les clefs qui sont plus petites que la valeur de la clef du n\oe{}ud sont
stockées du côté gauche.

Toutes les clefs qui sont plus grandes que la valeur de la clef du n\oe{}ud sont
stockées du côté droit.

Ainsi, l'algorithme de recherche est simple: si la valeur que vous cherchez est plus
petite que la valeur de la clef du n\oe{}ud courant:
déplacez à gauche, si elle plus grande: déplacez à droite, stoppez si la valeur cherchée
est égale à la valeur de la clef du n\oe{}ud.

C'est pourquoi l'algorithme de recherche peut chercher des nombres, des chaînes de
texte, etc., tant que la fonction de comparaison de clef est fourni.

Toutes les clefs ont des valeurs uniques.

En ayant cela, il faut $\approx \log_{2} n$ itérations pour trouver une clef dans
un arbre binaire équilibré avec $n$ clef.
Ceci implique que $\approx 10$ itérations sont pour $\approx 1000$ keys, ou $\approx 13$
itérations pour $\approx 10000$ clefs.

Pas mal, mais l'arbre doit toujours être équilibré pour cela: i.e., les clefs doivent
être distribuées uniformément à chaque niveaux.
Les opérations d'insertion et de suppression font un peut de maintenance pour garder
l'arbre dans un état équilibré.

Il y a plusieurs algorithme d'équilibrage disponible, incluant l'arbre AVL et l'arbre red-black.

La dernière étend chaque n\oe{}ud avec une valeur de \q{couleur} opur simplifier
le processus de rééquilibrage, de ce fait, chaque n\oe{}ud peut être rouge ou noir.

À la fois les implémentations des templates \TT{std::map} et \TT{std::set} de GCC
et MSVC utilisent les arbres red-black.

\TT{std::set} a seulement des clefs.
\TT{std::map} est la version étendue de \TT{std::set}: il a aussi une valeur à chaque
n\oe{}ud.

\myparagraph{MSVC}

\lstinputlisting[style=customc]{\CURPATH/STL/map_set/MSVC_FR.cpp}

\lstinputlisting[caption=MSVC 2012]{\CURPATH/STL/map_set/MSVC.txt}

La structure n'est pas paquée, donc chaque valeur \Tchar occupe 4 octets.

\TT{std::set} 




[Cormen, Thomas H. and Leiserson, Charles E. and Rivest, Ronald L. and Stein, Clifford,
\emph{Introduction to Algorithms, Third Edition}, (2009)].

\footnote{\url{http://www.ethoberon.ethz.ch/WirthPubl/AD.pdf}}.


\myparagraph{GCC}

\lstinputlisting[style=customc]{\CURPATH/STL/map_set/GCC.cpp}

\lstinputlisting[caption=GCC 4.8.1]{\CURPATH/STL/map_set/GCC.txt}

\footnote{\url{http://go.yurichev.com/17084}}.


\myparagraph{Démo de rééquilibrage (GCC)}

Il y a aussi une démo nous montrant comment un arbre est rééquilibré après quelques
insertions.

\lstinputlisting[caption=GCC,style=customc]{\CURPATH/STL/map_set/GCC_rebalancing_demo.cpp}

\lstinputlisting[caption=GCC 4.8.1]{\CURPATH/STL/map_set/GCC_rebalancing_demo.txt}

