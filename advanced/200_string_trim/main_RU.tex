\mysection{Обрезка строк}
\newcommand{\CRLF}{\ac{CR}/\ac{LF}}

Весьма востребованная операция со строками --- это удаление некоторых символов в начале и/или конце
строки.

В этом примере, мы будем работать с функцией, удаляющей все символы перевода строки 
(\CRLF{}) в конце входной строки:

\lstinputlisting[style=customc]{\CURPATH/strtrim_RU.c}

Входной аргумент всегда возвращается на выходе, это удобно, когда вам нужно объединять
функции обработки строк в цепочки, как это сделано здесь в функции \main.

\myindex{\CLanguageElements!Short-circuit}
Вторая часть for() (\TT{str\_len>0 \&\& (c=s[str\_len-1])}) называется в \CCpp \q{short-circuit} 
(короткое замыкание) и это очень удобно: \InSqBrackets{\CNotes 1.3.8}.

Компиляторы \CCpp гарантируют последовательное вычисление слева направо.

Так что если первое условие не истинно после вычисления, второе никогда не будет
вычисляться.

% subsections
\subsection{x64: \Optimizing MSVC 2013}

\lstinputlisting[caption=\Optimizing MSVC 2013 x64,style=customasmx86]{\CURPATH/MSVC2013_x64_Ox_RU.asm}

В начале, MSVC вставил тело функции \strlen{} прямо в код, потому что решил, что так будет
быстрее чем обычная работа \strlen{} + время на вызов её и возврат из нее.

Это также называется \emph{inlining}: \myref{inline_code}.

\myindex{x86!\Instructions!OR}
\myindex{\CStandardLibrary!strlen()}
\label{using_OR_instead_of_MOV}
Первая инструкция функции \strlen{} вставленная здесь,\\
это \TT{OR RAX, 0xFFFFFFFFFFFFFFFF}. 
MSVC часто использует \TT{OR} вместо \TT{MOV RAX, 0xFFFFFFFFFFFFFFFF}, потому что опкод получается короче.

И конечно, это эквивалентно друг другу: все биты просто выставляются, а все выставленные
биты это -1 в дополнительном коде (two's complement): \myref{sec:signednumbers}.

Кто-то мог бы спросить, зачем вообще нужно использовать число -1 в функции \strlen{}?

Вследствие оптимизации, конечно.
Вот что сделал MSVC:

\lstinputlisting[caption=Вставленная \strlen{} сгенерированная MSVC 2013 x64,style=customasmx86]{\CURPATH/strlen_MSVC_RU.asm}

Попробуйте написать короче, если хотите инициализировать счетчик нулем!

Ну, например:

\lstinputlisting[caption=Наша версия \strlen{},style=customasmx86]{\CURPATH/my_strlen_RU.asm}

Не получилось. Нам придется вводить дополнительную инструкцию JMP!

Что сделал MSVC 2013, так это передвинул инструкцию \TT{INC} в место перед загрузкой символа.

Если самый первый символ --- нулевой, всё нормально, \RAX содержит 0 в этот момент, так что
итоговая длина строки будет 0.

Остальную часть функции проще понять.

\subsection{x64: \NonOptimizing GCC 4.9.1}

\lstinputlisting[style=customasmx86]{\CURPATH/GCC491_x64_O0_RU.asm}

Комментарии автора.
После исполнения \strlen{}, управление передается на метку L2,
и там проверяются два выражения, одно после другого.

\myindex{\CLanguageElements!Short-circuit}
Второе никогда не будет проверяться, если первое выражение не истинно (\emph{str\_len==0})
(это \q{short-circuit}).

Теперь посмотрим на эту функцию в коротком виде:

\begin{itemize}
\item Первая часть for() (вызов \strlen{})
\item goto L2
\item L5: Тело for(). переход на выход, если нужно
\item Третья часть for() (декремент str\_len)
\item L2: Вторая часть for(): проверить первое выражение, затем второе. 
переход на начало тела цикла, или выход.

\item L4: // выход
\item return s
\end{itemize}

\subsection{x64: \Optimizing GCC 4.9.1}
\label{string_trim_GCC_x64_O3}

\lstinputlisting[style=customasmx86]{\CURPATH/GCC491_x64_O3_RU.asm}

Тут более сложный результат.
Код перед циклом исполняется только один раз, но также содержит проверку символов \CRLF{}!

Зачем нужна это дублирование кода?

Обычная реализация главного цикла это, наверное, такая:

\begin{itemize}
\item (начало цикла) проверить символы \CRLF{}, принять решения

\item записать нулевой символ
\end{itemize}

Но GCC решил поменять местами эти два шага. 
Конечно, шаг \emph{записать нулевой символ} не может быть первым, так что нужна еще одна
проверка:


\begin{itemize}
\item обработать первый символ. сравнить его с \CRLF{}, выйти если символ не равен \CRLF{}

\item (начало цикла) записать нулевой символ

\item проверить символы \CRLF{}, принять решения

\end{itemize}

Теперь основной цикл очень короткий, а это очень хорошо для современных процессоров.

Код не использует переменную str\_len, но str\_len-1.

Так что это больше похоже на индекс в буфере.
Должно быть, GCC заметил, что выражение str\_len-1 используется дважды.

Так что будет лучше выделить переменную, которая всегда содержит значение равное 
текущей длине строки минус 1, и уменьшать его на 1 (это тот же эффект, что и уменьшать
переменную str\_len).


\subsubsection{ARM64}

\myparagraph{GCC}

Компилируем пример в GCC 4.8.1 для ARM64:

\lstinputlisting[numbers=left,label=hw_ARM64_GCC,caption=\NonOptimizing GCC 4.8.1 + objdump,style=customasmARM]{patterns/01_helloworld/ARM/hw.lst}

В ARM64 нет режима Thumb и Thumb-2, только ARM, так что тут только 32-битные инструкции.

Регистров тут в 2 раза больше: \myref{ARM64_GPRs}.
64-битные регистры теперь имеют префикс 
\TT{X-}, а их 32-битные части --- \TT{W-}.

\myindex{ARM!\Instructions!STP}
Инструкция \TT{STP} (\emph{Store Pair}) 
сохраняет в стеке сразу два регистра: \RegX{29} и \RegX{30}.
Конечно, эта инструкция может сохранять эту пару где угодно в памяти, но здесь указан регистр \ac{SP}, так что
пара сохраняется именно в стеке.

Регистры в ARM64 64-битные, каждый имеет длину в 8 байт, так что для хранения двух регистров нужно именно 16 байт.

Восклицательный знак (``!'') после операнда означает, что сначала от \ac{SP} будет отнято 16 и только затем
значения из пары регистров будут записаны в стек.

Это называется \emph{pre-index}.
Больше о разнице между \emph{post-index} и \emph{pre-index} 
описано здесь: \myref{ARM_postindex_vs_preindex}.

Таким образом, в терминах более знакомого всем процессора x86, первая инструкция~--- это просто аналог 
пары инструкций \TT{PUSH X29} и \TT{PUSH X30}.
\RegX{29} в ARM64 используется как \ac{FP}, а \RegX{30} 
как \ac{LR}, поэтому они сохраняются в прологе функции и
восстанавливаются в эпилоге.

Вторая инструкция копирует \ac{SP} в \RegX{29} (или \ac{FP}).
Это нужно для установки стекового фрейма функции.

\label{pointers_ADRP_and_ADD}
\myindex{ARM!\Instructions!ADRP/ADD pair}
Инструкции \TT{ADRP} и \ADD нужны для формирования адреса строки \q{Hello!} в регистре \RegX{0}, 
ведь первый аргумент функции передается через этот регистр.
Но в ARM нет инструкций, при помощи которых можно записать в регистр длинное число 
(потому что сама длина инструкции ограничена 4-я байтами. Больше об этом здесь: \myref{ARM_big_constants_loading}).
Так что нужно использовать несколько инструкций.
Первая инструкция (\TT{ADRP}) записывает в \RegX{0} адрес 4-килобайтной страницы где находится строка, 
а вторая (\ADD) просто прибавляет к этому адресу остаток.
Читайте больше об этом: \myref{ARM64_relocs}.

\TT{0x400000 + 0x648 = 0x400648}, и мы видим, что в секции данных \TT{.rodata} по этому адресу как раз находится наша
Си-строка \q{Hello!}.

\myindex{ARM!\Instructions!BL}
Затем при помощи инструкции \TT{BL} вызывается \puts. Это уже рассматривалось ранее: \myref{puts}.

Инструкция \MOV записывает 0 в \RegW{0}. 
\RegW{0} это младшие 32 бита 64-битного регистра \RegX{0}:

\begin{center}
\begin{tabular}{ | l | l | }
\hline
\RU{Старшие 32 бита}\EN{High 32-bit part}\ES{Parte alta de 32 bits}\PTBRph{}\PLph{}\ITph{}\DE{Oberer 32-Bit-Teil}\THAph{}\NLph{}\FR{Partie 32 bits haute} & \RU{младшие 32 бита}\EN{low 32-bit part}\ES{parte baja de 32 bits}\PTBRph{}\PL{Starsze 32 bity}\ITph{}\DE{Unterer 32-Bit-Teil}\THAph{}\NLph{}\FR{Partie 32 bits basse} \\
\hline
\multicolumn{2}{ | c | }{X0} \\
\hline
\multicolumn{1}{ | c | }{} & \multicolumn{1}{ c | }{W0} \\
\hline
\end{tabular}
\end{center}


А результат функции возвращается через \RegX{0}, и \main возвращает 0, 
так что вот так готовится возвращаемый результат.

Почему именно 32-битная часть?
Потому что в ARM64, как и в x86-64, тип \Tint оставили 32-битным, для лучшей совместимости.

Следовательно, раз уж функция возвращает 32-битный \Tint, то нужно заполнить только 32 младших бита регистра \RegX{0}.

Для того, чтобы удостовериться в этом, немного отредактируем этот пример и перекомпилируем его.%

Теперь \main возвращает 64-битное значение:

\begin{lstlisting}[caption=\main возвращающая значение типа \TT{uint64\_t},style=customc]
#include <stdio.h>
#include <stdint.h>

uint64_t main()
{
        printf ("Hello!\n");
        return 0;
}
\end{lstlisting}

Результат точно такой же, только \MOV в той строке теперь выглядит так:

\begin{lstlisting}[caption=\NonOptimizing GCC 4.8.1 + objdump]
  4005a4:       d2800000        mov     x0, #0x0      // #0
\end{lstlisting}

\myindex{ARM!\Instructions!LDP}
Далее при помощи инструкции \INS{LDP} (\emph{Load Pair}) восстанавливаются регистры \RegX{29} и \RegX{30}.

Восклицательного знака после инструкции нет. Это означает, что сначала значения достаются из стека, и только потом \ac{SP} увеличивается на 16.

Это называется \emph{post-index}.

\myindex{ARM!\Instructions!RET}
В ARM64 есть новая инструкция: \RET. 
Она работает так же как и \INS{BX LR}, но там добавлен специальный бит,
подсказывающий процессору, что это именно выход из функции, а не просто переход, чтобы процессор
мог более оптимально исполнять эту инструкцию.

Из-за простоты этой функции оптимизирующий GCC генерирует точно такой же код.


\subsubsection{ARM + \OptimizingKeilVI (\ARMMode)}

\begin{lstlisting}[caption=\OptimizingKeilVI (\ARMMode),style=customasmARM]
02 0C C0 E3          BIC     R0, R0, #0x200
01 09 80 E3          ORR     R0, R0, #0x4000
1E FF 2F E1          BX      LR
\end{lstlisting}

\myindex{ARM!\Instructions!BIC}
\INS{BIC} (\emph{BItwise bit Clear}) это инструкция сбрасывающая заданные биты. 
Это как аналог \AND, но только с инвертированным операндом.

Т.е. это аналог инструкций \NOT+\AND.

\myindex{ARM!\Instructions!ORR}
\INS{ORR} это \q{логическое или}, аналог \OR в x86.

Пока всё понятно.

\subsubsection{ARM + \OptimizingKeilVI (\ThumbMode)}

\begin{lstlisting}[caption=\OptimizingKeilVI (\ThumbMode),style=customasmARM]
01 21 89 03          MOVS    R1, 0x4000
08 43                ORRS    R0, R1
49 11                ASRS    R1, R1, #5   ; сгенерировать 0x200 и записать в R1
88 43                BICS    R0, R1
70 47                BX      LR
\end{lstlisting}

Вероятно, Keil решил, что код в режиме Thumb,
получающий \TT{0x200} из \TT{0x4000}, 
более компактный, нежели код, 
записывающий \TT{0x200} в какой-нибудь регистр.
% TODO1 пример, как компилятор при помощи сдвигов оптизирует такое: a=0x1000; b=0x2000; c=0x4000, etc

\myindex{ARM!\Instructions!ASRS}
Поэтому при помощи инструкции \INS{ASRS} (\ASRdesc), это значение вычисляется как $\TT{0x4000} \gg 5$.

\subsubsection{ARM + \OptimizingXcodeIV (\ARMMode)}
\label{anomaly:LLVM}
\myindex{\CompilerAnomaly}

\begin{lstlisting}[caption=\OptimizingXcodeIV (\ARMMode),label=ARM_leaf_example3,style=customasmARM]
42 0C C0 E3          BIC             R0, R0, #0x4200
01 09 80 E3          ORR             R0, R0, #0x4000
1E FF 2F E1          BX              LR
\end{lstlisting}

Код, который был сгенерирован LLVM, в исходном коде, на самом деле, выглядел бы так:

\begin{lstlisting}[style=customc]
    REMOVE_BIT (rt, 0x4200);
    SET_BIT (rt, 0x4000);
\end{lstlisting}

И он делает в точности что нам нужно. 
Но почему \TT{0x4200}? 
Возможно, это артефакт оптимизатора LLVM
\footnote{Это был LLVM build 2410.2.00 входящий в состав Apple Xcode 4.6.3}.
Возможно, ошибка оптимизатора компилятора, но создаваемый код всё же работает верно.

Об аномалиях компиляторов, подробнее читайте здесь ~(\myref{anomaly:Intel}).

\OptimizingXcodeIV для режима Thumb генерирует точно такой же код.

\subsubsection{ARM: ещё об инструкции \INS{BIC}}
\myindex{ARM!\Instructions!BIC}

Если немного переделать пример:

\begin{lstlisting}[style=customc]
int f(int a)
{
    int rt=a;

    REMOVE_BIT (rt, 0x1234);

    return rt;
};
\end{lstlisting}

То оптимизирующий Keil 5.03 в режиме ARM сделает такое:

\begin{lstlisting}[style=customasmARM]
f PROC
        BIC      r0,r0,#0x1000
        BIC      r0,r0,#0x234
        BX       lr
        ENDP
\end{lstlisting}

Здесь две инструкции \INS{BIC}, т.е. биты \TT{0x1234} сбрасываются в два прохода.

Это потому что в инструкции \INS{BIC} нельзя закодировать значение \TT{0x1234}, 
но можно закодировать \TT{0x1000} либо \TT{0x234}.

\subsubsection{ARM64: \Optimizing GCC (Linaro) 4.9}

\Optimizing GCC, компилирующий для ARM64, может использовать \AND вместо \INS{BIC}:

\begin{lstlisting}[caption=\Optimizing GCC (Linaro) 4.9,style=customasmARM]
f:
	and	w0, w0, -513	; 0xFFFFFFFFFFFFFDFF
	orr	w0, w0, 16384	; 0x4000
	ret
\end{lstlisting}

\subsubsection{ARM64: \NonOptimizing GCC (Linaro) 4.9}

\NonOptimizing GCC генерирует больше избыточного кода, но он работает также:

\begin{lstlisting}[caption=\NonOptimizing GCC (Linaro) 4.9,style=customasmARM]
f:
	sub	sp, sp, #32
	str	w0, [sp,12]
	ldr	w0, [sp,12]
	str	w0, [sp,28]
	ldr	w0, [sp,28]
	orr	w0, w0, 16384	; 0x4000
	str	w0, [sp,28]
	ldr	w0, [sp,28]
	and	w0, w0, -513	; 0xFFFFFFFFFFFFFDFF
	str	w0, [sp,28]
	ldr	w0, [sp,28]
	add	sp, sp, 32
	ret
\end{lstlisting}


\subsubsection{MIPS}
% FIXME better start at non-optimizing version?
Функция использует много S-регистров, которые должны быть сохранены. Вот почему их значения сохраняются
в прологе функции и восстанавливаются в эпилоге.

\lstinputlisting[caption=\Optimizing GCC 4.4.5 (IDA),style=customasmMIPS]{patterns/13_arrays/1_simple/MIPS_O3_IDA_RU.lst}

Интересная вещь: здесь два цикла и в первом не нужна переменная $i$, а нужна только переменная
$i*2$ (скачущая через 2 на каждой итерации) и ещё адрес в памяти (скачущий через 4 на каждой итерации).

Так что мы видим здесь две переменных: одна (в \$V0) увеличивается на 2 каждый раз, и вторая (в \$V1) --- на 4.

Второй цикл содержит вызов \printf. Он должен показывать значение $i$ пользователю,
поэтому здесь есть переменная, увеличивающаяся на 1 каждый раз (в \$S0), а также адрес в памяти (в \$S1) 
увеличивающийся на 4 каждый раз.

Это напоминает нам оптимизацию циклов: \myref{loop_iterators}.
Цель оптимизации в том, чтобы избавиться от операций умножения.



