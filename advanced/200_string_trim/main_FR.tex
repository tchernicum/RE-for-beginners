\mysection{Ajustement de chaînes}
\newcommand{\CRLF}{\ac{CR}/\ac{LF}}

Un traitement de chaîne très courant est la suppression de certains caractères au
début et/où à la fin.

Dans cet exemple, nous allons travailler avec une fonction qui supprime tous les
caractères newline (\CRLF{}) à la fin de la chaîne entrée:

\lstinputlisting[style=customc]{\CURPATH/strtrim_FR.c}

L'argument en entrée est toujours renvoyé en sortie, ceci est pratique lorsque vous
voulez chaîner les fonctions de traitement de chaîne, comme c'est fait ici dans la
fonction \main.

\myindex{\CLanguageElements!Short-circuit}
La seconde partie de for() (\TT{str\_len>0 \&\& (c=s[str\_len-1])}) est appelé le
\q{short-circuit} en \CCpp et est très pratique \InSqBrackets{\CNotes 1.3.8}.

Les compilateurs \CCpp garantissent une séquence d'évaluation de gauche à droite.

Donc, si la première clause est fausse après l'évaluation, la seconde n'est pas évaluée.

% subsections
\subsubsection{x64: 8 arguments}

\myindex{x86-64}
\label{example_printf8_x64}
Pour voir comment les autres arguments sont passés par la pile, changeons encore
notre exemple en augmentant le nombre d'arguments à 9 (chaîne de format de
\printf + 8 variables \Tint):

\lstinputlisting[style=customc]{patterns/03_printf/2.c}

\myparagraph{MSVC}

Comme il a déjà été mentionné, les 4 premiers arguments sont passés par les registres
\RCX, \RDX, \Reg{8}, \Reg{9} sous Win64, tandis les autres le sont---par la pile.
C'est exactement de que l'on voit ici.
Toutefois, l'instruction \MOV est utilisée ici à la place de \PUSH, donc les valeurs
sont stockées sur la pile d'une manière simple.

\lstinputlisting[caption=MSVC 2012 x64,style=customasmx86]{patterns/03_printf/x86/2_MSVC_x64_FR.asm}

Le lecteur observateur pourrait demander pourquoi 8 octets sont alloués sur la
pile pour les valeurs \Tint, alors que 4 suffisent?
Oui, il faut se rappeler: 8 octets sont alloués pour tout type de données plus
petit que 64 bits.
Ceci est instauré pour des raisons de commodités: cela rend facile le calcul
de l'adresse de n'importe quel argument.
En outre, ils sont tous situés à des adresses mémoires alignées.
Il en est de même dans les environnements 32-bit: 4 octets sont réservés pour tout
types de données.

% also for local variables?

\myparagraph{GCC}

Le tableau est similaire pour les OS x86-64 *NIX, excepté que les 6 premiers arguments
sont passés par les registres \RDI, \RSI, \RDX, \RCX, \Reg{8}, \Reg{9}.
Tout les autres---par la pile.
GCC génère du code stockant le pointeur de chaîne dans \EDI au lieu de \RDI{}---nous
l'avons noté précédemment:
\myref{hw_EDI_instead_of_RDI}.

Nous avions également noté que le registre \EAX a été vidé avant l'appel à
\printf: \myref{SysVABI_input_EAX}.

\lstinputlisting[caption=GCC 4.4.6 x64 \Optimizing,style=customasmx86]{patterns/03_printf/x86/2_GCC_x64_FR.s}

\myparagraph{GCC + GDB}
\myindex{GDB}

Essayons cet exemple dans \ac{GDB}.

\begin{lstlisting}
$ gcc -g 2.c -o 2
\end{lstlisting}

\begin{lstlisting}
$ gdb 2
GNU gdb (GDB) 7.6.1-ubuntu
...
Reading symbols from /home/dennis/polygon/2...done.
\end{lstlisting}

\begin{lstlisting}[caption=mettons le point d'arrêt à \printf{,} et lançons]
(gdb) b printf
Breakpoint 1 at 0x400410
(gdb) run
Starting program: /home/dennis/polygon/2 

Breakpoint 1, __printf (format=0x400628 "a=%d; b=%d; c=%d; d=%d; e=%d; f=%d; g=%d; h=%d\n") at printf.c:29
29	printf.c: No such file or directory.
\end{lstlisting}

Les registres \RSI/\RDX/\RCX/\Reg{8}/\Reg{9} ont les valeurs attendues.
\RIP contient l'adresse de la toute première instruction de la fonction \printf.

\begin{lstlisting}
(gdb) info registers
rax            0x0	0
rbx            0x0	0
rcx            0x3	3
rdx            0x2	2
rsi            0x1	1
rdi            0x400628	4195880
rbp            0x7fffffffdf60	0x7fffffffdf60
rsp            0x7fffffffdf38	0x7fffffffdf38
r8             0x4	4
r9             0x5	5
r10            0x7fffffffdce0	140737488346336
r11            0x7ffff7a65f60	140737348263776
r12            0x400440	4195392
r13            0x7fffffffe040	140737488347200
r14            0x0	0
r15            0x0	0
rip            0x7ffff7a65f60	0x7ffff7a65f60 <__printf>
...
\end{lstlisting}

\begin{lstlisting}[caption=inspectons la chaîne de format]
(gdb) x/s $rdi
0x400628:	"a=%d; b=%d; c=%d; d=%d; e=%d; f=%d; g=%d; h=%d\n"
\end{lstlisting}

Affichons la pile avec la commande x/g cette fois---\emph{g} est l'unité pour \emph{giant words}, i.e., mots de 64-bit.

\begin{lstlisting}
(gdb) x/10g $rsp
0x7fffffffdf38:	0x0000000000400576	0x0000000000000006
0x7fffffffdf48:	0x0000000000000007	0x00007fff00000008
0x7fffffffdf58:	0x0000000000000000	0x0000000000000000
0x7fffffffdf68:	0x00007ffff7a33de5	0x0000000000000000
0x7fffffffdf78:	0x00007fffffffe048	0x0000000100000000
\end{lstlisting}

Le tout premier élément de la pile, comme dans le cas précédent, est la \ac{RA}.
3 valeurs sont aussi passées par la pile: 6, 7, 8.
Nous voyons également que 8 est passé avec les 32-bits de poids fort non
effacés: \GTT{0x00007fff00000008}.
C'est en ordre, car les valeurs sont d'un type \Tint, qui est 32-bit.
Donc, la partie haute du registre ou l'élément de la pile peuvent contenir des
\q{restes de données aléatoires}.

Si vous regardez où le contrôle reviendra après l'exécution de \printf,
\ac{GDB} affiche la fonction \main en entier:

\begin{lstlisting}[style=customasmx86]
(gdb) set disassembly-flavor intel
(gdb) disas 0x0000000000400576
Dump of assembler code for function main:
   0x000000000040052d <+0>:	push   rbp
   0x000000000040052e <+1>:	mov    rbp,rsp
   0x0000000000400531 <+4>:	sub    rsp,0x20
   0x0000000000400535 <+8>:	mov    DWORD PTR [rsp+0x10],0x8
   0x000000000040053d <+16>:	mov    DWORD PTR [rsp+0x8],0x7
   0x0000000000400545 <+24>:	mov    DWORD PTR [rsp],0x6
   0x000000000040054c <+31>:	mov    r9d,0x5
   0x0000000000400552 <+37>:	mov    r8d,0x4
   0x0000000000400558 <+43>:	mov    ecx,0x3
   0x000000000040055d <+48>:	mov    edx,0x2
   0x0000000000400562 <+53>:	mov    esi,0x1
   0x0000000000400567 <+58>:	mov    edi,0x400628
   0x000000000040056c <+63>:	mov    eax,0x0
   0x0000000000400571 <+68>:	call   0x400410 <printf@plt>
   0x0000000000400576 <+73>:	mov    eax,0x0
   0x000000000040057b <+78>:	leave  
   0x000000000040057c <+79>:	ret    
End of assembler dump.
\end{lstlisting}

Laissons se terminer l'exécution de \printf, exécutez l'instruction mettant \EAX
à zéro, et notez que le registre \EAX à une valeur d'exactement zéro.
\RIP pointe maintenant sur l'instruction \INS{LEAVE}, i.e, la pénultième de la
fonction \main.

\begin{lstlisting}
(gdb) finish
Run till exit from #0  __printf (format=0x400628 "a=%d; b=%d; c=%d; d=%d; e=%d; f=%d; g=%d; h=%d\n") at printf.c:29
a=1; b=2; c=3; d=4; e=5; f=6; g=7; h=8
main () at 2.c:6
6		return 0;
Value returned is $1 = 39
(gdb) next
7	};
(gdb) info registers
rax            0x0	0
rbx            0x0	0
rcx            0x26	38
rdx            0x7ffff7dd59f0	140737351866864
rsi            0x7fffffd9	2147483609
rdi            0x0	0
rbp            0x7fffffffdf60	0x7fffffffdf60
rsp            0x7fffffffdf40	0x7fffffffdf40
r8             0x7ffff7dd26a0	140737351853728
r9             0x7ffff7a60134	140737348239668
r10            0x7fffffffd5b0	140737488344496
r11            0x7ffff7a95900	140737348458752
r12            0x400440	4195392
r13            0x7fffffffe040	140737488347200
r14            0x0	0
r15            0x0	0
rip            0x40057b	0x40057b <main+78>
...
\end{lstlisting}

\subsubsection{ARM64}

\myparagraph{GCC (Linaro) 4.9 \Optimizing}

\myindex{Multiplication-addition fusionnées} % TODO FIXME verify
\myindex{ARM!\Instructions!MADD}
Tout ce qu'il y a ici est simple.
\TT{MADD} est juste une instruction qui effectue une multiplication/addition fusionnées
(similaire à l'instruction \TT{MLA} que nous avons déjà vue).
Tous les 3 arguments sont passés dans la partie 32-bit de X-registres.
Effectivement, le type des arguments est \emph{int} 32-bit.
Le résultat est renvoyé dans \TT{W0}.

\lstinputlisting[caption=GCC (Linaro) 4.9 \Optimizing,style=customasmARM]{patterns/05_passing_arguments/ARM/ARM64_O3_FR.s}

Étendons le type de toutes les données à 64-bit \TT{uint64\_t} et testons:

\lstinputlisting[style=customc]{patterns/05_passing_arguments/ex64.c}

\begin{lstlisting}[style=customasmARM]
f:
	madd	x0, x0, x1, x2
	ret
main:
	mov	x1, 13396
	adrp	x0, .LC8
	stp	x29, x30, [sp, -16]!
	movk	x1, 0x27d0, lsl 16
	add	x0, x0, :lo12:.LC8
	movk	x1, 0x122, lsl 32
	add	x29, sp, 0
	movk	x1, 0x58be, lsl 48
	bl	printf
	mov	w0, 0
	ldp	x29, x30, [sp], 16
	ret

.LC8:
	.string	"%lld\n"
\end{lstlisting}

La fonction \ttf{} est la même, seulement les X-registres 64-bit sont utilisés entièrement
maintenant.
Les valeurs longues sur 64-bit sont chargées dans les registres par partie, c'est
également décrit ici: \myref{ARM_big_constants_loading}.

\myparagraph{GCC (Linaro) 4.9 \NonOptimizing}

Le code sans optimisation est plus redondant:

\begin{lstlisting}[style=customasmARM]
f:
	sub	sp, sp, #16
	str	w0, [sp,12]
	str	w1, [sp,8]
	str	w2, [sp,4]
	ldr	w1, [sp,12]
	ldr	w0, [sp,8]
	mul	w1, w1, w0
	ldr	w0, [sp,4]
	add	w0, w1, w0
	add	sp, sp, 16
	ret
\end{lstlisting}

Le code sauve ses arguments en entrée dans la pile locale, dans le cas où quelqu'un
(ou quelque chose) dans cette fonction aurait besoin d'utiliser les registres \TT{W0...W2}.
Cela évite d'écraser les arguments originels de la fonction, qui pourraient être
de nouveau utilisés par la suite.

Cela est appelé \emph{Zone de sauvegarde de registre.} (\ARMPCS).
L'appelée, toutefois, n'est pas obligée de les sauvegarder.
C'est un peu similaire au \q{Shadow Space}: \myref{shadow_space}.

Pourquoi est-ce que GCC 4.9 avec l'option d'optimisation supprime ce code de sauvegarde?
Parce qu'il a fait plus d'optimisation et en a conclu que les arguments de la fonction
n'allaient pas être utilisés par la suite et donc que les registres \TT{W0...W2}
ne vont pas être utilisés.

\myindex{ARM!\Instructions!MUL}
\myindex{ARM!\Instructions!ADD}

Nous avons donc une paire d'instructions \TT{MUL}/\TT{ADD} au lieu d'un seul \TT{MADD}.

\subsubsection{ARM}

\myindex{Linux}

Le bit \TT{O\_CREAT} est testé différemment dans le noyau Linux 3.8.0.

\begin{lstlisting}[caption=noyau Linux 3.8.0,style=customc]
struct file *do_filp_open(int dfd, struct filename *pathname,
		const struct open_flags *op)
{
...
	filp = path_openat(dfd, pathname, &nd, op, flags | LOOKUP_RCU);
...
}

static struct file *path_openat(int dfd, struct filename *pathname,
		struct nameidata *nd, const struct open_flags *op, int flags)
{
...
	error = do_last(nd, &path, file, op, &opened, pathname);
...
}


static int do_last(struct nameidata *nd, struct path *path,
		   struct file *file, const struct open_flags *op,
		   int *opened, struct filename *name)
{
...
	if (!(open_flag & O_CREAT)) {
    ...
		error = lookup_fast(nd, path, &inode);
    ...
	} else {
    ...
		error = complete_walk(nd);
        }
...
}
\end{lstlisting}

Voici à quoi ressemble le noyau compilé pour le mode ARM dans \IDA:

\lstinputlisting[caption=do\_last() dans vmlinux (IDA),style=customasmARM]{patterns/14_bitfields/1_check/IDA.lst}

\myindex{ARM!\Instructions!TST}
\myindex{x86!\Instructions!TEST}
\TT{TST} est analogue à l'instruction \TEST en x86.
Nous pouvons \q{pointer} visuellement ce morceau de code grâce au fait que la fonction
\TT{lookup\_fast()} doit être exécutée dans un cas et \TT{complete\_walk()} dans
l'autre.
Ceci correspond au code source de la fonction \TT{do\_last()}.
La macro \TT{O\_CREAT} vaut \TT{0x40} ici aussi.


\subsection{MIPS}

\lstinputlisting[caption=GCC 4.4.5 \Optimizing (IDA),style=customasmMIPS]{\CURPATH/MIPS_O3_IDA_FR.lst}

Les registres préfixés avec S- sont aussi appelés  \q{saved temporaries} (sauvé temporairement),
donc la valeur de \$S0 est sauvée dans la pile locale et restaurée à la fin.


