\subsection{ARM64: GCC (Linaro) 4.9 \NonOptimizing}

Cette implémentation est simple:

\lstinputlisting[caption=GCC (Linaro) 4.9 \NonOptimizing,style=customasmARM]{\CURPATH/GCC49_ARM64_O0_FR.s}

\subsection{ARM64: GCC (Linaro) 4.9 \Optimizing}

Ceci est une optimisation plus avancée.

Le premier caractère est chargé au début, et comparé avec 10 (le caractère \ac{LF}).

Les caractères sont ensuite chargés dans la boucle principale, pour les caractères
après le premier.

Ceci est quelque peu similaire à l'exemple \myref{string_trim_GCC_x64_O3}.

\lstinputlisting[caption=GCC (Linaro) 4.9 \Optimizing,style=customasmARM]{\CURPATH/GCC49_ARM64_O3_FR.s}

