\subsection{MIPS}

Pour une raison quelconque, GCC 4.4.5 avec optimisation génère seulement une instruction
de division:

\lstinputlisting[caption=\Optimizing GCC 4.4.5 (IDA),style=customasmMIPS]{\CURPATH/MIPS_O3_IDA_FR.lst}

\myindex{MIPS!\Instructions!BREAK}
Ici, nous voyons une nouvelle instruction: BREAK. Elle lève simplement une exception.

Dans ce cas, une exception est levée si le diviseur est zéro (il n'est pas possible
de diviser par zéro dans les mathématiques conventionnelles).

Mais GCC n'a probablement pas fait correctement le travail d'optimisation et n'a
pas vu que \$V0 ne vaut jamais zéro.

Donc le test est laissé ici.
Donc, si \$V0 est zéro, BREAK est exécuté, signalant l'exception à l'\ac{OS}.

\myindex{MIPS!\Instructions!MFLO}
Autrement, MFLO s'exécute, qui prend le résultat de la division depuis le registre
LO et le copie dans \$V0.

\myindex{MIPS!\Registers!LO}
\myindex{MIPS!\Registers!HI}
À propos, comme on devrait le savoir, l'instruction MUL laisse les 32bits hauts du
résultat dans le registre HI et les 32 bits bas dans le registre LO.

DIV laisse le résultat dans le registre LO, et le reste dans le registre HI.

\myindex{MIPS!\Instructions!MFHI}
Si nous modifions la déclaration en \q{a \% 9},
l'instruction MFHI est utilisée au lieu de MFLO.
