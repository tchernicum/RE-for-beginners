\mysection{Temperature converting}

Another very popular example in 
programming books for beginners is a small program that converts Fahrenheit temperature to Celsius or back.

\[
	C=\frac{5 \cdot (F-32)}{9}
\]

We can also add simple error handling:
1) we must check if the user has entered a correct number;
2) we must check if the Celsius temperature is not below $-273$ 
(which is below absolute zero, as we may recall from school physics lessons).

\myindex{\CStandardLibrary!exit()}
The \TT{exit()} function terminates 
the program instantly, without returning to the \gls{caller} function.

\subsection{Integer values}

\lstinputlisting[style=customc]{\CURPATH/i.c}

\subsubsection{\Optimizing MSVC 2012 x86}

\lstinputlisting[caption=\Optimizing MSVC 2012 x86,style=customasmx86]{\CURPATH/i_MSVC_2012_Ox_x86_EN.asm}

What we can say about it:

\begin{itemize}
\item The address of \printf is first loaded in the 
\ESI register, so the subsequent
\printf calls are done just by the \TT{CALL ESI} instruction.
It's a very popular compiler 
technique, possible if several consequent calls to the same function are present
in the code, and/or if there is a free register which can be used for this.

\item We see the \TT{ADD EAX, -32} 
instruction at the place where 32 has to be subtracted from the value.
$EAX=EAX+(-32)$ is equivalent to $EAX=EAX-32$ 
and somehow, the compiler decided to use \TT{ADD} instead of \TT{SUB}.
Maybe it's worth it, it's hard to be sure.

\myindex{x86!\Instructions!LEA}
\item The \LEA instruction is used when 
the value is to be multiplied by 5: \TT{lea ecx, DWORD PTR [eax+eax*4]}.
Yes, $i+i*4$ is equivalent to $i*5$ and \LEA 
works faster then \TT{IMUL}.\\
By the way, the \TT{SHL EAX, 2 / ADD EAX, EAX} instruction pair could be also used here instead---
some compilers do it like.

\item The division by multiplication trick (\myref{sec:divisionbymult}) 
is also used here.

\item \main returns 0 if we don't have \TT{return 0} 
at its end.
The C99 standard tells us \InSqBrackets{\CNineNineStd 5.1.2.2.3} that \main 
will return 0 in case the 
\TT{return} statement is missing.
This rule works only for the \main function.

Though, MSVC doesn't officially support C99, but maybe it support it partially?
\end{itemize}

\subsubsection{\Optimizing MSVC 2012 x64}

The code is almost the same, but we can find \TT{INT 3} instructions after each \TT{exit()} call.

\begin{lstlisting}[style=customasmx86]
	xor	ecx, ecx
	call	QWORD PTR __imp_exit
	int	3
\end{lstlisting}

\TT{INT 3} is a debugger breakpoint.

It is known that \TT{exit()} is one of the functions which can never return
\footnote{another popular one is \TT{longjmp()}},
so if it does, something really odd has happened and it's time to load the debugger.

\subsection{Floating-point values}

\lstinputlisting[style=customc]{\CURPATH/f.c}

MSVC 2010 x86 uses \ac{FPU} instructions\dots

\lstinputlisting[caption=\Optimizing MSVC 2010 x86,style=customasmx86]{\CURPATH/f_MSVC_2010_x86_Ox_EN.asm}

\dots but MSVC  2012 uses \ac{SIMD} instructions instead:

\lstinputlisting[caption=\Optimizing MSVC 2010 x86,style=customasmx86]{\CURPATH/f_MSVC_2012_x86_Ox_EN.asm}

Of course, \ac{SIMD} instructions are available in x86 mode, 
including those working with floating point numbers.

It's somewhat easier to use them for calculations, so the new Microsoft compiler uses them.

We can also see that the $-273$ value 
is loaded into \XMM{0} register too early.
And that's OK, because the compiler may emit instructions not in 
the order they are in the source code.
