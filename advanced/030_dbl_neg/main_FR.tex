\mysection{Double négation}

Une façon répandue\footnote{C'est sujet à controverse, car ça conduit à du code difficile
à lire} de convertir des valeurs différentes de zéro en 1 (ou le booléen \emph{true})
et la valeur zéro en 0 (ou le booléen \emph{false}) est la déclaration \emph{!!variable}:

\begin{lstlisting}[style=customc]
int convert_to_bool(int a)
{
	return !!a;
};
\end{lstlisting}

GCC 5.4 x86 avec optimisation:

\begin{lstlisting}[style=customasmx86]
convert_to_bool:
        mov     edx, DWORD PTR [esp+4]
        xor     eax, eax
        test    edx, edx
        setne   al
        ret
\end{lstlisting}

\INS{XOR} efface toujours la valeur de retour dans \EAX, même si \INS{SETNE} n'est
pas déclenché. I.e., \INS{XOR} met la valeur de retour par défaut à zéro.

\myindex{x86!\Instructions!SET}
Si la valeur en entrée n'est pas égale à zéro (le suffixe \TT{-NE} dans l'instruction
\INS{SET}), 1 est mis dans \AL, autrement \AL n'est pas modifié.

Pourquoi est-ce que \INS{SETNE} opère sur la partie 8-bit basse du registre \EAX?
Parce que ce qui compte c'est juste le dernier bit (0 or 1), puisque les autres bits
sont mis à zéro par \INS{XOR}.

Ainsi, ce code C/C++ peut être récrit comme ceci:

\begin{lstlisting}[style=customc]
int convert_to_bool(int a)
{
	if (a!=0)
		return 1;
	else
		return 0;
};
\end{lstlisting}

\dots ou même:

\begin{lstlisting}[style=customc]
int convert_to_bool(int a)
{
	if (a)
		return 1;
	else
		return 0;
};
\end{lstlisting}

Les compilateurs visant des \ac{CPU}s n'ayant pas d'instructions similaires à \INS{SET},
génèrent dans ce cas des instructions de branchement, etc.
