\mysection{Exemple strstr()}
\label{strstr_example}
\myindex{\CStandardLibrary!strstr()}

Revenons au fait que GCC peut parfois utiliser une partie d'une chaîne de caractères:
\myref{use_parts_of_C_strings}.

La fonction \emph{strstr()} de la bibliothèque standard \CCpp est utilisée pour trouver
une occurrence dans une chaîne. C'est ce que nous voulons faire:

\begin{lstlisting}[style=customc]
#include <string.h>
#include <stdio.h>

int main()
{
	char *s="Hello, world!";
	char *w=strstr(s, "world");

	printf ("%p, [%s]\n", s, s);
	printf ("%p, [%s]\n", w, w);
};
\end{lstlisting}

La sortie est:

\begin{lstlisting}
0x8048530, [Hello, world!]
0x8048537, [world!]
\end{lstlisting}

La différence entre l'adresse de la chaîne originale et l'adresse de la sous-chaîne
que \emph{strstr()} a renvoyé est 7.
En effet, la chaîne \q{Hello, } a une longueur de 7 caractères.

La fonction \printf{} lors du second appel n'a aucune idée qu'il y a des autres caractères
avant la chaîne passée et elle affiche des caractères depuis le milieu de la chaîne
originale jusqu'à la fin (marquée par un octet à zéro).

