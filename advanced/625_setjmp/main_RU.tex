\mysection{setjmp/longjmp}

\myindex{\CStandardLibrary!setjmp}
\myindex{\CStandardLibrary!longjmp}

setjmp/longjmp это механизм в Си, очень похожий на throw/catch в Си++ и других высокоуровневых \ac{PL}.
\myindex{zlib}
Вот пример из zlib:

\begin{lstlisting}[style=customc]
...

    /* return if bits() or decode() tries to read past available input */
    if (setjmp(s.env) != 0)             /* if came back here via longjmp(), */
        err = 2;                        /*  then skip decomp(), return error */
    else
	err = decomp(&s); /* decompress */

...

    /* load at least need bits into val */
    val = s->bitbuf;
    while (s->bitcnt < need) {
        if (s->left == 0) {
            s->left = s->infun(s->inhow, &(s->in));
	    if (s->left == 0) longjmp(s->env, 1); /* out of input */

...

        if (s->left == 0) {
            s->left = s->infun(s->inhow, &(s->in));
	    if (s->left == 0) longjmp(s->env, 1); /* out of input */
\end{lstlisting}
( zlib/contrib/blast/blast.c )

Вызов \TT{setjmp()} сохраняет текущие \ac{PC}, \ac{SP} и другие регистры в структуре \TT{env}, затем возвращает 0.

В слуае ошибки, \TT{longjmp()} \emph{телепортирует} вас в место точно после вызова \TT{setjmp()},
как если бы вызов \TT{setjmp()} вернул ненулевое значение (которое было передано в \TT{longjmp()}).
\myindex{UNIX!fork}
Это напоминает нам сисколл \TT{fork()} в UNIX.

Посмотрим в более дистиллированный пример:

\lstinputlisting[style=customc]{advanced/625_setjmp/1.c}

Если запустим, то увидим:

\begin{lstlisting}
f1() begin
f2() begin
Error 1234
\end{lstlisting}

Структура \TT{jmp\_buf} обычно недокументирована, чтобы сохранить прямую совместимость.

Посмотрим, как setjmp() реализован в MSVC 2013 x64:

\lstinputlisting[style=customasmx86]{advanced/625_setjmp/setjmp_VC2013_x64.asm}

Она просто заполняет структуру jmp\_buf текущими значениями почти всех регистров.
Также, текущее значение \ac{RA} берется из стека и сохраняется в jmp\_buf:
В будущем, оно будет испоьзовано как новое значение \ac{PC}.

Теперь longjmp():

\lstinputlisting[style=customasmx86]{advanced/625_setjmp/longjmp_VC2013_x64.asm}

Она просто восстанавливает (почти) все регистры, берет из структуры \ac{RA} и переходит туда.
Эффект такой же, как если бы setjmp() вернула управление в вызывающую ф-цию.
Также, \TT{RAX} выставляется такой же, как и второй аргумент longjmp().
Это работает, как если бы setjmp() вернуло ненулевое значение в самом начале.

Как побочный эффект восстановления \ac{SP}, все значения в стеке, которые были установлены и использованы между
вызовами setjmp() и longjmp(), просто выкидываются.
Они больше не будут использоваться.
Следовательно, longjmp() обычно делает переход назад
\footnote{Впрочем, существуют люди, которые используют всё это для куда более сложных вещей, включая имитацию копроцедур, итд:
\url{https://www.embeddedrelated.com/showarticle/455.php},
\url{http://fanf.livejournal.com/105413.html}}.

Это подразумевает, что в отличии от механизма throw/catch в Си++, память не будет освобождаться,
деструкторы не будут вызываться, итд.
Следовательно, эта техника иногда опасна.
Тем не менее, всё это довольно популярно, до сих пор. Это все еще используется в \oracle.

Это также имеет неожиданный побочный эффект: если некий буфер был перезаписан внутри ф-ции (может даже из-за удаленной атаки),
и ф-ция хочет сообщить об ошибке, и вызывает longjmp(), перезаписанная часть стека становится просто неиспользованной.

В качестве упражнения, попробуйте понять, почему не все регистры сохраняются.
Почему пропускаются регистры XMM0-XMM5 и другие?

