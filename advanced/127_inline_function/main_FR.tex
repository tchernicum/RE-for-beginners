\mysection{Fonctions inline}
\myindex{Inline code}
\label{inline_code}

Le code inline, c'est lorsque le compilateur, au lieu de mettre une instruction
d'appel à une petite ou à une minuscule fonction, copie son corps à la place.

\lstinputlisting[caption=Un exemple simple,style=customc]{\CURPATH/1.c}

\dots est compilée de façon très prédictive, toutefois, si nous utilisons l'option
d'optimisation de GCC (\Othree), nous voyons:

\lstinputlisting[caption=GCC 4.8.1 \Optimizing,style=customasmx86]{\CURPATH/1.s}

(Ici la division est effectuée avec une multiplication(\myref{sec:divisionbymult}).)

Oui, notre petite fonction \TT{celsius\_to\_fahrenheit()} a été placée juste avant
l'appel à \printf.

Pourquoi? C'est plus rapide que d'exécuter la code de cette fonction plus le surcoût
de l'appel/retour.

Les optimiseurs des compilateurs modernes choisissent de mettre en ligne les petites
fonctions automatiquement.
Mais il est possible de forcer le compilateur à mettre en ligne automatiquement certaines
fonctions, en les marquants avec le mot clef \q{inline} dans sa déclaration.

% sections
\EN{\mysection{Unions}

\CCpp \emph{union} is mostly used for interpreting a variable (or memory block) of one data type as a variable of another data type.

% sections
\EN{\input{patterns/17_unions/FPU_PRNG/main_EN}}
\RU{\input{patterns/17_unions/FPU_PRNG/main_RU}}
\DE{\input{patterns/17_unions/FPU_PRNG/main_DE}}
\FR{\input{patterns/17_unions/FPU_PRNG/main_FR}}
\JA{\input{patterns/17_unions/FPU_PRNG/main_JA}}


\EN{\input{patterns/17_unions/epsilon/main_EN}}
\RU{\input{patterns/17_unions/epsilon/main_RU}}
\DE{\input{patterns/17_unions/epsilon/main_DE}}
\FR{\input{patterns/17_unions/epsilon/main_FR}}
\JA{\input{patterns/17_unions/epsilon/main_JA}}

\subsection{FSCALE instruction replacement}
\myindex{x86!\Instructions!FSCALE}

Agner Fog in his \emph{Optimizing subroutines in assembly language / An optimization guide for x86 platforms} work
\footnote{\url{http://www.agner.org/optimize/optimizing_assembly.pdf}} states that \INS{FSCALE} \ac{FPU} instruction
(calculating $2^n$) may be slow on many CPUs, and he offers faster replacement.

Here is my translation of his assembly code to \CCpp:

\lstinputlisting[style=customc]{patterns/17_unions/FSCALE.c}

\INS{FSCALE} instruction may be faster in your environment, but still, it's a good example of \emph{union}'s and the fact
that exponent is stored in $2^n$ form,
so an input $n$ value is shifted to the exponent in IEEE 754 encoded number.
Then exponent is then corrected with addition of 0x3f800000 or 0x3ff0000000000000.

The same can be done without shift using \emph{struct}, but internally, shift operations still occurred.



\subsection{Fast square root calculation}

Another well-known algorithm where \Tfloat is interpreted as integer is fast calculation of square root.

\begin{lstlisting}[caption=The source code is taken from Wikipedia: \url{http://go.yurichev.com/17364},style=customc]
/* Assumes that float is in the IEEE 754 single precision floating point format
 * and that int is 32 bits. */
float sqrt_approx(float z)
{
    int val_int = *(int*)&z; /* Same bits, but as an int */
    /*
     * To justify the following code, prove that
     *
     * ((((val_int / 2^m) - b) / 2) + b) * 2^m = ((val_int - 2^m) / 2) + ((b + 1) / 2) * 2^m)
     *
     * where
     *
     * b = exponent bias
     * m = number of mantissa bits
     *
     * .
     */
 
    val_int -= 1 << 23; /* Subtract 2^m. */
    val_int >>= 1; /* Divide by 2. */
    val_int += 1 << 29; /* Add ((b + 1) / 2) * 2^m. */
 
    return *(float*)&val_int; /* Interpret again as float */
}
\end{lstlisting}

As an exercise, you can try to compile this function and to understand, how it works. \\
\\
There is also well-known algorithm of fast calculation of $\frac{1}{\sqrt{x}}$.
\myindex{Quake III Arena}
Algorithm became popular, supposedly, because it was used in Quake III Arena.

Algorithm description can be found in Wikipedia: \url{http://go.yurichev.com/17360}.

}
\RU{\mysection{Объединения (union)}

\emph{union} в \CCpp используется в основном для интерпретации переменной (или блока памяти) одного типа как переменной другого типа.

% sections
\EN{\input{patterns/17_unions/FPU_PRNG/main_EN}}
\RU{\input{patterns/17_unions/FPU_PRNG/main_RU}}
\DE{\input{patterns/17_unions/FPU_PRNG/main_DE}}
\FR{\input{patterns/17_unions/FPU_PRNG/main_FR}}
\JA{\input{patterns/17_unions/FPU_PRNG/main_JA}}


\EN{\input{patterns/17_unions/epsilon/main_EN}}
\RU{\input{patterns/17_unions/epsilon/main_RU}}
\DE{\input{patterns/17_unions/epsilon/main_DE}}
\FR{\input{patterns/17_unions/epsilon/main_FR}}
\JA{\input{patterns/17_unions/epsilon/main_JA}}

\subsection{Замена инструкции FSCALE}
\myindex{x86!\Instructions!FSCALE}

Agner Fog в своей работе \emph{Optimizing subroutines in assembly language / An optimization guide for x86 platforms}
\footnote{\url{http://www.agner.org/optimize/optimizing_assembly.pdf}} указывает, что инструкция \ac{FPU} \INS{FSCALE}
(вычисление $2^n$) может быть медленной на многих CPU, и он предлагает более быструю замену.

Вот мой перевод его кода на ассемблер на \CCpp:

\lstinputlisting[style=customc]{patterns/17_unions/FSCALE.c}

Инструкция \INS{FSCALE} в вашей среде может быть быстрее, но всё же, это хорошая демонстрация \emph{union}-а и того факта,
что экспонента хранится в виде $2^n$,
так что входное значение $n$ сдвигается в экспоненту закодированного в IEEE 754 числа.
Потом экспонента корректируется прибавлением 0x3f800000 или 0x3ff0000000000000.

То же самое можно сделать без сдвигов, при помощи \emph{struct}, но всё равно, внутри будет операция.



\subsection{Быстрое вычисление квадратного корня}

Вот где еще можно на практике применить трактовку типа \Tfloat как целочисленного, это быстрое вычисление квадратного корня.

\begin{lstlisting}[caption=Исходный код взят из Wikipedia: \url{http://go.yurichev.com/17364},style=customc]
/* Assumes that float is in the IEEE 754 single precision floating point format
 * and that int is 32 bits. */
float sqrt_approx(float z)
{
    int val_int = *(int*)&z; /* Same bits, but as an int */
    /*
     * To justify the following code, prove that
     *
     * ((((val_int / 2^m) - b) / 2) + b) * 2^m = ((val_int - 2^m) / 2) + ((b + 1) / 2) * 2^m)
     *
     * where
     *
     * b = exponent bias
     * m = number of mantissa bits
     *
     * .
     */
 
    val_int -= 1 << 23; /* Subtract 2^m. */
    val_int >>= 1; /* Divide by 2. */
    val_int += 1 << 29; /* Add ((b + 1) / 2) * 2^m. */
 
    return *(float*)&val_int; /* Interpret again as float */
}
\end{lstlisting}

В качестве упражнения, вы можете попробовать скомпилировать эту функцию и разобраться, как она работает. \\
\\
Имеется также известный алгоритм быстрого вычисления $\frac{1}{\sqrt{x}}$.
\myindex{Quake III Arena}
Алгоритм стал известным, вероятно потому, что был применен в Quake III Arena.

Описание алгоритма есть в Wikipedia: \url{http://go.yurichev.com/17361}.
}
\DE{\mysection{Unions}

Die \\Cpp \emph{union} wird hauptsächlich verwendet um eine Variable (oder einen Speicherblock) eines Datentyps als
Variable eines anderen Datentyps zu interpretieren.

% sections
\EN{\input{patterns/17_unions/FPU_PRNG/main_EN}}
\RU{\input{patterns/17_unions/FPU_PRNG/main_RU}}
\DE{\input{patterns/17_unions/FPU_PRNG/main_DE}}
\FR{\input{patterns/17_unions/FPU_PRNG/main_FR}}
\JA{\input{patterns/17_unions/FPU_PRNG/main_JA}}


\EN{\input{patterns/17_unions/epsilon/main_EN}}
\RU{\input{patterns/17_unions/epsilon/main_RU}}
\DE{\input{patterns/17_unions/epsilon/main_DE}}
\FR{\input{patterns/17_unions/epsilon/main_FR}}
\JA{\input{patterns/17_unions/epsilon/main_JA}}

\subsection{FSCALE Ersatz}
\myindex{x86!\Instructions!FSCALE}
Agner Fog schreibt in seiner Abhandlung \emph{Optimizing subroutines in assembly language / An optimization guide for x86
platforms} \footnote{\url{http://www.agner.org/optimize/optimizing_assembly.pdf}} , dass der Befehl \INS{FSCALE}
\ac{FPU} (der $2^n$ berechnet) auf vielen CPUs langsam ist und bietet einen schnelleren Ersatz an.

Hier ist meine Übersetzung von seinem Assemblercode in \CCpp:

\lstinputlisting[style=customc]{patterns/17_unions/FSCALE.c}
Der Befehl \INS{FSCALE} kann zwar in bestimmten Umgebungen schneller sein, ist aber vor allem ein gutes Beispiel für
\emph{unions} und die Tatsache, dass der Exponent in der Form $2^n$ gespeichert wird, sodass ein Eingabewert $n$ zum
Exponenten nach IEEE 754 Standard verschoben wird.
Der Exponent wird dann durch Addition von 0x3f800000 oder 0x3ff0000000000000 korrigiert.

Das gleiche kann ohne Verschiebung durch ein \emph{struct} erreicht werden, aber intern werden stets Schiebebefehle
verwendet.


\subsection{Schnelle Berechnung der Quadratwurzel}

Ein anderer bekannter Algorithmus, in dem \Tfloat als \Tint interpretiert wird, ist die schnelle Berechnung einer
Quadratwurzel.

\begin{lstlisting}[caption=Quellcode stammt aus der Wikipedia: \url{http://go.yurichev.com/17364},style=customc]
/* Assumes that float is in the IEEE 754 single precision floating point format
 * and that int is 32 bits. */
float sqrt_approx(float z)
{
    int val_int = *(int*)&z; /* Same bits, but as an int */
    /*
     * To justify the following code, prove that
     *
     * ((((val_int / 2^m) - b) / 2) + b) * 2^m = ((val_int - 2^m) / 2) + ((b + 1) / 2) * 2^m)
     *
     * where
     *
     * b = exponent bias
     * m = number of mantissa bits
     *
     * .
     */
 
    val_int -= 1 << 23; /* Subtract 2^m. */
    val_int >>= 1; /* Divide by 2. */
    val_int += 1 << 29; /* Add ((b + 1) / 2) * 2^m. */
 
    return *(float*)&val_int; /* Interpret again as float */
}
\end{lstlisting}

Versuchen Sie als Übung, diese Funktion zu kompilieren und zu verstehen wie sie funktioniert.\\\\
Es gibt auch einen bekannten Algorithmus zur schnellen Berechnung von $\frac{1}{\sqrt{x}}$.
\myindex{Quake III Arena}
Der Algorithmus wurde vermutlich so populär, weil er in Quake III Arena verwendet wurde.
Eine Beschreibung des Algorithmus' findet man bei Wikipedia: \url{http://go.yurichev.com/17360}.
}
\FR{\mysection{Unions}

Les \emph{unions} en \CCpp sont utilisées principalement pour interpréter une variable
(ou un bloc de mémoire) d'un type de données comme une variable d'un autre type de données.

% sections
\EN{\input{patterns/17_unions/FPU_PRNG/main_EN}}
\RU{\input{patterns/17_unions/FPU_PRNG/main_RU}}
\DE{\input{patterns/17_unions/FPU_PRNG/main_DE}}
\FR{\input{patterns/17_unions/FPU_PRNG/main_FR}}
\JA{\input{patterns/17_unions/FPU_PRNG/main_JA}}


\EN{\input{patterns/17_unions/epsilon/main_EN}}
\RU{\input{patterns/17_unions/epsilon/main_RU}}
\DE{\input{patterns/17_unions/epsilon/main_DE}}
\FR{\input{patterns/17_unions/epsilon/main_FR}}
\JA{\input{patterns/17_unions/epsilon/main_JA}}

\subsection{Remplacement de FSCALE}
\myindex{x86!\Instructions!FSCALE}

Agner Fog dans son travail\footnote{\url{http://www.agner.org/optimize/optimizing_assembly.pdf}}
\emph{Optimizing subroutines in assembly language / An optimization guide for x86 platforms}
indique que l'instruction \ac{FPU} \INS{FSCALE} (qui calcule $2^n$) peut être lente
sur de nombreux CPUs, et propose un remplacement plus rapide.

Voici ma conversion de son code assembleur en \CCpp:

\lstinputlisting[style=customc]{patterns/17_unions/FSCALE.c}

L'instruction \INS{FSCALE} peut être plus rapide dans votre environnement, mais néanmoins,
c'est un bon exemple d'\emph{union} et du fait que l'exposant est stocké sous la forme
$2^n$, donc une valeur $n$ en entrée est décalée à l'exposant dans le nombre encodé
en IEEE 754.
Ensuite, l'exposant est corrigé avec l'ajout de 0x3f800000 ou de 0x3ff0000000000000.

La même chose peut être faite sans décalage utilisant \emph{struct}, mais en interne,
l'opération de décalage aura toujours lieu.



\subsection{\FRph{}}

Un autre algorithme connu où un \Tfloat est interprété comme un entier est celui
de calcul rapide de racine carrée.

\begin{lstlisting}[caption=Le code source provient de Wikipedia: \url{http://go.yurichev.com/17364},style=customc]
/* Assumes that float is in the IEEE 754 single precision floating point format
 * and that int is 32 bits. */
float sqrt_approx(float z)
{
    int val_int = *(int*)&z; /* Same bits, but as an int */
    /*
     * To justify the following code, prove that
     *
     * ((((val_int / 2^m) - b) / 2) + b) * 2^m = ((val_int - 2^m) / 2) + ((b + 1) / 2) * 2^m)
     *
     * where
     *
     * b = exponent bias
     * m = number of mantissa bits
     *
     * .
     */
 
    val_int -= 1 << 23; /* Subtract 2^m. */
    val_int >>= 1; /* Divide by 2. */
    val_int += 1 << 29; /* Add ((b + 1) / 2) * 2^m. */
 
    return *(float*)&val_int; /* Interpret again as float */
}
\end{lstlisting}

À titre d'exercice, vous pouvez essayez de compiler cette fonction et de comprendre
comme elle fonctionne.\\
\\
C'est un algorithme connu de calcul rapide de $\frac{1}{\sqrt{x}}$.
\myindex{Quake III Arena}
L'algorithme devînt connu, supposément, car il a été utilisé dans Quake III Arena.

La description de l'algorithme peut être trouvée sur Wikipédia: \url{http://go.yurichev.com/17360}.

}
\JA{\mysection{共用体}

\CCpp \emph{共用体}は、あるデータ型の変数(またはメモリブロック)を別のデータ型の変数として解釈するために使用されます。

% sections
\EN{\input{patterns/17_unions/FPU_PRNG/main_EN}}
\RU{\input{patterns/17_unions/FPU_PRNG/main_RU}}
\DE{\input{patterns/17_unions/FPU_PRNG/main_DE}}
\FR{\input{patterns/17_unions/FPU_PRNG/main_FR}}
\JA{\input{patterns/17_unions/FPU_PRNG/main_JA}}


\EN{\input{patterns/17_unions/epsilon/main_EN}}
\RU{\input{patterns/17_unions/epsilon/main_RU}}
\DE{\input{patterns/17_unions/epsilon/main_DE}}
\FR{\input{patterns/17_unions/epsilon/main_FR}}
\JA{\input{patterns/17_unions/epsilon/main_JA}}

\subsection{FSCALE replacement}
\myindex{x86!\Instructions!FSCALE}

Agner Fog氏による\emph{Optimizing subroutines in assembly language / An optimization guide for x86 platforms}では
\footnote{\url{http://www.agner.org/optimize/optimizing_assembly.pdf}}、多くのCPUでは\INS{FSCALE} \ac{FPU}命令
($2^n$の計算)が遅くなる可能性があると述べ、より速いものを提案しています。

これが私のアセンブリコードの \CCpp への翻訳です。

\lstinputlisting[style=customc]{patterns/17_unions/FSCALE.c}

\INS{FSCALE}命令はあなたの環境ではより速いかもしれませんが、それでも、それは\emph{共用体}の良い例であり、
指数が$2^n$形式で格納されるという事実です。
そのため、入力された$n$の値はIEEE 754形式で符号化された数の指数にシフトされます。
その後、0x3f800000または0x3ff0000000000000を追加して指数を補正します。

\emph{構造体}を使用してシフトなしで同じことを実行できますが、それでも内部ではシフト操作が発生しました。


\subsection{\JAph{}}

\Tfloat が整数として解釈される別のよく知られたアルゴリズムは平方根の高速計算です。

\begin{lstlisting}[caption=ソースコードはウィキペディアから取りました: \url{http://go.yurichev.com/17364},style=customc]
/* Assumes that float is in the IEEE 754 single precision floating point format
 * and that int is 32 bits. */
float sqrt_approx(float z)
{
    int val_int = *(int*)&z; /* Same bits, but as an int */
    /*
     * To justify the following code, prove that
     *
     * ((((val_int / 2^m) - b) / 2) + b) * 2^m = ((val_int - 2^m) / 2) + ((b + 1) / 2) * 2^m)
     *
     * where
     *
     * b = exponent bias
     * m = number of mantissa bits
     *
     * .
     */
 
    val_int -= 1 << 23; /* Subtract 2^m. */
    val_int >>= 1; /* Divide by 2. */
    val_int += 1 << 29; /* Add ((b + 1) / 2) * 2^m. */
 
    return *(float*)&val_int; /* Interpret again as float */
}
\end{lstlisting}

演習として、この関数をコンパイルして、その機能を理解することを試みることができます。\\
\\
$\frac{1}{\sqrt{x}}$の高速計算のよく知られたアルゴリズムもあります。
\myindex{Quake III Arena}
Quake III Arenaで使用されていたため、アルゴリズムが普及したと考えられます。

アルゴリズムの説明はWikipediaにあります:\url{http://go.yurichev.com/17360}

}

