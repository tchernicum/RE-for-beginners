\subsection{Pointeur sur une fonction}

Un nom de fonction en \CCpp sans parenthèses, comme ``printf'' est un pointeur sur
une fonction du type \emph{void (*)()}.
Essayons de lire le contenu de la fonction et de la patcher:

\lstinputlisting[style=customc]{advanced/450_more_ptrs/6.c}

Ça dit que les 3 premiers octets de la fonction sont \TT{55 89 e5}.
En effet, ce sont les opcodes des instructions \INS{PUSH EBP} et \INS{MOV EBP, ESP}
(se sont des opcodes x86).
Mais alors notre programme plante, car la section \emph{text} est en lecture seule.

Nous pouvons recompiler notre exemple et rendre la section \emph{text} modifiable
\footnote{\url{http://stackoverflow.com/questions/27581279/make-text-segment-writable-elf}}:

\begin{lstlisting}
gcc --static -g -Wl,--omagic -o example example.c
\end{lstlisting}

Ça fonctionne!

\begin{lstlisting}
we are in print_something()
first 3 bytes: 55 89 e5...
going to call patched print_something():
it must exit at this point
\end{lstlisting}

