\subsection{Abus de pointeurs dans le noyau Windows}

La section ressource d'un exécutable PE dans les OS Windows est une section contenant
des images, des icônes, des chaînes, etc.
Les premières versions de Windows permettaient seulement d'adresser les ressources
par ID, mais Microsoft a ajouté un moyen de les adresser en utilisant des chaînes.

Donc, il doit être alors possible de passer un ID ou une chaîne a la fonction
Elle est déclarée comme ceci:
\href{https://msdn.microsoft.com/en-us/library/windows/desktop/ms648042i\%28v=vs.85\%29.aspx}{FindResource()}.

\myindex{win32!FindResource()}

\begin{lstlisting}[style=customc]
HRSRC WINAPI FindResource(
  _In_opt_ HMODULE hModule,
  _In_     LPCTSTR lpName,
  _In_     LPCTSTR lpType
);
\end{lstlisting}

\emph{lpName} et \emph{lpType} ont un type \emph{char*} ou \emph{wchar*}, et lorsque quelqu'un 
veut encore passer un ID, il doit utiliser la macro
\href{https://msdn.microsoft.com/en-us/library/windows/desktop/ms648029\%28v=vs.85\%29.aspx}{MAKEINTRESOURCE},
comme ceci:

\myindex{win32!MAKEINTRESOURCE()}

\begin{lstlisting}[style=customc]
result = FindResource(..., MAKEINTRESOURCE(1234), ...);
\end{lstlisting}

C'est un fait intéressant que MAKEINTRESOURCE est juste un casting d'entier vers un pointeur.
Dans MSVC 2013, dans le fichier\\\emph{Microsoft SDKs\textbackslash{}Windows\textbackslash{}v7.1A\textbackslash{}Include\textbackslash{}Ks.h}
nous pouvons voir ceci:

\begin{lstlisting}[style=customc]
...

#if (!defined( MAKEINTRESOURCE )) 
#define MAKEINTRESOURCE( res ) ((ULONG_PTR) (USHORT) res)
#endif

...
\end{lstlisting}

Ça semble fou. Regardons dans l'ancien code source de Windows NT4 qui avait fuité.\\
Dans \emph{private/windows/base/client/module.c} nous pouvons trouver le source code
de \emph{FindResource()}:

\begin{lstlisting}[style=customc]
HRSRC
FindResourceA(
    HMODULE hModule,
    LPCSTR lpName,
    LPCSTR lpType
    )

...

{
    NTSTATUS Status;
    ULONG IdPath[ 3 ];
    PVOID p;

    IdPath[ 0 ] = 0;
    IdPath[ 1 ] = 0;
    try {
        if ((IdPath[ 0 ] = BaseDllMapResourceIdA( lpType )) == -1) {
            Status = STATUS_INVALID_PARAMETER;
            }
        else
        if ((IdPath[ 1 ] = BaseDllMapResourceIdA( lpName )) == -1) {
            Status = STATUS_INVALID_PARAMETER;
...
\end{lstlisting}

Continuons avec \emph{BaseDllMapResourceIdA()} dans le même fichier source:

\begin{lstlisting}[style=customc]
ULONG
BaseDllMapResourceIdA(
    LPCSTR lpId
    )
{
    NTSTATUS Status;
    ULONG Id;
    UNICODE_STRING UnicodeString;
    ANSI_STRING AnsiString;
    PWSTR s;

    try {
        if ((ULONG)lpId & LDR_RESOURCE_ID_NAME_MASK) {
            if (*lpId == '#') {
                Status = RtlCharToInteger( lpId+1, 10, &Id );
                if (!NT_SUCCESS( Status ) || Id & LDR_RESOURCE_ID_NAME_MASK) {
                    if (NT_SUCCESS( Status )) {
                        Status = STATUS_INVALID_PARAMETER;
                        }
                    BaseSetLastNTError( Status );
                    Id = (ULONG)-1;
                    }
                }
            else {
                RtlInitAnsiString( &AnsiString, lpId );
                Status = RtlAnsiStringToUnicodeString( &UnicodeString,
                                                       &AnsiString,
                                                       TRUE
                                                     );
                if (!NT_SUCCESS( Status )){
                    BaseSetLastNTError( Status );
                    Id = (ULONG)-1;
                    }
                else {
                    s = UnicodeString.Buffer;
                    while (*s != UNICODE_NULL) {
                        *s = RtlUpcaseUnicodeChar( *s );
                        s++;
                        }

                    Id = (ULONG)UnicodeString.Buffer;
                    }
                }
            }
        else {
            Id = (ULONG)lpId;
            }
        }
    except (EXCEPTION_EXECUTE_HANDLER) {
        BaseSetLastNTError( GetExceptionCode() );
        Id =  (ULONG)-1;
        }
    return Id;
}
\end{lstlisting}

\emph{lpId} est ANDé avec \emph{LDR\_RESOURCE\_ID\_NAME\_MASK}. \\
 Que nous pouvons trouver dans \emph{public/sdk/inc/ntldr.h}:

\begin{lstlisting}[style=customc]
...

#define LDR_RESOURCE_ID_NAME_MASK 0xFFFF0000

...
\end{lstlisting}

Donc \emph{lpId} est ANDé avec \emph{0xFFFF0000} et si des bits sont encore présents
dans la partie 16-bit basse, la première partie de la fonction est exécutée (\emph{lpId} 
est traité comme une adresse de chaîne).
Autrement---la seconde moitié (\emph{lpId} est traitée comme une valeur 16-bit).

Encore, ce code peut être trouvé dans le fichier kernel32.dll de Windows 7:

\lstinputlisting[style=customasmx86]{advanced/450_more_ptrs/tmp1.lst}

Si la valeur du pointeur en entrée est plus grande que 0x10000, un saut au traitement
de chaînes se produit.
Autrement, la valeur en entrée du \emph{lpId} est renvoyée telle quelle.
Le masque \emph{0xFFFF0000} n'est plus utilisé ici, car ceci est du code 64-bit, mais
encore, \emph{0xFFFFFFFFFFFF0000} pourrait fonctionner ici.

Le lecteur attentif pourrait demander ce qui se passe si l'adresse de la chaîne en
entrée est plus petite que 0x10000?
Ce code se base sur le fait que dans Windows, il n'y a rien aux adresses en dessous
de 0x10000, au moins en Win32 realm.

Raymond Chen \href{https://blogs.msdn.microsoft.com/oldnewthing/20130925-00/?p=3123}{écrit} à propos de ceci:

\begin{framed}
\begin{quotation}
How does MAKE­INT­RESOURCE work? It just stashes the integer in the bottom 16 bits of a pointer, leaving the upper bits zero. This relies on the convention that the first 64KB of address space is never mapped to valid memory, a convention that is enforced starting in Windows 7.
\end{quotation}
\end{framed}

En quelques mots, ceci est un sale hack et probablement qu'il ne devrait être utilisé
qu'en cas de réelle nécessité.
Peut-être que la fonction \emph{FindResource()} avait un type \emph{SHORT} pour ses arguments,
et puis Microsoft a ajouté un moyen de passer des chaînes ici, mais que le code ancien
devait toujours être supporté.

Maintenant, voici un exemple distillé:

\begin{lstlisting}[style=customc]
#include <stdio.h>
#include <stdint.h>

void f(char* a)
{
	if (((uint64_t)a)>0x10000)
		printf ("Pointer to string has been passed: %s\n", a);
	else
		printf ("16-bit value has been passed: %d\n", (uint64_t)a);
};

int main()
{
	f("Hello!"); // pass string
	f((char*)1234); // pass 16-bit value
};
\end{lstlisting}

Ça fonctionne!

\subsubsection{Abus de pointeurs dans le noyau Linux}

Comme ça a déjà été pointé dans des \href{https://news.ycombinator.com/item?id=11823647}{commentaires sur Hacker News},
le noyau Linux comporte aussi des choses comme ça.

Par exemple, cette fonction peut renvoyer un code erreur ou un pointeur:

\begin{lstlisting}[style=customc]
struct kernfs_node *kernfs_create_link(struct kernfs_node *parent,
				       const char *name,
				       struct kernfs_node *target)
{
	struct kernfs_node *kn;
	int error;

	kn = kernfs_new_node(parent, name, S_IFLNK|S_IRWXUGO, KERNFS_LINK);
	if (!kn)
		return ERR_PTR(-ENOMEM);

	if (kernfs_ns_enabled(parent))
		kn->ns = target->ns;
	kn->symlink.target_kn = target;
	kernfs_get(target);	/* ref owned by symlink */

	error = kernfs_add_one(kn);
	if (!error)
		return kn;

	kernfs_put(kn);
	return ERR_PTR(error);
}
\end{lstlisting}

( \url{https://github.com/torvalds/linux/blob/fceef393a538134f03b778c5d2519e670269342f/fs/kernfs/symlink.c#L25} )

\emph{ERR\_PTR} est une macro pour caster un entier en un pointeur:

\begin{lstlisting}[style=customc]
static inline void * __must_check ERR_PTR(long error)
{
	return (void *) error;
}
\end{lstlisting}

( \url{https://github.com/torvalds/linux/blob/61d0b5a4b2777dcf5daef245e212b3c1fa8091ca/tools/virtio/linux/err.h} )

Ce fichier d'en-tête contient aussi une macro d'aide pour distinguer un code d'erreur
d'un pointeur:

\begin{lstlisting}[style=customc]
#define IS_ERR_VALUE(x) unlikely((x) >= (unsigned long)-MAX_ERRNO)
\end{lstlisting}

Ceci signifie que les codes erreurs sont les ``pointeurs'' qui sont très proche de
-1, et, heureusement, il n'y a rien dans la mémoire du noyau à des adresses comme
0xFFFFFFFFFFFFFFFF, 0xFFFFFFFFFFFFFFFE, 0xFFFFFFFFFFFFFFFD, etc.

Une méthode bien plus répandue est de renvoyer \emph{NULL} en cas d'erreur et de passer
le code d'erreur par un argument supplémentaire.
Les auteurs du noyau Linux ne font pas ça, mais quiconque veut utiliser ces fonctions
doit toujours garder en mémoire que le pointeur renvoyé doit toujours être testé avec
\emph{IS\_ERR\_VALUE} avant d'être déréférencé.

Par exemple:

\begin{lstlisting}[style=customc]
	fman->cam_offset = fman_muram_alloc(fman->muram, fman->cam_size);
	if (IS_ERR_VALUE(fman->cam_offset)) {
		dev_err(fman->dev, "%s: MURAM alloc for DMA CAM failed\n",
			__func__);
		return -ENOMEM;
	}
\end{lstlisting}

( \url{https://github.com/torvalds/linux/blob/aa00edc1287a693eadc7bc67a3d73555d969b35d/drivers/net/ethernet/freescale/fman/fman.c#L826} )

\subsubsection{Abus de pointeurs dans l'espace utilisateur UNIX}

\myindex{UNIX!mmap()}
La fonction mmap() renvoie -1 en cas d'erreur (ou \TT{MAP\_FAILED}, qui vaut -1).
Certaines personnes disent que mmap() peut mapper une zone mémoire à l'adresse zéro
dans de rares situations, donc elle ne peut pas utiliser 0 ou NULL comme code d'erreur.

