\subsection{Случай с Pin}

\myindex{Pin}
Интересно посмотреть, как некоторые ф-ции из \ac{DBI} Pin берут на вход несколько аргументов:

\begin{lstlisting}[style=customc]
            INS_InsertPredicatedCall(
                ins, IPOINT_BEFORE, (AFUNPTR)RecordMemRead,
                IARG_INST_PTR,
                IARG_MEMORYOP_EA, memOp,
                IARG_END);
\end{lstlisting}
( \TT{pinatrace.cpp} )

И вот как объявлена ф-ция \TT{INS\_InsertPredicatedCall()}:

\begin{lstlisting}[style=customc]
extern VOID INS_InsertPredicatedCall(INS ins, IPOINT ipoint, AFUNPTR funptr, ...);
\end{lstlisting}
( \TT{pin\_client.PH} )

Следовательно, константы с именами начинающимися с \TT{IARG\_} это что-то вроде аргументов для ф-ции,
которая обрабатывается внутри \TT{INS\_InsertPredicatedCall()}.
Вы можете передавать столько аргументов, сколько нужно.
Некоторые команды имеют дополнительные аргументы, некоторые другие --- нет.
Полный список аргументов:
\url{https://software.intel.com/sites/landingpage/pintool/docs/58423/Pin/html/group__INST__ARGS.html}.
И должен быть какой-то способ узнать, закончился ли список аргументов, так что список должен быть оконечен при помощи
константы \TT{IARG\_END}, без которой ф-ция будет (пытаться) обрабатывать случайный шум из локального стека,
принимая его за дополнительные аргументы.

\index{Python}
Также, в [\RobPikePractice] можно найти прекрасный пример ф-ций на \CCpp{}, очень похожих на 
\emph{pack/unpack}\footnote{\url{https://docs.python.org/3/library/struct.html}} в Python.

