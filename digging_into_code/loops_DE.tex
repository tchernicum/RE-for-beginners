\mysection{Schleifen}

Wann immer ein Programm mit einer Datei oder einem Puffer bestimmter Größe
zu tun hat, muss dies eine Art von Verabeitungsschleife im code haben.


Dies ist ein reales Beispiel der \tracer-Tool-Ausgabe, bei dem der Code auf
irgendeine Weise codierte Datei von 258 Byte lud.
Das Tool lief mit der Absicht die Zahl der Anweisungen zukommen
(ein \ac{DBI}-Tool würde dies heutzutage sehr viel besser machen).
Ich fand sehr schnell ein Code-Stück, welches 259/258 mal ausgeführt wurde.

\lstinputlisting{digging_into_code/crypto_loop.txt}

Wie sich herausstellte war dies auch eine Decodier-Schleife.
