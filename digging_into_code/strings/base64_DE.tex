\subsubsection{Base64}
\myindex{Base64}

Die Base64 Codierung ist sehr weit verbreitet f\"ur f\"alle in denen man Bin\"ardaten als Textstring \"ubertragen will.

Im Grunde, codiert dieser Algorithmus 3 Bin\"ar Bytes in 4 druckbare Zeichen: 
Alle 26 Latein Zeichen (beides klein und groß Buchstaben), Ziffern, plus Zeichen (\q{+}) und slash Zeichen (\q{/}),
64 Zeichen insgesamt. 

Ein charakteristisches Feature von Base64 Strings ist das sie oft (aber nicht immer) mit 1 oder 2  \gls{padding}
Gleichheitszeichen (\q{=}) Enden, zum Beispiel: 

\begin{lstlisting}
AVjbbVSVfcUMu1xvjaMgjNtueRwBbxnyJw8dpGnLW8ZW8aKG3v4Y0icuQT+qEJAp9lAOuWs=
\end{lstlisting}

\begin{lstlisting}
WVjbbVSVfcUMu1xvjaMgjNtueRwBbxnyJw8dpGnLW8ZW8aKG3v4Y0icuQT+qEJAp9lAOuQ==
\end{lstlisting}

Das Gleichheitszeichen Symbol (q{=}) wird man niemals in der Mitte eines Base64-codierten
Strings sehen.

Jetzt ein Beispiel wie man per Hand Base64 codieren kann.
Lasst uns 0x00, 0x11 , 0x22 und 0x33 in Hexadezimalzahlen in einen Base64
String umwandeln: 

\lstinputlisting{digging_into_code/strings/base64_ex.sh}

Lasst uns alle 4 Bytes in Bin\"ar Form bringen und dann neu gruppieren in 6-Bit Gruppen:

\begin{lstlisting}
|  00  ||  11  ||  22  ||  33  ||      ||      |
00000000000100010010001000110011????????????????
| A  || B  || E  || i  || M  || w  || =  || =  |
\end{lstlisting}

Die ersten drei Bytes (0x00, 0x11, 0x22) k\"onnen in 4 Base64 Zeichen umgewandelt werden (``ABEi''),
aber nicht das letzte Byte (0x33), also wird das Byte codiert indem man zwei Buchstaben 
benutzt (``Mw'') und das \gls{padding} Symbol (``='') wird zweimal hinzugef\"ugt um die letzte
Gruppe auf 4 Zeichen zu erweitern. Das bedeutet das die L\"ange aller korrekten Base64 Strings
sich immer durch 4 Teilen l\"asst. 

\myindex{XML}
\myindex{PGP}
Base64 wird oft benutzt wenn es darum geht Bin\"ardaten in  XML Dateien zu speichern.
``Armored'' (z.B, in Text Form) PGP Cookie und Signaturen werden codiert mit Base64.

Manche Leute versuchen auch Base64 zu benutzen um Strings zu verschleiern. 
\url{http://blog.sec-consult.com/2016/01/deliberately-hidden-backdoor-account-in.html}
\footnote{\url{http://archive.is/nDCas}}.

\myindex{base64scanner}
Es gibt Werkzeuge zum scannen von beliebigen Bin\"ardateien nach Base64 Strings.
Ein solch ein Scanner ist base64scanner\footnote{\url{https://github.com/DennisYurichev/base64scanner}}.

\myindex{UseNet}
\myindex{FidoNet}
\myindex{Uuencoding}
\myindex{Phrack}
Ein weiteres Codierungssystem welches im UseNet und FidoNet sehr weit verbreitet
war, ist UUencoding. Binärdateien sind in Phrack Magazine immernoch mit UUencoding
codiert. Es hat eigentlich die gleichen Features, unterscheidet sich von Base64
jedoch insofern, dass der Dateiname auch im Header gespeichert wird.

\myindex{Tor}
\myindex{base32}
By the Way: Es gibt auch einen nahen Verwandten zu Base64: Base32., ein Alphabet das ~10 Zeichen und ~26 Latein Zeichen hat. 
Eine verbreitete Anwendung ist Onion Adressen zu codieren. 
\footnote{\url{https://trac.torproject.org/projects/tor/wiki/doc/HiddenServiceNames}},
z.B: \\
\url{http://3g2upl4pq6kufc4m.onion/}.
\ac{URL} kann keine mixed-case Latein Zeichen beinhalten, deshalb haben Tor Entwickler sich f\"ur Base32 entschieden.
