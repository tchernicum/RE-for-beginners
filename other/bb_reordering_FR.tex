\mysection{Réordonnancement des blocs élémentaires}

% TODO __builtin_expect in GCC?

\subsection{Optimisation guidée par profil}
\label{PGO}

\myindex{\oracle}
\myindex{Intel C++}

Cette méthode d'optimisation déplace certains \gls{basic block}s vers d'autres sections du fichier 
binaire exécutable.

Il est évident que certaines parties d'une fonction sont exécutées plus fréquemment que d'autres
(ex: le corps d'une boucle) et d'autres moins souvent (ex: gestionnaire d'erreur, gestionnaires 
d'exception).

Le compilateur ajoute dans le code exécutable des instructions d'instrumentation. Le développeur 
exécute ensuite un nombre important de tests, ce qui permet de collecter des statistiques.

A l'aider de ces dernières, le compilateur prépare le fichier exécutable final en déplacant les 
fragments de code les moins exécutés vers une autre section.

Tous les fragments de code les plus souvent exécutés sont ainsi regroupés, ce qui constitue un 
facteur important pour la rapidité d'exécution et l'utilisation du cache.

Voici un exemple de code \oracle produit par le compilateur Intel C++:

\lstinputlisting[caption=orageneric11.dll (win32),style=customasmx86]{other/orageneric.lst}

La distance entre ces deux fragments de code avoisine les 9 Mo.

Tous les fragments de code rarement exécutés sont regroupés à la fin de la section de code de la DLL.

La partie de la fonction qui a été déplacée était marquée par le compilateur Intel C++ avec le 
préfixe \TT{VInfreq}.

Nous voyons donc qu'une partie de la fonction qui écrit dans un fichier journal (probablement à la 
suite d'une erreur ou d'un avertissement) n'a sans doute pas été exécuté très souvent durant les 
tests effectués par les développeurs Oracle lors de la collecte des statistiques. Il n'est même pas 
dit qu'elle ait jamais été exécutée.

Le bloc élémentaire qui écrit dans le journal s'achève par un retour à la partie \q{hot} de la 
fonction.

Un autre bloc élémentaire \q{infrequent} est celui qui retourne le code erreur 27050.

Pour ce qui est des fichiers Linux au format ELF, le compilateur Intel C++ déplace tous les 
fragments de code rarement exécutés vers une section séparée nommée \TT{text.unlikely}. 
Les fragments les plus souvent exécutés sont quant à eux regroupés dans la section \TT{text.hot}.

Cette information peut aider le rétro ingénieur à distinguer la partie principale d'une fonction 
des parties qui assurent la gestion d'erreurs.
