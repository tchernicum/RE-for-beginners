\subsection{10 PRINT CHR\$(205.5+RND(1)); : GOTO 10}

All examples here are MS-DOS .COM files.
%FIXME1 -> about .COM files

\myindex{MS-DOS}
In [Nick Montfort et al, \emph{10 PRINT CHR\$(205.5+RND(1)); : GOTO 10}, (The MIT Press:2012)]
\footnote{\AlsoAvailableAs \url{http://go.yurichev.com/17286}}

we can read about one of the most simple possible random maze generators.

It just prints a slash or backslash characters randomly and endlessly, resulting in something like this:

\begin{figure}[H]
\centering
\includegraphics[width=0.6\textwidth]{examples/demos/10print/10print.png}
\end{figure}

There are a few known implementations for 16-bit x86.

\subsubsection{Trixter's 42 byte version}

\newcommand{\FNURLTRIXTER}{\footnote{\url{http://go.yurichev.com/17305}}}

The listing was taken from his website\FNURLTRIXTER, 
but the comments are mine.

\lstinputlisting[style=customasmx86]{examples/demos/10print/10print_42_EN.lst}

\myindex{Intel!8253}
The pseudo-random value here is in fact the time 
that has passed from the system's boot, taken from the 8253 time chip, the value increases by one 18.2 times per second.

By writing zero to port \TT{43h}, 
we send the command \q{select counter 0}, 
"counter latch", 
"binary counter" (not a \ac{BCD} value).

\myindex{x86!\Instructions!POPF}
The interrupts are enabled back with the \TT{POPF} instruction, which restores the \TT{IF} flag as well.

\myindex{x86!\Instructions!IN}
It is not possible to 
use the \TT{IN} instruction with registers other than \TT{AL}, 
hence the shuffling.

\subsubsection{
My attempt to reduce Trixter's version: 27 bytes}

We can say that since we use the timer not 
to get a precise time value, but a pseudo-random one, we do not need
to spend time (and code) to disable the interrupts.

Another thing we can say is that we need only one bit from the low 8-bit part, so let's read only it.

We can reduced the code slightly and we've got 27 bytes:

\lstinputlisting[style=customasmx86]{examples/demos/10print/10print_27_EN.lst}

\subsubsection{
Taking random memory garbage as a source of randomness}

Since it is MS-DOS, there is no memory protection at all, we can read from whatever address we want.
\myindex{x86!\Instructions!LODSB}
Even more than that: a simple \TT{LODSB} 
instruction reads a byte from the \TT{DS:SI} address, but it's not a problem
if the registers' values are not set up, let it read 1) random bytes; 2) from a random place in memory!

It is suggested in Trixter's webpage\FNURLTRIXTER 
to use \TT{LODSB} without any setup.

\myindex{x86!\Instructions!SCASB}
It is also suggested that the \TT{SCASB} 
instruction can be used instead, because it sets a flag according to the byte it reads.

Another idea to minimize the code is to use the \TT{INT 29h} DOS syscall, which just prints the character stored in the \TT{AL} register.

That is what Peter Ferrie did
\footnote{\url{http://go.yurichev.com/17087}}:

\lstinputlisting[caption=Peter Ferrie: 10 bytes,style=customasmx86]{examples/demos/10print/ferrie_10_EN.lst}

So it is possible to get rid of conditional jumps at all.
The \ac{ASCII} code of backslash (\q{\textbackslash{}}) 
is \TT{0x5C} and \TT{0x2F} for slash (\q{/}).
So we have to convert one (pseudo-random) bit in the \TT{CF} flag to a value of \TT{0x5C} or \TT{0x2F}.

This is done easily: by \TT{AND}-ing all bits in \TT{AL} (where all 8 bits are set or cleared) with \TT{0x2D} we have just 0 or \TT{0x2D}.

By adding \TT{0x2F} to this value, we get \TT{0x5C} or \TT{0x2F}.

Then we just output it to the screen.

\subsubsection{\Conclusion{}}

\myindex{DOSBox}
It is also worth mentioning that the result may 
be different in DOSBox, \gls{Windows NT} and even MS-DOS, 

due to different
conditions: the timer chip can be emulated differently and the initial register contents may be different as well.
