\mysection{ARM}
\myindex{ARM}

\subsection{Терминология}

ARM изначально разрабатывался как 32-битный \ac{CPU}, 
поэтому \emph{слово} здесь, в отличие от x86, 32-битное.

\begin{description}
	\item[byte] 8-бит.
		Для определения переменных и массива байт используется директива ассемблера DCB.
	\item[halfword] 16-бит. \dittoclosing директива ассемблера DCW.
	\item[word] 32-бит. \dittoclosing директива ассемблера DCD.
	\item[doubleword] 64-бит.
	\item[quadword] 128-бит.
\end{description}

\subsection{Версии}

\begin{itemize}
\item ARMv4: появился режим Thumb.

\item ARMv6: использовался в iPhone 1st gen., iPhone 3G 
(Samsung 32-bit RISC ARM 1176JZ(F)-S поддерживающий Thumb-2)

\item ARMv7: появился Thumb-2 (2003).
Использовался в iPhone 3GS, iPhone 4, iPad 1st gen. (ARM Cortex-A8), iPad 2 (Cortex-A9),
iPad 3rd gen.

\item ARMv7s: Добавлены новые инструкции.
Использовался в iPhone 5, iPhone 5c, iPad 4th gen. (Apple A6).

\item ARMv8: 64-битный процессор, \ac{AKA} ARM64 \ac{AKA} AArch64.
Использовался в iPhone 5S, iPad Air (Apple A7).
В 64-битном режиме, режима Thumb больше нет, только режим ARM (4-байтные инструкции).
\end{itemize}

% sections
\subsection{32-\RU{битный}\EN{bit} ARM (AArch32)}

\subsubsection{\RU{Регистры общего пользования}\EN{General purpose registers}}

\begin{itemize}
\myindex{ARM!\Registers!R0}
	\item R0 --- \RU{результат функции обычно возвращается через R0}
		\EN{function result is usually returned using R0}
	\item R1...R12 --- \ac{GPR}s
	\item R13 --- \ac{AKA} SP (\gls{stack pointer})
\myindex{ARM!\Registers!Link Register}
	\item R14 --- \ac{AKA} LR (\gls{link register})
	\item R15 --- \ac{AKA} PC (program counter)
\end{itemize}

\myindex{ARM!\Registers!scratch registers}
\Reg{0}-\Reg{3} \RU{называются также \q{scratch registers}: аргументы функции обычно передаются через них,
и эти значения не обязательно восстанавливать перед выходом из функции}
\EN{are also called \q{scratch registers}: the function's arguments are usually passed in them,
and the values in them are not required to be restored upon the function's exit}.

\subsubsection{Current Program Status Register (CPSR)}

\begin{center}
\begin{tabular}{ | l | l | }
\hline
\headercolor\ \RU{Бит}\EN{Bit} &
\headercolor\ \RU{Описание}\EN{Description} \\
\hline
0..4           & M --- processor mode \\
\hline
5              & T --- Thumb state \\
\hline
6              & F --- FIQ disable \\
\hline
7              & I --- IRQ disable \\
\hline
8              & A --- imprecise data abort disable \\
\hline
9              & E --- data endianness \\
\hline
10..15, 25, 26 & IT --- if-then state \\
\hline
16..19         & GE --- greater-than-or-equal-to \\
\hline
20..23         & DNM --- do not modify \\
\hline
24             & J --- Java state \\
\hline
27             & Q --- sticky overflow \\
\hline
28             & V --- overflow \\
\hline
29             & C --- carry/borrow/extend \\
\hline
\myindex{ARM!\Registers!Z}
30             & Z --- zero bit \\
\hline
31             & N --- negative/less than \\
\hline
\end{tabular}
\end{center}

% TODO
% \myindex{ARM!\Registers!APSR}
% \subsubsection{Application Program Status Register (APSR)}

% TODO
% \myindex{ARM!\Registers!FPSCR}
% \subsubsection{Floating-Point Status and Control Register (FPPSR)}
% http://infocenter.arm.com/help/index.jsp?topic=/com.arm.doc.ddi0344b/Chdfafia.html

\subsubsection{\RU{Регистры VPF (для чисел с плавающей точкой) и NEON}
\EN{VFP (floating point) and NEON registers}}
\label{ARM_VFP_registers}

% http://infocenter.arm.com/help/index.jsp?topic=/com.arm.doc.dht0002a/ch01s03s02.html

\myindex{ARM!D-\registers{}}
\myindex{ARM!S-\registers{}}
\begin{center}
\begin{tabular}{ | l | l | l | l | }
\hline
0..31\textsuperscript{bits} & 32..64 & 65..96 & 97..127 \\
\hline
\multicolumn{4}{ | c | }{Q0\textsuperscript{128 bits}} \\
\hline
\multicolumn{2}{ | c | }{D0\textsuperscript{64 bits}} & \multicolumn{2}{ c | }{D1} \\
\hline
S0\textsuperscript{32 bits} & S1 & S2 & S3 \\
\hline
\end{tabular}
\end{center}

\RU{S-регистры 32-битные, используются для хранения чисел с одинарной точностью}
\EN{S-registers are 32-bit, used for the storage of single precision numbers}.

\RU{D-регистры 64-битные, используются для хранения чисел с двойной точностью}
\EN{D-registers are 64-bit ones, used for the storage of double precision numbers}.

\RU{D- и S-регистры занимают одно и то же место в памяти CPU --- можно
обращаться к D-регистрам через S-регистры (хотя это и бессмысленно)}
\EN{D- and S-registers share the same physical space in the CPU---it is possible to access 
a D-register via the S-registers (it is senseless though)}.

\RU{Точно также, \gls{NEON} Q-регистры имеют размер 128 бит и занимают то же физическое место 
в памяти CPU что и остальные регистры, предназначенные для чисел с плавающей точкой}
\EN{Likewise, the \gls{NEON} Q-registers are 128-bit ones and share the same physical space in the CPU 
with the other floating point registers}.

\RU{В VFP присутствует 32 S-регистров: S0..S31}
\EN{In VFP 32 S-registers are present: S0..S31}.

\RU{В VPFv2 были добавлены 16 D-регистров, которые занимают то же место что и S0..S31}
\EN{In VFPv2 there 16 D-registers are added, which in fact occupy the same space as S0..S31}.

\EN{In VFPv3 (\gls{NEON} or \q{Advanced SIMD}) there are 16 more D-registers, D0..D31, but the D16..D31 registers are not sharing space with any other S-registers.}%
\RU{В VFPv3 (\gls{NEON} или \q{Advanced SIMD}) добавили еще 16 D-регистров, в итоге это D0..D31, но регистры D16..D31 не делят место с другими S-регистрами.}

\EN{In \gls{NEON} or \q{Advanced SIMD} another 16 128-bit Q-registers were added, which share the same space as D0..D31.}%
\RU{В \gls{NEON} или \q{Advanced SIMD} были добавлены также 16 128-битных Q-регистров, делящих место с регистрами D0..D31.}

\subsection{64-\RU{битный}\EN{bit} ARM (AArch64)}

\subsubsection{\RU{Регистры общего пользования}\EN{General purpose registers}}
\label{ARM64_GPRs}

\RU{Количество регистров было удвоено со времен}\EN{The number of registers was doubled since} AArch32.

\begin{itemize}
\myindex{ARM!\Registers!X0}
	\item X0 --- \RU{результат функции обычно возвращается через X0}
		\EN{function result is usually returned using X0}
        \item X0...X7 --- \RU{Здесь передаются аргументы функции}\EN{Function arguments are passed here}.
	\item X8
	\item X9...X15 --- \RU{временные регистры, вызываемая функция может их использовать и не восстанавливать 
их}\EN{are temporary registers, the callee function can use and not restore them}.
	\item X16
	\item X17
	\item X18
	\item X19...X29 --- \RU{вызываемая функция может их использовать, но должна восстанавливать их по 
завершению}\EN{callee function can use them, but must restore them upon exit}.
	\item X29 --- \EN{used as}\RU{используется как} \ac{FP} (\EN{at least}\RU{как минимум в} GCC)
	\item X30 --- \q{Procedure Link Register} \ac{AKA} \ac{LR} (\gls{link register}).
	\EN{\item X31---register always contains zero \ac{AKA} XZR or \q{Zero Register}. It's 32-bit part is called WZR.}%
	\RU{\item X31 --- регистр, всегда содержащий ноль \ac{AKA} XZR или \q{Zero Register}. Его 32-битная часть называется WZR.}
	\item \ac{SP}, \RU{больше не регистр общего пользования}\EN{not a general purpose register anymore}.
\end{itemize}

\RU{См.также}\EN{See also}: \ARMPCS.

\EN{The 32-bit part of each X-register is also accessible via W-registers (W0, W1, etc.).}
\RU{32-битная часть каждого X-регистра также доступна как W-регистр (W0, W1, итд).}

\begin{center}
\begin{tabular}{ | l | l | }
\hline
\RU{Старшие 32 бита}\EN{High 32-bit part}\ES{Parte alta de 32 bits}\PTBRph{}\PLph{}\ITph{}\DE{Oberer 32-Bit-Teil}\THAph{}\NLph{}\FR{Partie 32 bits haute} & \RU{младшие 32 бита}\EN{low 32-bit part}\ES{parte baja de 32 bits}\PTBRph{}\PL{Starsze 32 bity}\ITph{}\DE{Unterer 32-Bit-Teil}\THAph{}\NLph{}\FR{Partie 32 bits basse} \\
\hline
\multicolumn{2}{ | c | }{X0} \\
\hline
\multicolumn{1}{ | c | }{} & \multicolumn{1}{ c | }{W0} \\
\hline
\end{tabular}
\end{center}


\subsection{\RU{Инструкции}\EN{Instructions}}

\RU{В ARM имеется также для некоторых инструкций суффикс \emph{-S}, указывающий, 
что эта инструкция будет модифицировать флаги.}
\EN{There is a \emph{-S} suffix for some instructions in ARM,
indicating that the instruction sets the flags according to the result.}
\RU{Инструкции без этого суффикса не модифицируют флаги.}
\EN{Instructions which lacks this suffix are not modify flags.}
\myindex{ARM!\Instructions!ADD}
\myindex{ARM!\Instructions!ADDS}
\myindex{ARM!\Instructions!CMP}
\RU{Например, инструкция}\EN{For example} \TT{ADD} \RU{в отличие от}\EN{unlike} \TT{ADDS}
\RU{сложит два числа, но флаги не изменит}
\EN{will add two numbers, but the flags will not be touched}.
\RU{Такие инструкции удобно использовать
между \CMP где выставляются флаги и, например, инструкциями перехода, где флаги используются.}
\EN{Such instructions are convenient to use between \CMP where the flags are set and, 
e.g. conditional jumps, where the flags are used.}
\EN{They are also better in terms of data dependency analysis 
(because less number of registers are modified during execution).}
\RU{Они также лучше в смысле анализа зависимостей данных (data dependency analysis) 
(потому что меньшее количество регистров модифицируется во время исполнения).}

% ADD
% ADDAL
% ADDCC
% ADDS
% ADR
% ADREQ
% ADRGT
% ADRHI
% ADRNE
% ASRS
% B
% BCS
% BEQ
% BGE
% BIC
% BL
% BLE
% BLEQ
% BLGT
% BLHI
% BLS
% BLT
% BLX
% BNE
% BX
% CMP
% IDIV
% IT
% LDMCSFD
% LDMEA
% LDMED
% LDMFA
% LDMFD
% LDMGEFD
% LDR.W
% LDR
% LDRB.W
% LDRB
% LDRSB
% LSL.W
% LSL
% LSLS
% MLA
% MOV
% MOVT.W
% MOVT
% MOVW
% MULS
% MVNS
% ORR
% POP
% PUSH
% RSB
% SMMUL
% STMEA
% STMED
% STMFA
% STMFD
% STMIA
% STMIB
% STR
% SUB
% SUBEQ
% SXTB
% TEST
% TST
% VADD
% VDIV
% VLDR
% VMOV
% VMOVGT
% VMRS
% VMUL
%\myindex{ARM!Optional operators!ASR
%\myindex{ARM!Optional operators!LSL
%\myindex{ARM!Optional operators!LSR
%\myindex{ARM!Optional operators!ROR
%\myindex{ARM!Optional operators!RRX

% AArch64
% RET is BR X30 or BR LR but with additional hint to CPU

\subsubsection{\RU{Таблица условных кодов}\EN{Conditional codes table}}

% TODO rework this!
\small
\begin{center}
\begin{tabular}{ | l | l | l | }
\hline
\HeaderColor \RU{Код}\EN{Code} & 
\HeaderColor \RU{Описание}\EN{Description} & 
\HeaderColor \RU{Флаги}\EN{Flags} \\
\hline
EQ & \EN{Equal}\RU{равно} & Z == 1 \\
\hline
NE & \EN{Not equal}\RU{не равно} & Z == 0 \\
\hline
CS \ac{AKA} HS (Higher or Same) & \EN{Carry set}\RU{перенос} / \EN{Unsigned, Greater than, equal}\RU{беззнаковое, больше или равно} & C == 1 \\
\hline
CC \ac{AKA} LO (LOwer) & \EN{Carry clear}\RU{нет переноса} / \EN{Unsigned, Less than}\RU{беззнаковое, меньше чем} & C == 0 \\
\hline
MI & \EN{Minus, negative}\RU{минус, отрицательный знак} / \EN{Less than}\RU{меньше чем} & N == 1 \\
\hline
PL & \EN{Plus, positive or zero}\RU{плюс, положительный знак или ноль} / \EN{Greater than, equal}\RU{больше чем или равно} & N == 0 \\
\hline
VS & \EN{Overflow}\RU{переполнение} & V == 1 \\
\hline
VC & \EN{No overflow}\RU{нет переполнения} & V == 0 \\
\hline
HI & \EN{Unsigned higher}\RU{беззнаковое, больше чем} / \EN{Greater than} & C == 1 \AndENRU \\
 & & Z == 0 \\
\hline
LS & \EN{Unsigned lower or same}\RU{беззнаковое, меньше или равно} / \EN{Less than or equal} & C == 0 \OrENRU \\
 & & Z == 1 \\
\hline
GE & \EN{Signed greater than or equal}\RU{знаковое, больше чем или равно} / \EN{Greater than or equal} & N == V \\
\hline
LT & \EN{Signed less than}\RU{знаковое, меньше чем} / \EN{Less than} & N != V \\
\hline
GT & \EN{Signed greater than}\RU{знаковое, больше чем} / \EN{Greater than} & Z == 0 \AndENRU \\
 & & N == V \\
\hline
LE & \EN{Signed less than or equal}\RU{знаковое, меньше чем или равно} / \EN{Less than, equal} & Z == 1 \OrENRU \\
 & & N != V \\
\hline
None / AL & \RU{Всегда}\EN{Always} & \RU{Любые}\EN{Any} \\
\hline
\end{tabular}
\end{center}
\normalsize


