\myindex{\CStandardLibrary!memset()}
\myindex{x86!\Instructions!STOSB}
\myindex{x86!\Instructions!STOSW}
\myindex{x86!\Instructions!STOSD}
\myindex{x86!\Instructions!STOSQ}
\item[STOSB/STOSW/STOSD/STOSQ] stocke un octet/
un mot 16-bit/
un mot 32-bit/
un mot 64-bit de AX/EAX/RAX à l'adresse se trouvant dans DI/EDI/RDI.

\label{REP_STOSx}
\myindex{x86!\Prefixes!REP}
Couplée avec le préfixe REP, elle est répétée en boucle, le compteur étant dans le
registre CX/ECX/RCX:
elle fonctionne comme memset() en C.
Si la taille du bloc est connue lors de la compilation, memset() est souvent mise
en ligne dans un petit morceau de code en utilisant REP MOVSx, parfois même avec
plusieurs instructions.

\myindex{\CStandardLibrary!memset()}
memset(EDI, 0xAA, 15) est équivalent à:

\lstinputlisting[style=customasmx86]{appendix/x86/instructions/STOSB_ex1_FR.asm}

(Apparemment, ça fonctionne plus vite que de de stocker 15 octets avec un seul REP STOSB).
