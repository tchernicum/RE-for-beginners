\myindex{x86!\Instructions!NOP}
\myindex{x86!\Instructions!XCHG}
  \item[NOP] \ac{NOP}. Son opcode est 0x90, qui est en fait l'instruction sans effet
  \TT{XCHG EAX,EAX}.
  Ceci implique que le x86 n'a pas d'instruction \ac{NOP} dédiée (comme dans de nombreux \ac{RISC}).
  Ce livre contient au moins un listing où GDB affiche NOP comme l'instruction 16-bit XCHG:
  \myref{NOP_as_XCHG_example}.

  Plus d'exemples de telles opérations:
  (\myref{sec:npad}).

  \ac{NOP} peut être généré par le compilateur pour aligner des labels sur une limite
  de 16-octets.
  Un autre usage très répandu de \ac{NOP} est de remplacer manuellement (patcher)
  une instruction, comme un saut conditionnel, par \ac{NOP}, afin de désactiver cette
  exécution.

