\subsection{Buffer overflow protection methods}
\label{subsec:BO_protection}

There are several methods to protect against this scourge, regardless of the \CCpp programmers' negligence.
MSVC has options like\footnote{compiler-side buffer overflow protection methods:
\href{http://go.yurichev.com/17133}{wikipedia.org/wiki/Buffer\_overflow\_protection}}:

\begin{lstlisting}
 /RTCs Stack Frame runtime checking
 /GZ Enable stack checks (/RTCs)
\end{lstlisting}

\myindex{x86!\Instructions!RET}
\myindex{Function prologue}
\myindex{Security cookie}

One of the methods is to write a random value between the local variables in stack at function prologue 
and to check it in function epilogue before the function exits.
If value is not the same, do not execute the last instruction \RET, but stop (or hang).
The process will halt, but that is much better than a remote attack to your host.
    
\newcommand{\CANARYURL}{\href{http://go.yurichev.com/17134}{wikipedia.org/wiki/Domestic\_canary\#Miner.27s\_canary}}

\myindex{Canary}

This random value is called a \q{canary} sometimes, it is related to the miners' canary\footnote{\CANARYURL},
they were used by miners in the past days in order to detect poisonous gases quickly.

Canaries are very sensitive to mine gases, they become very agitated in case of danger, or even die.

If we compile our very simple array example~(\myref{arrays_simple}) in \ac{MSVC}
with RTC1 and RTCs option,\\
you can see a call to \TT{@\_RTC\_CheckStackVars@8} a function at the end of the function that checks if the \q{canary} is correct.

Let's see how GCC handles this. 
Let's take an \TT{alloca()}~(\myref{alloca}) example:

\lstinputlisting[style=customc]{patterns/02_stack/04_alloca/2_1.c}

By default, without any additional options, GCC 4.7.3 inserts a \q{canary} check into the code:

\lstinputlisting[caption=GCC 4.7.3,style=customasmx86]{patterns/13_arrays/3_BO_protection/gcc_canary_EN.asm}

\myindex{x86!\Registers!GS}
The random value is located in \TT{gs:20}. 
It gets written on the stack and then at the end of the function
the value in the stack is compared with the correct \q{canary} in \TT{gs:20}. 
If the values are not equal, the 
\TT{\_\_stack\_chk\_fail} 
function is called and we can see in the console something like that (Ubuntu 13.04 x86):

\begin{lstlisting}
*** buffer overflow detected ***: ./2_1 terminated
======= Backtrace: =========
/lib/i386-linux-gnu/libc.so.6(__fortify_fail+0x63)[0xb7699bc3]
/lib/i386-linux-gnu/libc.so.6(+0x10593a)[0xb769893a]
/lib/i386-linux-gnu/libc.so.6(+0x105008)[0xb7698008]
/lib/i386-linux-gnu/libc.so.6(_IO_default_xsputn+0x8c)[0xb7606e5c]
/lib/i386-linux-gnu/libc.so.6(_IO_vfprintf+0x165)[0xb75d7a45]
/lib/i386-linux-gnu/libc.so.6(__vsprintf_chk+0xc9)[0xb76980d9]
/lib/i386-linux-gnu/libc.so.6(__sprintf_chk+0x2f)[0xb7697fef]
./2_1[0x8048404]
/lib/i386-linux-gnu/libc.so.6(__libc_start_main+0xf5)[0xb75ac935]
======= Memory map: ========
08048000-08049000 r-xp 00000000 08:01 2097586    /home/dennis/2_1
08049000-0804a000 r--p 00000000 08:01 2097586    /home/dennis/2_1
0804a000-0804b000 rw-p 00001000 08:01 2097586    /home/dennis/2_1
094d1000-094f2000 rw-p 00000000 00:00 0          [heap]
b7560000-b757b000 r-xp 00000000 08:01 1048602    /lib/i386-linux-gnu/libgcc_s.so.1
b757b000-b757c000 r--p 0001a000 08:01 1048602    /lib/i386-linux-gnu/libgcc_s.so.1
b757c000-b757d000 rw-p 0001b000 08:01 1048602    /lib/i386-linux-gnu/libgcc_s.so.1
b7592000-b7593000 rw-p 00000000 00:00 0
b7593000-b7740000 r-xp 00000000 08:01 1050781    /lib/i386-linux-gnu/libc-2.17.so
b7740000-b7742000 r--p 001ad000 08:01 1050781    /lib/i386-linux-gnu/libc-2.17.so
b7742000-b7743000 rw-p 001af000 08:01 1050781    /lib/i386-linux-gnu/libc-2.17.so
b7743000-b7746000 rw-p 00000000 00:00 0
b775a000-b775d000 rw-p 00000000 00:00 0
b775d000-b775e000 r-xp 00000000 00:00 0          [vdso]
b775e000-b777e000 r-xp 00000000 08:01 1050794    /lib/i386-linux-gnu/ld-2.17.so
b777e000-b777f000 r--p 0001f000 08:01 1050794    /lib/i386-linux-gnu/ld-2.17.so
b777f000-b7780000 rw-p 00020000 08:01 1050794    /lib/i386-linux-gnu/ld-2.17.so
bff35000-bff56000 rw-p 00000000 00:00 0          [stack]
Aborted (core dumped)
\end{lstlisting}

\myindex{MS-DOS}
gs is the so-called segment register. These registers were used widely in MS-DOS and DOS-extenders
times.
Today, its function is different.
\myindex{TLS}
\myindex{Windows!TIB}

To say it briefly, the \TT{gs} register in Linux always points to the
\ac{TLS}~(\myref{TLS})---some information specific to thread is stored there.
By the way, in win32 the \TT{fs} register plays the same role, pointing to
\ac{TIB} \footnote{\href{http://go.yurichev.com/17104}{wikipedia.org/wiki/Win32\_Thread\_Information\_Block}}. 

More information can be found in the Linux kernel source code (at least in 3.11 version),\\
in \emph{arch/x86/include/asm/stackprotector.h} this variable is described in the comments.

\subsubsection{ARM: \OptimizingKeilVI (\ARMMode)}
\myindex{\CLanguageElements!switch}

\lstinputlisting[style=customasmARM]{patterns/08_switch/1_few/few_ARM_ARM_O3.asm}

Again, by investigating this code we cannot say if it was a switch() in the original source code, 
or just a pack of if() statements.

\myindex{ARM!\Instructions!ADRcc}

Anyway, we see here predicated instructions again (like \ADREQ (\emph{Equal}))
which is triggered only in case $R0=0$, and then loads the address of the string \emph{<<zero\textbackslash{}n>>}
into \Reg{0}.
\myindex{ARM!\Instructions!BEQ}
The next instruction \ac{BEQ} redirects control flow to \TT{loc\_170}, if $R0=0$.

An astute reader may ask, will \ac{BEQ} trigger correctly since \ADREQ it
has already filled the \Reg{0} register before with another value?

Yes, it will since \ac{BEQ} checks the flags set by the \CMP instruction, 
and \ADREQ does not modify any flags at all.

The rest of the instructions are already familiar to us. 
There is only one call to \printf , at the end, and we have already examined this trick here~(\myref{ARM_B_to_printf}).
At the end, there are three paths to \printf{}.

\myindex{ARM!\Instructions!ADRcc}
\myindex{ARM!\Instructions!CMP}
The last instruction, \TT{CMP R0, \#2}, is needed to check if $a=2$.

If it is not true, then \ADRNE loads a pointer to the string \emph{<<something unknown \textbackslash{}n>>}
into \Reg{0}, since $a$ has already been checked to be equal to 0 or 1,
and we can sure that the $a$ variable is not equal to these numbers at this point.
And if $R0=2$, 
a pointer to the string \emph{<<two\textbackslash{}n>>}
will be loaded by \ADREQ into \Reg{0}.

\subsubsection{ARM: \OptimizingKeilVI (\ThumbMode)}

\lstinputlisting[style=customasmARM]{patterns/08_switch/1_few/few_ARM_thumb_O3.asm}

% FIXME а каким можно? к каким нельзя? \myref{} ->

As was already mentioned, it is not possible to add conditional predicates to most instructions in Thumb
mode, so the Thumb-code here is somewhat similar to the easily understandable x86 \ac{CISC}-style code.

\subsubsection{ARM64: \NonOptimizing GCC (Linaro) 4.9}

\lstinputlisting[style=customasmARM]{patterns/08_switch/1_few/ARM64_GCC_O0_EN.lst}

The type of the input value is \Tint, hence register \RegW{0} is used to hold it instead of the whole
\RegX{0} register.

The string pointers are passed to \puts using an \INS{ADRP}/\INS{ADD} instructions pair just like it was demonstrated in the 
\q{\HelloWorldSectionName} example:~\myref{pointers_ADRP_and_ADD}.

\subsubsection{ARM64: \Optimizing GCC (Linaro) 4.9}

\lstinputlisting[style=customasmARM]{patterns/08_switch/1_few/ARM64_GCC_O3_EN.lst}

Better optimized piece of code.
\TT{CBZ} (\emph{Compare and Branch on Zero}) instruction does jump if \RegW{0} is zero.
There is also a direct jump to \puts instead of calling it, like it was explained before:~\myref{JMP_instead_of_RET}.



