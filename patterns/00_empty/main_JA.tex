\mysection{空関数}
\label{empty_func}

可能な限り単純な関数は、何もしない関数です。

\lstinputlisting[caption=\EN{\CCpp Code},style=customc]{patterns/00_empty/1.c}

コンパイルしましょう!

\subsection{x86}

x86プラットフォーム上でGCCコンパイラとMSVCコンパイラの両方の生成物は次のとおりです。

\lstinputlisting[caption=\Optimizing GCC/MSVC (\assemblyOutput),style=customasmx86]{patterns/00_empty/1.s}

\myindex{x86!\Instructions!RET}
\RET 命令のみです。 \gls{caller}に戻る命令です。

\subsection{ARM}

\lstinputlisting[caption=\OptimizingKeilVI (\ARMMode) \assemblyOutput,style=customasmARM]{patterns/00_empty/1_Keil_ARM_O3.s}

リターンアドレスはARM \ac{ISA}のローカルスタックには保存されず、リンクレジスタに保存されるため、
\INS{BX LR}命令は実行をそのアドレスにジャンプさせるため、\gls{caller}への実行が効果的に戻ります。

\subsection{MIPS}

レジスタの命名には、数値($ 0〜$ 31)または擬似名($ V0、$ A0など)の2つの命名規則がMIPSの世界で使用されています。

以下のGCCアセンブリ出力は、レジスタを番号順にリストしています。

\lstinputlisting[caption=\Optimizing GCC 4.4.5 (\assemblyOutput),style=customasmMIPS]{patterns/00_empty/MIPS.s}

\dots \IDA は疑似名を使います

\lstinputlisting[caption=\Optimizing GCC 4.4.5 (IDA),style=customasmMIPS]{patterns/00_empty/MIPS_IDA.lst}

\myindex{MIPS!\Instructions!J}
最初の命令は、実行フローを\gls{caller}に返すジャンプ命令(JまたはJR)で、\$31(または\$RA)レジスタのアドレスにジャンプします。

これはARMの\ac{LR}に類似したレジスタです。

2番目の命令は\ac{NOP}で、何もしません。 私たちは今それを無視することができます。

\subsubsection{MIPS命令とレジスタ名についての注意}

MIPSの世界では、レジスタと命令の名前は、伝統的に小文字で書かれています。
しかし、一貫性を保つために、この本は大文字の使用で通します。

\subsection{実際の空関数}

空の関数は役に立たないように見えますが、低レベルのコードでは頻繁に使用されます。

まず第一に、次のようなデバッグ機能でとてもポピュラーです。

\lstinputlisting[caption=\CCpp Code,style=customc]{patterns/00_empty/dbg_print_EN.c}

非デバッグビルドでは(greleasehのように)、\TT{\_DEBUG}は定義されていないので、
\TT{dbg\_print()}関数は実行中に呼び出されているにもかかわらず、空になります。

同様に、ソフトウェア保護の一般的な方法は、合法な顧客向けに1つのビルドを作成し、
他にはデモビルドを作成することです。 デモビルドには、この例のようにいくつかの重要な機能が欠けています。

\lstinputlisting[caption=\CCpp Code,style=customc]{patterns/00_empty/demo_EN.c}

\TT{save\_file()}関数は、ユーザーがメニューの\TT{File->Save}をクリックすると呼び出すことができます。 
デモ版は、このメニュー項目を無効にしてお届けできますが、ソフトウェアクラッカーが有効にしても、役に立つコードのない空の関数だけが呼び出されます。

IDAは、\TT{nullsub\_00}、\TT{nullsub\_01}などの名前でこのような機能をマークします。
