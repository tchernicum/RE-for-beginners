\subsection{ARM}
\label{sec:hw_ARM}

\myindex{\idevices}
\myindex{Raspberry Pi}
\myindex{Xcode}
\myindex{LLVM}
\myindex{Keil}
Per gli esperimenti con i processori ARM, sono stati utilizzati diversi compilatori:

\begin{itemize}
\item Diffuso nel settore embedded: Keil Release 6/2013.

\item Apple Xcode 4.6.3 IDE con il compilatore LLVM-GCC 4.2
\footnote{Apple Xcode 4.6.3 utilizza il compilatore open-source GCC come compilatore front-end ed il generatore di codice LLVM}.

\item GCC 4.9 (Linaro) (per ARM64), disponibile per win32 su \url{http://go.yurichev.com/17325}.

\end{itemize}

Il codice ARM a 32 bit è utilizzato (incluse le modalità Thumb e Thumb-2) in tutti i casi in questo libro, se non specificato differentemente.
Quando parliamo di ARM a 64 bit, lo chiamiamo ARM64.

% subsections
\subsubsection{\NonOptimizingKeilVI (\ARMMode)}

Iniziamo a compilare il nostro esempio in Keil:

\begin{lstlisting}
armcc.exe --arm --c90 -O0 1.c
\end{lstlisting}

\myindex{\IntelSyntax}
Il compilatore \emph{armcc} produce un listato assembly con sintassi Intel,
e utilizza macro di alto livello legate al processore ARM
\footnote{ad esempio, la modalità ARM è priva delle istruzioni \PUSH/\POP},
tuttavia è più importante per noi vedere le istruzioni \q{così come sono}, quindi guardiamo in \IDA il risultato compilato.

\begin{lstlisting}[caption=\NonOptimizingKeilVI (\ARMMode) \IDA,style=customasmARM]
.text:00000000             main
.text:00000000 10 40 2D E9    STMFD   SP!, {R4,LR}
.text:00000004 1E 0E 8F E2    ADR     R0, aHelloWorld ; "hello, world"
.text:00000008 15 19 00 EB    BL      __2printf
.text:0000000C 00 00 A0 E3    MOV     R0, #0
.text:00000010 10 80 BD E8    LDMFD   SP!, {R4,PC}

.text:000001EC 68 65 6C 6C+aHelloWorld  DCB "hello, world",0    ; DATA XREF: main+4
\end{lstlisting}

Nell'esempio possiamo facilmente vedere che ogni istruzione ha lunghezza pari a 4 byte.
Difatti abbiamo compilato il codice per la modalità ARM e non Thumb.

\myindex{ARM!\Instructions!STMFD}
\myindex{ARM!\Instructions!POP}
La prima istruzione, \TT{STMFD SP!, \{R4,LR\}}\footnote{\ac{STMFD}},
funziona come l'istruzione \PUSH in x86, scrivendo i valori di due registri (\Reg{4} e \ac{LR}) nello stack.

Infatti il listato di output prodotto dal compilatore \emph{armcc}, per semplificazione, mostra l'istruzione \INS{PUSH \{r4,lr\}}.
Ma questo non è del tutto esatto. L'istruzione \PUSH esiste solo in modalità Thumb.
Utilizziamo quindi \IDA per non fare confusione.

Questa istruzione dapprima \glslink{decrement}{decrementa} il valore di \ac{SP} così da farlo puntare alla porzione dello stack che è libera di ospitare nuovi dati,
quindi salva il valore dei registri \Reg{4} e \ac{LR} all'indirizzo memorizzato
nel registro \ac{SP} appena modificato.

Questa istruzione (esattamente come \PUSH in Thumb mode) è in grado di salvare il valore di più registri contemporaneamente, cosa che può risultare molto utile.
A proposito, non esiste un equivalente in x86.
Si può notare anche che l'istruzione \TT{STMFD} è una generalizzazione
dell'istruzione \PUSH (estendendone le sue funzionalità), poichè può funzionare con qualunque registro, e non solo \ac{SP}.
In altre parole, \TT{STMFD} può essere usata per memorizzare un insieme di registri all'indirizzo di memoria specificato.

\myindex{\PICcode}
\myindex{ARM!\Instructions!ADR}
L'istruzione \INS{ADR R0, aHelloWorld}
aggiunge o sottrae il valore del registro \ac{PC} all'offset in cui è memorizzata la stringa \TT{hello, world}.
Ci si potrebbe chiedere, come è utilizzato in questo caso il registro \TT{PC}?
Questo viene detto \q{\PICcode}\footnote{Maggiori informazioni sono fornite nella relativa sezione~(\myref{sec:PIC})}.

Questo tipo di codice può essere eseguito ad un indirizzo non fisso in memoria.
In altre parole, si tratta di un indirizzamento relativo a \ac{PC} (\ac{PC}-relative addressing).
L'istruzione \TT{ADR} tiene conto della differenza tra l'indirizzo di questa istruzione e l'indirizzo dove si trova la stringa.
Questa differenza (offset) dovrà sempre essere la stessa, a prescindere dall'indirizzo in cui nostro codice sarà caricato dall'\ac{OS}.
Ciò spiega perchè bisogna soltanto aggiungere l'indirizzo dell'istruzione corrente (tramite \ac{PC}) per ottenere l'indirizzo assoluto in memoria della nostra stringa C.

\myindex{ARM!\Registers!Link Register}
\myindex{ARM!\Instructions!BL}
L'istruzione \INS{BL \_\_2printf}\footnote{Branch with Link} chiama la funzione \printf.
Questa istruzione funziona così:

\begin{itemize}
\item memorizza l'indirizzo successivo all'istruzione \INS{BL} (\TT{0xC}) nel registro \ac{LR};
\item quindi passa il controllo a \printf scrivendo il suo indirizzo nel registro \ac{PC}.
\end{itemize}

Quando la funzione \printf termina la sua esecuzione, deve sapere a chi restituire il controllo (dove ritornare).
Per questo motivo ogni funzione passa il controllo all'indirizzo memorizzato nel registro \ac{LR}.

Questa è una differenza tra processori \ac{RISC} \q{puri} come ARM e processori \ac{CISC} come x86,
nei quali il return address viene solitamente memorizzato nello stack
Maggiori informazioni si trovano nella prossima sezione~(\myref{sec:stack}).

A proposito, un indirizzo assoluto o un offset a 32-bit non può essere codificato nell'istruzione a 32-bit \TT{BL}
poichè ha solo spazio per 24 bit.
Come potremmo ricordare, tutte le istruzioni in ARM-mode hanno dimensione fissa di 4 byte (32 bit).
Dunque possono essere collocate solo su indirizzi allineati su 4-byte.
Ciò implica che gli ultimi 2 bit dell'indirizzo dell'istruzione (che sono sempre zero) possono essere omessi.
Abbiamo in definitiva 26 bit per la codifica dell'offset (offset encoding). E cio è sufficiente per codificare $current\_PC \pm{} \approx{}32M$.

\myindex{ARM!\Instructions!MOV}
L'istruzione successiva, \INS{MOV R0, \#0}\footnote{abbreviazione di MOVe} scrive semplicemente 0 nel registro \Reg{0}.
Questo perchè la nostra funzione C restituisce 0, ed il valore di ritorno deve essere memorizzato nel registro \Reg{0}.

\myindex{ARM!\Registers!Link Register}
\myindex{ARM!\Instructions!LDMFD}
\myindex{ARM!\Instructions!POP}
L'ultima istruzione \INS{LDMFD SP!, {R4,PC}}\footnote{\ac{LDMFD} è l'istruzione inversa rispetto a \ac{STMFD}}.
Carica valori dallo stack (o qualunque altra zona di memoria) per salvarli nei registri \Reg{4} e \ac{PC}, e \glslink{increment}{incrementa} lo \gls{stack pointer} \ac{SP}.
In questo caso funziona come \POP.\\
N.B. La prima istruzione \TT{STMFD} aveva salvato la coppia di registri \Reg{4} e \ac{LR} sullo stack, ma \Reg{4} e \ac{PC} vengono \emph{ripristinati} durante l'esecuzione di \TT{LDMFD}.

Come già sappiamo, l'indirizzo del posto a cui ogni funzione devere restituire il controllo è solitamente memorizzato nel registro \ac{LR}.
La prima istruzione salva il suo valore nello stack perchè lo stesso registro sarà utilizzato dalla nostra funzione
\main per la chiamata a \printf.
Al termine della funzione, questo valore può essere scritto direttamente nel registro \ac{PC}, passando di fatto il controllo al punto in cui la nostra funzione era stata chiamata.

Dal momento che \main è solitamente la funzione principale in \CCpp,
il controllo verrà restituito al loader dell' \ac{OS} oppure ad un punto in una \ac{CRT},
o qualcosa del genere.

Tutto questo consente di omettere l'istruzione \TT{BX LR} alla fine della funzione.

\myindex{ARM!DCB}
\TT{DCB} è una direttiva assembly che definisce un array di byte o una stringa ASCII, analoga alla direttiva DB
nel linguaggio assembly x86.

\subsubsection{\NonOptimizingKeilVI (\ThumbMode)}

Compiliamo lo stesso esempio usando Keil in Thumb mode:

\begin{lstlisting}
armcc.exe --thumb --c90 -O0 1.c
\end{lstlisting}

Otteniamo (in \IDA):

\begin{lstlisting}[caption=\NonOptimizingKeilVI (\ThumbMode) + \IDA,style=customasmARM]
.text:00000000             main
.text:00000000 10 B5          PUSH    {R4,LR}
.text:00000002 C0 A0          ADR     R0, aHelloWorld ; "hello, world"
.text:00000004 06 F0 2E F9    BL      __2printf
.text:00000008 00 20          MOVS    R0, #0
.text:0000000A 10 BD          POP     {R4,PC}

.text:00000304 68 65 6C 6C+aHelloWorld  DCB "hello, world",0    ; DATA XREF: main+2
\end{lstlisting}

Possiamo facilmente individuare gli opcode a 2-byte (16-bit). Questo è, come già detto, Thumb.
\myindex{ARM!\Instructions!BL}
L'istruzione \TT{BL} , tuttavia, è composta da due istruzioni a 16-bit.
Ciò accade perchè è impossibile caricare un offset per la funzione \printf usando il poco spazio a disposizione in un opcode a 16-bit.
Pertanto la prima istruzione a 16-bit carica i 10 bit alti dell'offset e la seconda istruzione carica
gli 11 bit più bassi dell'offset.

% TODO:
% BL has space for 11 bits, so if we don't encode the lowest bit,
% then we should get 11 bits for the upper half, and 12 bits for the lower half.
% And the highest bit encodes the sign, so the destination has to be within
% \pm 4M of current_PC.
% This may be less if adding the lower half does not carry over,
% but I'm not sure --all my programs have 0 for the upper half,
% and don't carry over for the lower half.
% It would be interesting to check where __2printf is located relative to 0x8
% (I think the program counter is the next instruction on a multiple of 4
% for THUMB).
% The lower 11 bytes of the BL instructions and the even bit are
% 000 0000 0110 | 001 0010 1110 0 = 000 0000 0110 0010 0101 1100 = 0x00625c,
% so __2printf should be at 0x006264.
% But if we only have 10 and 11 bits, then the offset would be:
% 00 0000 0110 | 01 0010 1110 0 = 0 0000 0011 0010 0101 1100 = 0x00325c,
% so __2printf should be at 0x003264.
% In this case, though, the new program counter can only be 1M away,
% because of the highest bit is used for the sign.

Come già detto, tutte le istruzione in Thumb mode hanno dimensione pari a 2 bytes (o 16 bit).
Ciò implica che è impossibile trovare un'istruzione Thumb ad un indirizzo dispari.
Per questo motivo, l'ultimo bit dell'indirizzo può essere omesso nell'encoding delle istruzioni.

Per riassumere, l'istruzione Thumb \TT{BL} può codificare un indirizzo in $current\_PC \pm{}\approx{}2M$.

\myindex{ARM!\Instructions!PUSH}
\myindex{ARM!\Instructions!POP}
Riguardo le altre istruzioni nella funzione: \PUSH e \POP qui funzionano come \TT{STMFD}/\TT{LDMFD} con
l'unica differenza che il registro \ac{SP} in questo caso non viene menzionato esplicitamente.
\TT{ADR} funziona esattamente come nell'esempio precedente.
\TT{MOVS} scrive 0 nel registro \Reg{0} per restituire zero.

\subsubsection{\OptimizingXcodeIV (\ARMMode)}

Xcode 4.6.3 senza ottimizzazioni produce un sacco di codice ridondante, perciò studieremo l'output ottimizzato in cui le
le istruzioni sono il meno possibile, settando lo switch del compilatore \Othree.

\begin{lstlisting}[caption=\OptimizingXcodeIV (\ARMMode),style=customasmARM]
__text:000028C4             _hello_world
__text:000028C4 80 40 2D E9   STMFD           SP!, {R7,LR}
__text:000028C8 86 06 01 E3   MOV             R0, #0x1686
__text:000028CC 0D 70 A0 E1   MOV             R7, SP
__text:000028D0 00 00 40 E3   MOVT            R0, #0
__text:000028D4 00 00 8F E0   ADD             R0, PC, R0
__text:000028D8 C3 05 00 EB   BL              _puts
__text:000028DC 00 00 A0 E3   MOV             R0, #0
__text:000028E0 80 80 BD E8   LDMFD           SP!, {R7,PC}

__cstring:00003F62 48 65 6C 6C+aHelloWorld_0  DCB "Hello world!",0
\end{lstlisting}

Le istruzioni \TT{STMFD} e \TT{LDMFD} ci sono già familiari.

\myindex{ARM!\Instructions!MOV}

L'istruzione \MOV scrive il numero \TT{0x1686} nel registro \Reg{0} .
Questo è l'offset che punta alla stringa \q{Hello world!} .

Il registro \TT{R7} (per come standardizzato in \IOSABI) è un puntatore di frame (frame pointer). Maggiori informazioni in basso.

\myindex{ARM!\Instructions!MOVT}
L'istruzione \TT{MOVT R0, \#0} (MOVe Top) scrive 0 nei 16 bit alti (higher 16 bits) del registro.
Il problema qui è che l'istruzione generica \MOV in ARM mode potrebbe scrivere solo i 16 bit bassi del registro.

Ricorda che tutti gli opcode delle istruzioni in ARM mode sono limitati ad una lunghezza di 32 bit. Ovviamente questa limitazione non riguarda lo spostamento dei dati tra registri.
Per questo motivo esiste l'istruzione aggiuntiva \TT{MOVT} per scrivere nelle parti alte dei registri (nei bit da 16 a 31, inclusi).
Il suo uso qui è comunque ridondante, perchè l'istruzione \TT{MOV R0, \#0x1686} di sopra ha azzerato la parte alta del registro.
Si tratta probabilmente di un difetto/svista del compilatore.
% TODO:
% I think, more specifically, the string is not put in the text section,
% ie. the compiler is actually not using position-independent code,
% as mentioned in the next paragraph.
% MOVT is used because the assembly code is generated before the relocation,
% so the location of the string is not yet known,
% and the high bits may still be needed.

\myindex{ARM!\Instructions!ADD}
L'istruzione \TT{ADD R0, PC, R0} aggiunge il valore in \ac{PC} al valore in \Reg{0}, per calcolare l'indirizzo assoluto della stringa \q{Hello world!}.
Come sappiamo, si tratta di \q{\PICcode} e quindi questa correzione risulta essenziale in questo caso.

L'istruzione \INS{BL} chiama la funzione \puts ivece di \printf.

\label{puts}
\myindex{\CStandardLibrary!puts()}
\myindex{puts() instead of printf()}

GCC ha sostituito la prima chiamata a \printf con \puts.
Infatti: \printf con un solo argomento è quasi analoga a \puts.

\emph{Quasi}, perchè le due funzioni producono lo stesso risultato solo nel caso in cui la stringa non contiene
identificatori di formato (format identifiers) che iniziano con \emph{\%}.
In caso contrario l'effetto di queste due funzioni sarebbe diverso
\footnote{Bisogna anche notare che \puts non richiede un simbolo new line `\textbackslash{}n' alla fine della stringa,
per questo non lo vediamo qui.}.

Perchè il compilatore ha sostituito \printf con \puts? Probabilmente perchè \puts è più veloce
\footnote{\href{http://go.yurichev.com/17063}{ciselant.de/projects/gcc\_printf/gcc\_printf.html}}.

Poichè passa direttamente i caratteri a \gls{stdout} senza confrontare ciascuno di essi con il simbolo \emph{\%}.

Andando avanti, vediamo la familiare istruzione \TT{MOV R0, \#0} che imposta il registro \Reg{0} a 0.

\subsubsection{\OptimizingXcodeIV (\ThumbTwoMode)}

Di default Xcode 4.6.3 genera codice per Thumb-2 in questo modo:

\begin{lstlisting}[caption=\OptimizingXcodeIV (\ThumbTwoMode),style=customasmARM]
__text:00002B6C                   _hello_world
__text:00002B6C 80 B5          PUSH            {R7,LR}
__text:00002B6E 41 F2 D8 30    MOVW            R0, #0x13D8
__text:00002B72 6F 46          MOV             R7, SP
__text:00002B74 C0 F2 00 00    MOVT.W          R0, #0
__text:00002B78 78 44          ADD             R0, PC
__text:00002B7A 01 F0 38 EA    BLX             _puts
__text:00002B7E 00 20          MOVS            R0, #0
__text:00002B80 80 BD          POP             {R7,PC}

...

__cstring:00003E70 48 65 6C 6C 6F 20+aHelloWorld  DCB "Hello world!",0xA,0
\end{lstlisting}

% Q: If you subtract 0x13D8 from 0x3E70,
% you actually get a location that is not in this function, or in _puts.
% How is PC-relative addressing done in THUMB2?
% A: it's not Thumb-related. there are just mess with two different segments. TODO: rework this listing.

\myindex{\ThumbTwoMode}
\myindex{ARM!\Instructions!BL}
\myindex{ARM!\Instructions!BLX}

Le istruzioni \TT{BL} e \TT{BLX} in Thumb mode, come ricordiamo, sono codificate con una coppia di istruzioni 16-bit.
In Thumb-2 questi opcode \emph{surrogati} sono estesi in modo tale che le nuove istruzioni possano essere codificate in istruzioni a 32-bit.

Ciò appare ovvio considerando che che gli opcodes delle istruzioni Thumb-2 iniziano sempre con \TT{0xFx} o \TT{0xEx}.

Ma nel listato \IDA
i byte degli opcode sono invertiti poichè per i processori ARM le istruzioni sono codificate secondo il seguente principio:
l'ultimo byte viene prima ed è seguito dal primo byte (per le modalità Thumb e Thumb-2)
oppure, per istruzioni in ARM mode il quarto byte viene prima, seguito dal terzo, dal secondo ed infine dal primo (a causa
della diversa \gls{endianness}).

Quindi i byte nei listati IDA sono collocati cosi':
\begin{itemize}
\item per ARM and ARM64 modes: 4-3-2-1;
\item per Thumb mode: 2-1;
\item per coppie di istruzioni a 16-bit in Thumb-2 mode: 2-1-4-3.
\end{itemize}

\myindex{ARM!\Instructions!MOVW}
\myindex{ARM!\Instructions!MOVT.W}
\myindex{ARM!\Instructions!BLX}

Come possiamo vedere, le istruzioni \TT{MOVW}, \TT{MOVT.W} e \TT{BLX} iniziano con \TT{0xFx}.

Ona delle istruzioni Thumb-2 è \TT{MOVW R0, \#0x13D8} ~---memorizza un valore a 16-bit nella parte bassa del registro \Reg{0} ,
azzerando i bit più alti.

Allo stesso modo, \TT{MOVT.W R0, \#0} ~funziona come \TT{MOVT} nel precedente esempio, ma in Thumb-2.

\myindex{ARM!mode switching}
\myindex{ARM!\Instructions!BLX}

Tra le altre differenze, l'istruzione \TT{BLX} in questo caso è usata al posto di \TT{BL}.

La differenza sta nel fatto che, oltre a salvare \ac{RA} nel registro \ac{LR} e passare il controllo alla funzione \puts,
il processore passa dalla modalità Thumb/Thumb-2 alla modalità ARM mode (o viceversa).

Questa istruzione è posta qui poichè l'istruzione a cui il controllo viene passato appare così (è codificata in ARM mode):

\begin{lstlisting}[style=customasmARM]
__symbolstub1:00003FEC _puts           ; CODE XREF: \_hello\_world+E
__symbolstub1:00003FEC 44 F0 9F E5     LDR  PC, =__imp__puts
\end{lstlisting}

Si tratta essenzialmente di un jump alla zona dove è scritto l'indirizzo di \puts nella imports section.

Il lettore attento potrebbe chiedere: perchè non chiamare \puts proprio nel punto del codice dove serve effettivamente?

Perchè non è efficiente in termini di spazio.

\myindex{Dynamically loaded libraries}
Quasi tutti i programmi utilizzano librerie esterne dinamiche (come le DLL in Windows, .so in *NIX o .dylib in \MacOSX).
Le librerie dinamiche contengono funzioni usate di frequente, inclusa la funzione C standard \puts.

\myindex{Relocation}
In un file eseguibile (Windows PE .exe, ELF o Mach-O) è presente una sezione di import (import section).
Si tratta di una lista di simboli (symbols - funzioni o variabili globali) importata da moduli esterni insieme ai nomi dei moduli stessi.

Il loader dell' \ac{OS} carica tutti i moduli necessari e, mentre enumera gli import symbols nel modulo primario, determina gli indirizzi
corretti per ciascun simbolo.

Nel nostro caso, \emph{\_\_imp\_\_puts} è una variabile a 32-bit usata dal loader dell'\ac{OS} per memorizzare l'indirizzo corretto
della funzione in una libreria esterna.
Successivamente l'istruzione \TT{LDR} legge semplicemente il valore a 32-bit da questa variabile e lo scrive nel registro \ac{PC},
passando il controllo ad esso.

Quindi, per ridurre il tempo necessario al loader dell'\ac{OS} per completare questa procedura, è una buona idea scrivere l'indirizzo
di ogni simbolo solo una volta, in un punto dedicato.

\myindex{thunk-functions}
Inoltre, come abbiamo già capito, è impossibile caricare un valore a 32-bit in un registro utilizzando solo una istruzione senza
accedere alla memoria.

Pertanto, la soluzione ottimale è quella di allocare una funzione separata, che funziona in ARM mode, con il solo scopo di passare
il controllo alla libreria dinamica e quindi saltare dal codice Thumb a questa piccola funzione di una sola istruzione
(la cosiddetta \gls{thunk function}).

\myindex{ARM!\Instructions!BL}
A proposito, nel precedente esempio (compilato per ARM mode) il controllo viene passato da \TT{BL} alla stessa \gls{thunk function}.
La modalità del processore però non viene cambiata (da cui l'assenza di una \q{X} nella instruction mnemonic).

\myparagraph{Altre informazioni sulle funzioni thunk}
\myindex{thunk-functions}

Le thunk-functions sono difficili da comprendere, apparentemente, a causa di una denominazione impropria.
Il modo migliore per capirle è pensarle come adattatori o convertitori da un tipo di jack ad un altro.
Ad esempio, un adattatore che consente l'inserimento di una spina elettrica Inglese in una presa Americana, o viceversa.
Le thunk functions sono a volte anche dette \emph{wrappers}.

Seguono altre descrizioni di queste funzioni:

\begin{framed}
\begin{quotation}
“Un pezzo di codice che fornisce un indirizzo:”, secondo to P. Z. Ingerman,
che ha inventato le thunk nel 1961 come un modo per legare i parameters alle loro definizioni formali
nelle chiamate a procedura in Algol-60. Se una procedura viene chiamata con un'espressione al posto di un parametro formale,
il compilatore genera una thunk che calcola l'espressione e lascia l'indirizzo del risultato in una posizione standard.

\dots

Microsoft e IBM hanno definito, nei loro sistemi basati su Intel, un “ambiente a 16-bit”
(con orrendi registri di segmento e limitazioni di indirizzi a 64K) e un “ambiente a 32-bit”
(con indirizzamento piatto (flat) e gestione della memoria semi reale). I due ambienti possono girare contemporaneamente
sullo stesso computer e OS (grazie a quello che, nel mondo Microsoft, è chiamato WOW, acronimo per Windows On Windows).
MS e IBM hanno entrambi deciso che il processo di passare da 16 a 32 bit e viceversa è detto un “thunk”; in Windows 95,
esiste anche un tool, THUNK.EXE, detto “thunk compiler”.
\end{quotation}
\end{framed}
% TODO FIXME move to bibliography and quote properly above the quote
( \href{http://go.yurichev.com/17362}{The Jargon File} )

\myindex{LAPACK}
\myindex{FORTRAN}
Un ulteriore esempio possiamo trovarlo all'interno della libreria LAPACK---un ``Linear Algebra PACKage'' scritto in FORTRAN.
Anche gli sviluppatori \CCpp vogliono utilizzare LAPACK, ma non è pensabile riscriverla in \CCpp e mantenere diverse versioni.
Esistono quindi delle piccole funzioni C chiamabili da un ambiente \CCpp, che a loro volta chiamano le funzioni FORTRAN,
e non fanno quasi nient'altro:

\begin{lstlisting}[style=customc]
double Blas_Dot_Prod(const LaVectorDouble &dx, const LaVectorDouble &dy)
{
    assert(dx.size()==dy.size());
    integer n = dx.size();
    integer incx = dx.inc(), incy = dy.inc();

    return F77NAME(ddot)(&n, &dx(0), &incx, &dy(0), &incy);
}
\end{lstlisting}

Anche questo tipo di funzioni vengono chiamate ``wrapper''.

\subsubsection{ARM64}

\myparagraph{GCC}

Compiliamo l'esempio con GCC 4.8.1 per ARM64:

\lstinputlisting[numbers=left,label=hw_ARM64_GCC,caption=\NonOptimizing GCC 4.8.1 + objdump,style=customasmARM]{patterns/01_helloworld/ARM/hw.lst}

In ARM64 non ci sono le modalità Thumb e Thumb-2, ma solo ARM, quindi esistono soltanto istruzioni a 32-bit.
Il numero di registri è raddoppiato: \myref{ARM64_GPRs}.
I registri a 64-bit hanno il prefisso \TT{X-} prefixes, mentre le loro parti a 32-bit hanno il prefisso --- \TT{W-}.

\myindex{ARM!\Instructions!STP}
L'istruzione \TT{STP} (\emph{Store Pair})
salva simultaneamente due registri nello stack: \RegX{29} e \RegX{30}.

Questa istruzione può ovviamente salvare la coppia di valori in una posizione arbitraria in memoria, tuttavia in questo caso è
specificato il registro \ac{SP}, e di conseguenza la coppia viene salvata nello stack.

I registri ARM64 sono a 64-bit, ognuno di essi ha dimensione pari a 8 byte, quindi sono necessari 16 byte per salvare i due registri.

Il punto esclamativo (``!'') dopo l'operando sta a significare che 16 deve esse prima sottratto da \ac{SP}, e solo successivamente
i valori devono essere scritti nello stack.

Questo è anche detto \emph{pre-index}.
Per le differenze tra \emph{post-index} e \emph{pre-index}
leggere qui: \myref{ARM_postindex_vs_preindex}.

Quindi, in termini del più familiare x86, la prima istruzione è semplicamente l'analogo della coppia
\TT{PUSH X29} e \TT{PUSH X30}.
\RegX{29} in ARM64 è usato come \ac{FP}, e \RegX{30}
come \ac{LR}, e questo spiega perchè sono salvati nel prologo della funzione e ripristinati nell'epilogo.

La seconda istruzione copia \ac{SP} in \RegX{29} (o \ac{FP}).
Ciò viene fatto per impostare lo stack frame della funzione.

\label{pointers_ADRP_and_ADD}
\myindex{ARM!\Instructions!ADRP/ADD pair}
Le istruzioni \TT{ADRP} e \ADD sono usate per inserire
l'indirizzo della stringa \q{Hello!} nel registro \RegX{0},
poichè il primo argomento della funzione viene passato in questo registro.

Non esiste alcun tipo di istruzione in ARM in grado di salvare un numero molto grande in un registro (perchè la lunghezza delle
istruzioni è limitata a 4 byte, maggiori informazioni qui: \myref{ARM_big_constants_loading}).
Perciò devono essere utilizzate più istruzioni. La prima (\TT{ADRP}), scrive l'indirizzo della pagina di 4KiB (4KiB page)
in cui si trova la stringa, nel registro \RegX{0}, e la seconda (\ADD) aggiunge semplicemente il resto dell'indirizzo.
Maggiori informazioni su questo tema: \myref{ARM64_relocs}.

\TT{0x400000 + 0x648 = 0x400648}, e vediamo la nostra C-string \q{Hello!} nel \TT{.rodata} data segment a questo indirizzo.

\myindex{ARM!\Instructions!BL}

\puts viene chiamata subito dopo usando l'istruzione \TT{BL}. Questo è già stato discusso: \myref{puts}.

\MOV scrive 0 in \RegW{0}.
\RegW{0} è la parte bassa a 32 bits del registro a 64-bit \RegX{0}:

\begin{center}
\begin{tabular}{ | l | l | }
\hline
\RU{Старшие 32 бита}\EN{High 32-bit part}\ES{Parte alta de 32 bits}\PTBRph{}\PLph{}\ITph{}\DE{Oberer 32-Bit-Teil}\THAph{}\NLph{}\FR{Partie 32 bits haute} & \RU{младшие 32 бита}\EN{low 32-bit part}\ES{parte baja de 32 bits}\PTBRph{}\PL{Starsze 32 bity}\ITph{}\DE{Unterer 32-Bit-Teil}\THAph{}\NLph{}\FR{Partie 32 bits basse} \\
\hline
\multicolumn{2}{ | c | }{X0} \\
\hline
\multicolumn{1}{ | c | }{} & \multicolumn{1}{ c | }{W0} \\
\hline
\end{tabular}
\end{center}


Il risultato della funzione viene restituito tramite \RegX{0} e \main restituisce 0, quindi è in questo modo che viene preparato
il valore da restituire.
Ma perchè usare la parte a 32-bit?

Perchè il tipo \Tint in ARM64, esattamente come in x86-64, è ancora a 32 bit, per maggiore compatibilità.
Quindi se una funzione restituisce un \Tint a 32 bit, solo la parte più bassa a 32 bit del registro \RegX{0} verrà utilizzata.

Per verificare quanto detto, cambiamo leggermente l'esempio e ricompiliamolo.
Adesso \main restituisce un valore a 64-bit:

\begin{lstlisting}[caption=\main returning a value of \TT{uint64\_t} type,style=customc]
#include <stdio.h>
#include <stdint.h>

uint64_t main()
{
        printf ("Hello!\n");
        return 0;
}
\end{lstlisting}

Il risultato è lo stesso, ma quell'istruzione MOV adesso appare così:

\begin{lstlisting}[caption=\NonOptimizing GCC 4.8.1 + objdump]
  4005a4:       d2800000        mov     x0, #0x0      // #0
\end{lstlisting}

\myindex{ARM!\Instructions!LDP}

\INS{LDP} (\emph{Load Pair}) infine riprisina i registri \RegX{29} e \RegX{30}.

Non c'è il punto esclamativo dopo l'istruzione: ciò implica che il valore viene prima caricato dallo stack, e solo successivamente
\ac{SP} è incrementato di 16.
Questo viene detto \emph{post-index}.

\myindex{ARM!\Instructions!RET}
Una nuova istruzione è apparsa in ARM64: \RET.
Funziona esattamente come \TT{BX LR}, con l'aggiunta di uno speciale \emph{hint} bit, che informa la \ac{CPU}
del fatto che si tratta di un ritorno da una funzione, e non soltanto una normale istruzione jump, in questo modo
può venire eseguita in modo più ottimale.

A causa della semplicità della funzione, GCC con le opzioni di ottimizzazione genera esattamente lo stesso codice.

