\subsubsection{MSVC}

Das Beispiel wird jezt in MSVC 2010 kompiliert:

\begin{lstlisting}
cl 1.cpp /Fa1.asm
\end{lstlisting}

(Die \TT{/Fa}-Option weist den Compiler an, Assembler-Code auszugeben.)

\begin{lstlisting}[caption=MSVC 2010,style=customasmx86]
CONST	SEGMENT
$SG3830	DB	'hello, world', 0AH, 00H
CONST	ENDS
PUBLIC	_main
EXTRN	_printf:PROC
; Function compile flags: /Odtp
_TEXT	SEGMENT
_main	PROC
	push	ebp
	mov	ebp, esp
	push	OFFSET $SG3830
	call	_printf
	add	esp, 4
	xor	eax, eax
	pop	ebp
	ret	0
_main	ENDP
_TEXT	ENDS
\end{lstlisting}

MSVC erstellt Assembler-Code im Intel-Syntax.
Der Unterschied zum AT\&T-Syntax wird später in \myref{ATT_syntax} behandelt.

Der Compiler generiert die Datei \TT{1.obj}, die anschließend zu \TT{1.exe} gelinkt wird.
In diesem Fall besteht die Datei aus zwei Segmenten: \TT{CONST} (für konstante Daten) und \TT{\_TEXT} (für Quellcode).

\myindex{\CLanguageElements!const}
\label{string_is_const_char}
Die Zeichenkette \TT{hello, world} hat in \CCpp den Typ \TT{const char[]}\InSqBrackets{\TCPPPL p176, 7.3.2}, aber keinen eigenen Bezeichner.
Da der Compiler jedoch irgendwie auf diese Zeichenkette zugreifen muss, definiert er den internen Namen \TT{\$SG3830}.

Aus diesem Grund kann das Beispiel auch wie folgt geschrieben werden:

\lstinputlisting[style=customc]{patterns/01_helloworld/hw_2.c}

Nochmal zurück zum Assembler-Listing: wie man sehen kann ist die Zeichenkette gemäß dem \CCpp-Standard mit einem 0-Byte abgeschlossen.
Mehr über \CCpp-Zeichenketten ist im Abschnitt \myref{C_strings} zu finden.

% Not sure what the technically precise translation of prologue and epilogue is
In dem Code-Segment \TT{\_TEXT} ist lediglich eine Funktion: \main{}.
Diese startet mit einem Prolog-Teil und endet mit einem Epilog-Teil (wie fast alle Funktionen)
\footnote{Mehr darüber in dem Abschnitt über Funktions-Prologe und -Epiloge ~(\myref{sec:prologepilog}).}.

\myindex{x86!\Instructions!CALL}
Nach dem Funktions-Prolog ist der Aufruf der \printf{}-Funktion zu sehen:\\
\INS{CALL \_printf}.
\myindex{x86!\Instructions!PUSH}
Vor dem Aufruf wird die Adresse der Zeichenkette (oder ein Zeiger darauf) mit dem Inhalt unserer Begrüßung
auf dem Stack gespeichert. Dies geschieht durch die \PUSH-Anweisung.

Wenn \printf die Ausführung wieder an \main übergibt, befindet sich die Adresse der Zeichenkette (oder ein Zeiger darauf) immer noch auf dem Stack.
Da diese jedoch nicht mehr benötigt wird, muss der \gls{stack pointer} (das \ESP-Register) korrigiert werden.

\myindex{x86!\Instructions!ADD}
\INS{ADD ESP, 4} bedeutet, dass der Wert 4 zu dem \ESP-Rregister-Wert addiert wird.

Warum 4? Da dies ein 32-Bit-Programm ist, werden exakt 4 Byte benötigt um Adressen auf dem Stack abzulegen. Wenn dies x64-Code wäre, würden 8 Byte benötigt.
\INS{ADD ESP, 4} ist quasi gleichbedeutend mit \INS{POP Register} jedoch ohne die Verwendung von Registern\footnote{Statusregister der CPU können sich jedoch ändern}.

\myindex{Intel C++}
\myindex{\oracle}
\myindex{x86!\Instructions!POP}

Aus dem gleichen Grund generieren einige Compiler (wie der Intel C++-Compiler) die Anweisung \TT{POP ECX}
anstatt \ADD (dieses Muster kann zum Beispiel im \oracle{}-Code gefunden werden, da dieser mit dem Intel-Compiler erstellt wurde).
Diese Anweisung hat nahezu den gleichen Effekt, nur dass die Inhalte des \ECX-Registers überschrieben werden.
Der Intel C++-Compiler nutzt \TT{POP ECX} vermutlich, da der OpCode für diese Anweisung kürzer ist als \TT{ADD ESP, x} (1 Byte für \TT{POP} und 3 Byte für \TT{ADD}),

Nachfolgend ein Beispiel unter der Verwendung von \POP anstatt \ADD aus \oracle{}:

\begin{lstlisting}[caption=\oracle 10.2 Linux (app.o file),style=customasmx86]
.text:0800029A                 push    ebx
.text:0800029B                 call    qksfroChild
.text:080002A0                 pop     ecx
\end{lstlisting}

%Mehr über den Stack gibt es im Abschnitt
% ~(\myref{sec:stack}).
\myindex{\CLanguageElements!return}
Nachdem \printf aufgerufen wurde, enthält der Original-\CCpp-Code die Anweisung \TT{return 0} als Rückgabewert der \main-Funktion.

\myindex{x86!\Instructions!XOR}
In dem hier gezeigten Code ist dies durch die Anweisung \INS{XOR EAX, EAX} realisiert.

\myindex{x86!\Instructions!MOV}

XOR ist lediglich ein \q{exklusives Oder}\footnote{\href{http://go.yurichev.com/17118}{wikipedia}} aber der Compiler nutzt dies oft anstatt
\INS{MOV EAX, 0}---auch hier wieder aufgrund des leicht kürzeren OpCodes (2 Byte für \XOR und 5 Byte für \MOV).

\myindex{x86!\Instructions!SUB}
Einige Compiler erzeugen \INS{SUB EAX, EAX}, was \emph{Subtrahiere den Wert in} \EAX \emph{vom Wert in} \EAX bedeutet.
In jedem Fall erzeugt dies einen Wert von Null.

\myindex{x86!\Instructions!RET}
Die letzte Anweisung \RET gibt die Ausführungskontrolle wieder an die aufrufende Funktion \gls{caller}.
Üblicherweise ist dies \CCpp \ac{CRT}-Code welcher wiederum die Kontrolle an das Betriebssystem (\ac{OS}) übergibt.
