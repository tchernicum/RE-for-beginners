\subsubsection{GCC: x86-64}

\myindex{x86-64}
Proviamo anche GCC in Linux a 64-bit:

\lstinputlisting[caption=GCC 4.4.6 x64,,style=customasmx86]{patterns/01_helloworld/GCC_x64_EN.s}

% I think I got the intent right on the following line - Renaissance
Linux, *BSD e \MacOSX usano anche un metodo per passare argomenti di funzione nei registri. \SysVABI.

I primi 6 argomenti sono passati nei registri \RDI, \RSI, \RDX, \RCX, \Reg{8}, \Reg{9}, ed il resto---tramite lo stack.

Quindi il puntatore alla stringa viene passato in \EDI (la parte a 32-bit del registro).
Ma perchè non utilizza la parte a 64-bit, \RDI?

E' importante ricordare che tutte le istruzioni \MOV in modalità 64-bit che scrivono qualcosa nella parte dei 32-bit bassa di un registro, azzera anche la parte a 32-bit alta (come indicato nei manuali Intel: \myref{x86_manuals}).\\
Ad esempio, \INS{MOV EAX, 011223344h} scrive correttamente un valore in \RAX, poichè i bit della parte alta verranno azzerati.

Se apriamo il file oggetto compilato (.o), possiamo anche vedere gli opcode di tutte le istruzioni
\footnote{Questo deve essere abilitato in \textbf{Options $\rightarrow$ Disassembly $\rightarrow$ Number of opcode bytes}}:

\lstinputlisting[caption=GCC 4.4.6 x64,style=customasmx86]{patterns/01_helloworld/GCC_x64.lst}

\label{hw_EDI_instead_of_RDI}
Come possiamo notare, l'istruzione che scrive in \EDI all'indirizzo \TT{0x4004D4} occupa 5 byte.
La stessa istruzione che scrive un valore a 64-bit in \RDI occupa 7 bytes.
Apparentemente, GCC sta cercando di risparmiare un po' di spazio.
Inoltre, può essere sicuro che il segmento dati contenente la stringa non sarà allocato ad indirizzi maggiori di 4\gls{GiB}.

\label{SysVABI_input_EAX}
% There isn't an ABI acronym in acronyms.tex - I figure the intent is to Application Binary Interface,
% so I put it in there (in the EN section, commented out)
Notiamo anche che il registro \EAX è stato azzerato prima della chiamata alla funzione \printf.
Questo viene fatto perché, in base allo standard \ac{ABI} citato in precedenza,
il numero dei registri vettore usati deve essere passato in \EAX nei sistemi *NIX su x86-64.
