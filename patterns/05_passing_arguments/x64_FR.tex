\subsection{x64}

\myindex{x86-64}

Le scénario est un peu différent en x86-64. Les arguments de la fonction (les 4 ou
6 premiers d'entre eux) sont passés dans des registres i.e. l'\glslink{callee}{appelée}
les lit depuis des registres au lieu de les lire dans la pile.

\subsubsection{MSVC}

MSVC \Optimizing:

\lstinputlisting[caption=MSVC 2012 x64 \Optimizing,style=customasmx86]{patterns/05_passing_arguments/x64_MSVC_Ox_FR.asm}

Comme on peut le voir, la fonction compacte \ttf prend tous ses arguments dans des registres.

La fonction \LEA est utilisée ici pour l'addition,
apparemment le compilateur considère qu'elle plus rapide que \TT{ADD}.
\myindex{x86!\Instructions!LEA}

\LEA est aussi utilisée dans la fonction \main pour préparer le premier et le troisième
argument de \ttf. Le compilateur doit avoir décidé que cela s'exécutera plus vite
que la façon usuelle ce charger des valeurs dans les registres, qui utilise l'instruction
\MOV.

Regardons ce qu'a généré MSVC sans optimisation:

\lstinputlisting[caption=MSVC 2012 x64,style=customasmx86]{patterns/05_passing_arguments/x64_MSVC_IDA_FR.asm}

C'est un peu déroutant, car les 3 arguments dans des registres sont sauvegardés
sur la pile pour une certaine raison.
\myindex{Shadow space}
\label{shadow_space}
Ceci est appelé \q{shadow space} % TODO dont know translation, mabe better to keep as is
\footnote{\href{http://go.yurichev.com/17256}{MSDN}}: 
chaque Win64 peut (mais ce n'est pas requis) y sauver les 4 registres. % TODO clarify Win64
Ceci est fait pour deux raisons:
1) c'est trop généreux d'allouer un registre complet (et même 4 registres) pour
un argument en entrée, donc il sera accédé par la pile;
2) le debugger sait toujours où trouver les arguments de la fonction lors d'un arrêt
\footnote{\href{http://go.yurichev.com/17257}{MSDN}}.

Donc, de grosses fonctions peuvent sauvegarder leurs arguments en entrée dans le \q{shadows space}
si elle veulent les utiliser pendant l'exécution, mais quelques petites fonctions
(comme la notre) peuvent ne pas le faire.

C'est la responsabilité de l'\glslink{caller}{appelant} d'allouer le \q{shadow space}
sur la pile.

\subsubsection{GCC}

GCC \Optimizing génère du code plus ou moins compréhensible:

\lstinputlisting[caption=GCC 4.4.6 x64 \Optimizing,style=customasmx86]{patterns/05_passing_arguments/x64_GCC_O3_FR.s}

GCC \NonOptimizing:

\lstinputlisting[caption=GCC 4.4.6 x64,style=customasmx86]{patterns/05_passing_arguments/x64_GCC_FR.s}

\myindex{Shadow space}

Il n'y a pas d'exigeance de \q{shadow space} en System V *NIX (\SysVABI), mais
l'\glslink{callee}{appelée} peut vouloir sauvegarder ses arguments quelque part
en cas de manque de registres.

\subsubsection{GCC: uint64\_t au lieu de int}


Notre exemple fonctionne avec des \Tint 32-bit, c'est pourquoi c'est la partie 32-bit
des registres qui est utilisée (préfixée par \TT{E-}).

Il peut être légèrement modifié pour utiliser des valeurs 64-bit:

\lstinputlisting[style=customc]{patterns/05_passing_arguments/ex64.c}

\lstinputlisting[caption=GCC 4.4.6 x64 \Optimizing,style=customasmx86]{patterns/05_passing_arguments/ex64_GCC_O3_IDA_FR.asm}

Le code est le même, mais cette fois les registres \emph{complets} (préfixés par \TT{R-}) sont utilisés.

