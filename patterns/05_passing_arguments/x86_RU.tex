\subsection{x86}

\subsubsection{MSVC}

Рассмотрим пример, скомпилированный в (MSVC 2010 Express):

\lstinputlisting[label=src:passing_arguments_ex_MSVC_cdecl,caption=MSVC 2010 Express,style=customasmx86]{patterns/05_passing_arguments/msvc_RU.asm}

\myindex{x86!\Registers!EBP}
Итак, здесь видно: в функции \main заталкиваются три числа в стек и вызывается функция \\
\TT{f(int,int,int)}.
 
Внутри \ttf доступ к аргументам, также как и к локальным переменным, происходит через макросы: 
\TT{\_a\$ = 8}, но разница в том, что эти смещения со знаком \emph{плюс}, 
таким образом если прибавить макрос \TT{\_a\$} к указателю на \EBP, то адресуется \emph{внешняя} 
часть \glslink{stack frame}{фрейма} стека относительно \EBP.

\myindex{x86!\Instructions!IMUL}
\myindex{x86!\Instructions!ADD}
Далее всё более-менее просто: значение $a$ помещается в \EAX. 
Далее \EAX умножается при помощи инструкции \IMUL на то, что лежит в \TT{\_b}, 
и в \EAX остается \glslink{product}{произведение} этих двух значений.

Далее к регистру \EAX прибавляется то, что лежит в \TT{\_c}.

Значение из \EAX никуда не нужно перекладывать, оно уже лежит где надо. 
Возвращаем управление вызывающей функции~--- она возьмет значение из \EAX и отправит его в \printf.

\clearpage
\subsubsection{\olly}
\myindex{\olly}

Компилируем этот пример в MSVC 2010 с ключами \TT{/GS- /MD} и запускаем в \olly.
Открываем окна данных и стека по адресу, который передается в качестве первого аргумента в функцию \TT{GetSystemTime()}, 
ждем пока эта функция исполнится, и видим следующее:

\begin{figure}[H]
\centering
\myincludegraphics{patterns/15_structs/1_systemtime/olly_systemtime1.png}
\caption{\olly: \TT{GetSystemTime()} только что исполнилась}
\label{fig:struct_olly_1}
\end{figure}

Точное системное время на моем компьютере, в которое исполнилась функция, это 9 декабря 2014, 22:29:52:

\lstinputlisting[caption=Вывод \printf]{patterns/15_structs/1_systemtime/console.txt}

Таким образом, в окне данных мы видим следующие 16 байт: 
\begin{lstlisting}
DE 07 0C 00 02 00 09 00 16 00 1D 00 34 00 D4 03
\end{lstlisting}

Каждые два байта отражают одно поле структуры. 
А так как порядок байт (\gls{endianness}) \emph{little endian},
то в начале следует младший байт, затем старший.
Следовательно, вот какие 16-битные числа сейчас записаны в памяти:

\begin{center}
\begin{tabular}{ | l | l | l | }
\hline
\headercolor{} Шестнадцатеричное число & 
\headercolor{} десятичное число & 
\headercolor{} имя поля \\
\hline
0x07DE & 2014	& wYear \\
\hline
0x000C & 12	& wMonth \\
\hline
0x0002 & 2	& wDayOfWeek \\
\hline
0x0009 & 9	& wDay \\
\hline
0x0016 & 22	& wHour \\
\hline
0x001D & 29	& wMinute \\
\hline
0x0034 & 52	& wSecond \\
\hline	
0x03D4 & 980	& wMilliseconds \\
\hline
\end{tabular}
\end{center}

В окне стека, видны те же значения, только они сгруппированы как 32-битные значения.

Затем \printf просто берет нужные значения и выводит их на консоль.

Некоторые поля \printf не выводит (\TT{wDayOfWeek} и
\TT{wMilliseconds}), но они находятся в памяти и доступны для использования.



\subsubsection{GCC}

Скомпилируем то же в GCC 4.4.1 и посмотрим результат в \IDA:

\lstinputlisting[caption=GCC 4.4.1,style=customasmx86]{patterns/05_passing_arguments/gcc_RU.asm}

Практически то же самое, если не считать мелких отличий описанных ранее.

После вызова обоих функций \glslink{stack pointer}{указатель стека} не возвращается назад, 
потому что предпоследняя инструкция \TT{LEAVE} (\myref{x86_ins:LEAVE}) делает это за один раз, в конце исполнения.

