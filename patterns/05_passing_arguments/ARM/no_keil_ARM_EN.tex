\subsubsection{\NonOptimizingKeilVI (\ARMMode)}

\lstinputlisting[style=customasmARM]{patterns/05_passing_arguments/ARM/tmp1.asm}

The \main function simply calls two other functions, with three values passed to the 
first one~---(\ttf).

As was noted before, in ARM the first 4 values are usually passed in the first 4 registers (\Reg{0}-\Reg{3}).

The \ttf function, as it seems, uses the first 3 registers (\Reg{0}-\Reg{2}) as arguments.

\myindex{ARM!\Instructions!MLA}
The \TT{MLA} (\emph{Multiply Accumulate}) 
instruction multiplies its first two operands (\Reg{3} and \Reg{1}), adds the third operand (\Reg{2}) to the product and stores
the result into the zeroth register (\Reg{0}), via which, by standard, functions return values.

\myindex{Fused multiply–add}
Multiplication and addition at once (\emph{Fused multiply–add}) 
is a very useful operation. By the way, there was no such instruction in x86 
before FMA-instructions appeared in SIMD
\footnote{\href{http://go.yurichev.com/17103}{wikipedia}}.

The very first \TT{MOV R3, R0}, 
instruction is, apparently, redundant (a single \TT{MLA} instruction could be used here instead). 
The compiler has not optimized it, since this is non-optimizing compilation.

\myindex{ARM!Mode switching}
\myindex{ARM!\Instructions!BX}

The \TT{BX} instruction returns the control to the address stored in the \ac{LR} register and, if necessary, 
switches the processor mode from Thumb to ARM or vice versa.
This can be necessary since, as we can see, function \ttf is not aware from what kind of code it may be
called, ARM or Thumb.
Thus, if it gets called from Thumb code, 
\TT{BX} is not only returns control to the calling function,
but also switches the processor mode to Thumb.
Or not switch, if the function has been called from ARM code \InSqBrackets{\ARMSevenRef A2.3.2}.
% look for "BXWritePC()" in manual
