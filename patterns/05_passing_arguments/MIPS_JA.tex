\subsection{MIPS}

\lstinputlisting[caption=\Optimizing GCC 4.4.5,style=customasmMIPS]{patterns/05_passing_arguments/MIPS_O3_IDA_JA.lst}

最初の4つの関数引数は、A-が前に付いた4つのレジスタに渡されます。

\myindex{MIPS!\Instructions!MULT}

MIPSには、HIとLOの2つの特殊レジスタがあり、\TT{MULT}命令の実行中に乗算の64ビット結果が格納されます。

\myindex{MIPS!\Instructions!MFLO}
\myindex{MIPS!\Instructions!MFHI}

これらのレジスタは、\TT{MFLO}および\TT{MFHI}命令を使用することによってのみアクセスできます。
ここでは\TT{MFLO}は乗算結果の低部分をとり、それを \$V0 に格納します。
そのため、乗算結果の上位32ビット部分が削除されます(HIレジスタの内容は使用されません)。
確かに、ここでは32ビットの \Tint データ型を扱います。

\myindex{MIPS!\Instructions!ADDU}

最後に、\TT{ADDU}(\q{Add Unsigned})は3番目の引数の値を結果に加えます。

\myindex{MIPS!\Instructions!ADD}
\myindex{MIPS!\Instructions!ADDU}
\myindex{Ada}
\myindex{Integer overflow}

MIPSには、\TT{ADD}と\TT{ADDU}の2種類の加算命令があります。
それらの違いは、署名性に関連するものではなく、例外に対するものです。 \TT{ADD}では、オーバーフローに関する例外が発生することがあります。これは、たとえばAda \ac{PL}で有用\footnote{\url{http://go.yurichev.com/17326}}であり、サポートされることもあります。 
\TT{ADDU}は、オーバーフロー時に例外を発生させません。

\CCpp ではこれをサポートしていないため、この例では\TT{ADD}の代わりに\TT{ADDU}が表示されています。

32ビットの結果は \$V0 に残ります。

\myindex{MIPS!\Instructions!JAL}
\myindex{MIPS!\Instructions!JALR}

\main に新しい命令があります(\TT{JAL} (\q{Jump and Link}))。

\INS{JAL}と\INS{JALR}の違いは、相対オフセットが最初の命令でエンコードされる一方、
\INS{JALR}はレジスタに格納された絶対アドレスにジャンプすることです(\q{ジャンプレジスタとリンクレジスタ})。

\ttf と \main の両方の関数は同じオブジェクトファイルに配置されるので、 
\ttf の相対アドレスは既知で固定されています。
