\subsection{x86}

\subsubsection{MSVC}

Voici ce que l'on obtient après compilation (MSVC 2010 Express) :

\lstinputlisting[label=src:passing_arguments_ex_MSVC_cdecl,caption=MSVC 2010 Express,style=customasmx86]{patterns/05_passing_arguments/msvc_FR.asm}

\myindex{x86!\Registers!EBP}

Ce que l'on voit, c'est que la fonction \main pousse 3 nombres sur la pile et appelle
\TT{f(int,int,int)}.

L'accès aux arguments à l'intérieur de \ttf est organisé à l'aide de macros
comme:\\
\TT{\_a\$ = 8},
de la même façon que pour les variables locales, mais avec des offsets positifs
(accédés avec \emph{plus}).
Donc, nous accédons à la partie \emph{hors} de la \glslink{stack frame}{structure locale de pile}
en ajoutant la macro \TT{\_a\$} à la valeur du registre \EBP.

\myindex{x86!\Instructions!IMUL}
\myindex{x86!\Instructions!ADD}

Ensuite, la valeur de $a$ est stockée dans \EAX. Après l'exécution de l'instruction
\IMUL, la valeur de \EAX est le \glslink{product}{produit} de la valeur de \EAX
et du contenu de \TT{\_b}.

Après cela, \ADD ajoute la valeur dans \TT{\_c} à \EAX.

La valeur dans \EAX n'a pas besoin d'être déplacée/copiée : elle est déjà là
où elle doit être.
Lors du retour dans la fonction \glslink{caller}{appelante}, elle prend la valeur dans
\EAX et l'utilise comme argument pour \printf.

\clearpage
\myparagraph{MSVC + \olly}
\myindex{\olly}

Chargeons notre exemple dans \olly et voyons quelles valeurs sont présentes dans EAX/EBX/ECX/EDX après 
exécution de l'instruction CPUID: 

\begin{figure}[H]
\centering
\myincludegraphics{patterns/15_structs/6_bitfields/cpuid/olly.png}
\caption{\olly: Après exécution de CPUID}
\label{fig:cpuid_olly_1}
\end{figure}

La valeur de EAX est \TT{0x000206A7} (ma \ac{CPU} est un Intel Xeon E3-1220).\\
Cette valeur exprimée en binaire vaut $0b0000 0000 0000 0010 0000 0110 1010 0111$.

Voici la manière dont les bits sont répartis sur les différents champs:

\begin{center}
\begin{tabular}{ | l | l | l | }
\hline
\headercolor{} champ &
\headercolor{} format binaire &
\headercolor{} format décimal \\
\hline
reserved2		& 0000 & 0 \\
\hline
extended\_family\_id	& 00000000 & 0 \\
\hline
extended\_model\_id	& 0010 & 2 \\
\hline
reserved1		& 00 & 0 \\
\hline
processor\_id		& 00 & 0 \\
\hline
family\_id		& 0110 & 6 \\
\hline
model			& 1010 & 10 \\
\hline
stepping		& 0111 & 7 \\
\hline
\end{tabular}
\end{center}

\lstinputlisting[caption=Console output]{patterns/15_structs/6_bitfields/cpuid/console.txt}


\subsubsection{GCC}

Compilons le même code avec GCC 4.4.1 et regardons le résultat dans \IDA :

\lstinputlisting[caption=GCC 4.4.1,style=customasmx86]{patterns/05_passing_arguments/gcc_FR.asm}

Le résultat est presque le même, avec quelques différences mineures discutées
précédemment.

Le \glslink{stack pointer}{pointeur de pile} n'est pas remis après les deux appels
de fonction (f et printf), car la pénultième instruction \TT{LEAVE} (\myref{x86_ins:LEAVE})
s'en occupe à la fin.
