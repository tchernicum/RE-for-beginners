\subsection{x86}

\subsubsection{MSVC}

Das ist das Ergebnis nach dem kompilieren (MSVC 2010 Express):

\lstinputlisting[label=src:passing_arguments_ex_MSVC_cdecl,caption=MSVC 2010 Express,style=customasmx86]{patterns/05_passing_arguments/msvc_DE.asm}

\myindex{x86!\Registers!EBP}

Was wir hier sehen ist das die \main Funktion drei Zahlen auf den Stack schiebt und \TT{f(int,int,int).} aufruft

Der Argument zugriff innerhalb von \ttf wird organisiert mit der Hilfe von Makros wie zum Beispiel:\\
\TT{\_a\$ = 8}, 
auf die gleiche weise wie Lokale Variablen allerdings mit positiven Offsets (adressiert mit \emph{plus}).

Also adressieren wir die \emph{äussere} Seite des \glslink{stack frame}{Stack frame} indem wir \TT{\_a\$} Makros zum Wert des \EBP Registers addieren  

\myindex{x86!\Instructions!IMUL}
\myindex{x86!\Instructions!ADD}

Dann wird der Wert von $a$ in \EAX gespeichert. Nachdem die \IMUL Instruktion ausgeführt wurde, ist
der Wert in \EAX ein Produkt des Wertes aus \EAX und dem Inhalt von \TT{\_b}.

Nun addiert \ADD den Wert in \TT{\_c} auf \EAX

Der Wert in \EAX muss nicht verschoben werden: Der Wert von \EAX befindet sich schon wo er sein muss

Beim zurück kehren zur \gls{caller} Funktion, wird der Wert aus \EAX genommen und als Argument 
für den \printf Aufruf benutzt.


\clearpage
\subsubsection{MSVC: x86 + \olly}
\myindex{\olly}

Hier sehen die Dinge noch einfacher aus:

\begin{figure}[H]
\centering
\myincludegraphics{patterns/04_scanf/2_global/ex2_olly_1.png}
\caption{\olly: nach Ausführung von \scanf}
\label{fig:scanf_ex2_olly_1}
\end{figure}

Die Variable befindet sich im Datensegment.
Nachdem der \PUSH Befehl (der die Adresse von $x$ speichert) ausgeführt worden ist,
erscheint die Adresse im Stackfenster. Wir machen einen Rechtsklick auf die Zeile und wählen \q{Follow in dump}.
Die Variable erscheint nun im Speicherfenster auf der linken Seite. 
Nachdem wir in der Konsole 123 eingegeben haben, erscheint \TT{0x7B} im Speicherfenster (siehe markiertes Feld im
Screenshot).

Warum ist das erste Byte \TT{7B}?
Logisch gedacht müsste dort \TT{00 00 00 7B} sein. 
Der Grund dafür ist die sogenannte \glslink{endianness}{Endianess} und x86 verwendet \emph{litte Endian}. 
Dies bedeutet, dass das niederwertigste Byte zuerst und das höchstwertigste zuletzt geschrieben werden.
Für mehr Informationen dazu siehe: \myref{sec:endianness}.
Zurück zu Beispiel: der 32-Bit-Wert wird von dieser Speicheradresse nach \EAX geladen und an \printf übergeben. 

Die Speicheradresse von $x$ ist \TT{0x00C53394}.

\clearpage
In \olly können wir die Speicherzuordnung des Prozesses nachvollziehen (Alt-M) und wir erkennen, dass sich diese Adresse
innerhalb des \TT{.data} PE-Segments von unserem Programm befindet:

\label{olly_memory_map_example}
\begin{figure}[H]
\centering
\myincludegraphics{patterns/04_scanf/2_global/ex2_olly_2.png}
\caption{\olly: Speicherzuordnung}
\label{fig:scanf_ex2_olly_2}
\end{figure}



\subsubsection{GCC}


Lasst uns das gleiche in GCC kompilieren und die Ergebnisse in \IDA betrachten:

\lstinputlisting[caption=GCC 4.4.1,style=customasmx86]{patterns/05_passing_arguments/gcc_DE.asm}

Das Ergebnis ist fast das gleiche aber mit kleineren Unterschieden die wir bereits früher
besprochen haben.

Der \gls{stack pointer} wird nicht zurück gesetzt nach den beiden Funktion aufrufen (f und printf),
weil die vorletzte \TT{LEAVE} Instruktion (\myref{x86_ins:LEAVE}) sich um das zurück setzen kümmert.
