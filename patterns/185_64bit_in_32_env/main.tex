\ifdefined\ENGLISH
\mysection{64-bit values in 32-bit environment}
\label{sec:64bit_in_32_env}

In a 32-bit environment, \ac{GPR}'s are 32-bit, so 64-bit values are stored and passed as 32-bit value pairs
\footnote{By the way, 32-bit values are passed as pairs in 16-bit environment in the same way: \myref{win16_32bit_values}}.
\fi

\ifdefined\RUSSIAN
\mysection{64-битные значения в 32-битной среде}
\label{sec:64bit_in_32_env}

В среде, где \ac{GPR}-ы 32-битные, 64-битные значения хранятся и передаются как пары 32-битных значений
\footnote{Кстати, в 16-битной среде, 32-битные значения передаются 16-битными парами точно так же: \myref{win16_32bit_values}}.
\fi

\ifdefined\GERMAN
\mysection{64-Bit-Werte in 32-Bit-Umgebungen}
\label{sec:64bit_in_32_env}

In einer 32-Bit-Umgebung sind \ac{GPR} 32 Bit groß. Also werden 64-Bit-Werte in
32-Bit-Wertepaaren gespeichert und übergeben\footnote{Übrigens, 32-Bit-Werte werden
als Paare in 16--Bit-Umgebungen auf der gleiche Art übergeben: \myref{win16_32bit_values}}.
\fi

\ifdefined\FRENCH
\mysection{Valeurs 64-bit dans un environnement 32-bit}
\label{sec:64bit_in_32_env}

Dans un environnement 32-bit, les \ac{GPR} sont 32-bit, donc les valeurs 64-bit sont
stockées et passées comme une paire de registres 32-bit\footnote{A propos, les valeurs
32-bit sont passées en tant que paire dans les environnements 16-bit de la même manière:
\myref{win16_32bit_values}}.
\fi

\ifdefined\JAPANESE
\mysection{32ビット環境での64ビット値}
\label{sec:64bit_in_32_env}

32ビット環境では、\ac{GPR}は32ビットなので、64ビット値は32ビット値ペアとして格納され、渡されます。
\footnote{ちなみに、32ビット値は16ビット環境でも同様にペアとして渡されます: \myref{win16_32bit_values}}.
\fi

\EN{\subsection{Win32 PE}
\label{win32_pe}
\myindex{Windows!Win32}

\acs{PE} is an executable file format used in Windows.
The difference between .exe, .dll and .sys is that .exe and .sys usually do not have exports, only imports.

\myindex{OEP}

A \ac{DLL}, just like any other PE-file, has an entry point (\ac{OEP}) (the function DllMain() is located there)
but this function usually does nothing.
.sys is usually a device driver.
As of drivers, Windows requires the checksum to be present in the PE file and for it to be correct
\footnote{For example, Hiew(\myref{Hiew}) can calculate it}.

\myindex{Windows!Windows Vista}
Starting at Windows Vista, a driver's files must also be signed with a digital signature. It will fail to load otherwise.

\myindex{MS-DOS}
Every PE file begins with tiny DOS program that prints a
message like \q{This program cannot be run in DOS mode.}---if you run this program in DOS or Windows 3.1 (\ac{OS}-es which are not aware of the PE format),
this message will be printed.

\subsubsection{Terminology}

\myindex{VA}
\myindex{Base address}
\myindex{RVA}
\myindex{Windows!IAT}
\myindex{Windows!INT}

\begin{itemize}
\item Module---a separate file, .exe or .dll.

\item Process---a program loaded into memory and currently running.  Commonly consists of one .exe file and bunch of .dll files.

\item Process memory---the memory a process works with.  Each process has its own.
There usually are loaded modules, memory of the stack, \gls{heap}(s), etc.

\item \ac{VA}---an address which is to be used in program while runtime.

\item Base address (of module)---the address within the process memory at which the module is to be loaded.
\ac{OS} loader may change it, if the base address is already occupied by another module just loaded before.

\item \ac{RVA}---the \ac{VA}-address minus the base address.

Many addresses in PE-file tables use \ac{RVA}-addresses.

%\item
%Data directory --- ...

\item \ac{IAT}---an array of addresses of imported symbols \footnote{\PietrekPE}.
Sometimes, the \TT{IMAGE\_DIRECTORY\_ENTRY\_IAT} data directory points at the \ac{IAT}.
\label{IDA_idata}
It is worth noting that \ac{IDA} (as of 6.1) may allocate a pseudo-section named \TT{.idata} for
\ac{IAT}, even if the \ac{IAT} is a part of another section!

\item \ac{INT}---an array of names of symbols to be imported\footnote{\PietrekPE}.
\end{itemize}

\subsubsection{Base address}

The problem is that several module authors can prepare DLL files for others to use and it is not possible
to reach an agreement which addresses is to be assigned to whose modules.

So that is why if two necessary DLLs for a process have the same base address,
one of them will be loaded at this base address, and the other---at some other free space in process memory,
and each virtual addresses in the second DLL will be corrected.

\par With \ac{MSVC} the linker often generates the .exe files with a base address of \TT{0x400000}
\footnote{The origin of this address choice is described here: \href{http://go.yurichev.com/17041}{MSDN}},
and with the code section starting at \TT{0x401000}.
This means that the \ac{RVA} of the start of the code section is \TT{0x1000}.

DLLs are often generated by MSVC's linker with a base address of \TT{0x10000000}
\footnote{This can be changed by the /BASE linker option}.

\myindex{ASLR}

There is also another reason to load modules at various base addresses, in this case random ones.
It is \ac{ASLR}\footnote{\href{http://go.yurichev.com/17140}{wikipedia}}.

\myindex{Shellcode}

A shellcode trying to get executed on a compromised system must call system functions, hence, know their addresses.

In older \ac{OS} (in \gls{Windows NT} line: before Windows Vista),
system DLL (like kernel32.dll, user32.dll) were always loaded at known addresses,
and if we also recall
that their versions rarely changed, the addresses of functions were
fixed and shellcode could call them directly.

In order to avoid this, the \ac{ASLR}
method loads your program and all modules it needs at random base addresses, different every time.

\ac{ASLR} support is denoted in a PE file by setting the flag
\par \TT{IMAGE\_DLL\_CHARACTERISTICS\_DYNAMIC\_BASE} \InSqBrackets{see \Russinovich}.

\subsubsection{Subsystem}

There is also a \emph{subsystem} field, usually it is:

\myindex{Native API}

\begin{itemize}
\item native\footnote{Meaning, the module use Native API instead of Win32} (.sys-driver),

\item console (console application) or

\item \ac{GUI} (non-console).
\end{itemize}

\subsubsection{OS version}

A PE file also specifies the minimal Windows version it needs in order to be loadable.

The table of version numbers stored in the PE file and corresponding Windows codenames is
here\footnote{\href{http://go.yurichev.com/17044}{wikipedia}}.

\myindex{Windows!Windows NT4}
\myindex{Windows!Windows 2000}
For example, \ac{MSVC} 2005 compiles .exe files for running on Windows NT4 (version 4.00), but \ac{MSVC} 2008 does not
(the generated files have a version of 5.00, at least Windows 2000 is needed to run them).

\myindex{Windows!Windows XP}

\ac{MSVC} 2012 generates .exe files of version 6.00 by default,
targeting at least Windows Vista.
However, by changing the compiler's options\footnote{\href{http://go.yurichev.com/17045}{MSDN}},
it is possible to force it to compile for Windows XP.

\subsubsection{Sections}

Division in sections, as it seems, is present in all executable file formats.

It is devised in order to separate code from data, and data---from constant data.

\begin{itemize}
\item Either the \emph{IMAGE\_SCN\_CNT\_CODE} or \emph{IMAGE\_SCN\_MEM\_EXECUTE} flags will be set on the code section---this is executable code.

\item On data section---\emph{IMAGE\_SCN\_CNT\_INITIALIZED\_DATA},\\
\emph{IMAGE\_SCN\_MEM\_READ} and \emph{IMAGE\_SCN\_MEM\_WRITE} flags.

\item On an empty section with uninitialized data---\\
\emph{IMAGE\_SCN\_CNT\_UNINITIALIZED\_DATA}, \emph{IMAGE\_SCN\_MEM\_READ} \\
        and \emph{IMAGE\_SCN\_MEM\_WRITE}.

\item On a constant data section (one that's protected from writing), the flags \\
\emph{IMAGE\_SCN\_CNT\_INITIALIZED\_DATA} and \emph{IMAGE\_SCN\_MEM\_READ} can be set, \\
but not \emph{IMAGE\_SCN\_MEM\_WRITE}.
A process going to crash if it tries to write to this section.
\end{itemize}

\myindex{TLS}
\myindex{BSS}
Each section in PE-file may have a name, however, it is not very important.
Often (but not always) the code section is named \TT{.text},
the data section---\TT{.data}, the constant data section --- \TT{.rdata} \emph{(readable data)}.
Other popular section names are:

\myindex{MIPS}
\begin{itemize}
\item \TT{.idata}---imports section.
\ac{IDA} may create a pseudo-section named like this: \myref{IDA_idata}.
\item \TT{.edata}---exports section (rare)
\item \TT{.pdata}---section holding all information about exceptions in Windows NT for MIPS, \ac{IA64} and x64: \myref{SEH_win64}
\item \TT{.reloc}---relocs section
\item \TT{.bss}---uninitialized data (\ac{BSS})
\item \TT{.tls}---thread local storage (\ac{TLS})
\item \TT{.rsrc}---resources
\item \TT{.CRT}---may present in binary files compiled by ancient MSVC versions
\end{itemize}

PE file packers/encryptors often garble section names or replace the names with their own.

\ac{MSVC} allows you to declare data in arbitrarily named section
\footnote{\href{http://go.yurichev.com/17047}{MSDN}}.

Some compilers and linkers can add a section with debugging symbols and
other debugging information (MinGW for instance).
\myindex{Windows!PDB}
However it is not so in latest versions of \ac{MSVC} (separate \gls{PDB} files are used there for this purpose).\\
\\
That is how a PE section is described in the file:

\begin{lstlisting}[style=customc]
typedef struct _IMAGE_SECTION_HEADER {
  BYTE  Name[IMAGE_SIZEOF_SHORT_NAME];
  union {
    DWORD PhysicalAddress;
    DWORD VirtualSize;
  } Misc;
  DWORD VirtualAddress;
  DWORD SizeOfRawData;
  DWORD PointerToRawData;
  DWORD PointerToRelocations;
  DWORD PointerToLinenumbers;
  WORD  NumberOfRelocations;
  WORD  NumberOfLinenumbers;
  DWORD Characteristics;
} IMAGE_SECTION_HEADER, *PIMAGE_SECTION_HEADER;
\end{lstlisting}
\footnote{\href{http://go.yurichev.com/17048}{MSDN}}

\myindex{Hiew}
A word about terminology: \emph{PointerToRawData} is called \q{Offset} in Hiew
and \emph{VirtualAddress} is called \q{RVA} there.

\subsubsection{Data section}

Data section in file can be smaller than in memory.
For example, some variables can be initialized, some are not.
Compiler and linker will collect them all into one section, but the first part of it is initialized and allocated in file,
while another is absent in file (of course, to make it smaller).
\emph{VirtualSize} will be equal to the size of section in memory, and \emph{SizeOfRawData} --- to
size of section in file.

IDA can show the border between initialized and not initialized parts like that:

\begin{lstlisting}[style=customasmx86]
...

.data:10017FFA                 db    0
.data:10017FFB                 db    0
.data:10017FFC                 db    0
.data:10017FFD                 db    0
.data:10017FFE                 db    0
.data:10017FFF                 db    0
.data:10018000                 db    ? ;
.data:10018001                 db    ? ;
.data:10018002                 db    ? ;
.data:10018003                 db    ? ;
.data:10018004                 db    ? ;
.data:10018005                 db    ? ;

...
\end{lstlisting}

\subsubsection{Relocations (relocs)}
\label{subsec:relocs}

\ac{AKA} FIXUP-s (at least in Hiew).

They are also present in almost all executable file formats
\footnote{Even in .exe files for MS-DOS}.
Exceptions are shared dynamic libraries compiled with \ac{PIC}, or any other \ac{PIC}-code.

What are they for?

Obviously, modules can be loaded on various base addresses, but how to deal with global variables, for example?
They must be accessed by address.  One solution is \PICcode{} (\myref{sec:PIC}).
But it is not always convenient.

That is why a relocations table is present.
There the addresses of points that must be corrected are enumerated,
in case of loading at a different base address.

% TODO тут бы пример с HIEW или objdump..
For example, there is a global variable at address \TT{0x410000} and this is how it is accessed:

\begin{lstlisting}[style=customasmx86]
A1 00 00 41 00         mov         eax,[000410000]
\end{lstlisting}

The base address of the module is \TT{0x400000}, the \ac{RVA} of the global variable is \TT{0x10000}.

If the module is loaded at base address \TT{0x500000}, the real address of the global variable must be \TT{0x510000}.

\myindex{x86!\Instructions!MOV}

As we can see, the address of variable is encoded in the instruction \TT{MOV}, after the byte \TT{0xA1}.

That is why the address of the 4 bytes after \TT{0xA1}, is written in the relocs table.

If the module is loaded at a different base address, the \ac{OS} loader enumerates all addresses in the table,

finds each 32-bit word the address points to, subtracts the original base address from it
(we get the \ac{RVA} here), and adds the new base address to it.

If a module is loaded at its original base address, nothing happens.

All global variables can be treated like that.

Relocs may have various types, however, in Windows for x86 processors, the type is usually \\
\emph{IMAGE\_REL\_BASED\_HIGHLOW}.

\myindex{Hiew}

By the way, relocs are darkened in Hiew, for example: \figref{fig:scanf_ex3_hiew_1}.

\myindex{\olly}
\olly underlines the places in memory to which relocs are to be applied, for example: \figref{fig:switch_lot_olly3}.

\subsubsection{Exports and imports}

\label{PE_exports_imports}
As we all know, any executable program must use the \ac{OS}'s services and other DLL-libraries somehow.

It can be said that functions from one module (usually DLL) must be connected somehow to the points of their
calls in other modules (.exe-file or another DLL).

For this, each DLL has an \q{exports} table, which consists of functions plus their addresses in a module.

And every .exe file or DLL has \q{imports}, a table of functions it needs for execution including
list of DLL filenames.

After loading the main .exe-file, the \ac{OS} loader processes imports table:
it loads the additional DLL-files, finds function names
among the DLL exports and writes their addresses down in the \ac{IAT} of the main .exe-module.

\myindex{Windows!Win32!Ordinal}

As we can see, during loading the loader must compare a lot of function names, but string comparison is not a very
fast procedure, so there is a support for \q{ordinals} or \q{hints},
which are function numbers stored in the table, instead of their names.

That is how they can be located faster when loading a DLL.
Ordinals are always present in the \q{export} table.

\myindex{MFC}
For example, a program using the \ac{MFC} library usually loads mfc*.dll by ordinals,
and in such programs there are no \ac{MFC} function names in \ac{INT}.

% TODO example!
When loading such programs in \IDA, it will ask for a path to the mfc*.dll files
in order to determine the function names.

If you don't tell \IDA the path to these DLLs, there will be \emph{mfc80\_123} instead of function names.

\myparagraph{Imports section}

Often a separate section is allocated for the imports table and everything related to it (with name like \TT{.idata}),
however, this is not a strict rule.

Imports are also a confusing subject because of the terminological mess. Let's try to collect all information in one place.

\begin{figure}[H]
\centering
\myincludegraphics{OS/PE/unnamed0.png}
\caption{
A scheme that unites all PE-file structures related to imports}
\end{figure}

The main structure is the array \emph{IMAGE\_IMPORT\_DESCRIPTOR}.
Each element for each DLL being imported.

Each element holds the \ac{RVA} address of the text string (DLL name) (\emph{Name}).

\emph{OriginalFirstThunk} is the \ac{RVA} address of the \ac{INT} table.
This is an array of \ac{RVA} addresses, each of which points to a text string with a function name.
Each string is prefixed by a 16-bit integer
(\q{hint})---\q{ordinal} of function.

While loading, if it is possible to find a function by ordinal,
then the strings comparison will not occur. The array is terminated by zero.

There is also a pointer to the \ac{IAT} table named \emph{FirstThunk}, it is just the \ac{RVA} address
of the place where the loader writes the addresses of the resolved functions.

The points where the loader writes addresses are marked by \IDA like this: \emph{\_\_imp\_CreateFileA}, etc.

There are at least two ways to use the addresses written by the loader.

\myindex{x86!\Instructions!CALL}
\begin{itemize}
\item The code will have instructions like \emph{call \_\_imp\_CreateFileA},
and since the field with the address of the imported function is a global variable in some sense,
the address of the \emph{call} instruction (plus 1 or 2) is to be added to the relocs table,
for the case when the module is loaded at a different base address.

But, obviously, this may enlarge relocs table significantly.

Because there are might be a lot of calls to imported functions in the module.

Furthermore, large relocs table slows down the process of loading modules.

\myindex{x86!\Instructions!JMP}
\myindex{thunk-functions}
\item For each imported function, there is only one jump allocated, using the \JMP instruction
plus a reloc to it.
Such points are also called \q{thunks}.

All calls to the imported functions are just \CALL instructions to the corresponding \q{thunk}.
In this case, additional relocs are not necessary because these \CALL{}-s
have relative addresses and do not need to be corrected.
\end{itemize}

These two methods can be combined.

Possible, the linker creates individual \q{thunk}s if there are too many calls to the function,
but not done by default. \\
\\
By the way, the array of function addresses to which FirstThunk is pointing is not necessary to be located in the \ac{IAT} section.
For example, the author of these lines once wrote the PE\_add\_import\footnote{\href{http://go.yurichev.com/17049}{yurichev.com}}
utility for adding imports to an existing .exe-file.

Some time earlier, in the previous versions of the utility,
at the place of the function you want to substitute with a call to another DLL,
my utility wrote the following code:

\begin{lstlisting}[style=customasmx86]
MOV EAX, [yourdll.dll!function]
JMP EAX
\end{lstlisting}

FirstThunk points to the first instruction. In other words, when loading yourdll.dll,
the loader writes the address of the \emph{function} function right in the code.

It also worth noting that a code section is usually write-protected, so my utility adds the \\
\emph{IMAGE\_SCN\_MEM\_WRITE}
flag for code section. Otherwise, the program to crash while loading with error code
5 (access denied). \\
\\
One might ask: what if I supply a program with a set of DLL files which is not supposed to change (including addresses of all DLL functions),
is it possible to speed up the loading process?

Yes, it is possible to write the addresses of the functions to be imported into the FirstThunk arrays in advance.
The \emph{Timestamp} field is present in the \\
\emph{IMAGE\_IMPORT\_DESCRIPTOR} structure.

If a value is present there, then the loader compares this value with the date-time of the DLL file.

If the values are equal, then the loader does not do anything, and the loading of the process can be faster.
This is called \q{old-style binding}
\footnote{\href{http://go.yurichev.com/17050}{MSDN}. There is also the \q{new-style binding}.}.
\myindex{BIND.EXE}

The BIND.EXE utility in Windows SDK is for this.
For speeding up the loading of your program, Matt Pietrek in \PietrekPEURL, suggests to do the binding shortly after your program
installation on the computer of the end user. \\
\\
PE-files packers/encryptors may also compress/encrypt imports table.

In this case, the Windows loader, of course, will not load all necessary DLLs.
\myindex{Windows!Win32!LoadLibrary}
\myindex{Windows!Win32!GetProcAddress}

Therefore, the packer/encryptor does this on its own, with the help of
\emph{LoadLibrary()} and the \emph{GetProcAddress()} functions.

That is why these two functions are often present in \ac{IAT} in packed files.\\
\\
In the standard DLLs from the Windows installation, \ac{IAT} often is located right at the beginning of the PE file.
Supposedly, it is made so for optimization.

While loading, the .exe file is not loaded into memory as a whole (recall huge install programs which are
started suspiciously fast), it is \q{mapped}, and loaded into memory in parts as they are accessed.

Probably, Microsoft developers decided it will be faster.

\subsubsection{Resources}

\label{PEresources}

Resources in a PE file are just a set of icons, pictures, text strings, dialog descriptions.

Perhaps they were separated from the main code, so all these things could be multilingual,
and it would be simpler to pick text or picture for the language that is currently set in the \ac{OS}. \\
\\
As a side effect, they can be edited easily and saved back to the executable file, even if one does not have special knowledge,
by using the ResHack editor, for example (\myref{ResHack}).

\subsubsection{.NET}

\myindex{.NET}

.NET programs are not compiled into machine code but into a special bytecode.
\myindex{OEP}
Strictly speaking, there is bytecode instead of the usual x86 code
in the .exe file, however, the entry point (\ac{OEP}) points to this tiny fragment of x86 code:

\begin{lstlisting}[style=customasmx86]
jmp         mscoree.dll!_CorExeMain
\end{lstlisting}

The .NET loader is located in mscoree.dll, which processes the PE file.
\myindex{Windows!Windows XP}

It was so in all pre-Windows XP \ac{OS}es. Starting from XP, the \ac{OS} loader is able to detect the .NET file
and run it without executing that \JMP instruction
\footnote{\href{http://go.yurichev.com/17051}{MSDN}}.

\myindex{TLS}
\subsubsection{TLS}

This section holds initialized data for the \ac{TLS}(\myref{TLS}) (if needed).
When a new thread start, its \ac{TLS} data is initialized using the data from this section. \\
\\
\myindex{TLS!Callbacks}

Aside from that, the PE file specification also provides initialization of the
\ac{TLS} section, the so-called TLS callbacks.

If they are present, they are to be called before the control is passed to the main entry point (\ac{OEP}).

This is used widely in the PE file packers/encryptors.

\subsubsection{Tools}

\myindex{objdump}
\myindex{Cygwin}
\myindex{Hiew}
\label{ResHack}

\begin{itemize}
\item objdump (present in cygwin) for dumping all PE-file structures.

\item Hiew(\myref{Hiew}) as editor.

\item pefile---Python-library for PE-file processing \footnote{\url{http://go.yurichev.com/17052}}.

\item ResHack \acs{AKA} Resource Hacker---resources editor\footnote{\url{http://go.yurichev.com/17052}}.

\item PE\_add\_import\footnote{\url{http://go.yurichev.com/17049}}---
simple tool for adding symbol(s) to PE executable import table.

\item PE\_patcher\footnote{\href{http://go.yurichev.com/17054}{yurichev.com}}---simple tool for patching PE executables.

\item PE\_search\_str\_refs\footnote{\href{http://go.yurichev.com/17055}{yurichev.com}}---simple tool for searching for a function in PE executables which use some text string.
\end{itemize}

\subsubsection{Further reading}

% FIXME: bibliography per chapter or section
\begin{itemize}
\item Daniel Pistelli---The .NET File Format \footnote{\url{http://go.yurichev.com/17056}}
\end{itemize}

}
\RU{\mysection{Пример вычисления адреса сети}

Как мы знаем, TCP/IP-адрес (IPv4) состоит из четырех чисел в пределах $0 \ldots 255$, т.е. 4 байта.

4 байта легко помещаются в 32-битную переменную, так что адрес хоста в IPv4, сетевая маска или адрес сети
могут быть 32-битными числами.

С точки зрения пользователя, маска сети определяется четырьмя числами в формате вроде \\
255.255.255.0,
но сетевые инженеры (сисадмины) используют более компактную нотацию (\ac{CIDR}),
вроде  \q{/8}, \q{/16}, итд.

Эта нотация просто определяет количество бит в сетевой маске, начиная с \ac{MSB}.

\small
\begin{center}
\begin{tabular}{ | l | l | l | l | l | l | }
\hline
\HeaderColor Маска & 
\HeaderColor Хосты & 
\HeaderColor Свободно &
\HeaderColor Сетевая маска &
\HeaderColor В шест.виде &
\HeaderColor \\
\hline
/30  & 4        & 2        & 255.255.255.252  & 0xfffffffc  & \\
\hline
/29  & 8        & 6        & 255.255.255.248  & 0xfffffff8  & \\
\hline
/28  & 16       & 14       & 255.255.255.240  & 0xfffffff0  & \\
\hline
/27  & 32       & 30       & 255.255.255.224  & 0xffffffe0  & \\
\hline
/26  & 64       & 62       & 255.255.255.192  & 0xffffffc0  & \\
\hline
/24  & 256      & 254      & 255.255.255.0    & 0xffffff00  & сеть класса C \\
\hline
/23  & 512      & 510      & 255.255.254.0    & 0xfffffe00  & \\
\hline
/22  & 1024     & 1022     & 255.255.252.0    & 0xfffffc00  & \\
\hline
/21  & 2048     & 2046     & 255.255.248.0    & 0xfffff800  & \\
\hline
/20  & 4096     & 4094     & 255.255.240.0    & 0xfffff000  & \\
\hline
/19  & 8192     & 8190     & 255.255.224.0    & 0xffffe000  & \\
\hline
/18  & 16384    & 16382    & 255.255.192.0    & 0xffffc000  & \\
\hline
/17  & 32768    & 32766    & 255.255.128.0    & 0xffff8000  & \\
\hline
/16  & 65536    & 65534    & 255.255.0.0      & 0xffff0000  & сеть класса B \\
\hline
/8   & 16777216 & 16777214 & 255.0.0.0        & 0xff000000  & сеть класса A \\
\hline
\end{tabular}
\end{center}
\normalsize

Вот простой пример, вычисляющий адрес сети используя сетевую маску и адрес хоста.

\lstinputlisting[style=customc]{\CURPATH/netmask.c}

\subsection{calc\_network\_address()}

Функция \TT{calc\_network\_address()} самая простая: 

она просто умножает (логически, используя \AND) адрес хоста на сетевую маску, в итоге давая адрес
сети.

\lstinputlisting[caption=\Optimizing MSVC 2012 /Ob0,numbers=left,style=customasmx86]{\CURPATH/calc_network_address_MSVC_2012_Ox.asm}

На строке 22 мы видим самую важную инструкцию \AND --- так вычисляется адрес сети.

\subsection{form\_IP()}

Функция \TT{form\_IP()} просто собирает все 4 байта в одно 32-битное значение.

Вот как это обычно происходит:

\begin{itemize}
\item Выделите переменную для возвращаемого значения. Обнулите её.

\item 
Возьмите четвертый (самый младший) байт, сложите его (логически, инструкцией \OR) с возвращаемым
значением. Оно содержит теперь 4-й байт.

\item Возьмите третий байт, сдвиньте его на 8 бит влево.
Получится значение в виде \TT{0x0000bb00}, где \TT{bb} это третий байт.
Сложите итоговое значение (логически, инструкцией \OR) с возвращаемым значением.
Возвращаемое значение пока что содержит \TT{0x000000aa}, так что логическое сложение
в итоге выдаст значение вида \TT{0x0000bbaa}.

\item 
Возьмите второй байт, сдвиньте его на 16 бит влево.
Вы получите значение вида \TT{0x00cc0000}, где \TT{cc} это второй байт.
Сложите (логически) результат и возвращаемое значение.
Выходное значение содержит пока что \TT{0x0000bbaa}, так что логическое сложение
в итоге выдаст значение вида \TT{0x00ccbbaa}.

\item 
Возьмите первый байт, сдвиньте его на 24 бита влево.
Вы получите значение вида \TT{0xdd000000}, где \TT{dd} это первый байт.
Сложите (логически) результат и выходное значение.
Выходное значение содержит пока что \TT{0x00ccbbaa}, так что сложение выдаст в итоге значение
вида \TT{0xddccbbaa}.

\end{itemize}

И вот как работает неоптимизирующий MSVC 2012:

\lstinputlisting[caption=\NonOptimizing MSVC 2012,style=customasmx86]{\CURPATH/form_IP_MSVC_2012_RU.asm}

Хотя, порядок операций другой, но, конечно, порядок роли не играет.

\Optimizing MSVC 2012 делает то же самое, но немного иначе:

\lstinputlisting[caption=\Optimizing MSVC 2012 /Ob0,style=customasmx86]{\CURPATH/form_IP_MSVC_2012_Ox_RU.asm}

Можно сказать, что каждый байт записывается в младшие 8 бит возвращаемого значения,
и затем возвращаемое значение сдвигается на один байт влево на каждом шаге.

Повторять 4 раза, для каждого байта.

\par
Вот и всё! 
К сожалению, наверное, нет способа делать это иначе.
Не существует более-менее популярных \ac{CPU} или \ac{ISA}, где имеется инструкция для сборки значения из бит или байт.
Обычно всё это делает сдвигами бит и логическим сложением (OR).

\subsection{print\_as\_IP()}

\TT{print\_as\_IP()} делает наоборот: расщепляет 32-битное значение на 4 байта.

Расщепление работает немного проще: просто сдвигайте входное значение на 24, 16, 8 или 0 бит,
берите биты с нулевого по седьмой (младший байт), вот и всё:

\lstinputlisting[caption=\NonOptimizing MSVC 2012,style=customasmx86]{\CURPATH/print_as_IP_MSVC_2012_RU.asm}

\Optimizing MSVC 2012 делает почти всё то же самое, только без ненужных перезагрузок входного значения:

\lstinputlisting[caption=\Optimizing MSVC 2012 /Ob0,style=customasmx86]{\CURPATH/print_as_IP_MSVC_2012_Ox.asm}

\subsection{form\_netmask() и set\_bit()}

\TT{form\_netmask()} делает сетевую маску из \ac{CIDR}-нотации.

Конечно, было бы куда эффективнее использовать здесь какую-то уже готовую таблицу, но мы рассматриваем
это именно так, сознательно, для демонстрации битовых сдвигов.
Мы также сделаем отдельную функцию \TT{set\_bit()}. 

Не очень хорошая идея выделять отдельную функцию для такой примитивной операции, но так будет проще понять,
как это всё работает.

\lstinputlisting[caption=\Optimizing MSVC 2012 /Ob0,style=customasmx86]{\CURPATH/form_netmask_MSVC_2012_Ox.asm}

\TT{set\_bit()} примитивна: просто сдвигает единицу на нужное количество бит, затем складывает (логически) с
входным значением \q{input}.
\TT{form\_netmask()} имеет цикл: он выставит столько бит (начиная с \ac{MSB}), 
сколько передано в аргументе \TT{netmask\_bits}.

\subsection{Итог}

Вот и всё!
Мы запускаем и видим:

\begin{lstlisting}
netmask=255.255.255.0
network address=10.1.2.0
netmask=255.0.0.0
network address=10.0.0.0
netmask=255.255.255.128
network address=10.1.2.0
netmask=255.255.255.192
network address=10.1.2.64
\end{lstlisting}
}
\DE{\mysection{\Stack}
\label{sec:stack}
\myindex{\Stack}

Der Stack ist eine der fundamentalen Datenstrukturen in der Informatik.
\footnote{\href{http://go.yurichev.com/17119}{wikipedia.org/wiki/Call\_Stack}}.
\ac{AKA} \ac{LIFO}.

Technisch betrachtet ist es ein Stapelspeicher innerhalb des Prozessspeichers der zusammen mit den \ESP (x86), \RSP (x64) oder dem \ac{SP} (ARM) Register als ein Zeiger in diesem Speicherblock fungiert.

\myindex{ARM!\Instructions!PUSH}
\myindex{ARM!\Instructions!POP}
\myindex{x86!\Instructions!PUSH}
\myindex{x86!\Instructions!POP}

Die häufigsten Stack-Zugriffsinstruktionen sind die \PUSH- und \POP-Instruktionen (in beidem x86 und ARM Thumb-Modus). \PUSH subtrahiert vom \ESP/\RSP/\ac{SP} 4 Byte im 32-Bit Modus (oder 8 im 64-Bit Modus) und schreibt dann den Inhalt des Zeigers an die Adresse auf die von \ESP/\RSP/\ac{SP} gezeigt wird.

\POP ist die umgekehrte Operation: Die Daten des Zeigers für die Speicherregion auf die von \ac{SP}
gezeigt wird werden ausgelesen und die Inhalte in den Instruktionsoperanden geschreiben (oft ist das ein Register). Dann werden 4 (beziehungsweise 8) Byte zum \gls{stack pointer} addiert.

Nach der Stackallokation, zeigt der \gls{stack pointer} auf den Boden des Stacks.
\PUSH verringert den \gls{stack pointer} und \POP erhöht ihn.
Der Boden des Stacks ist eigentlich der Anfang der Speicherregion die für den Stack reserviert wurde.
Das wirkt zunächst seltsam, aber so funktioniert es.

ARM unterstützt beides, aufsteigende und absteigende Stacks.

\myindex{ARM!\Instructions!STMFD}
\myindex{ARM!\Instructions!LDMFD}
\myindex{ARM!\Instructions!STMED}
\myindex{ARM!\Instructions!LDMED}
\myindex{ARM!\Instructions!STMFA}
\myindex{ARM!\Instructions!LDMFA}
\myindex{ARM!\Instructions!STMEA}
\myindex{ARM!\Instructions!LDMEA}

Zum Beispiel die \ac{STMFD}/\ac{LDMFD} und \ac{STMED}/\ac{LDMED} Instruktionen sind alle dafür gedacht mit einem absteigendem Stack zu arbeiten ( wächst nach unten, fängt mit hohen Adressen an und entwickelt sich zu niedrigeren Adressen). Die \ac{STMFA}/\ac{LDMFA} und \ac{STMEA}/\ac{LDMEA} Instruktionen sind dazu gedacht mit einem aufsteigendem Stack zu arbeiten (wächst nach oben und fängt mit niedrigeren Adressen an und wächst nach oben).

% It might be worth mentioning that STMED and STMEA write first,
% and then move the pointer, and that LDMED and LDMEA move the pointer first, and then read.
% In other words, ARM not only lets the stack grow in a non-standard direction,
% but also in a non-standard order.
% Maybe this can be in the glossary, which would explain why E stands for "empty".

\subsection{Warum wächst der Stack nach unten?}
\label{stack_grow_backwards}

Intuitiv, würden man annehmen das der Stack nach oben wächst z.B Richtung höherer Adressen, so wie bei jeder anderen Datenstruktur.

Der Grund das der Stack rückwärts wächst ist wohl historisch bedingt. Als Computer so groß waren das sie einen ganzen Raum beansprucht haben war es einfach Speicher in zwei Sektionen zu unterteilen, einen Teil für den \gls{heap} und einen Teil für den Stack. Sicher war zu dieser Zeit nicht bekannt wie groß der \gls{heap} und der Stack wachsen würden, während der Programm Laufzeit, also war die Lösung die einfachste mögliche.

\input{patterns/02_stack/stack_and_heap}

In \RitchieThompsonUNIX können wir folgendes lesen:

\begin{framed}
\begin{quotation}
Der user-core eines Programm Images wird in drei logische Segmente unterteilt. Das Programm-Text Segment beginnt bei 0 im virtuellen Adress Speicher. Während der Ausführung wird das Segment als schreibgeschützt markiert und eine einzelne Kopie des Segments wird unter allen Prozessen geteilt die das Programm ausführen. An der ersten 8K grenze über dem Programm Text Segment im Virtuellen Speicher, fängt der ``nonshared'' Bereich an, der nach Bedarf von Syscalls erweitert werden kann. Beginnend bei der höchsten Adresse im Virtuellen Speicher ist das Stack Segment, das Automatisch nach unten wächst während der Hardware Stackpointer sich ändert.
\end{quotation}
\end{framed}

Das erinnert daran wie manche Schüler Notizen zu  zwei Vorträgen in einem Notebook dokumentieren:
Notizen für den ersten Vortrag werden normal notiert, und Notizen zur zum zweiten Vortrag werden 
ans Ende des Notizbuches geschrieben, indem man das Notizbuch umdreht. Die Notizen treffen sich irgendwann
im Notizbuch aufgrund des fehlenden Freien Platzes.

% I think if we want to expand on this analogy,
% one might remember that the line number increases as as you go down a page.
% So when you decrease the address when pushing to the stack, visually,
% the stack does grow upwards.
% Of course, the problem is that in most human languages,
% just as with computers,
% we write downwards, so this direction is what makes buffer overflows so messy.

\subsection{Für was wird der Stack benutzt?}

% subsections
\EN{\input{patterns/02_stack/01_saving_ret_addr_EN}}
\RU{\input{patterns/02_stack/01_saving_ret_addr_RU}}
\DE{\input{patterns/02_stack/01_saving_ret_addr_DE}}
\FR{\input{patterns/02_stack/01_saving_ret_addr_FR}}
\PTBR{\input{patterns/02_stack/01_saving_ret_addr_PTBR}}
\IT{\input{patterns/02_stack/01_saving_ret_addr_IT}}
\PL{\input{patterns/02_stack/01_saving_ret_addr_PL}}
\JA{\input{patterns/02_stack/01_saving_ret_addr_JA}}

\EN{\input{patterns/02_stack/02_args_passing_EN}}
\RU{\input{patterns/02_stack/02_args_passing_RU}}
\PTBR{\input{patterns/02_stack/02_args_passing_PTBR}}
\DE{\input{patterns/02_stack/02_args_passing_DE}}
\IT{\input{patterns/02_stack/02_args_passing_IT}}
\FR{\input{patterns/02_stack/02_args_passing_FR}}
\JA{\input{patterns/02_stack/02_args_passing_JA}}
\PL{\input{patterns/02_stack/02_args_passing_PL}}


\EN{\input{patterns/02_stack/03_local_vars_EN}}
\RU{\input{patterns/02_stack/03_local_vars_RU}}
\DE{\input{patterns/02_stack/03_local_vars_DE}}
\PTBR{\input{patterns/02_stack/03_local_vars_PTBR}}
\EN{\input{patterns/02_stack/04_alloca/main_EN}}
\FR{\input{patterns/02_stack/04_alloca/main_FR}}
\RU{\input{patterns/02_stack/04_alloca/main_RU}}
\PTBR{\input{patterns/02_stack/04_alloca/main_PTBR}}
\IT{\input{patterns/02_stack/04_alloca/main_IT}}
\DE{\input{patterns/02_stack/04_alloca/main_DE}}
\PL{\input{patterns/02_stack/04_alloca/main_PL}}
\JA{\input{patterns/02_stack/04_alloca/main_JA}}

\subsubsection{(Windows) SEH}
\myindex{Windows!Structured Exception Handling}

\ifdefined\RUSSIAN
В стеке хранятся записи \ac{SEH} для функции (если они присутствуют).
Читайте больше о нем здесь: (\myref{sec:SEH}).
\fi % RUSSIAN

\ifdefined\ENGLISH
\ac{SEH} records are also stored on the stack (if they are present).
Read more about it: (\myref{sec:SEH}).
\fi % ENGLISH

\ifdefined\BRAZILIAN
\ac{SEH} também são guardados na pilha (se estiverem presentes).
\PTBRph{}: (\myref{sec:SEH}).
\fi % BRAZILIAN

\ifdefined\ITALIAN
I record \ac{SEH}, se presenti, sono anch'essi memorizzati nello stack.
Maggiori informazioni qui: (\myref{sec:SEH}).
\fi % ITALIAN

\ifdefined\FRENCH
Les enregistrements \ac{SEH} sont aussi stockés dans la pile (s'ils sont présents).
Lire à ce propos: (\myref{sec:SEH}).
\fi % FRENCH


\ifdefined\POLISH
Na stosie są przechowywane wpisy \ac{SEH} dla funkcji (jeśli są one obecne).
Więcej o tym tutaj: (\myref{sec:SEH}).
\fi % POLISH

\ifdefined\JAPANESE
\ac{SEH}レコードはスタックにも格納されます(存在する場合)。
それについてもっと読む:(\myref{sec:SEH})
\fi % JAPANESE

\ifdefined\ENGLISH
\subsubsection{Buffer overflow protection}

More about it here~(\myref{subsec:bufferoverflow}).
\fi

\ifdefined\RUSSIAN
\subsubsection{Защита от переполнений буфера}

Здесь больше об этом~(\myref{subsec:bufferoverflow}).
\fi

\ifdefined\BRAZILIAN
\subsubsection{Proteção contra estouro de buffer}

Mais sobre aqui~(\myref{subsec:bufferoverflow}).
\fi

\ifdefined\ITALIAN
\subsubsection{Protezione da buffer overflow}

Maggiori informazioni qui~(\myref{subsec:bufferoverflow}).
\fi

\ifdefined\FRENCH
\subsubsection{Protection contre les débordements de tampon}

Lire à ce propos~(\myref{subsec:bufferoverflow}).
\fi


\ifdefined\POLISH
\subsubsection{Metody zabiezpieczenia przed przepełnieniem stosu}

Więcej o tym tutaj~(\myref{subsec:bufferoverflow}).
\fi

\ifdefined\JAPANESE
\subsubsection{バッファオーバーフロー保護}

詳細はこちら~(\myref{subsec:bufferoverflow})
\fi

\subsubsection{Automatisches deallokieren der Daten auf dem Stack}

Vielleicht ist der Grund warum man lokale Variablen und SEH Einträge auf dem Stack speichert, weil sie beim 
verlassen der Funktion automatisch aufgeräumt werden. Man braucht dabei nur eine Instruktion um die Position
des Stackpointers zu korrigieren (oftmals ist es die \ADD Instruktion). Funktions Argumente, könnte man sagen 
werden auch am Ende der Funktion deallokiert. Im Kontrast dazu, alles was auf dem \emph{heap} gespeichert wird muss
explizit deallokiert werden. 

% sections
\EN{\input{patterns/02_stack/07_layout_EN}}
\RU{\input{patterns/02_stack/07_layout_RU}}
\DE{\input{patterns/02_stack/07_layout_DE}}
\PTBR{\input{patterns/02_stack/07_layout_PTBR}}
\EN{\input{patterns/02_stack/08_noise/main_EN}}
\FR{\input{patterns/02_stack/08_noise/main_FR}}
\RU{\input{patterns/02_stack/08_noise/main_RU}}
\IT{\input{patterns/02_stack/08_noise/main_IT}}
\DE{\input{patterns/02_stack/08_noise/main_DE}}
\PL{\input{patterns/02_stack/08_noise/main_PL}}
\JA{\input{patterns/02_stack/08_noise/main_JA}}

\input{patterns/02_stack/exercises}
}
\FR{\subsection{Renvoyer une valeur 64-bit}

\lstinputlisting[style=customc]{patterns/185_64bit_in_32_env/ret/0.c}

\subsubsection{x86}

Dans un environnement 32-bit, les valeurs 64-bit sont renvoyées des fonctions dans
la paire de registres \EDX{}:\EAX{}.

\lstinputlisting[caption=MSVC 2010 \Optimizing,style=customasmx86]{patterns/185_64bit_in_32_env/ret/0_MSVC_2010_Ox.asm}

\subsubsection{ARM}

Une valeur 64-bit est renvoyée dans la paire de registres \Reg{0}-\Reg{1} (\Reg{1}
est pour la partie haute et \Reg{0} pour la partie basse):

\lstinputlisting[caption=\OptimizingKeilVI (\ARMMode),style=customasmARM]{patterns/185_64bit_in_32_env/ret/Keil_ARM_O3.s}

\subsubsection{MIPS}

Une valeur 64-bit est renvoyée dans la paire de registres \TT{V0}-\TT{V1} (\$2-\$3)
(\TT{V0} (\$2) est pour la partie haute et \TT{V1} (\$3) pour la partie basse):

\lstinputlisting[caption=GCC 4.4.5 \Optimizing (listing assembleur),style=customasmMIPS]{patterns/185_64bit_in_32_env/ret/0_MIPS.s}

\lstinputlisting[caption=GCC 4.4.5 \Optimizing (IDA),style=customasmMIPS]{patterns/185_64bit_in_32_env/ret/0_MIPS_IDA.lst}
}
\JA{\subsection{64ビットの値を返す}

\lstinputlisting[style=customc]{patterns/185_64bit_in_32_env/ret/0.c}

\subsubsection{x86}

32ビット環境では、64ビットの値は \EDX{}:\EAX{} レジスタペアを使って関数から返されます。

\lstinputlisting[caption=\Optimizing MSVC 2010,style=customasmx86]{patterns/185_64bit_in_32_env/ret/0_MSVC_2010_Ox.asm}

\subsubsection{ARM}

64ビットの値は \Reg{0}-\Reg{1} レジスタペアを使って返されます(\Reg{1}は高位の部分を\Reg{0}は低位の部分です)。

\lstinputlisting[caption=\OptimizingKeilVI (\ARMMode),style=customasmARM]{patterns/185_64bit_in_32_env/ret/Keil_ARM_O3.s}

\subsubsection{MIPS}

64ビットの値は\TT{V0}-\TT{V1} (\$2-\$3)レジスタペアを使って返されます(\TT{V0} (\$2)は高位の部分を\TT{V1} (\$3)は低位の部分です)。

\lstinputlisting[caption=\Optimizing GCC 4.4.5 (assembly listing),style=customasmMIPS]{patterns/185_64bit_in_32_env/ret/0_MIPS.s}

\lstinputlisting[caption=\Optimizing GCC 4.4.5 (IDA),style=customasmMIPS]{patterns/185_64bit_in_32_env/ret/0_MIPS_IDA.lst}
}

\EN{\subsection{Win32 PE}
\label{win32_pe}
\myindex{Windows!Win32}

\acs{PE} is an executable file format used in Windows.
The difference between .exe, .dll and .sys is that .exe and .sys usually do not have exports, only imports.

\myindex{OEP}

A \ac{DLL}, just like any other PE-file, has an entry point (\ac{OEP}) (the function DllMain() is located there)
but this function usually does nothing.
.sys is usually a device driver.
As of drivers, Windows requires the checksum to be present in the PE file and for it to be correct
\footnote{For example, Hiew(\myref{Hiew}) can calculate it}.

\myindex{Windows!Windows Vista}
Starting at Windows Vista, a driver's files must also be signed with a digital signature. It will fail to load otherwise.

\myindex{MS-DOS}
Every PE file begins with tiny DOS program that prints a
message like \q{This program cannot be run in DOS mode.}---if you run this program in DOS or Windows 3.1 (\ac{OS}-es which are not aware of the PE format),
this message will be printed.

\subsubsection{Terminology}

\myindex{VA}
\myindex{Base address}
\myindex{RVA}
\myindex{Windows!IAT}
\myindex{Windows!INT}

\begin{itemize}
\item Module---a separate file, .exe or .dll.

\item Process---a program loaded into memory and currently running.  Commonly consists of one .exe file and bunch of .dll files.

\item Process memory---the memory a process works with.  Each process has its own.
There usually are loaded modules, memory of the stack, \gls{heap}(s), etc.

\item \ac{VA}---an address which is to be used in program while runtime.

\item Base address (of module)---the address within the process memory at which the module is to be loaded.
\ac{OS} loader may change it, if the base address is already occupied by another module just loaded before.

\item \ac{RVA}---the \ac{VA}-address minus the base address.

Many addresses in PE-file tables use \ac{RVA}-addresses.

%\item
%Data directory --- ...

\item \ac{IAT}---an array of addresses of imported symbols \footnote{\PietrekPE}.
Sometimes, the \TT{IMAGE\_DIRECTORY\_ENTRY\_IAT} data directory points at the \ac{IAT}.
\label{IDA_idata}
It is worth noting that \ac{IDA} (as of 6.1) may allocate a pseudo-section named \TT{.idata} for
\ac{IAT}, even if the \ac{IAT} is a part of another section!

\item \ac{INT}---an array of names of symbols to be imported\footnote{\PietrekPE}.
\end{itemize}

\subsubsection{Base address}

The problem is that several module authors can prepare DLL files for others to use and it is not possible
to reach an agreement which addresses is to be assigned to whose modules.

So that is why if two necessary DLLs for a process have the same base address,
one of them will be loaded at this base address, and the other---at some other free space in process memory,
and each virtual addresses in the second DLL will be corrected.

\par With \ac{MSVC} the linker often generates the .exe files with a base address of \TT{0x400000}
\footnote{The origin of this address choice is described here: \href{http://go.yurichev.com/17041}{MSDN}},
and with the code section starting at \TT{0x401000}.
This means that the \ac{RVA} of the start of the code section is \TT{0x1000}.

DLLs are often generated by MSVC's linker with a base address of \TT{0x10000000}
\footnote{This can be changed by the /BASE linker option}.

\myindex{ASLR}

There is also another reason to load modules at various base addresses, in this case random ones.
It is \ac{ASLR}\footnote{\href{http://go.yurichev.com/17140}{wikipedia}}.

\myindex{Shellcode}

A shellcode trying to get executed on a compromised system must call system functions, hence, know their addresses.

In older \ac{OS} (in \gls{Windows NT} line: before Windows Vista),
system DLL (like kernel32.dll, user32.dll) were always loaded at known addresses,
and if we also recall
that their versions rarely changed, the addresses of functions were
fixed and shellcode could call them directly.

In order to avoid this, the \ac{ASLR}
method loads your program and all modules it needs at random base addresses, different every time.

\ac{ASLR} support is denoted in a PE file by setting the flag
\par \TT{IMAGE\_DLL\_CHARACTERISTICS\_DYNAMIC\_BASE} \InSqBrackets{see \Russinovich}.

\subsubsection{Subsystem}

There is also a \emph{subsystem} field, usually it is:

\myindex{Native API}

\begin{itemize}
\item native\footnote{Meaning, the module use Native API instead of Win32} (.sys-driver),

\item console (console application) or

\item \ac{GUI} (non-console).
\end{itemize}

\subsubsection{OS version}

A PE file also specifies the minimal Windows version it needs in order to be loadable.

The table of version numbers stored in the PE file and corresponding Windows codenames is
here\footnote{\href{http://go.yurichev.com/17044}{wikipedia}}.

\myindex{Windows!Windows NT4}
\myindex{Windows!Windows 2000}
For example, \ac{MSVC} 2005 compiles .exe files for running on Windows NT4 (version 4.00), but \ac{MSVC} 2008 does not
(the generated files have a version of 5.00, at least Windows 2000 is needed to run them).

\myindex{Windows!Windows XP}

\ac{MSVC} 2012 generates .exe files of version 6.00 by default,
targeting at least Windows Vista.
However, by changing the compiler's options\footnote{\href{http://go.yurichev.com/17045}{MSDN}},
it is possible to force it to compile for Windows XP.

\subsubsection{Sections}

Division in sections, as it seems, is present in all executable file formats.

It is devised in order to separate code from data, and data---from constant data.

\begin{itemize}
\item Either the \emph{IMAGE\_SCN\_CNT\_CODE} or \emph{IMAGE\_SCN\_MEM\_EXECUTE} flags will be set on the code section---this is executable code.

\item On data section---\emph{IMAGE\_SCN\_CNT\_INITIALIZED\_DATA},\\
\emph{IMAGE\_SCN\_MEM\_READ} and \emph{IMAGE\_SCN\_MEM\_WRITE} flags.

\item On an empty section with uninitialized data---\\
\emph{IMAGE\_SCN\_CNT\_UNINITIALIZED\_DATA}, \emph{IMAGE\_SCN\_MEM\_READ} \\
        and \emph{IMAGE\_SCN\_MEM\_WRITE}.

\item On a constant data section (one that's protected from writing), the flags \\
\emph{IMAGE\_SCN\_CNT\_INITIALIZED\_DATA} and \emph{IMAGE\_SCN\_MEM\_READ} can be set, \\
but not \emph{IMAGE\_SCN\_MEM\_WRITE}.
A process going to crash if it tries to write to this section.
\end{itemize}

\myindex{TLS}
\myindex{BSS}
Each section in PE-file may have a name, however, it is not very important.
Often (but not always) the code section is named \TT{.text},
the data section---\TT{.data}, the constant data section --- \TT{.rdata} \emph{(readable data)}.
Other popular section names are:

\myindex{MIPS}
\begin{itemize}
\item \TT{.idata}---imports section.
\ac{IDA} may create a pseudo-section named like this: \myref{IDA_idata}.
\item \TT{.edata}---exports section (rare)
\item \TT{.pdata}---section holding all information about exceptions in Windows NT for MIPS, \ac{IA64} and x64: \myref{SEH_win64}
\item \TT{.reloc}---relocs section
\item \TT{.bss}---uninitialized data (\ac{BSS})
\item \TT{.tls}---thread local storage (\ac{TLS})
\item \TT{.rsrc}---resources
\item \TT{.CRT}---may present in binary files compiled by ancient MSVC versions
\end{itemize}

PE file packers/encryptors often garble section names or replace the names with their own.

\ac{MSVC} allows you to declare data in arbitrarily named section
\footnote{\href{http://go.yurichev.com/17047}{MSDN}}.

Some compilers and linkers can add a section with debugging symbols and
other debugging information (MinGW for instance).
\myindex{Windows!PDB}
However it is not so in latest versions of \ac{MSVC} (separate \gls{PDB} files are used there for this purpose).\\
\\
That is how a PE section is described in the file:

\begin{lstlisting}[style=customc]
typedef struct _IMAGE_SECTION_HEADER {
  BYTE  Name[IMAGE_SIZEOF_SHORT_NAME];
  union {
    DWORD PhysicalAddress;
    DWORD VirtualSize;
  } Misc;
  DWORD VirtualAddress;
  DWORD SizeOfRawData;
  DWORD PointerToRawData;
  DWORD PointerToRelocations;
  DWORD PointerToLinenumbers;
  WORD  NumberOfRelocations;
  WORD  NumberOfLinenumbers;
  DWORD Characteristics;
} IMAGE_SECTION_HEADER, *PIMAGE_SECTION_HEADER;
\end{lstlisting}
\footnote{\href{http://go.yurichev.com/17048}{MSDN}}

\myindex{Hiew}
A word about terminology: \emph{PointerToRawData} is called \q{Offset} in Hiew
and \emph{VirtualAddress} is called \q{RVA} there.

\subsubsection{Data section}

Data section in file can be smaller than in memory.
For example, some variables can be initialized, some are not.
Compiler and linker will collect them all into one section, but the first part of it is initialized and allocated in file,
while another is absent in file (of course, to make it smaller).
\emph{VirtualSize} will be equal to the size of section in memory, and \emph{SizeOfRawData} --- to
size of section in file.

IDA can show the border between initialized and not initialized parts like that:

\begin{lstlisting}[style=customasmx86]
...

.data:10017FFA                 db    0
.data:10017FFB                 db    0
.data:10017FFC                 db    0
.data:10017FFD                 db    0
.data:10017FFE                 db    0
.data:10017FFF                 db    0
.data:10018000                 db    ? ;
.data:10018001                 db    ? ;
.data:10018002                 db    ? ;
.data:10018003                 db    ? ;
.data:10018004                 db    ? ;
.data:10018005                 db    ? ;

...
\end{lstlisting}

\subsubsection{Relocations (relocs)}
\label{subsec:relocs}

\ac{AKA} FIXUP-s (at least in Hiew).

They are also present in almost all executable file formats
\footnote{Even in .exe files for MS-DOS}.
Exceptions are shared dynamic libraries compiled with \ac{PIC}, or any other \ac{PIC}-code.

What are they for?

Obviously, modules can be loaded on various base addresses, but how to deal with global variables, for example?
They must be accessed by address.  One solution is \PICcode{} (\myref{sec:PIC}).
But it is not always convenient.

That is why a relocations table is present.
There the addresses of points that must be corrected are enumerated,
in case of loading at a different base address.

% TODO тут бы пример с HIEW или objdump..
For example, there is a global variable at address \TT{0x410000} and this is how it is accessed:

\begin{lstlisting}[style=customasmx86]
A1 00 00 41 00         mov         eax,[000410000]
\end{lstlisting}

The base address of the module is \TT{0x400000}, the \ac{RVA} of the global variable is \TT{0x10000}.

If the module is loaded at base address \TT{0x500000}, the real address of the global variable must be \TT{0x510000}.

\myindex{x86!\Instructions!MOV}

As we can see, the address of variable is encoded in the instruction \TT{MOV}, after the byte \TT{0xA1}.

That is why the address of the 4 bytes after \TT{0xA1}, is written in the relocs table.

If the module is loaded at a different base address, the \ac{OS} loader enumerates all addresses in the table,

finds each 32-bit word the address points to, subtracts the original base address from it
(we get the \ac{RVA} here), and adds the new base address to it.

If a module is loaded at its original base address, nothing happens.

All global variables can be treated like that.

Relocs may have various types, however, in Windows for x86 processors, the type is usually \\
\emph{IMAGE\_REL\_BASED\_HIGHLOW}.

\myindex{Hiew}

By the way, relocs are darkened in Hiew, for example: \figref{fig:scanf_ex3_hiew_1}.

\myindex{\olly}
\olly underlines the places in memory to which relocs are to be applied, for example: \figref{fig:switch_lot_olly3}.

\subsubsection{Exports and imports}

\label{PE_exports_imports}
As we all know, any executable program must use the \ac{OS}'s services and other DLL-libraries somehow.

It can be said that functions from one module (usually DLL) must be connected somehow to the points of their
calls in other modules (.exe-file or another DLL).

For this, each DLL has an \q{exports} table, which consists of functions plus their addresses in a module.

And every .exe file or DLL has \q{imports}, a table of functions it needs for execution including
list of DLL filenames.

After loading the main .exe-file, the \ac{OS} loader processes imports table:
it loads the additional DLL-files, finds function names
among the DLL exports and writes their addresses down in the \ac{IAT} of the main .exe-module.

\myindex{Windows!Win32!Ordinal}

As we can see, during loading the loader must compare a lot of function names, but string comparison is not a very
fast procedure, so there is a support for \q{ordinals} or \q{hints},
which are function numbers stored in the table, instead of their names.

That is how they can be located faster when loading a DLL.
Ordinals are always present in the \q{export} table.

\myindex{MFC}
For example, a program using the \ac{MFC} library usually loads mfc*.dll by ordinals,
and in such programs there are no \ac{MFC} function names in \ac{INT}.

% TODO example!
When loading such programs in \IDA, it will ask for a path to the mfc*.dll files
in order to determine the function names.

If you don't tell \IDA the path to these DLLs, there will be \emph{mfc80\_123} instead of function names.

\myparagraph{Imports section}

Often a separate section is allocated for the imports table and everything related to it (with name like \TT{.idata}),
however, this is not a strict rule.

Imports are also a confusing subject because of the terminological mess. Let's try to collect all information in one place.

\begin{figure}[H]
\centering
\myincludegraphics{OS/PE/unnamed0.png}
\caption{
A scheme that unites all PE-file structures related to imports}
\end{figure}

The main structure is the array \emph{IMAGE\_IMPORT\_DESCRIPTOR}.
Each element for each DLL being imported.

Each element holds the \ac{RVA} address of the text string (DLL name) (\emph{Name}).

\emph{OriginalFirstThunk} is the \ac{RVA} address of the \ac{INT} table.
This is an array of \ac{RVA} addresses, each of which points to a text string with a function name.
Each string is prefixed by a 16-bit integer
(\q{hint})---\q{ordinal} of function.

While loading, if it is possible to find a function by ordinal,
then the strings comparison will not occur. The array is terminated by zero.

There is also a pointer to the \ac{IAT} table named \emph{FirstThunk}, it is just the \ac{RVA} address
of the place where the loader writes the addresses of the resolved functions.

The points where the loader writes addresses are marked by \IDA like this: \emph{\_\_imp\_CreateFileA}, etc.

There are at least two ways to use the addresses written by the loader.

\myindex{x86!\Instructions!CALL}
\begin{itemize}
\item The code will have instructions like \emph{call \_\_imp\_CreateFileA},
and since the field with the address of the imported function is a global variable in some sense,
the address of the \emph{call} instruction (plus 1 or 2) is to be added to the relocs table,
for the case when the module is loaded at a different base address.

But, obviously, this may enlarge relocs table significantly.

Because there are might be a lot of calls to imported functions in the module.

Furthermore, large relocs table slows down the process of loading modules.

\myindex{x86!\Instructions!JMP}
\myindex{thunk-functions}
\item For each imported function, there is only one jump allocated, using the \JMP instruction
plus a reloc to it.
Such points are also called \q{thunks}.

All calls to the imported functions are just \CALL instructions to the corresponding \q{thunk}.
In this case, additional relocs are not necessary because these \CALL{}-s
have relative addresses and do not need to be corrected.
\end{itemize}

These two methods can be combined.

Possible, the linker creates individual \q{thunk}s if there are too many calls to the function,
but not done by default. \\
\\
By the way, the array of function addresses to which FirstThunk is pointing is not necessary to be located in the \ac{IAT} section.
For example, the author of these lines once wrote the PE\_add\_import\footnote{\href{http://go.yurichev.com/17049}{yurichev.com}}
utility for adding imports to an existing .exe-file.

Some time earlier, in the previous versions of the utility,
at the place of the function you want to substitute with a call to another DLL,
my utility wrote the following code:

\begin{lstlisting}[style=customasmx86]
MOV EAX, [yourdll.dll!function]
JMP EAX
\end{lstlisting}

FirstThunk points to the first instruction. In other words, when loading yourdll.dll,
the loader writes the address of the \emph{function} function right in the code.

It also worth noting that a code section is usually write-protected, so my utility adds the \\
\emph{IMAGE\_SCN\_MEM\_WRITE}
flag for code section. Otherwise, the program to crash while loading with error code
5 (access denied). \\
\\
One might ask: what if I supply a program with a set of DLL files which is not supposed to change (including addresses of all DLL functions),
is it possible to speed up the loading process?

Yes, it is possible to write the addresses of the functions to be imported into the FirstThunk arrays in advance.
The \emph{Timestamp} field is present in the \\
\emph{IMAGE\_IMPORT\_DESCRIPTOR} structure.

If a value is present there, then the loader compares this value with the date-time of the DLL file.

If the values are equal, then the loader does not do anything, and the loading of the process can be faster.
This is called \q{old-style binding}
\footnote{\href{http://go.yurichev.com/17050}{MSDN}. There is also the \q{new-style binding}.}.
\myindex{BIND.EXE}

The BIND.EXE utility in Windows SDK is for this.
For speeding up the loading of your program, Matt Pietrek in \PietrekPEURL, suggests to do the binding shortly after your program
installation on the computer of the end user. \\
\\
PE-files packers/encryptors may also compress/encrypt imports table.

In this case, the Windows loader, of course, will not load all necessary DLLs.
\myindex{Windows!Win32!LoadLibrary}
\myindex{Windows!Win32!GetProcAddress}

Therefore, the packer/encryptor does this on its own, with the help of
\emph{LoadLibrary()} and the \emph{GetProcAddress()} functions.

That is why these two functions are often present in \ac{IAT} in packed files.\\
\\
In the standard DLLs from the Windows installation, \ac{IAT} often is located right at the beginning of the PE file.
Supposedly, it is made so for optimization.

While loading, the .exe file is not loaded into memory as a whole (recall huge install programs which are
started suspiciously fast), it is \q{mapped}, and loaded into memory in parts as they are accessed.

Probably, Microsoft developers decided it will be faster.

\subsubsection{Resources}

\label{PEresources}

Resources in a PE file are just a set of icons, pictures, text strings, dialog descriptions.

Perhaps they were separated from the main code, so all these things could be multilingual,
and it would be simpler to pick text or picture for the language that is currently set in the \ac{OS}. \\
\\
As a side effect, they can be edited easily and saved back to the executable file, even if one does not have special knowledge,
by using the ResHack editor, for example (\myref{ResHack}).

\subsubsection{.NET}

\myindex{.NET}

.NET programs are not compiled into machine code but into a special bytecode.
\myindex{OEP}
Strictly speaking, there is bytecode instead of the usual x86 code
in the .exe file, however, the entry point (\ac{OEP}) points to this tiny fragment of x86 code:

\begin{lstlisting}[style=customasmx86]
jmp         mscoree.dll!_CorExeMain
\end{lstlisting}

The .NET loader is located in mscoree.dll, which processes the PE file.
\myindex{Windows!Windows XP}

It was so in all pre-Windows XP \ac{OS}es. Starting from XP, the \ac{OS} loader is able to detect the .NET file
and run it without executing that \JMP instruction
\footnote{\href{http://go.yurichev.com/17051}{MSDN}}.

\myindex{TLS}
\subsubsection{TLS}

This section holds initialized data for the \ac{TLS}(\myref{TLS}) (if needed).
When a new thread start, its \ac{TLS} data is initialized using the data from this section. \\
\\
\myindex{TLS!Callbacks}

Aside from that, the PE file specification also provides initialization of the
\ac{TLS} section, the so-called TLS callbacks.

If they are present, they are to be called before the control is passed to the main entry point (\ac{OEP}).

This is used widely in the PE file packers/encryptors.

\subsubsection{Tools}

\myindex{objdump}
\myindex{Cygwin}
\myindex{Hiew}
\label{ResHack}

\begin{itemize}
\item objdump (present in cygwin) for dumping all PE-file structures.

\item Hiew(\myref{Hiew}) as editor.

\item pefile---Python-library for PE-file processing \footnote{\url{http://go.yurichev.com/17052}}.

\item ResHack \acs{AKA} Resource Hacker---resources editor\footnote{\url{http://go.yurichev.com/17052}}.

\item PE\_add\_import\footnote{\url{http://go.yurichev.com/17049}}---
simple tool for adding symbol(s) to PE executable import table.

\item PE\_patcher\footnote{\href{http://go.yurichev.com/17054}{yurichev.com}}---simple tool for patching PE executables.

\item PE\_search\_str\_refs\footnote{\href{http://go.yurichev.com/17055}{yurichev.com}}---simple tool for searching for a function in PE executables which use some text string.
\end{itemize}

\subsubsection{Further reading}

% FIXME: bibliography per chapter or section
\begin{itemize}
\item Daniel Pistelli---The .NET File Format \footnote{\url{http://go.yurichev.com/17056}}
\end{itemize}

}
\RU{\mysection{Пример вычисления адреса сети}

Как мы знаем, TCP/IP-адрес (IPv4) состоит из четырех чисел в пределах $0 \ldots 255$, т.е. 4 байта.

4 байта легко помещаются в 32-битную переменную, так что адрес хоста в IPv4, сетевая маска или адрес сети
могут быть 32-битными числами.

С точки зрения пользователя, маска сети определяется четырьмя числами в формате вроде \\
255.255.255.0,
но сетевые инженеры (сисадмины) используют более компактную нотацию (\ac{CIDR}),
вроде  \q{/8}, \q{/16}, итд.

Эта нотация просто определяет количество бит в сетевой маске, начиная с \ac{MSB}.

\small
\begin{center}
\begin{tabular}{ | l | l | l | l | l | l | }
\hline
\HeaderColor Маска & 
\HeaderColor Хосты & 
\HeaderColor Свободно &
\HeaderColor Сетевая маска &
\HeaderColor В шест.виде &
\HeaderColor \\
\hline
/30  & 4        & 2        & 255.255.255.252  & 0xfffffffc  & \\
\hline
/29  & 8        & 6        & 255.255.255.248  & 0xfffffff8  & \\
\hline
/28  & 16       & 14       & 255.255.255.240  & 0xfffffff0  & \\
\hline
/27  & 32       & 30       & 255.255.255.224  & 0xffffffe0  & \\
\hline
/26  & 64       & 62       & 255.255.255.192  & 0xffffffc0  & \\
\hline
/24  & 256      & 254      & 255.255.255.0    & 0xffffff00  & сеть класса C \\
\hline
/23  & 512      & 510      & 255.255.254.0    & 0xfffffe00  & \\
\hline
/22  & 1024     & 1022     & 255.255.252.0    & 0xfffffc00  & \\
\hline
/21  & 2048     & 2046     & 255.255.248.0    & 0xfffff800  & \\
\hline
/20  & 4096     & 4094     & 255.255.240.0    & 0xfffff000  & \\
\hline
/19  & 8192     & 8190     & 255.255.224.0    & 0xffffe000  & \\
\hline
/18  & 16384    & 16382    & 255.255.192.0    & 0xffffc000  & \\
\hline
/17  & 32768    & 32766    & 255.255.128.0    & 0xffff8000  & \\
\hline
/16  & 65536    & 65534    & 255.255.0.0      & 0xffff0000  & сеть класса B \\
\hline
/8   & 16777216 & 16777214 & 255.0.0.0        & 0xff000000  & сеть класса A \\
\hline
\end{tabular}
\end{center}
\normalsize

Вот простой пример, вычисляющий адрес сети используя сетевую маску и адрес хоста.

\lstinputlisting[style=customc]{\CURPATH/netmask.c}

\subsection{calc\_network\_address()}

Функция \TT{calc\_network\_address()} самая простая: 

она просто умножает (логически, используя \AND) адрес хоста на сетевую маску, в итоге давая адрес
сети.

\lstinputlisting[caption=\Optimizing MSVC 2012 /Ob0,numbers=left,style=customasmx86]{\CURPATH/calc_network_address_MSVC_2012_Ox.asm}

На строке 22 мы видим самую важную инструкцию \AND --- так вычисляется адрес сети.

\subsection{form\_IP()}

Функция \TT{form\_IP()} просто собирает все 4 байта в одно 32-битное значение.

Вот как это обычно происходит:

\begin{itemize}
\item Выделите переменную для возвращаемого значения. Обнулите её.

\item 
Возьмите четвертый (самый младший) байт, сложите его (логически, инструкцией \OR) с возвращаемым
значением. Оно содержит теперь 4-й байт.

\item Возьмите третий байт, сдвиньте его на 8 бит влево.
Получится значение в виде \TT{0x0000bb00}, где \TT{bb} это третий байт.
Сложите итоговое значение (логически, инструкцией \OR) с возвращаемым значением.
Возвращаемое значение пока что содержит \TT{0x000000aa}, так что логическое сложение
в итоге выдаст значение вида \TT{0x0000bbaa}.

\item 
Возьмите второй байт, сдвиньте его на 16 бит влево.
Вы получите значение вида \TT{0x00cc0000}, где \TT{cc} это второй байт.
Сложите (логически) результат и возвращаемое значение.
Выходное значение содержит пока что \TT{0x0000bbaa}, так что логическое сложение
в итоге выдаст значение вида \TT{0x00ccbbaa}.

\item 
Возьмите первый байт, сдвиньте его на 24 бита влево.
Вы получите значение вида \TT{0xdd000000}, где \TT{dd} это первый байт.
Сложите (логически) результат и выходное значение.
Выходное значение содержит пока что \TT{0x00ccbbaa}, так что сложение выдаст в итоге значение
вида \TT{0xddccbbaa}.

\end{itemize}

И вот как работает неоптимизирующий MSVC 2012:

\lstinputlisting[caption=\NonOptimizing MSVC 2012,style=customasmx86]{\CURPATH/form_IP_MSVC_2012_RU.asm}

Хотя, порядок операций другой, но, конечно, порядок роли не играет.

\Optimizing MSVC 2012 делает то же самое, но немного иначе:

\lstinputlisting[caption=\Optimizing MSVC 2012 /Ob0,style=customasmx86]{\CURPATH/form_IP_MSVC_2012_Ox_RU.asm}

Можно сказать, что каждый байт записывается в младшие 8 бит возвращаемого значения,
и затем возвращаемое значение сдвигается на один байт влево на каждом шаге.

Повторять 4 раза, для каждого байта.

\par
Вот и всё! 
К сожалению, наверное, нет способа делать это иначе.
Не существует более-менее популярных \ac{CPU} или \ac{ISA}, где имеется инструкция для сборки значения из бит или байт.
Обычно всё это делает сдвигами бит и логическим сложением (OR).

\subsection{print\_as\_IP()}

\TT{print\_as\_IP()} делает наоборот: расщепляет 32-битное значение на 4 байта.

Расщепление работает немного проще: просто сдвигайте входное значение на 24, 16, 8 или 0 бит,
берите биты с нулевого по седьмой (младший байт), вот и всё:

\lstinputlisting[caption=\NonOptimizing MSVC 2012,style=customasmx86]{\CURPATH/print_as_IP_MSVC_2012_RU.asm}

\Optimizing MSVC 2012 делает почти всё то же самое, только без ненужных перезагрузок входного значения:

\lstinputlisting[caption=\Optimizing MSVC 2012 /Ob0,style=customasmx86]{\CURPATH/print_as_IP_MSVC_2012_Ox.asm}

\subsection{form\_netmask() и set\_bit()}

\TT{form\_netmask()} делает сетевую маску из \ac{CIDR}-нотации.

Конечно, было бы куда эффективнее использовать здесь какую-то уже готовую таблицу, но мы рассматриваем
это именно так, сознательно, для демонстрации битовых сдвигов.
Мы также сделаем отдельную функцию \TT{set\_bit()}. 

Не очень хорошая идея выделять отдельную функцию для такой примитивной операции, но так будет проще понять,
как это всё работает.

\lstinputlisting[caption=\Optimizing MSVC 2012 /Ob0,style=customasmx86]{\CURPATH/form_netmask_MSVC_2012_Ox.asm}

\TT{set\_bit()} примитивна: просто сдвигает единицу на нужное количество бит, затем складывает (логически) с
входным значением \q{input}.
\TT{form\_netmask()} имеет цикл: он выставит столько бит (начиная с \ac{MSB}), 
сколько передано в аргументе \TT{netmask\_bits}.

\subsection{Итог}

Вот и всё!
Мы запускаем и видим:

\begin{lstlisting}
netmask=255.255.255.0
network address=10.1.2.0
netmask=255.0.0.0
network address=10.0.0.0
netmask=255.255.255.128
network address=10.1.2.0
netmask=255.255.255.192
network address=10.1.2.64
\end{lstlisting}
}
\DE{\mysection{\Stack}
\label{sec:stack}
\myindex{\Stack}

Der Stack ist eine der fundamentalen Datenstrukturen in der Informatik.
\footnote{\href{http://go.yurichev.com/17119}{wikipedia.org/wiki/Call\_Stack}}.
\ac{AKA} \ac{LIFO}.

Technisch betrachtet ist es ein Stapelspeicher innerhalb des Prozessspeichers der zusammen mit den \ESP (x86), \RSP (x64) oder dem \ac{SP} (ARM) Register als ein Zeiger in diesem Speicherblock fungiert.

\myindex{ARM!\Instructions!PUSH}
\myindex{ARM!\Instructions!POP}
\myindex{x86!\Instructions!PUSH}
\myindex{x86!\Instructions!POP}

Die häufigsten Stack-Zugriffsinstruktionen sind die \PUSH- und \POP-Instruktionen (in beidem x86 und ARM Thumb-Modus). \PUSH subtrahiert vom \ESP/\RSP/\ac{SP} 4 Byte im 32-Bit Modus (oder 8 im 64-Bit Modus) und schreibt dann den Inhalt des Zeigers an die Adresse auf die von \ESP/\RSP/\ac{SP} gezeigt wird.

\POP ist die umgekehrte Operation: Die Daten des Zeigers für die Speicherregion auf die von \ac{SP}
gezeigt wird werden ausgelesen und die Inhalte in den Instruktionsoperanden geschreiben (oft ist das ein Register). Dann werden 4 (beziehungsweise 8) Byte zum \gls{stack pointer} addiert.

Nach der Stackallokation, zeigt der \gls{stack pointer} auf den Boden des Stacks.
\PUSH verringert den \gls{stack pointer} und \POP erhöht ihn.
Der Boden des Stacks ist eigentlich der Anfang der Speicherregion die für den Stack reserviert wurde.
Das wirkt zunächst seltsam, aber so funktioniert es.

ARM unterstützt beides, aufsteigende und absteigende Stacks.

\myindex{ARM!\Instructions!STMFD}
\myindex{ARM!\Instructions!LDMFD}
\myindex{ARM!\Instructions!STMED}
\myindex{ARM!\Instructions!LDMED}
\myindex{ARM!\Instructions!STMFA}
\myindex{ARM!\Instructions!LDMFA}
\myindex{ARM!\Instructions!STMEA}
\myindex{ARM!\Instructions!LDMEA}

Zum Beispiel die \ac{STMFD}/\ac{LDMFD} und \ac{STMED}/\ac{LDMED} Instruktionen sind alle dafür gedacht mit einem absteigendem Stack zu arbeiten ( wächst nach unten, fängt mit hohen Adressen an und entwickelt sich zu niedrigeren Adressen). Die \ac{STMFA}/\ac{LDMFA} und \ac{STMEA}/\ac{LDMEA} Instruktionen sind dazu gedacht mit einem aufsteigendem Stack zu arbeiten (wächst nach oben und fängt mit niedrigeren Adressen an und wächst nach oben).

% It might be worth mentioning that STMED and STMEA write first,
% and then move the pointer, and that LDMED and LDMEA move the pointer first, and then read.
% In other words, ARM not only lets the stack grow in a non-standard direction,
% but also in a non-standard order.
% Maybe this can be in the glossary, which would explain why E stands for "empty".

\subsection{Warum wächst der Stack nach unten?}
\label{stack_grow_backwards}

Intuitiv, würden man annehmen das der Stack nach oben wächst z.B Richtung höherer Adressen, so wie bei jeder anderen Datenstruktur.

Der Grund das der Stack rückwärts wächst ist wohl historisch bedingt. Als Computer so groß waren das sie einen ganzen Raum beansprucht haben war es einfach Speicher in zwei Sektionen zu unterteilen, einen Teil für den \gls{heap} und einen Teil für den Stack. Sicher war zu dieser Zeit nicht bekannt wie groß der \gls{heap} und der Stack wachsen würden, während der Programm Laufzeit, also war die Lösung die einfachste mögliche.

\input{patterns/02_stack/stack_and_heap}

In \RitchieThompsonUNIX können wir folgendes lesen:

\begin{framed}
\begin{quotation}
Der user-core eines Programm Images wird in drei logische Segmente unterteilt. Das Programm-Text Segment beginnt bei 0 im virtuellen Adress Speicher. Während der Ausführung wird das Segment als schreibgeschützt markiert und eine einzelne Kopie des Segments wird unter allen Prozessen geteilt die das Programm ausführen. An der ersten 8K grenze über dem Programm Text Segment im Virtuellen Speicher, fängt der ``nonshared'' Bereich an, der nach Bedarf von Syscalls erweitert werden kann. Beginnend bei der höchsten Adresse im Virtuellen Speicher ist das Stack Segment, das Automatisch nach unten wächst während der Hardware Stackpointer sich ändert.
\end{quotation}
\end{framed}

Das erinnert daran wie manche Schüler Notizen zu  zwei Vorträgen in einem Notebook dokumentieren:
Notizen für den ersten Vortrag werden normal notiert, und Notizen zur zum zweiten Vortrag werden 
ans Ende des Notizbuches geschrieben, indem man das Notizbuch umdreht. Die Notizen treffen sich irgendwann
im Notizbuch aufgrund des fehlenden Freien Platzes.

% I think if we want to expand on this analogy,
% one might remember that the line number increases as as you go down a page.
% So when you decrease the address when pushing to the stack, visually,
% the stack does grow upwards.
% Of course, the problem is that in most human languages,
% just as with computers,
% we write downwards, so this direction is what makes buffer overflows so messy.

\subsection{Für was wird der Stack benutzt?}

% subsections
\EN{\input{patterns/02_stack/01_saving_ret_addr_EN}}
\RU{\input{patterns/02_stack/01_saving_ret_addr_RU}}
\DE{\input{patterns/02_stack/01_saving_ret_addr_DE}}
\FR{\input{patterns/02_stack/01_saving_ret_addr_FR}}
\PTBR{\input{patterns/02_stack/01_saving_ret_addr_PTBR}}
\IT{\input{patterns/02_stack/01_saving_ret_addr_IT}}
\PL{\input{patterns/02_stack/01_saving_ret_addr_PL}}
\JA{\input{patterns/02_stack/01_saving_ret_addr_JA}}

\EN{\input{patterns/02_stack/02_args_passing_EN}}
\RU{\input{patterns/02_stack/02_args_passing_RU}}
\PTBR{\input{patterns/02_stack/02_args_passing_PTBR}}
\DE{\input{patterns/02_stack/02_args_passing_DE}}
\IT{\input{patterns/02_stack/02_args_passing_IT}}
\FR{\input{patterns/02_stack/02_args_passing_FR}}
\JA{\input{patterns/02_stack/02_args_passing_JA}}
\PL{\input{patterns/02_stack/02_args_passing_PL}}


\EN{\input{patterns/02_stack/03_local_vars_EN}}
\RU{\input{patterns/02_stack/03_local_vars_RU}}
\DE{\input{patterns/02_stack/03_local_vars_DE}}
\PTBR{\input{patterns/02_stack/03_local_vars_PTBR}}
\EN{\input{patterns/02_stack/04_alloca/main_EN}}
\FR{\input{patterns/02_stack/04_alloca/main_FR}}
\RU{\input{patterns/02_stack/04_alloca/main_RU}}
\PTBR{\input{patterns/02_stack/04_alloca/main_PTBR}}
\IT{\input{patterns/02_stack/04_alloca/main_IT}}
\DE{\input{patterns/02_stack/04_alloca/main_DE}}
\PL{\input{patterns/02_stack/04_alloca/main_PL}}
\JA{\input{patterns/02_stack/04_alloca/main_JA}}

\subsubsection{(Windows) SEH}
\myindex{Windows!Structured Exception Handling}

\ifdefined\RUSSIAN
В стеке хранятся записи \ac{SEH} для функции (если они присутствуют).
Читайте больше о нем здесь: (\myref{sec:SEH}).
\fi % RUSSIAN

\ifdefined\ENGLISH
\ac{SEH} records are also stored on the stack (if they are present).
Read more about it: (\myref{sec:SEH}).
\fi % ENGLISH

\ifdefined\BRAZILIAN
\ac{SEH} também são guardados na pilha (se estiverem presentes).
\PTBRph{}: (\myref{sec:SEH}).
\fi % BRAZILIAN

\ifdefined\ITALIAN
I record \ac{SEH}, se presenti, sono anch'essi memorizzati nello stack.
Maggiori informazioni qui: (\myref{sec:SEH}).
\fi % ITALIAN

\ifdefined\FRENCH
Les enregistrements \ac{SEH} sont aussi stockés dans la pile (s'ils sont présents).
Lire à ce propos: (\myref{sec:SEH}).
\fi % FRENCH


\ifdefined\POLISH
Na stosie są przechowywane wpisy \ac{SEH} dla funkcji (jeśli są one obecne).
Więcej o tym tutaj: (\myref{sec:SEH}).
\fi % POLISH

\ifdefined\JAPANESE
\ac{SEH}レコードはスタックにも格納されます(存在する場合)。
それについてもっと読む:(\myref{sec:SEH})
\fi % JAPANESE

\ifdefined\ENGLISH
\subsubsection{Buffer overflow protection}

More about it here~(\myref{subsec:bufferoverflow}).
\fi

\ifdefined\RUSSIAN
\subsubsection{Защита от переполнений буфера}

Здесь больше об этом~(\myref{subsec:bufferoverflow}).
\fi

\ifdefined\BRAZILIAN
\subsubsection{Proteção contra estouro de buffer}

Mais sobre aqui~(\myref{subsec:bufferoverflow}).
\fi

\ifdefined\ITALIAN
\subsubsection{Protezione da buffer overflow}

Maggiori informazioni qui~(\myref{subsec:bufferoverflow}).
\fi

\ifdefined\FRENCH
\subsubsection{Protection contre les débordements de tampon}

Lire à ce propos~(\myref{subsec:bufferoverflow}).
\fi


\ifdefined\POLISH
\subsubsection{Metody zabiezpieczenia przed przepełnieniem stosu}

Więcej o tym tutaj~(\myref{subsec:bufferoverflow}).
\fi

\ifdefined\JAPANESE
\subsubsection{バッファオーバーフロー保護}

詳細はこちら~(\myref{subsec:bufferoverflow})
\fi

\subsubsection{Automatisches deallokieren der Daten auf dem Stack}

Vielleicht ist der Grund warum man lokale Variablen und SEH Einträge auf dem Stack speichert, weil sie beim 
verlassen der Funktion automatisch aufgeräumt werden. Man braucht dabei nur eine Instruktion um die Position
des Stackpointers zu korrigieren (oftmals ist es die \ADD Instruktion). Funktions Argumente, könnte man sagen 
werden auch am Ende der Funktion deallokiert. Im Kontrast dazu, alles was auf dem \emph{heap} gespeichert wird muss
explizit deallokiert werden. 

% sections
\EN{\input{patterns/02_stack/07_layout_EN}}
\RU{\input{patterns/02_stack/07_layout_RU}}
\DE{\input{patterns/02_stack/07_layout_DE}}
\PTBR{\input{patterns/02_stack/07_layout_PTBR}}
\EN{\input{patterns/02_stack/08_noise/main_EN}}
\FR{\input{patterns/02_stack/08_noise/main_FR}}
\RU{\input{patterns/02_stack/08_noise/main_RU}}
\IT{\input{patterns/02_stack/08_noise/main_IT}}
\DE{\input{patterns/02_stack/08_noise/main_DE}}
\PL{\input{patterns/02_stack/08_noise/main_PL}}
\JA{\input{patterns/02_stack/08_noise/main_JA}}

\input{patterns/02_stack/exercises}
}
\FR{\subsection{Passage d'arguments, addition, soustraction}

\lstinputlisting[style=customc]{patterns/185_64bit_in_32_env/passing_add_sub/1.c}

\subsubsection{x86}

\lstinputlisting[caption=MSVC 2012 /Ob1 \Optimizing,style=customasmx86]{patterns/185_64bit_in_32_env/passing_add_sub/1_MSVC.asm}

Nous voyons dans la fonction \GTT{f\_add\_test()} que chaque valeur 64-bit est passée
en utilisant deux valeurs 32-bit, partie haute d'abord, puis partie basse.

L'addition et la soustraction se déroulent aussi par paire.

\myindex{x86!\Instructions!ADC}
Pour l'addition, la partie basse 32-bit est d'abord additionnée.
Si il y a eu une retenue pendant l'addition, le flag \TT{CF} est mis.

L'instruction suivante \INS{ADC} additionne les parties hautes, et ajoute aussi 1
si $CF=1$.

\myindex{x86!\Instructions!SBB}
La soustraction est aussi effectuée par paire.
Le premier \SUB peut aussi mettre le flag CF, qui doit être testé lors de l'instruction
\INS{SBB} suivante:
Si le flag de retenue est mis, alors 1 est soustrait du résultat.

Il est facile de voir comment le résultat de la fonction \GTT{f\_add()} est passé
à \printf{}.

\lstinputlisting[caption=GCC 4.8.1 -O1 -fno-inline,style=customasmx86]{patterns/185_64bit_in_32_env/passing_add_sub/1_GCC.asm}

Le code de GCC est le même.

\subsubsection{ARM}

\lstinputlisting[caption=\OptimizingKeilVI (\ARMMode),style=customasmARM]{patterns/185_64bit_in_32_env/passing_add_sub/Keil_ARM_O3.s}

\myindex{ARM!\Instructions!ADDS}
\myindex{ARM!\Instructions!SUBS}
\myindex{ARM!\Instructions!ADC}
\myindex{ARM!\Instructions!SBC}

La première valeur 64-bit est passée par la paire de registres \Reg{0} et \Reg{1},
la seconde dans la paire de registres \Reg{2} et \Reg{3}.
ARM a aussi l'instruction \INS{ADC} (qui compte le flag de retenue) et \INS{SBC}
(\q{soustraction avec retenue}).
Chose importante: lorsque les parties basses sont ajoutées/soustraites, les instructions
\INS{ADDS} et \INS{SUBS} avec le suffixe -S sont utilisées.
Le suffixe -S signifie \q{mettre les flags}, et les flags (en particulier le flag
de retenue) est ce dont les instructions suivantes \INS{ADC}/\INS{SBC} ont besoin.
Autrement, les instructions sans le suffixe -S feraient le travail (\ADD et \SUB).

\subsubsection{MIPS}

\lstinputlisting[caption=GCC 4.4.5 \Optimizing (IDA),style=customasmMIPS]{patterns/185_64bit_in_32_env/passing_add_sub/MIPS_O3_IDA_FR.lst}

MIPS n'a pas de registre de flags, donc il n'y a pas cette information après l'exécution
des opérations arithmétiques.
Donc il n'y a pas d'instructions comme \INS{ADC} et \INS{SBB} du x86.
Pour savoir si le flag de retenue serait mis, une comparaison est faite (en utilisant
l'instruction \INS{SLTU}), qui met le registre de destination à 1 ou 0.
Ce 1 ou ce 0 est ensuite ajouté ou soustrait au/du résultat final.

}
\JA{\subsection{Arguments passing, addition, subtraction}

\lstinputlisting[style=customc]{patterns/185_64bit_in_32_env/passing_add_sub/1.c}

\subsubsection{x86}

\lstinputlisting[caption=\Optimizing MSVC 2012 /Ob1,style=customasmx86]{patterns/185_64bit_in_32_env/passing_add_sub/1_MSVC.asm}

\GTT{f\_add\_test()}関数では、各64ビット値が2つの32ビット値を使用して渡されることを確認できます。
上位部分が最初に、次に下位部分になります。

足し算と引き算もペアで行われます。

\myindex{x86!\Instructions!ADC}
さらに、下位32ビット部分が最初に追加されます。 
加算中にキャリーが発生した場合は、\TT{CF}フラグが設定されます。

次の\INS{ADC}命令は、値の上位部分を加算し、 $CF=1$ の場合は1を加算します。

\myindex{x86!\Instructions!SBB}
減算もペアで行われます。 
最初の \SUB は、後続の \INS{SBB} 命令でチェックされるCFフラグをオンにすることもできます。
キャリーフラグがオンの場合は、結果から1も減算されます。

\GTT{f\_add()}関数の結果がどのように \printf{} に渡されるのかを理解するのは簡単です。

\lstinputlisting[caption=GCC 4.8.1 -O1 -fno-inline,style=customasmx86]{patterns/185_64bit_in_32_env/passing_add_sub/1_GCC.asm}

GCCのコードも同様です。

\subsubsection{ARM}

\lstinputlisting[caption=\OptimizingKeilVI (\ARMMode),style=customasmARM]{patterns/185_64bit_in_32_env/passing_add_sub/Keil_ARM_O3.s}

\myindex{ARM!\Instructions!ADDS}
\myindex{ARM!\Instructions!SUBS}
\myindex{ARM!\Instructions!ADC}
\myindex{ARM!\Instructions!SBC}

最初の64ビット値は\Reg{0}と\Reg{1}のレジスタペアに渡され、2番目の値は\Reg{2}と\Reg{3}のレジスタペアに渡されます。 
ARMには\INS{ADC}命令(キャリーフラグをカウントする)と\INS{SBC}(\q{subtract with carry})もあります。
重要なこと:下位部分が加算/減算されるとき、-S接尾辞付きの\INS{ADDS} および \INS{SUBS}命令が使用されます。 
-S接尾辞は\q{set flags}をあらわし、flags(特にキャリーフラグ)は、結果として生じる\INS{ADC}/\INS{SBC}命令が確実に必要とするものです。
そうでなければ、接尾辞-Sを付けずに命令を実行します( \ADD および \SUB )。

\subsubsection{MIPS}

\lstinputlisting[caption=\Optimizing GCC 4.4.5 (IDA),style=customasmMIPS]{patterns/185_64bit_in_32_env/passing_add_sub/MIPS_O3_IDA_JA.lst}

MIPSにはフラグレジスタがないため、算術演算の実行後にそのような情報は存在しません。
そのため、\INS{ADC} や \INS{SBB}ような命令はありません。 
キャリーフラグが設定されるかどうかを知るために、デスティネーションレジスタを1または0に
設定する比較{\INS{SLTU}命令を使用)も行われます。
その後、この1または0が最終結果に加算または減算されます。

}

\EN{\subsection{Win32 PE}
\label{win32_pe}
\myindex{Windows!Win32}

\acs{PE} is an executable file format used in Windows.
The difference between .exe, .dll and .sys is that .exe and .sys usually do not have exports, only imports.

\myindex{OEP}

A \ac{DLL}, just like any other PE-file, has an entry point (\ac{OEP}) (the function DllMain() is located there)
but this function usually does nothing.
.sys is usually a device driver.
As of drivers, Windows requires the checksum to be present in the PE file and for it to be correct
\footnote{For example, Hiew(\myref{Hiew}) can calculate it}.

\myindex{Windows!Windows Vista}
Starting at Windows Vista, a driver's files must also be signed with a digital signature. It will fail to load otherwise.

\myindex{MS-DOS}
Every PE file begins with tiny DOS program that prints a
message like \q{This program cannot be run in DOS mode.}---if you run this program in DOS or Windows 3.1 (\ac{OS}-es which are not aware of the PE format),
this message will be printed.

\subsubsection{Terminology}

\myindex{VA}
\myindex{Base address}
\myindex{RVA}
\myindex{Windows!IAT}
\myindex{Windows!INT}

\begin{itemize}
\item Module---a separate file, .exe or .dll.

\item Process---a program loaded into memory and currently running.  Commonly consists of one .exe file and bunch of .dll files.

\item Process memory---the memory a process works with.  Each process has its own.
There usually are loaded modules, memory of the stack, \gls{heap}(s), etc.

\item \ac{VA}---an address which is to be used in program while runtime.

\item Base address (of module)---the address within the process memory at which the module is to be loaded.
\ac{OS} loader may change it, if the base address is already occupied by another module just loaded before.

\item \ac{RVA}---the \ac{VA}-address minus the base address.

Many addresses in PE-file tables use \ac{RVA}-addresses.

%\item
%Data directory --- ...

\item \ac{IAT}---an array of addresses of imported symbols \footnote{\PietrekPE}.
Sometimes, the \TT{IMAGE\_DIRECTORY\_ENTRY\_IAT} data directory points at the \ac{IAT}.
\label{IDA_idata}
It is worth noting that \ac{IDA} (as of 6.1) may allocate a pseudo-section named \TT{.idata} for
\ac{IAT}, even if the \ac{IAT} is a part of another section!

\item \ac{INT}---an array of names of symbols to be imported\footnote{\PietrekPE}.
\end{itemize}

\subsubsection{Base address}

The problem is that several module authors can prepare DLL files for others to use and it is not possible
to reach an agreement which addresses is to be assigned to whose modules.

So that is why if two necessary DLLs for a process have the same base address,
one of them will be loaded at this base address, and the other---at some other free space in process memory,
and each virtual addresses in the second DLL will be corrected.

\par With \ac{MSVC} the linker often generates the .exe files with a base address of \TT{0x400000}
\footnote{The origin of this address choice is described here: \href{http://go.yurichev.com/17041}{MSDN}},
and with the code section starting at \TT{0x401000}.
This means that the \ac{RVA} of the start of the code section is \TT{0x1000}.

DLLs are often generated by MSVC's linker with a base address of \TT{0x10000000}
\footnote{This can be changed by the /BASE linker option}.

\myindex{ASLR}

There is also another reason to load modules at various base addresses, in this case random ones.
It is \ac{ASLR}\footnote{\href{http://go.yurichev.com/17140}{wikipedia}}.

\myindex{Shellcode}

A shellcode trying to get executed on a compromised system must call system functions, hence, know their addresses.

In older \ac{OS} (in \gls{Windows NT} line: before Windows Vista),
system DLL (like kernel32.dll, user32.dll) were always loaded at known addresses,
and if we also recall
that their versions rarely changed, the addresses of functions were
fixed and shellcode could call them directly.

In order to avoid this, the \ac{ASLR}
method loads your program and all modules it needs at random base addresses, different every time.

\ac{ASLR} support is denoted in a PE file by setting the flag
\par \TT{IMAGE\_DLL\_CHARACTERISTICS\_DYNAMIC\_BASE} \InSqBrackets{see \Russinovich}.

\subsubsection{Subsystem}

There is also a \emph{subsystem} field, usually it is:

\myindex{Native API}

\begin{itemize}
\item native\footnote{Meaning, the module use Native API instead of Win32} (.sys-driver),

\item console (console application) or

\item \ac{GUI} (non-console).
\end{itemize}

\subsubsection{OS version}

A PE file also specifies the minimal Windows version it needs in order to be loadable.

The table of version numbers stored in the PE file and corresponding Windows codenames is
here\footnote{\href{http://go.yurichev.com/17044}{wikipedia}}.

\myindex{Windows!Windows NT4}
\myindex{Windows!Windows 2000}
For example, \ac{MSVC} 2005 compiles .exe files for running on Windows NT4 (version 4.00), but \ac{MSVC} 2008 does not
(the generated files have a version of 5.00, at least Windows 2000 is needed to run them).

\myindex{Windows!Windows XP}

\ac{MSVC} 2012 generates .exe files of version 6.00 by default,
targeting at least Windows Vista.
However, by changing the compiler's options\footnote{\href{http://go.yurichev.com/17045}{MSDN}},
it is possible to force it to compile for Windows XP.

\subsubsection{Sections}

Division in sections, as it seems, is present in all executable file formats.

It is devised in order to separate code from data, and data---from constant data.

\begin{itemize}
\item Either the \emph{IMAGE\_SCN\_CNT\_CODE} or \emph{IMAGE\_SCN\_MEM\_EXECUTE} flags will be set on the code section---this is executable code.

\item On data section---\emph{IMAGE\_SCN\_CNT\_INITIALIZED\_DATA},\\
\emph{IMAGE\_SCN\_MEM\_READ} and \emph{IMAGE\_SCN\_MEM\_WRITE} flags.

\item On an empty section with uninitialized data---\\
\emph{IMAGE\_SCN\_CNT\_UNINITIALIZED\_DATA}, \emph{IMAGE\_SCN\_MEM\_READ} \\
        and \emph{IMAGE\_SCN\_MEM\_WRITE}.

\item On a constant data section (one that's protected from writing), the flags \\
\emph{IMAGE\_SCN\_CNT\_INITIALIZED\_DATA} and \emph{IMAGE\_SCN\_MEM\_READ} can be set, \\
but not \emph{IMAGE\_SCN\_MEM\_WRITE}.
A process going to crash if it tries to write to this section.
\end{itemize}

\myindex{TLS}
\myindex{BSS}
Each section in PE-file may have a name, however, it is not very important.
Often (but not always) the code section is named \TT{.text},
the data section---\TT{.data}, the constant data section --- \TT{.rdata} \emph{(readable data)}.
Other popular section names are:

\myindex{MIPS}
\begin{itemize}
\item \TT{.idata}---imports section.
\ac{IDA} may create a pseudo-section named like this: \myref{IDA_idata}.
\item \TT{.edata}---exports section (rare)
\item \TT{.pdata}---section holding all information about exceptions in Windows NT for MIPS, \ac{IA64} and x64: \myref{SEH_win64}
\item \TT{.reloc}---relocs section
\item \TT{.bss}---uninitialized data (\ac{BSS})
\item \TT{.tls}---thread local storage (\ac{TLS})
\item \TT{.rsrc}---resources
\item \TT{.CRT}---may present in binary files compiled by ancient MSVC versions
\end{itemize}

PE file packers/encryptors often garble section names or replace the names with their own.

\ac{MSVC} allows you to declare data in arbitrarily named section
\footnote{\href{http://go.yurichev.com/17047}{MSDN}}.

Some compilers and linkers can add a section with debugging symbols and
other debugging information (MinGW for instance).
\myindex{Windows!PDB}
However it is not so in latest versions of \ac{MSVC} (separate \gls{PDB} files are used there for this purpose).\\
\\
That is how a PE section is described in the file:

\begin{lstlisting}[style=customc]
typedef struct _IMAGE_SECTION_HEADER {
  BYTE  Name[IMAGE_SIZEOF_SHORT_NAME];
  union {
    DWORD PhysicalAddress;
    DWORD VirtualSize;
  } Misc;
  DWORD VirtualAddress;
  DWORD SizeOfRawData;
  DWORD PointerToRawData;
  DWORD PointerToRelocations;
  DWORD PointerToLinenumbers;
  WORD  NumberOfRelocations;
  WORD  NumberOfLinenumbers;
  DWORD Characteristics;
} IMAGE_SECTION_HEADER, *PIMAGE_SECTION_HEADER;
\end{lstlisting}
\footnote{\href{http://go.yurichev.com/17048}{MSDN}}

\myindex{Hiew}
A word about terminology: \emph{PointerToRawData} is called \q{Offset} in Hiew
and \emph{VirtualAddress} is called \q{RVA} there.

\subsubsection{Data section}

Data section in file can be smaller than in memory.
For example, some variables can be initialized, some are not.
Compiler and linker will collect them all into one section, but the first part of it is initialized and allocated in file,
while another is absent in file (of course, to make it smaller).
\emph{VirtualSize} will be equal to the size of section in memory, and \emph{SizeOfRawData} --- to
size of section in file.

IDA can show the border between initialized and not initialized parts like that:

\begin{lstlisting}[style=customasmx86]
...

.data:10017FFA                 db    0
.data:10017FFB                 db    0
.data:10017FFC                 db    0
.data:10017FFD                 db    0
.data:10017FFE                 db    0
.data:10017FFF                 db    0
.data:10018000                 db    ? ;
.data:10018001                 db    ? ;
.data:10018002                 db    ? ;
.data:10018003                 db    ? ;
.data:10018004                 db    ? ;
.data:10018005                 db    ? ;

...
\end{lstlisting}

\subsubsection{Relocations (relocs)}
\label{subsec:relocs}

\ac{AKA} FIXUP-s (at least in Hiew).

They are also present in almost all executable file formats
\footnote{Even in .exe files for MS-DOS}.
Exceptions are shared dynamic libraries compiled with \ac{PIC}, or any other \ac{PIC}-code.

What are they for?

Obviously, modules can be loaded on various base addresses, but how to deal with global variables, for example?
They must be accessed by address.  One solution is \PICcode{} (\myref{sec:PIC}).
But it is not always convenient.

That is why a relocations table is present.
There the addresses of points that must be corrected are enumerated,
in case of loading at a different base address.

% TODO тут бы пример с HIEW или objdump..
For example, there is a global variable at address \TT{0x410000} and this is how it is accessed:

\begin{lstlisting}[style=customasmx86]
A1 00 00 41 00         mov         eax,[000410000]
\end{lstlisting}

The base address of the module is \TT{0x400000}, the \ac{RVA} of the global variable is \TT{0x10000}.

If the module is loaded at base address \TT{0x500000}, the real address of the global variable must be \TT{0x510000}.

\myindex{x86!\Instructions!MOV}

As we can see, the address of variable is encoded in the instruction \TT{MOV}, after the byte \TT{0xA1}.

That is why the address of the 4 bytes after \TT{0xA1}, is written in the relocs table.

If the module is loaded at a different base address, the \ac{OS} loader enumerates all addresses in the table,

finds each 32-bit word the address points to, subtracts the original base address from it
(we get the \ac{RVA} here), and adds the new base address to it.

If a module is loaded at its original base address, nothing happens.

All global variables can be treated like that.

Relocs may have various types, however, in Windows for x86 processors, the type is usually \\
\emph{IMAGE\_REL\_BASED\_HIGHLOW}.

\myindex{Hiew}

By the way, relocs are darkened in Hiew, for example: \figref{fig:scanf_ex3_hiew_1}.

\myindex{\olly}
\olly underlines the places in memory to which relocs are to be applied, for example: \figref{fig:switch_lot_olly3}.

\subsubsection{Exports and imports}

\label{PE_exports_imports}
As we all know, any executable program must use the \ac{OS}'s services and other DLL-libraries somehow.

It can be said that functions from one module (usually DLL) must be connected somehow to the points of their
calls in other modules (.exe-file or another DLL).

For this, each DLL has an \q{exports} table, which consists of functions plus their addresses in a module.

And every .exe file or DLL has \q{imports}, a table of functions it needs for execution including
list of DLL filenames.

After loading the main .exe-file, the \ac{OS} loader processes imports table:
it loads the additional DLL-files, finds function names
among the DLL exports and writes their addresses down in the \ac{IAT} of the main .exe-module.

\myindex{Windows!Win32!Ordinal}

As we can see, during loading the loader must compare a lot of function names, but string comparison is not a very
fast procedure, so there is a support for \q{ordinals} or \q{hints},
which are function numbers stored in the table, instead of their names.

That is how they can be located faster when loading a DLL.
Ordinals are always present in the \q{export} table.

\myindex{MFC}
For example, a program using the \ac{MFC} library usually loads mfc*.dll by ordinals,
and in such programs there are no \ac{MFC} function names in \ac{INT}.

% TODO example!
When loading such programs in \IDA, it will ask for a path to the mfc*.dll files
in order to determine the function names.

If you don't tell \IDA the path to these DLLs, there will be \emph{mfc80\_123} instead of function names.

\myparagraph{Imports section}

Often a separate section is allocated for the imports table and everything related to it (with name like \TT{.idata}),
however, this is not a strict rule.

Imports are also a confusing subject because of the terminological mess. Let's try to collect all information in one place.

\begin{figure}[H]
\centering
\myincludegraphics{OS/PE/unnamed0.png}
\caption{
A scheme that unites all PE-file structures related to imports}
\end{figure}

The main structure is the array \emph{IMAGE\_IMPORT\_DESCRIPTOR}.
Each element for each DLL being imported.

Each element holds the \ac{RVA} address of the text string (DLL name) (\emph{Name}).

\emph{OriginalFirstThunk} is the \ac{RVA} address of the \ac{INT} table.
This is an array of \ac{RVA} addresses, each of which points to a text string with a function name.
Each string is prefixed by a 16-bit integer
(\q{hint})---\q{ordinal} of function.

While loading, if it is possible to find a function by ordinal,
then the strings comparison will not occur. The array is terminated by zero.

There is also a pointer to the \ac{IAT} table named \emph{FirstThunk}, it is just the \ac{RVA} address
of the place where the loader writes the addresses of the resolved functions.

The points where the loader writes addresses are marked by \IDA like this: \emph{\_\_imp\_CreateFileA}, etc.

There are at least two ways to use the addresses written by the loader.

\myindex{x86!\Instructions!CALL}
\begin{itemize}
\item The code will have instructions like \emph{call \_\_imp\_CreateFileA},
and since the field with the address of the imported function is a global variable in some sense,
the address of the \emph{call} instruction (plus 1 or 2) is to be added to the relocs table,
for the case when the module is loaded at a different base address.

But, obviously, this may enlarge relocs table significantly.

Because there are might be a lot of calls to imported functions in the module.

Furthermore, large relocs table slows down the process of loading modules.

\myindex{x86!\Instructions!JMP}
\myindex{thunk-functions}
\item For each imported function, there is only one jump allocated, using the \JMP instruction
plus a reloc to it.
Such points are also called \q{thunks}.

All calls to the imported functions are just \CALL instructions to the corresponding \q{thunk}.
In this case, additional relocs are not necessary because these \CALL{}-s
have relative addresses and do not need to be corrected.
\end{itemize}

These two methods can be combined.

Possible, the linker creates individual \q{thunk}s if there are too many calls to the function,
but not done by default. \\
\\
By the way, the array of function addresses to which FirstThunk is pointing is not necessary to be located in the \ac{IAT} section.
For example, the author of these lines once wrote the PE\_add\_import\footnote{\href{http://go.yurichev.com/17049}{yurichev.com}}
utility for adding imports to an existing .exe-file.

Some time earlier, in the previous versions of the utility,
at the place of the function you want to substitute with a call to another DLL,
my utility wrote the following code:

\begin{lstlisting}[style=customasmx86]
MOV EAX, [yourdll.dll!function]
JMP EAX
\end{lstlisting}

FirstThunk points to the first instruction. In other words, when loading yourdll.dll,
the loader writes the address of the \emph{function} function right in the code.

It also worth noting that a code section is usually write-protected, so my utility adds the \\
\emph{IMAGE\_SCN\_MEM\_WRITE}
flag for code section. Otherwise, the program to crash while loading with error code
5 (access denied). \\
\\
One might ask: what if I supply a program with a set of DLL files which is not supposed to change (including addresses of all DLL functions),
is it possible to speed up the loading process?

Yes, it is possible to write the addresses of the functions to be imported into the FirstThunk arrays in advance.
The \emph{Timestamp} field is present in the \\
\emph{IMAGE\_IMPORT\_DESCRIPTOR} structure.

If a value is present there, then the loader compares this value with the date-time of the DLL file.

If the values are equal, then the loader does not do anything, and the loading of the process can be faster.
This is called \q{old-style binding}
\footnote{\href{http://go.yurichev.com/17050}{MSDN}. There is also the \q{new-style binding}.}.
\myindex{BIND.EXE}

The BIND.EXE utility in Windows SDK is for this.
For speeding up the loading of your program, Matt Pietrek in \PietrekPEURL, suggests to do the binding shortly after your program
installation on the computer of the end user. \\
\\
PE-files packers/encryptors may also compress/encrypt imports table.

In this case, the Windows loader, of course, will not load all necessary DLLs.
\myindex{Windows!Win32!LoadLibrary}
\myindex{Windows!Win32!GetProcAddress}

Therefore, the packer/encryptor does this on its own, with the help of
\emph{LoadLibrary()} and the \emph{GetProcAddress()} functions.

That is why these two functions are often present in \ac{IAT} in packed files.\\
\\
In the standard DLLs from the Windows installation, \ac{IAT} often is located right at the beginning of the PE file.
Supposedly, it is made so for optimization.

While loading, the .exe file is not loaded into memory as a whole (recall huge install programs which are
started suspiciously fast), it is \q{mapped}, and loaded into memory in parts as they are accessed.

Probably, Microsoft developers decided it will be faster.

\subsubsection{Resources}

\label{PEresources}

Resources in a PE file are just a set of icons, pictures, text strings, dialog descriptions.

Perhaps they were separated from the main code, so all these things could be multilingual,
and it would be simpler to pick text or picture for the language that is currently set in the \ac{OS}. \\
\\
As a side effect, they can be edited easily and saved back to the executable file, even if one does not have special knowledge,
by using the ResHack editor, for example (\myref{ResHack}).

\subsubsection{.NET}

\myindex{.NET}

.NET programs are not compiled into machine code but into a special bytecode.
\myindex{OEP}
Strictly speaking, there is bytecode instead of the usual x86 code
in the .exe file, however, the entry point (\ac{OEP}) points to this tiny fragment of x86 code:

\begin{lstlisting}[style=customasmx86]
jmp         mscoree.dll!_CorExeMain
\end{lstlisting}

The .NET loader is located in mscoree.dll, which processes the PE file.
\myindex{Windows!Windows XP}

It was so in all pre-Windows XP \ac{OS}es. Starting from XP, the \ac{OS} loader is able to detect the .NET file
and run it without executing that \JMP instruction
\footnote{\href{http://go.yurichev.com/17051}{MSDN}}.

\myindex{TLS}
\subsubsection{TLS}

This section holds initialized data for the \ac{TLS}(\myref{TLS}) (if needed).
When a new thread start, its \ac{TLS} data is initialized using the data from this section. \\
\\
\myindex{TLS!Callbacks}

Aside from that, the PE file specification also provides initialization of the
\ac{TLS} section, the so-called TLS callbacks.

If they are present, they are to be called before the control is passed to the main entry point (\ac{OEP}).

This is used widely in the PE file packers/encryptors.

\subsubsection{Tools}

\myindex{objdump}
\myindex{Cygwin}
\myindex{Hiew}
\label{ResHack}

\begin{itemize}
\item objdump (present in cygwin) for dumping all PE-file structures.

\item Hiew(\myref{Hiew}) as editor.

\item pefile---Python-library for PE-file processing \footnote{\url{http://go.yurichev.com/17052}}.

\item ResHack \acs{AKA} Resource Hacker---resources editor\footnote{\url{http://go.yurichev.com/17052}}.

\item PE\_add\_import\footnote{\url{http://go.yurichev.com/17049}}---
simple tool for adding symbol(s) to PE executable import table.

\item PE\_patcher\footnote{\href{http://go.yurichev.com/17054}{yurichev.com}}---simple tool for patching PE executables.

\item PE\_search\_str\_refs\footnote{\href{http://go.yurichev.com/17055}{yurichev.com}}---simple tool for searching for a function in PE executables which use some text string.
\end{itemize}

\subsubsection{Further reading}

% FIXME: bibliography per chapter or section
\begin{itemize}
\item Daniel Pistelli---The .NET File Format \footnote{\url{http://go.yurichev.com/17056}}
\end{itemize}

}
\RU{\mysection{Пример вычисления адреса сети}

Как мы знаем, TCP/IP-адрес (IPv4) состоит из четырех чисел в пределах $0 \ldots 255$, т.е. 4 байта.

4 байта легко помещаются в 32-битную переменную, так что адрес хоста в IPv4, сетевая маска или адрес сети
могут быть 32-битными числами.

С точки зрения пользователя, маска сети определяется четырьмя числами в формате вроде \\
255.255.255.0,
но сетевые инженеры (сисадмины) используют более компактную нотацию (\ac{CIDR}),
вроде  \q{/8}, \q{/16}, итд.

Эта нотация просто определяет количество бит в сетевой маске, начиная с \ac{MSB}.

\small
\begin{center}
\begin{tabular}{ | l | l | l | l | l | l | }
\hline
\HeaderColor Маска & 
\HeaderColor Хосты & 
\HeaderColor Свободно &
\HeaderColor Сетевая маска &
\HeaderColor В шест.виде &
\HeaderColor \\
\hline
/30  & 4        & 2        & 255.255.255.252  & 0xfffffffc  & \\
\hline
/29  & 8        & 6        & 255.255.255.248  & 0xfffffff8  & \\
\hline
/28  & 16       & 14       & 255.255.255.240  & 0xfffffff0  & \\
\hline
/27  & 32       & 30       & 255.255.255.224  & 0xffffffe0  & \\
\hline
/26  & 64       & 62       & 255.255.255.192  & 0xffffffc0  & \\
\hline
/24  & 256      & 254      & 255.255.255.0    & 0xffffff00  & сеть класса C \\
\hline
/23  & 512      & 510      & 255.255.254.0    & 0xfffffe00  & \\
\hline
/22  & 1024     & 1022     & 255.255.252.0    & 0xfffffc00  & \\
\hline
/21  & 2048     & 2046     & 255.255.248.0    & 0xfffff800  & \\
\hline
/20  & 4096     & 4094     & 255.255.240.0    & 0xfffff000  & \\
\hline
/19  & 8192     & 8190     & 255.255.224.0    & 0xffffe000  & \\
\hline
/18  & 16384    & 16382    & 255.255.192.0    & 0xffffc000  & \\
\hline
/17  & 32768    & 32766    & 255.255.128.0    & 0xffff8000  & \\
\hline
/16  & 65536    & 65534    & 255.255.0.0      & 0xffff0000  & сеть класса B \\
\hline
/8   & 16777216 & 16777214 & 255.0.0.0        & 0xff000000  & сеть класса A \\
\hline
\end{tabular}
\end{center}
\normalsize

Вот простой пример, вычисляющий адрес сети используя сетевую маску и адрес хоста.

\lstinputlisting[style=customc]{\CURPATH/netmask.c}

\subsection{calc\_network\_address()}

Функция \TT{calc\_network\_address()} самая простая: 

она просто умножает (логически, используя \AND) адрес хоста на сетевую маску, в итоге давая адрес
сети.

\lstinputlisting[caption=\Optimizing MSVC 2012 /Ob0,numbers=left,style=customasmx86]{\CURPATH/calc_network_address_MSVC_2012_Ox.asm}

На строке 22 мы видим самую важную инструкцию \AND --- так вычисляется адрес сети.

\subsection{form\_IP()}

Функция \TT{form\_IP()} просто собирает все 4 байта в одно 32-битное значение.

Вот как это обычно происходит:

\begin{itemize}
\item Выделите переменную для возвращаемого значения. Обнулите её.

\item 
Возьмите четвертый (самый младший) байт, сложите его (логически, инструкцией \OR) с возвращаемым
значением. Оно содержит теперь 4-й байт.

\item Возьмите третий байт, сдвиньте его на 8 бит влево.
Получится значение в виде \TT{0x0000bb00}, где \TT{bb} это третий байт.
Сложите итоговое значение (логически, инструкцией \OR) с возвращаемым значением.
Возвращаемое значение пока что содержит \TT{0x000000aa}, так что логическое сложение
в итоге выдаст значение вида \TT{0x0000bbaa}.

\item 
Возьмите второй байт, сдвиньте его на 16 бит влево.
Вы получите значение вида \TT{0x00cc0000}, где \TT{cc} это второй байт.
Сложите (логически) результат и возвращаемое значение.
Выходное значение содержит пока что \TT{0x0000bbaa}, так что логическое сложение
в итоге выдаст значение вида \TT{0x00ccbbaa}.

\item 
Возьмите первый байт, сдвиньте его на 24 бита влево.
Вы получите значение вида \TT{0xdd000000}, где \TT{dd} это первый байт.
Сложите (логически) результат и выходное значение.
Выходное значение содержит пока что \TT{0x00ccbbaa}, так что сложение выдаст в итоге значение
вида \TT{0xddccbbaa}.

\end{itemize}

И вот как работает неоптимизирующий MSVC 2012:

\lstinputlisting[caption=\NonOptimizing MSVC 2012,style=customasmx86]{\CURPATH/form_IP_MSVC_2012_RU.asm}

Хотя, порядок операций другой, но, конечно, порядок роли не играет.

\Optimizing MSVC 2012 делает то же самое, но немного иначе:

\lstinputlisting[caption=\Optimizing MSVC 2012 /Ob0,style=customasmx86]{\CURPATH/form_IP_MSVC_2012_Ox_RU.asm}

Можно сказать, что каждый байт записывается в младшие 8 бит возвращаемого значения,
и затем возвращаемое значение сдвигается на один байт влево на каждом шаге.

Повторять 4 раза, для каждого байта.

\par
Вот и всё! 
К сожалению, наверное, нет способа делать это иначе.
Не существует более-менее популярных \ac{CPU} или \ac{ISA}, где имеется инструкция для сборки значения из бит или байт.
Обычно всё это делает сдвигами бит и логическим сложением (OR).

\subsection{print\_as\_IP()}

\TT{print\_as\_IP()} делает наоборот: расщепляет 32-битное значение на 4 байта.

Расщепление работает немного проще: просто сдвигайте входное значение на 24, 16, 8 или 0 бит,
берите биты с нулевого по седьмой (младший байт), вот и всё:

\lstinputlisting[caption=\NonOptimizing MSVC 2012,style=customasmx86]{\CURPATH/print_as_IP_MSVC_2012_RU.asm}

\Optimizing MSVC 2012 делает почти всё то же самое, только без ненужных перезагрузок входного значения:

\lstinputlisting[caption=\Optimizing MSVC 2012 /Ob0,style=customasmx86]{\CURPATH/print_as_IP_MSVC_2012_Ox.asm}

\subsection{form\_netmask() и set\_bit()}

\TT{form\_netmask()} делает сетевую маску из \ac{CIDR}-нотации.

Конечно, было бы куда эффективнее использовать здесь какую-то уже готовую таблицу, но мы рассматриваем
это именно так, сознательно, для демонстрации битовых сдвигов.
Мы также сделаем отдельную функцию \TT{set\_bit()}. 

Не очень хорошая идея выделять отдельную функцию для такой примитивной операции, но так будет проще понять,
как это всё работает.

\lstinputlisting[caption=\Optimizing MSVC 2012 /Ob0,style=customasmx86]{\CURPATH/form_netmask_MSVC_2012_Ox.asm}

\TT{set\_bit()} примитивна: просто сдвигает единицу на нужное количество бит, затем складывает (логически) с
входным значением \q{input}.
\TT{form\_netmask()} имеет цикл: он выставит столько бит (начиная с \ac{MSB}), 
сколько передано в аргументе \TT{netmask\_bits}.

\subsection{Итог}

Вот и всё!
Мы запускаем и видим:

\begin{lstlisting}
netmask=255.255.255.0
network address=10.1.2.0
netmask=255.0.0.0
network address=10.0.0.0
netmask=255.255.255.128
network address=10.1.2.0
netmask=255.255.255.192
network address=10.1.2.64
\end{lstlisting}
}
\DE{\mysection{\Stack}
\label{sec:stack}
\myindex{\Stack}

Der Stack ist eine der fundamentalen Datenstrukturen in der Informatik.
\footnote{\href{http://go.yurichev.com/17119}{wikipedia.org/wiki/Call\_Stack}}.
\ac{AKA} \ac{LIFO}.

Technisch betrachtet ist es ein Stapelspeicher innerhalb des Prozessspeichers der zusammen mit den \ESP (x86), \RSP (x64) oder dem \ac{SP} (ARM) Register als ein Zeiger in diesem Speicherblock fungiert.

\myindex{ARM!\Instructions!PUSH}
\myindex{ARM!\Instructions!POP}
\myindex{x86!\Instructions!PUSH}
\myindex{x86!\Instructions!POP}

Die häufigsten Stack-Zugriffsinstruktionen sind die \PUSH- und \POP-Instruktionen (in beidem x86 und ARM Thumb-Modus). \PUSH subtrahiert vom \ESP/\RSP/\ac{SP} 4 Byte im 32-Bit Modus (oder 8 im 64-Bit Modus) und schreibt dann den Inhalt des Zeigers an die Adresse auf die von \ESP/\RSP/\ac{SP} gezeigt wird.

\POP ist die umgekehrte Operation: Die Daten des Zeigers für die Speicherregion auf die von \ac{SP}
gezeigt wird werden ausgelesen und die Inhalte in den Instruktionsoperanden geschreiben (oft ist das ein Register). Dann werden 4 (beziehungsweise 8) Byte zum \gls{stack pointer} addiert.

Nach der Stackallokation, zeigt der \gls{stack pointer} auf den Boden des Stacks.
\PUSH verringert den \gls{stack pointer} und \POP erhöht ihn.
Der Boden des Stacks ist eigentlich der Anfang der Speicherregion die für den Stack reserviert wurde.
Das wirkt zunächst seltsam, aber so funktioniert es.

ARM unterstützt beides, aufsteigende und absteigende Stacks.

\myindex{ARM!\Instructions!STMFD}
\myindex{ARM!\Instructions!LDMFD}
\myindex{ARM!\Instructions!STMED}
\myindex{ARM!\Instructions!LDMED}
\myindex{ARM!\Instructions!STMFA}
\myindex{ARM!\Instructions!LDMFA}
\myindex{ARM!\Instructions!STMEA}
\myindex{ARM!\Instructions!LDMEA}

Zum Beispiel die \ac{STMFD}/\ac{LDMFD} und \ac{STMED}/\ac{LDMED} Instruktionen sind alle dafür gedacht mit einem absteigendem Stack zu arbeiten ( wächst nach unten, fängt mit hohen Adressen an und entwickelt sich zu niedrigeren Adressen). Die \ac{STMFA}/\ac{LDMFA} und \ac{STMEA}/\ac{LDMEA} Instruktionen sind dazu gedacht mit einem aufsteigendem Stack zu arbeiten (wächst nach oben und fängt mit niedrigeren Adressen an und wächst nach oben).

% It might be worth mentioning that STMED and STMEA write first,
% and then move the pointer, and that LDMED and LDMEA move the pointer first, and then read.
% In other words, ARM not only lets the stack grow in a non-standard direction,
% but also in a non-standard order.
% Maybe this can be in the glossary, which would explain why E stands for "empty".

\subsection{Warum wächst der Stack nach unten?}
\label{stack_grow_backwards}

Intuitiv, würden man annehmen das der Stack nach oben wächst z.B Richtung höherer Adressen, so wie bei jeder anderen Datenstruktur.

Der Grund das der Stack rückwärts wächst ist wohl historisch bedingt. Als Computer so groß waren das sie einen ganzen Raum beansprucht haben war es einfach Speicher in zwei Sektionen zu unterteilen, einen Teil für den \gls{heap} und einen Teil für den Stack. Sicher war zu dieser Zeit nicht bekannt wie groß der \gls{heap} und der Stack wachsen würden, während der Programm Laufzeit, also war die Lösung die einfachste mögliche.

\input{patterns/02_stack/stack_and_heap}

In \RitchieThompsonUNIX können wir folgendes lesen:

\begin{framed}
\begin{quotation}
Der user-core eines Programm Images wird in drei logische Segmente unterteilt. Das Programm-Text Segment beginnt bei 0 im virtuellen Adress Speicher. Während der Ausführung wird das Segment als schreibgeschützt markiert und eine einzelne Kopie des Segments wird unter allen Prozessen geteilt die das Programm ausführen. An der ersten 8K grenze über dem Programm Text Segment im Virtuellen Speicher, fängt der ``nonshared'' Bereich an, der nach Bedarf von Syscalls erweitert werden kann. Beginnend bei der höchsten Adresse im Virtuellen Speicher ist das Stack Segment, das Automatisch nach unten wächst während der Hardware Stackpointer sich ändert.
\end{quotation}
\end{framed}

Das erinnert daran wie manche Schüler Notizen zu  zwei Vorträgen in einem Notebook dokumentieren:
Notizen für den ersten Vortrag werden normal notiert, und Notizen zur zum zweiten Vortrag werden 
ans Ende des Notizbuches geschrieben, indem man das Notizbuch umdreht. Die Notizen treffen sich irgendwann
im Notizbuch aufgrund des fehlenden Freien Platzes.

% I think if we want to expand on this analogy,
% one might remember that the line number increases as as you go down a page.
% So when you decrease the address when pushing to the stack, visually,
% the stack does grow upwards.
% Of course, the problem is that in most human languages,
% just as with computers,
% we write downwards, so this direction is what makes buffer overflows so messy.

\subsection{Für was wird der Stack benutzt?}

% subsections
\EN{\input{patterns/02_stack/01_saving_ret_addr_EN}}
\RU{\input{patterns/02_stack/01_saving_ret_addr_RU}}
\DE{\input{patterns/02_stack/01_saving_ret_addr_DE}}
\FR{\input{patterns/02_stack/01_saving_ret_addr_FR}}
\PTBR{\input{patterns/02_stack/01_saving_ret_addr_PTBR}}
\IT{\input{patterns/02_stack/01_saving_ret_addr_IT}}
\PL{\input{patterns/02_stack/01_saving_ret_addr_PL}}
\JA{\input{patterns/02_stack/01_saving_ret_addr_JA}}

\EN{\input{patterns/02_stack/02_args_passing_EN}}
\RU{\input{patterns/02_stack/02_args_passing_RU}}
\PTBR{\input{patterns/02_stack/02_args_passing_PTBR}}
\DE{\input{patterns/02_stack/02_args_passing_DE}}
\IT{\input{patterns/02_stack/02_args_passing_IT}}
\FR{\input{patterns/02_stack/02_args_passing_FR}}
\JA{\input{patterns/02_stack/02_args_passing_JA}}
\PL{\input{patterns/02_stack/02_args_passing_PL}}


\EN{\input{patterns/02_stack/03_local_vars_EN}}
\RU{\input{patterns/02_stack/03_local_vars_RU}}
\DE{\input{patterns/02_stack/03_local_vars_DE}}
\PTBR{\input{patterns/02_stack/03_local_vars_PTBR}}
\EN{\input{patterns/02_stack/04_alloca/main_EN}}
\FR{\input{patterns/02_stack/04_alloca/main_FR}}
\RU{\input{patterns/02_stack/04_alloca/main_RU}}
\PTBR{\input{patterns/02_stack/04_alloca/main_PTBR}}
\IT{\input{patterns/02_stack/04_alloca/main_IT}}
\DE{\input{patterns/02_stack/04_alloca/main_DE}}
\PL{\input{patterns/02_stack/04_alloca/main_PL}}
\JA{\input{patterns/02_stack/04_alloca/main_JA}}

\subsubsection{(Windows) SEH}
\myindex{Windows!Structured Exception Handling}

\ifdefined\RUSSIAN
В стеке хранятся записи \ac{SEH} для функции (если они присутствуют).
Читайте больше о нем здесь: (\myref{sec:SEH}).
\fi % RUSSIAN

\ifdefined\ENGLISH
\ac{SEH} records are also stored on the stack (if they are present).
Read more about it: (\myref{sec:SEH}).
\fi % ENGLISH

\ifdefined\BRAZILIAN
\ac{SEH} também são guardados na pilha (se estiverem presentes).
\PTBRph{}: (\myref{sec:SEH}).
\fi % BRAZILIAN

\ifdefined\ITALIAN
I record \ac{SEH}, se presenti, sono anch'essi memorizzati nello stack.
Maggiori informazioni qui: (\myref{sec:SEH}).
\fi % ITALIAN

\ifdefined\FRENCH
Les enregistrements \ac{SEH} sont aussi stockés dans la pile (s'ils sont présents).
Lire à ce propos: (\myref{sec:SEH}).
\fi % FRENCH


\ifdefined\POLISH
Na stosie są przechowywane wpisy \ac{SEH} dla funkcji (jeśli są one obecne).
Więcej o tym tutaj: (\myref{sec:SEH}).
\fi % POLISH

\ifdefined\JAPANESE
\ac{SEH}レコードはスタックにも格納されます(存在する場合)。
それについてもっと読む:(\myref{sec:SEH})
\fi % JAPANESE

\ifdefined\ENGLISH
\subsubsection{Buffer overflow protection}

More about it here~(\myref{subsec:bufferoverflow}).
\fi

\ifdefined\RUSSIAN
\subsubsection{Защита от переполнений буфера}

Здесь больше об этом~(\myref{subsec:bufferoverflow}).
\fi

\ifdefined\BRAZILIAN
\subsubsection{Proteção contra estouro de buffer}

Mais sobre aqui~(\myref{subsec:bufferoverflow}).
\fi

\ifdefined\ITALIAN
\subsubsection{Protezione da buffer overflow}

Maggiori informazioni qui~(\myref{subsec:bufferoverflow}).
\fi

\ifdefined\FRENCH
\subsubsection{Protection contre les débordements de tampon}

Lire à ce propos~(\myref{subsec:bufferoverflow}).
\fi


\ifdefined\POLISH
\subsubsection{Metody zabiezpieczenia przed przepełnieniem stosu}

Więcej o tym tutaj~(\myref{subsec:bufferoverflow}).
\fi

\ifdefined\JAPANESE
\subsubsection{バッファオーバーフロー保護}

詳細はこちら~(\myref{subsec:bufferoverflow})
\fi

\subsubsection{Automatisches deallokieren der Daten auf dem Stack}

Vielleicht ist der Grund warum man lokale Variablen und SEH Einträge auf dem Stack speichert, weil sie beim 
verlassen der Funktion automatisch aufgeräumt werden. Man braucht dabei nur eine Instruktion um die Position
des Stackpointers zu korrigieren (oftmals ist es die \ADD Instruktion). Funktions Argumente, könnte man sagen 
werden auch am Ende der Funktion deallokiert. Im Kontrast dazu, alles was auf dem \emph{heap} gespeichert wird muss
explizit deallokiert werden. 

% sections
\EN{\input{patterns/02_stack/07_layout_EN}}
\RU{\input{patterns/02_stack/07_layout_RU}}
\DE{\input{patterns/02_stack/07_layout_DE}}
\PTBR{\input{patterns/02_stack/07_layout_PTBR}}
\EN{\input{patterns/02_stack/08_noise/main_EN}}
\FR{\input{patterns/02_stack/08_noise/main_FR}}
\RU{\input{patterns/02_stack/08_noise/main_RU}}
\IT{\input{patterns/02_stack/08_noise/main_IT}}
\DE{\input{patterns/02_stack/08_noise/main_DE}}
\PL{\input{patterns/02_stack/08_noise/main_PL}}
\JA{\input{patterns/02_stack/08_noise/main_JA}}

\input{patterns/02_stack/exercises}
}
\FR{\subsection{Multiplication, division}

\lstinputlisting[style=customc]{patterns/185_64bit_in_32_env/multdiv/2.c}

\subsubsection{x86}

\lstinputlisting[caption=MSVC 2013 /Ob1 \Optimizing,style=customasmx86]{patterns/185_64bit_in_32_env/multdiv/2_MSVC_FR.asm}

La multiplication et la division sont des opérations plus complexes, donc en général
le compilateur embarque des appels à des fonctions de bibliothèque les effectuant.

Ces fonctions sont décrites ici: \myref{sec:MSVC_library_func}.

\lstinputlisting[caption=GCC 4.8.1 -fno-inline \Optimizing,style=customasmx86]{patterns/185_64bit_in_32_env/multdiv/2_GCC_FR.asm}

GCC fait ce que l'on attend, mais le code multiplication est mis en ligne (inlined)
directement dans la fonction, pensant que ça peut être plus efficace.
GCC a des noms de fonctions de bibliothèque différents: \myref{sec:GCC_library_func}.

\subsubsection{ARM}

Keil pour mode Thumb insère des appels à des sous-routines de bibliothèque:

\lstinputlisting[caption=\OptimizingKeilVI (\ThumbMode),style=customasmARM]{patterns/185_64bit_in_32_env/multdiv/Keil_thumb_O3.s}

Keil pour mode ARM, d'un autre côté, est capable de produire le code de la multiplication
64-bit:

\lstinputlisting[caption=\OptimizingKeilVI (\ARMMode),style=customasmARM]{patterns/185_64bit_in_32_env/multdiv/Keil_ARM_O3.s}
% TODO add explanation

\subsubsection{MIPS}

GCC \Optimizing pour MIPS peut générer du code pour la multiplication 64-bit, mais
doit appeler une routine de bibliothèque pour la division 64-bit:

\lstinputlisting[caption=GCC 4.4.5 \Optimizing (IDA),style=customasmMIPS]{patterns/185_64bit_in_32_env/multdiv/MIPS_O3_IDA.lst}

Il y a beaucoup de \ac{NOP}s, sans doute des slots de délai de remplissage après
l'instruction de multiplication (c'est plus lent que les autres instructions après
tout).

% TODO add explanation
}
\JA{\subsection{乗算、除算}

\lstinputlisting[style=customc]{patterns/185_64bit_in_32_env/multdiv/2.c}

\subsubsection{x86}

\lstinputlisting[caption=\Optimizing MSVC 2013 /Ob1,style=customasmx86]{patterns/185_64bit_in_32_env/multdiv/2_MSVC_JA.asm}

乗算と除算はより複雑な演算なので、通常、コンパイラはそれを行うライブラリ関数への
呼び出しを埋め込みます。

これらの機能はここに記述されています:\myref{sec:MSVC_library_func}

\lstinputlisting[caption=\Optimizing GCC 4.8.1 -fno-inline,style=customasmx86]{patterns/185_64bit_in_32_env/multdiv/2_GCC_JA.asm}

GCCは期待どおりに機能しますが、乗算コードは
関数内でインライン化されているため、より効率的になる可能性があります。
GCCには異なる関数名のライブラリあります:\myref{sec:GCC_library_func}

\subsubsection{ARM}

ThumbモードのKeilはライブラリサブルーチン呼び出しを挿入します。

\lstinputlisting[caption=\OptimizingKeilVI (\ThumbMode),style=customasmARM]{patterns/185_64bit_in_32_env/multdiv/Keil_thumb_O3.s}

一方、ARMモードのKeilでは64ビットの乗算コードを生成できます。

\lstinputlisting[caption=\OptimizingKeilVI (\ARMMode),style=customasmARM]{patterns/185_64bit_in_32_env/multdiv/Keil_ARM_O3.s}
% TODO add explanation

\subsubsection{MIPS}

MIPS用に \Optimizing GCC 64ビット乗算コードを生成できますが、64ビット除算用のライブラリルーチンを呼び出す必要があります。

\lstinputlisting[caption=\Optimizing GCC 4.4.5 (IDA),style=customasmMIPS]{patterns/185_64bit_in_32_env/multdiv/MIPS_O3_IDA.lst}

たくさんの\ac{NOP}があります。おそらく乗算命令の後に埋められた遅延スロットです(結局のところ、
それは他の命令より遅いです)。

% TODO add explanation
}

\EN{\subsection{Win32 PE}
\label{win32_pe}
\myindex{Windows!Win32}

\acs{PE} is an executable file format used in Windows.
The difference between .exe, .dll and .sys is that .exe and .sys usually do not have exports, only imports.

\myindex{OEP}

A \ac{DLL}, just like any other PE-file, has an entry point (\ac{OEP}) (the function DllMain() is located there)
but this function usually does nothing.
.sys is usually a device driver.
As of drivers, Windows requires the checksum to be present in the PE file and for it to be correct
\footnote{For example, Hiew(\myref{Hiew}) can calculate it}.

\myindex{Windows!Windows Vista}
Starting at Windows Vista, a driver's files must also be signed with a digital signature. It will fail to load otherwise.

\myindex{MS-DOS}
Every PE file begins with tiny DOS program that prints a
message like \q{This program cannot be run in DOS mode.}---if you run this program in DOS or Windows 3.1 (\ac{OS}-es which are not aware of the PE format),
this message will be printed.

\subsubsection{Terminology}

\myindex{VA}
\myindex{Base address}
\myindex{RVA}
\myindex{Windows!IAT}
\myindex{Windows!INT}

\begin{itemize}
\item Module---a separate file, .exe or .dll.

\item Process---a program loaded into memory and currently running.  Commonly consists of one .exe file and bunch of .dll files.

\item Process memory---the memory a process works with.  Each process has its own.
There usually are loaded modules, memory of the stack, \gls{heap}(s), etc.

\item \ac{VA}---an address which is to be used in program while runtime.

\item Base address (of module)---the address within the process memory at which the module is to be loaded.
\ac{OS} loader may change it, if the base address is already occupied by another module just loaded before.

\item \ac{RVA}---the \ac{VA}-address minus the base address.

Many addresses in PE-file tables use \ac{RVA}-addresses.

%\item
%Data directory --- ...

\item \ac{IAT}---an array of addresses of imported symbols \footnote{\PietrekPE}.
Sometimes, the \TT{IMAGE\_DIRECTORY\_ENTRY\_IAT} data directory points at the \ac{IAT}.
\label{IDA_idata}
It is worth noting that \ac{IDA} (as of 6.1) may allocate a pseudo-section named \TT{.idata} for
\ac{IAT}, even if the \ac{IAT} is a part of another section!

\item \ac{INT}---an array of names of symbols to be imported\footnote{\PietrekPE}.
\end{itemize}

\subsubsection{Base address}

The problem is that several module authors can prepare DLL files for others to use and it is not possible
to reach an agreement which addresses is to be assigned to whose modules.

So that is why if two necessary DLLs for a process have the same base address,
one of them will be loaded at this base address, and the other---at some other free space in process memory,
and each virtual addresses in the second DLL will be corrected.

\par With \ac{MSVC} the linker often generates the .exe files with a base address of \TT{0x400000}
\footnote{The origin of this address choice is described here: \href{http://go.yurichev.com/17041}{MSDN}},
and with the code section starting at \TT{0x401000}.
This means that the \ac{RVA} of the start of the code section is \TT{0x1000}.

DLLs are often generated by MSVC's linker with a base address of \TT{0x10000000}
\footnote{This can be changed by the /BASE linker option}.

\myindex{ASLR}

There is also another reason to load modules at various base addresses, in this case random ones.
It is \ac{ASLR}\footnote{\href{http://go.yurichev.com/17140}{wikipedia}}.

\myindex{Shellcode}

A shellcode trying to get executed on a compromised system must call system functions, hence, know their addresses.

In older \ac{OS} (in \gls{Windows NT} line: before Windows Vista),
system DLL (like kernel32.dll, user32.dll) were always loaded at known addresses,
and if we also recall
that their versions rarely changed, the addresses of functions were
fixed and shellcode could call them directly.

In order to avoid this, the \ac{ASLR}
method loads your program and all modules it needs at random base addresses, different every time.

\ac{ASLR} support is denoted in a PE file by setting the flag
\par \TT{IMAGE\_DLL\_CHARACTERISTICS\_DYNAMIC\_BASE} \InSqBrackets{see \Russinovich}.

\subsubsection{Subsystem}

There is also a \emph{subsystem} field, usually it is:

\myindex{Native API}

\begin{itemize}
\item native\footnote{Meaning, the module use Native API instead of Win32} (.sys-driver),

\item console (console application) or

\item \ac{GUI} (non-console).
\end{itemize}

\subsubsection{OS version}

A PE file also specifies the minimal Windows version it needs in order to be loadable.

The table of version numbers stored in the PE file and corresponding Windows codenames is
here\footnote{\href{http://go.yurichev.com/17044}{wikipedia}}.

\myindex{Windows!Windows NT4}
\myindex{Windows!Windows 2000}
For example, \ac{MSVC} 2005 compiles .exe files for running on Windows NT4 (version 4.00), but \ac{MSVC} 2008 does not
(the generated files have a version of 5.00, at least Windows 2000 is needed to run them).

\myindex{Windows!Windows XP}

\ac{MSVC} 2012 generates .exe files of version 6.00 by default,
targeting at least Windows Vista.
However, by changing the compiler's options\footnote{\href{http://go.yurichev.com/17045}{MSDN}},
it is possible to force it to compile for Windows XP.

\subsubsection{Sections}

Division in sections, as it seems, is present in all executable file formats.

It is devised in order to separate code from data, and data---from constant data.

\begin{itemize}
\item Either the \emph{IMAGE\_SCN\_CNT\_CODE} or \emph{IMAGE\_SCN\_MEM\_EXECUTE} flags will be set on the code section---this is executable code.

\item On data section---\emph{IMAGE\_SCN\_CNT\_INITIALIZED\_DATA},\\
\emph{IMAGE\_SCN\_MEM\_READ} and \emph{IMAGE\_SCN\_MEM\_WRITE} flags.

\item On an empty section with uninitialized data---\\
\emph{IMAGE\_SCN\_CNT\_UNINITIALIZED\_DATA}, \emph{IMAGE\_SCN\_MEM\_READ} \\
        and \emph{IMAGE\_SCN\_MEM\_WRITE}.

\item On a constant data section (one that's protected from writing), the flags \\
\emph{IMAGE\_SCN\_CNT\_INITIALIZED\_DATA} and \emph{IMAGE\_SCN\_MEM\_READ} can be set, \\
but not \emph{IMAGE\_SCN\_MEM\_WRITE}.
A process going to crash if it tries to write to this section.
\end{itemize}

\myindex{TLS}
\myindex{BSS}
Each section in PE-file may have a name, however, it is not very important.
Often (but not always) the code section is named \TT{.text},
the data section---\TT{.data}, the constant data section --- \TT{.rdata} \emph{(readable data)}.
Other popular section names are:

\myindex{MIPS}
\begin{itemize}
\item \TT{.idata}---imports section.
\ac{IDA} may create a pseudo-section named like this: \myref{IDA_idata}.
\item \TT{.edata}---exports section (rare)
\item \TT{.pdata}---section holding all information about exceptions in Windows NT for MIPS, \ac{IA64} and x64: \myref{SEH_win64}
\item \TT{.reloc}---relocs section
\item \TT{.bss}---uninitialized data (\ac{BSS})
\item \TT{.tls}---thread local storage (\ac{TLS})
\item \TT{.rsrc}---resources
\item \TT{.CRT}---may present in binary files compiled by ancient MSVC versions
\end{itemize}

PE file packers/encryptors often garble section names or replace the names with their own.

\ac{MSVC} allows you to declare data in arbitrarily named section
\footnote{\href{http://go.yurichev.com/17047}{MSDN}}.

Some compilers and linkers can add a section with debugging symbols and
other debugging information (MinGW for instance).
\myindex{Windows!PDB}
However it is not so in latest versions of \ac{MSVC} (separate \gls{PDB} files are used there for this purpose).\\
\\
That is how a PE section is described in the file:

\begin{lstlisting}[style=customc]
typedef struct _IMAGE_SECTION_HEADER {
  BYTE  Name[IMAGE_SIZEOF_SHORT_NAME];
  union {
    DWORD PhysicalAddress;
    DWORD VirtualSize;
  } Misc;
  DWORD VirtualAddress;
  DWORD SizeOfRawData;
  DWORD PointerToRawData;
  DWORD PointerToRelocations;
  DWORD PointerToLinenumbers;
  WORD  NumberOfRelocations;
  WORD  NumberOfLinenumbers;
  DWORD Characteristics;
} IMAGE_SECTION_HEADER, *PIMAGE_SECTION_HEADER;
\end{lstlisting}
\footnote{\href{http://go.yurichev.com/17048}{MSDN}}

\myindex{Hiew}
A word about terminology: \emph{PointerToRawData} is called \q{Offset} in Hiew
and \emph{VirtualAddress} is called \q{RVA} there.

\subsubsection{Data section}

Data section in file can be smaller than in memory.
For example, some variables can be initialized, some are not.
Compiler and linker will collect them all into one section, but the first part of it is initialized and allocated in file,
while another is absent in file (of course, to make it smaller).
\emph{VirtualSize} will be equal to the size of section in memory, and \emph{SizeOfRawData} --- to
size of section in file.

IDA can show the border between initialized and not initialized parts like that:

\begin{lstlisting}[style=customasmx86]
...

.data:10017FFA                 db    0
.data:10017FFB                 db    0
.data:10017FFC                 db    0
.data:10017FFD                 db    0
.data:10017FFE                 db    0
.data:10017FFF                 db    0
.data:10018000                 db    ? ;
.data:10018001                 db    ? ;
.data:10018002                 db    ? ;
.data:10018003                 db    ? ;
.data:10018004                 db    ? ;
.data:10018005                 db    ? ;

...
\end{lstlisting}

\subsubsection{Relocations (relocs)}
\label{subsec:relocs}

\ac{AKA} FIXUP-s (at least in Hiew).

They are also present in almost all executable file formats
\footnote{Even in .exe files for MS-DOS}.
Exceptions are shared dynamic libraries compiled with \ac{PIC}, or any other \ac{PIC}-code.

What are they for?

Obviously, modules can be loaded on various base addresses, but how to deal with global variables, for example?
They must be accessed by address.  One solution is \PICcode{} (\myref{sec:PIC}).
But it is not always convenient.

That is why a relocations table is present.
There the addresses of points that must be corrected are enumerated,
in case of loading at a different base address.

% TODO тут бы пример с HIEW или objdump..
For example, there is a global variable at address \TT{0x410000} and this is how it is accessed:

\begin{lstlisting}[style=customasmx86]
A1 00 00 41 00         mov         eax,[000410000]
\end{lstlisting}

The base address of the module is \TT{0x400000}, the \ac{RVA} of the global variable is \TT{0x10000}.

If the module is loaded at base address \TT{0x500000}, the real address of the global variable must be \TT{0x510000}.

\myindex{x86!\Instructions!MOV}

As we can see, the address of variable is encoded in the instruction \TT{MOV}, after the byte \TT{0xA1}.

That is why the address of the 4 bytes after \TT{0xA1}, is written in the relocs table.

If the module is loaded at a different base address, the \ac{OS} loader enumerates all addresses in the table,

finds each 32-bit word the address points to, subtracts the original base address from it
(we get the \ac{RVA} here), and adds the new base address to it.

If a module is loaded at its original base address, nothing happens.

All global variables can be treated like that.

Relocs may have various types, however, in Windows for x86 processors, the type is usually \\
\emph{IMAGE\_REL\_BASED\_HIGHLOW}.

\myindex{Hiew}

By the way, relocs are darkened in Hiew, for example: \figref{fig:scanf_ex3_hiew_1}.

\myindex{\olly}
\olly underlines the places in memory to which relocs are to be applied, for example: \figref{fig:switch_lot_olly3}.

\subsubsection{Exports and imports}

\label{PE_exports_imports}
As we all know, any executable program must use the \ac{OS}'s services and other DLL-libraries somehow.

It can be said that functions from one module (usually DLL) must be connected somehow to the points of their
calls in other modules (.exe-file or another DLL).

For this, each DLL has an \q{exports} table, which consists of functions plus their addresses in a module.

And every .exe file or DLL has \q{imports}, a table of functions it needs for execution including
list of DLL filenames.

After loading the main .exe-file, the \ac{OS} loader processes imports table:
it loads the additional DLL-files, finds function names
among the DLL exports and writes their addresses down in the \ac{IAT} of the main .exe-module.

\myindex{Windows!Win32!Ordinal}

As we can see, during loading the loader must compare a lot of function names, but string comparison is not a very
fast procedure, so there is a support for \q{ordinals} or \q{hints},
which are function numbers stored in the table, instead of their names.

That is how they can be located faster when loading a DLL.
Ordinals are always present in the \q{export} table.

\myindex{MFC}
For example, a program using the \ac{MFC} library usually loads mfc*.dll by ordinals,
and in such programs there are no \ac{MFC} function names in \ac{INT}.

% TODO example!
When loading such programs in \IDA, it will ask for a path to the mfc*.dll files
in order to determine the function names.

If you don't tell \IDA the path to these DLLs, there will be \emph{mfc80\_123} instead of function names.

\myparagraph{Imports section}

Often a separate section is allocated for the imports table and everything related to it (with name like \TT{.idata}),
however, this is not a strict rule.

Imports are also a confusing subject because of the terminological mess. Let's try to collect all information in one place.

\begin{figure}[H]
\centering
\myincludegraphics{OS/PE/unnamed0.png}
\caption{
A scheme that unites all PE-file structures related to imports}
\end{figure}

The main structure is the array \emph{IMAGE\_IMPORT\_DESCRIPTOR}.
Each element for each DLL being imported.

Each element holds the \ac{RVA} address of the text string (DLL name) (\emph{Name}).

\emph{OriginalFirstThunk} is the \ac{RVA} address of the \ac{INT} table.
This is an array of \ac{RVA} addresses, each of which points to a text string with a function name.
Each string is prefixed by a 16-bit integer
(\q{hint})---\q{ordinal} of function.

While loading, if it is possible to find a function by ordinal,
then the strings comparison will not occur. The array is terminated by zero.

There is also a pointer to the \ac{IAT} table named \emph{FirstThunk}, it is just the \ac{RVA} address
of the place where the loader writes the addresses of the resolved functions.

The points where the loader writes addresses are marked by \IDA like this: \emph{\_\_imp\_CreateFileA}, etc.

There are at least two ways to use the addresses written by the loader.

\myindex{x86!\Instructions!CALL}
\begin{itemize}
\item The code will have instructions like \emph{call \_\_imp\_CreateFileA},
and since the field with the address of the imported function is a global variable in some sense,
the address of the \emph{call} instruction (plus 1 or 2) is to be added to the relocs table,
for the case when the module is loaded at a different base address.

But, obviously, this may enlarge relocs table significantly.

Because there are might be a lot of calls to imported functions in the module.

Furthermore, large relocs table slows down the process of loading modules.

\myindex{x86!\Instructions!JMP}
\myindex{thunk-functions}
\item For each imported function, there is only one jump allocated, using the \JMP instruction
plus a reloc to it.
Such points are also called \q{thunks}.

All calls to the imported functions are just \CALL instructions to the corresponding \q{thunk}.
In this case, additional relocs are not necessary because these \CALL{}-s
have relative addresses and do not need to be corrected.
\end{itemize}

These two methods can be combined.

Possible, the linker creates individual \q{thunk}s if there are too many calls to the function,
but not done by default. \\
\\
By the way, the array of function addresses to which FirstThunk is pointing is not necessary to be located in the \ac{IAT} section.
For example, the author of these lines once wrote the PE\_add\_import\footnote{\href{http://go.yurichev.com/17049}{yurichev.com}}
utility for adding imports to an existing .exe-file.

Some time earlier, in the previous versions of the utility,
at the place of the function you want to substitute with a call to another DLL,
my utility wrote the following code:

\begin{lstlisting}[style=customasmx86]
MOV EAX, [yourdll.dll!function]
JMP EAX
\end{lstlisting}

FirstThunk points to the first instruction. In other words, when loading yourdll.dll,
the loader writes the address of the \emph{function} function right in the code.

It also worth noting that a code section is usually write-protected, so my utility adds the \\
\emph{IMAGE\_SCN\_MEM\_WRITE}
flag for code section. Otherwise, the program to crash while loading with error code
5 (access denied). \\
\\
One might ask: what if I supply a program with a set of DLL files which is not supposed to change (including addresses of all DLL functions),
is it possible to speed up the loading process?

Yes, it is possible to write the addresses of the functions to be imported into the FirstThunk arrays in advance.
The \emph{Timestamp} field is present in the \\
\emph{IMAGE\_IMPORT\_DESCRIPTOR} structure.

If a value is present there, then the loader compares this value with the date-time of the DLL file.

If the values are equal, then the loader does not do anything, and the loading of the process can be faster.
This is called \q{old-style binding}
\footnote{\href{http://go.yurichev.com/17050}{MSDN}. There is also the \q{new-style binding}.}.
\myindex{BIND.EXE}

The BIND.EXE utility in Windows SDK is for this.
For speeding up the loading of your program, Matt Pietrek in \PietrekPEURL, suggests to do the binding shortly after your program
installation on the computer of the end user. \\
\\
PE-files packers/encryptors may also compress/encrypt imports table.

In this case, the Windows loader, of course, will not load all necessary DLLs.
\myindex{Windows!Win32!LoadLibrary}
\myindex{Windows!Win32!GetProcAddress}

Therefore, the packer/encryptor does this on its own, with the help of
\emph{LoadLibrary()} and the \emph{GetProcAddress()} functions.

That is why these two functions are often present in \ac{IAT} in packed files.\\
\\
In the standard DLLs from the Windows installation, \ac{IAT} often is located right at the beginning of the PE file.
Supposedly, it is made so for optimization.

While loading, the .exe file is not loaded into memory as a whole (recall huge install programs which are
started suspiciously fast), it is \q{mapped}, and loaded into memory in parts as they are accessed.

Probably, Microsoft developers decided it will be faster.

\subsubsection{Resources}

\label{PEresources}

Resources in a PE file are just a set of icons, pictures, text strings, dialog descriptions.

Perhaps they were separated from the main code, so all these things could be multilingual,
and it would be simpler to pick text or picture for the language that is currently set in the \ac{OS}. \\
\\
As a side effect, they can be edited easily and saved back to the executable file, even if one does not have special knowledge,
by using the ResHack editor, for example (\myref{ResHack}).

\subsubsection{.NET}

\myindex{.NET}

.NET programs are not compiled into machine code but into a special bytecode.
\myindex{OEP}
Strictly speaking, there is bytecode instead of the usual x86 code
in the .exe file, however, the entry point (\ac{OEP}) points to this tiny fragment of x86 code:

\begin{lstlisting}[style=customasmx86]
jmp         mscoree.dll!_CorExeMain
\end{lstlisting}

The .NET loader is located in mscoree.dll, which processes the PE file.
\myindex{Windows!Windows XP}

It was so in all pre-Windows XP \ac{OS}es. Starting from XP, the \ac{OS} loader is able to detect the .NET file
and run it without executing that \JMP instruction
\footnote{\href{http://go.yurichev.com/17051}{MSDN}}.

\myindex{TLS}
\subsubsection{TLS}

This section holds initialized data for the \ac{TLS}(\myref{TLS}) (if needed).
When a new thread start, its \ac{TLS} data is initialized using the data from this section. \\
\\
\myindex{TLS!Callbacks}

Aside from that, the PE file specification also provides initialization of the
\ac{TLS} section, the so-called TLS callbacks.

If they are present, they are to be called before the control is passed to the main entry point (\ac{OEP}).

This is used widely in the PE file packers/encryptors.

\subsubsection{Tools}

\myindex{objdump}
\myindex{Cygwin}
\myindex{Hiew}
\label{ResHack}

\begin{itemize}
\item objdump (present in cygwin) for dumping all PE-file structures.

\item Hiew(\myref{Hiew}) as editor.

\item pefile---Python-library for PE-file processing \footnote{\url{http://go.yurichev.com/17052}}.

\item ResHack \acs{AKA} Resource Hacker---resources editor\footnote{\url{http://go.yurichev.com/17052}}.

\item PE\_add\_import\footnote{\url{http://go.yurichev.com/17049}}---
simple tool for adding symbol(s) to PE executable import table.

\item PE\_patcher\footnote{\href{http://go.yurichev.com/17054}{yurichev.com}}---simple tool for patching PE executables.

\item PE\_search\_str\_refs\footnote{\href{http://go.yurichev.com/17055}{yurichev.com}}---simple tool for searching for a function in PE executables which use some text string.
\end{itemize}

\subsubsection{Further reading}

% FIXME: bibliography per chapter or section
\begin{itemize}
\item Daniel Pistelli---The .NET File Format \footnote{\url{http://go.yurichev.com/17056}}
\end{itemize}

}
\RU{\mysection{Пример вычисления адреса сети}

Как мы знаем, TCP/IP-адрес (IPv4) состоит из четырех чисел в пределах $0 \ldots 255$, т.е. 4 байта.

4 байта легко помещаются в 32-битную переменную, так что адрес хоста в IPv4, сетевая маска или адрес сети
могут быть 32-битными числами.

С точки зрения пользователя, маска сети определяется четырьмя числами в формате вроде \\
255.255.255.0,
но сетевые инженеры (сисадмины) используют более компактную нотацию (\ac{CIDR}),
вроде  \q{/8}, \q{/16}, итд.

Эта нотация просто определяет количество бит в сетевой маске, начиная с \ac{MSB}.

\small
\begin{center}
\begin{tabular}{ | l | l | l | l | l | l | }
\hline
\HeaderColor Маска & 
\HeaderColor Хосты & 
\HeaderColor Свободно &
\HeaderColor Сетевая маска &
\HeaderColor В шест.виде &
\HeaderColor \\
\hline
/30  & 4        & 2        & 255.255.255.252  & 0xfffffffc  & \\
\hline
/29  & 8        & 6        & 255.255.255.248  & 0xfffffff8  & \\
\hline
/28  & 16       & 14       & 255.255.255.240  & 0xfffffff0  & \\
\hline
/27  & 32       & 30       & 255.255.255.224  & 0xffffffe0  & \\
\hline
/26  & 64       & 62       & 255.255.255.192  & 0xffffffc0  & \\
\hline
/24  & 256      & 254      & 255.255.255.0    & 0xffffff00  & сеть класса C \\
\hline
/23  & 512      & 510      & 255.255.254.0    & 0xfffffe00  & \\
\hline
/22  & 1024     & 1022     & 255.255.252.0    & 0xfffffc00  & \\
\hline
/21  & 2048     & 2046     & 255.255.248.0    & 0xfffff800  & \\
\hline
/20  & 4096     & 4094     & 255.255.240.0    & 0xfffff000  & \\
\hline
/19  & 8192     & 8190     & 255.255.224.0    & 0xffffe000  & \\
\hline
/18  & 16384    & 16382    & 255.255.192.0    & 0xffffc000  & \\
\hline
/17  & 32768    & 32766    & 255.255.128.0    & 0xffff8000  & \\
\hline
/16  & 65536    & 65534    & 255.255.0.0      & 0xffff0000  & сеть класса B \\
\hline
/8   & 16777216 & 16777214 & 255.0.0.0        & 0xff000000  & сеть класса A \\
\hline
\end{tabular}
\end{center}
\normalsize

Вот простой пример, вычисляющий адрес сети используя сетевую маску и адрес хоста.

\lstinputlisting[style=customc]{\CURPATH/netmask.c}

\subsection{calc\_network\_address()}

Функция \TT{calc\_network\_address()} самая простая: 

она просто умножает (логически, используя \AND) адрес хоста на сетевую маску, в итоге давая адрес
сети.

\lstinputlisting[caption=\Optimizing MSVC 2012 /Ob0,numbers=left,style=customasmx86]{\CURPATH/calc_network_address_MSVC_2012_Ox.asm}

На строке 22 мы видим самую важную инструкцию \AND --- так вычисляется адрес сети.

\subsection{form\_IP()}

Функция \TT{form\_IP()} просто собирает все 4 байта в одно 32-битное значение.

Вот как это обычно происходит:

\begin{itemize}
\item Выделите переменную для возвращаемого значения. Обнулите её.

\item 
Возьмите четвертый (самый младший) байт, сложите его (логически, инструкцией \OR) с возвращаемым
значением. Оно содержит теперь 4-й байт.

\item Возьмите третий байт, сдвиньте его на 8 бит влево.
Получится значение в виде \TT{0x0000bb00}, где \TT{bb} это третий байт.
Сложите итоговое значение (логически, инструкцией \OR) с возвращаемым значением.
Возвращаемое значение пока что содержит \TT{0x000000aa}, так что логическое сложение
в итоге выдаст значение вида \TT{0x0000bbaa}.

\item 
Возьмите второй байт, сдвиньте его на 16 бит влево.
Вы получите значение вида \TT{0x00cc0000}, где \TT{cc} это второй байт.
Сложите (логически) результат и возвращаемое значение.
Выходное значение содержит пока что \TT{0x0000bbaa}, так что логическое сложение
в итоге выдаст значение вида \TT{0x00ccbbaa}.

\item 
Возьмите первый байт, сдвиньте его на 24 бита влево.
Вы получите значение вида \TT{0xdd000000}, где \TT{dd} это первый байт.
Сложите (логически) результат и выходное значение.
Выходное значение содержит пока что \TT{0x00ccbbaa}, так что сложение выдаст в итоге значение
вида \TT{0xddccbbaa}.

\end{itemize}

И вот как работает неоптимизирующий MSVC 2012:

\lstinputlisting[caption=\NonOptimizing MSVC 2012,style=customasmx86]{\CURPATH/form_IP_MSVC_2012_RU.asm}

Хотя, порядок операций другой, но, конечно, порядок роли не играет.

\Optimizing MSVC 2012 делает то же самое, но немного иначе:

\lstinputlisting[caption=\Optimizing MSVC 2012 /Ob0,style=customasmx86]{\CURPATH/form_IP_MSVC_2012_Ox_RU.asm}

Можно сказать, что каждый байт записывается в младшие 8 бит возвращаемого значения,
и затем возвращаемое значение сдвигается на один байт влево на каждом шаге.

Повторять 4 раза, для каждого байта.

\par
Вот и всё! 
К сожалению, наверное, нет способа делать это иначе.
Не существует более-менее популярных \ac{CPU} или \ac{ISA}, где имеется инструкция для сборки значения из бит или байт.
Обычно всё это делает сдвигами бит и логическим сложением (OR).

\subsection{print\_as\_IP()}

\TT{print\_as\_IP()} делает наоборот: расщепляет 32-битное значение на 4 байта.

Расщепление работает немного проще: просто сдвигайте входное значение на 24, 16, 8 или 0 бит,
берите биты с нулевого по седьмой (младший байт), вот и всё:

\lstinputlisting[caption=\NonOptimizing MSVC 2012,style=customasmx86]{\CURPATH/print_as_IP_MSVC_2012_RU.asm}

\Optimizing MSVC 2012 делает почти всё то же самое, только без ненужных перезагрузок входного значения:

\lstinputlisting[caption=\Optimizing MSVC 2012 /Ob0,style=customasmx86]{\CURPATH/print_as_IP_MSVC_2012_Ox.asm}

\subsection{form\_netmask() и set\_bit()}

\TT{form\_netmask()} делает сетевую маску из \ac{CIDR}-нотации.

Конечно, было бы куда эффективнее использовать здесь какую-то уже готовую таблицу, но мы рассматриваем
это именно так, сознательно, для демонстрации битовых сдвигов.
Мы также сделаем отдельную функцию \TT{set\_bit()}. 

Не очень хорошая идея выделять отдельную функцию для такой примитивной операции, но так будет проще понять,
как это всё работает.

\lstinputlisting[caption=\Optimizing MSVC 2012 /Ob0,style=customasmx86]{\CURPATH/form_netmask_MSVC_2012_Ox.asm}

\TT{set\_bit()} примитивна: просто сдвигает единицу на нужное количество бит, затем складывает (логически) с
входным значением \q{input}.
\TT{form\_netmask()} имеет цикл: он выставит столько бит (начиная с \ac{MSB}), 
сколько передано в аргументе \TT{netmask\_bits}.

\subsection{Итог}

Вот и всё!
Мы запускаем и видим:

\begin{lstlisting}
netmask=255.255.255.0
network address=10.1.2.0
netmask=255.0.0.0
network address=10.0.0.0
netmask=255.255.255.128
network address=10.1.2.0
netmask=255.255.255.192
network address=10.1.2.64
\end{lstlisting}
}
\DE{\mysection{\Stack}
\label{sec:stack}
\myindex{\Stack}

Der Stack ist eine der fundamentalen Datenstrukturen in der Informatik.
\footnote{\href{http://go.yurichev.com/17119}{wikipedia.org/wiki/Call\_Stack}}.
\ac{AKA} \ac{LIFO}.

Technisch betrachtet ist es ein Stapelspeicher innerhalb des Prozessspeichers der zusammen mit den \ESP (x86), \RSP (x64) oder dem \ac{SP} (ARM) Register als ein Zeiger in diesem Speicherblock fungiert.

\myindex{ARM!\Instructions!PUSH}
\myindex{ARM!\Instructions!POP}
\myindex{x86!\Instructions!PUSH}
\myindex{x86!\Instructions!POP}

Die häufigsten Stack-Zugriffsinstruktionen sind die \PUSH- und \POP-Instruktionen (in beidem x86 und ARM Thumb-Modus). \PUSH subtrahiert vom \ESP/\RSP/\ac{SP} 4 Byte im 32-Bit Modus (oder 8 im 64-Bit Modus) und schreibt dann den Inhalt des Zeigers an die Adresse auf die von \ESP/\RSP/\ac{SP} gezeigt wird.

\POP ist die umgekehrte Operation: Die Daten des Zeigers für die Speicherregion auf die von \ac{SP}
gezeigt wird werden ausgelesen und die Inhalte in den Instruktionsoperanden geschreiben (oft ist das ein Register). Dann werden 4 (beziehungsweise 8) Byte zum \gls{stack pointer} addiert.

Nach der Stackallokation, zeigt der \gls{stack pointer} auf den Boden des Stacks.
\PUSH verringert den \gls{stack pointer} und \POP erhöht ihn.
Der Boden des Stacks ist eigentlich der Anfang der Speicherregion die für den Stack reserviert wurde.
Das wirkt zunächst seltsam, aber so funktioniert es.

ARM unterstützt beides, aufsteigende und absteigende Stacks.

\myindex{ARM!\Instructions!STMFD}
\myindex{ARM!\Instructions!LDMFD}
\myindex{ARM!\Instructions!STMED}
\myindex{ARM!\Instructions!LDMED}
\myindex{ARM!\Instructions!STMFA}
\myindex{ARM!\Instructions!LDMFA}
\myindex{ARM!\Instructions!STMEA}
\myindex{ARM!\Instructions!LDMEA}

Zum Beispiel die \ac{STMFD}/\ac{LDMFD} und \ac{STMED}/\ac{LDMED} Instruktionen sind alle dafür gedacht mit einem absteigendem Stack zu arbeiten ( wächst nach unten, fängt mit hohen Adressen an und entwickelt sich zu niedrigeren Adressen). Die \ac{STMFA}/\ac{LDMFA} und \ac{STMEA}/\ac{LDMEA} Instruktionen sind dazu gedacht mit einem aufsteigendem Stack zu arbeiten (wächst nach oben und fängt mit niedrigeren Adressen an und wächst nach oben).

% It might be worth mentioning that STMED and STMEA write first,
% and then move the pointer, and that LDMED and LDMEA move the pointer first, and then read.
% In other words, ARM not only lets the stack grow in a non-standard direction,
% but also in a non-standard order.
% Maybe this can be in the glossary, which would explain why E stands for "empty".

\subsection{Warum wächst der Stack nach unten?}
\label{stack_grow_backwards}

Intuitiv, würden man annehmen das der Stack nach oben wächst z.B Richtung höherer Adressen, so wie bei jeder anderen Datenstruktur.

Der Grund das der Stack rückwärts wächst ist wohl historisch bedingt. Als Computer so groß waren das sie einen ganzen Raum beansprucht haben war es einfach Speicher in zwei Sektionen zu unterteilen, einen Teil für den \gls{heap} und einen Teil für den Stack. Sicher war zu dieser Zeit nicht bekannt wie groß der \gls{heap} und der Stack wachsen würden, während der Programm Laufzeit, also war die Lösung die einfachste mögliche.

\input{patterns/02_stack/stack_and_heap}

In \RitchieThompsonUNIX können wir folgendes lesen:

\begin{framed}
\begin{quotation}
Der user-core eines Programm Images wird in drei logische Segmente unterteilt. Das Programm-Text Segment beginnt bei 0 im virtuellen Adress Speicher. Während der Ausführung wird das Segment als schreibgeschützt markiert und eine einzelne Kopie des Segments wird unter allen Prozessen geteilt die das Programm ausführen. An der ersten 8K grenze über dem Programm Text Segment im Virtuellen Speicher, fängt der ``nonshared'' Bereich an, der nach Bedarf von Syscalls erweitert werden kann. Beginnend bei der höchsten Adresse im Virtuellen Speicher ist das Stack Segment, das Automatisch nach unten wächst während der Hardware Stackpointer sich ändert.
\end{quotation}
\end{framed}

Das erinnert daran wie manche Schüler Notizen zu  zwei Vorträgen in einem Notebook dokumentieren:
Notizen für den ersten Vortrag werden normal notiert, und Notizen zur zum zweiten Vortrag werden 
ans Ende des Notizbuches geschrieben, indem man das Notizbuch umdreht. Die Notizen treffen sich irgendwann
im Notizbuch aufgrund des fehlenden Freien Platzes.

% I think if we want to expand on this analogy,
% one might remember that the line number increases as as you go down a page.
% So when you decrease the address when pushing to the stack, visually,
% the stack does grow upwards.
% Of course, the problem is that in most human languages,
% just as with computers,
% we write downwards, so this direction is what makes buffer overflows so messy.

\subsection{Für was wird der Stack benutzt?}

% subsections
\EN{\input{patterns/02_stack/01_saving_ret_addr_EN}}
\RU{\input{patterns/02_stack/01_saving_ret_addr_RU}}
\DE{\input{patterns/02_stack/01_saving_ret_addr_DE}}
\FR{\input{patterns/02_stack/01_saving_ret_addr_FR}}
\PTBR{\input{patterns/02_stack/01_saving_ret_addr_PTBR}}
\IT{\input{patterns/02_stack/01_saving_ret_addr_IT}}
\PL{\input{patterns/02_stack/01_saving_ret_addr_PL}}
\JA{\input{patterns/02_stack/01_saving_ret_addr_JA}}

\EN{\input{patterns/02_stack/02_args_passing_EN}}
\RU{\input{patterns/02_stack/02_args_passing_RU}}
\PTBR{\input{patterns/02_stack/02_args_passing_PTBR}}
\DE{\input{patterns/02_stack/02_args_passing_DE}}
\IT{\input{patterns/02_stack/02_args_passing_IT}}
\FR{\input{patterns/02_stack/02_args_passing_FR}}
\JA{\input{patterns/02_stack/02_args_passing_JA}}
\PL{\input{patterns/02_stack/02_args_passing_PL}}


\EN{\input{patterns/02_stack/03_local_vars_EN}}
\RU{\input{patterns/02_stack/03_local_vars_RU}}
\DE{\input{patterns/02_stack/03_local_vars_DE}}
\PTBR{\input{patterns/02_stack/03_local_vars_PTBR}}
\EN{\input{patterns/02_stack/04_alloca/main_EN}}
\FR{\input{patterns/02_stack/04_alloca/main_FR}}
\RU{\input{patterns/02_stack/04_alloca/main_RU}}
\PTBR{\input{patterns/02_stack/04_alloca/main_PTBR}}
\IT{\input{patterns/02_stack/04_alloca/main_IT}}
\DE{\input{patterns/02_stack/04_alloca/main_DE}}
\PL{\input{patterns/02_stack/04_alloca/main_PL}}
\JA{\input{patterns/02_stack/04_alloca/main_JA}}

\subsubsection{(Windows) SEH}
\myindex{Windows!Structured Exception Handling}

\ifdefined\RUSSIAN
В стеке хранятся записи \ac{SEH} для функции (если они присутствуют).
Читайте больше о нем здесь: (\myref{sec:SEH}).
\fi % RUSSIAN

\ifdefined\ENGLISH
\ac{SEH} records are also stored on the stack (if they are present).
Read more about it: (\myref{sec:SEH}).
\fi % ENGLISH

\ifdefined\BRAZILIAN
\ac{SEH} também são guardados na pilha (se estiverem presentes).
\PTBRph{}: (\myref{sec:SEH}).
\fi % BRAZILIAN

\ifdefined\ITALIAN
I record \ac{SEH}, se presenti, sono anch'essi memorizzati nello stack.
Maggiori informazioni qui: (\myref{sec:SEH}).
\fi % ITALIAN

\ifdefined\FRENCH
Les enregistrements \ac{SEH} sont aussi stockés dans la pile (s'ils sont présents).
Lire à ce propos: (\myref{sec:SEH}).
\fi % FRENCH


\ifdefined\POLISH
Na stosie są przechowywane wpisy \ac{SEH} dla funkcji (jeśli są one obecne).
Więcej o tym tutaj: (\myref{sec:SEH}).
\fi % POLISH

\ifdefined\JAPANESE
\ac{SEH}レコードはスタックにも格納されます(存在する場合)。
それについてもっと読む:(\myref{sec:SEH})
\fi % JAPANESE

\ifdefined\ENGLISH
\subsubsection{Buffer overflow protection}

More about it here~(\myref{subsec:bufferoverflow}).
\fi

\ifdefined\RUSSIAN
\subsubsection{Защита от переполнений буфера}

Здесь больше об этом~(\myref{subsec:bufferoverflow}).
\fi

\ifdefined\BRAZILIAN
\subsubsection{Proteção contra estouro de buffer}

Mais sobre aqui~(\myref{subsec:bufferoverflow}).
\fi

\ifdefined\ITALIAN
\subsubsection{Protezione da buffer overflow}

Maggiori informazioni qui~(\myref{subsec:bufferoverflow}).
\fi

\ifdefined\FRENCH
\subsubsection{Protection contre les débordements de tampon}

Lire à ce propos~(\myref{subsec:bufferoverflow}).
\fi


\ifdefined\POLISH
\subsubsection{Metody zabiezpieczenia przed przepełnieniem stosu}

Więcej o tym tutaj~(\myref{subsec:bufferoverflow}).
\fi

\ifdefined\JAPANESE
\subsubsection{バッファオーバーフロー保護}

詳細はこちら~(\myref{subsec:bufferoverflow})
\fi

\subsubsection{Automatisches deallokieren der Daten auf dem Stack}

Vielleicht ist der Grund warum man lokale Variablen und SEH Einträge auf dem Stack speichert, weil sie beim 
verlassen der Funktion automatisch aufgeräumt werden. Man braucht dabei nur eine Instruktion um die Position
des Stackpointers zu korrigieren (oftmals ist es die \ADD Instruktion). Funktions Argumente, könnte man sagen 
werden auch am Ende der Funktion deallokiert. Im Kontrast dazu, alles was auf dem \emph{heap} gespeichert wird muss
explizit deallokiert werden. 

% sections
\EN{\input{patterns/02_stack/07_layout_EN}}
\RU{\input{patterns/02_stack/07_layout_RU}}
\DE{\input{patterns/02_stack/07_layout_DE}}
\PTBR{\input{patterns/02_stack/07_layout_PTBR}}
\EN{\input{patterns/02_stack/08_noise/main_EN}}
\FR{\input{patterns/02_stack/08_noise/main_FR}}
\RU{\input{patterns/02_stack/08_noise/main_RU}}
\IT{\input{patterns/02_stack/08_noise/main_IT}}
\DE{\input{patterns/02_stack/08_noise/main_DE}}
\PL{\input{patterns/02_stack/08_noise/main_PL}}
\JA{\input{patterns/02_stack/08_noise/main_JA}}

\input{patterns/02_stack/exercises}
}
\FR{\subsection{Décalage à droite}

\lstinputlisting[style=customc]{patterns/185_64bit_in_32_env/shifting/3.c}

\subsubsection{x86}

\lstinputlisting[caption=MSVC 2012 /Ob1 \Optimizing,style=customasmx86]{patterns/185_64bit_in_32_env/shifting/3_MSVC.asm}

\lstinputlisting[caption=GCC 4.8.1 -fno-inline \Optimizing,style=customasmx86]{patterns/185_64bit_in_32_env/shifting/3_GCC.asm}

\myindex{x86!\Instructions!SHRD}

Le décalage se produit en deux passes: tout d'abord la partie basse est décalée,
puis la partie haute.
Mais la partie basse est décalée avec l'aide de l'instruction \INS{SHRD}, elle décale
la valeur de \EAX{} de 7 bits, mais tire les nouveaux bits de \EDX{}, i.e., de la
partie haute.
En d'autres mots, la valeur 64-bit dans la paire de registres \TT{EDX:EAX}, dans
son entier, est décalée de 7 bits et les 32 bits bas du résultat sont placés dans
\EAX{}.
La partie haute est décalée en utilisant l'instruction plus populaire \SHR{}: en
effet, les bits libérés dans la partie haute doivent être remplis avec des zéros.

\subsubsection{ARM}

ARM n'a pas une instruction telle que \INS{SHRD} en x86, donc le compilateur Keil
fait cela en utilisant des simples décalages et des opérations \INS{OR}:

\lstinputlisting[caption=\OptimizingKeilVI (\ARMMode),style=customasmARM]{patterns/185_64bit_in_32_env/shifting/Keil_ARM_O3.s}

\lstinputlisting[caption=\OptimizingKeilVI (\ThumbMode),style=customasmARM]{patterns/185_64bit_in_32_env/shifting/Keil_thumb_O3.s}
% TODO add explanationen 

\subsubsection{MIPS}

GCC pour MIPS suit le même algorithme que Keil fait pour le mode Thumb:

\lstinputlisting[caption=GCC 4.4.5 \Optimizing (IDA),style=customasmMIPS]{patterns/185_64bit_in_32_env/shifting/MIPS_O3_IDA.lst}

% TODO add explanation

}
\JA{\subsection{右シフト}

\lstinputlisting[style=customc]{patterns/185_64bit_in_32_env/shifting/3.c}

\subsubsection{x86}

\lstinputlisting[caption=\Optimizing MSVC 2012 /Ob1,style=customasmx86]{patterns/185_64bit_in_32_env/shifting/3_MSVC.asm}

\lstinputlisting[caption=\Optimizing GCC 4.8.1 -fno-inline,style=customasmx86]{patterns/185_64bit_in_32_env/shifting/3_GCC.asm}

\myindex{x86!\Instructions!SHRD}

シフトは2つのパスでも発生します:最初に下部がシフトされ、次に上部がシフトされます。 
しかし、下位部分は\INS{SHRD}命令の助けを借りてシフトされ、それは\EAX{}の値を7ビットだけシフトしますが、\EDX{}から
すなわち上位部分から新しいビットを引き出します。 つまり、\TT{EDX:EAX}レジスタのペアからの64ビット値は、全体として7ビットシフトされ、
結果の最下位32ビットが \EAX{} に格納されます。 
上位部分は、より一般的な \SHR{} 命令を使用してシフトされます。実際、上位部分の解放されたビットは
ゼロで埋められなければなりません。

\subsubsection{ARM}

ARMはx86では\INS{SHRD}のような命令を持っていないので、Keilコンパイラはこれを単純なシフトと\INS{OR}演算を使って行うべきです。

\lstinputlisting[caption=\OptimizingKeilVI (\ARMMode),style=customasmARM]{patterns/185_64bit_in_32_env/shifting/Keil_ARM_O3.s}

\lstinputlisting[caption=\OptimizingKeilVI (\ThumbMode),style=customasmARM]{patterns/185_64bit_in_32_env/shifting/Keil_thumb_O3.s}
% TODO add explanation

\subsubsection{MIPS}

MIPS向けのGCCは、KeilがThumbモードで行うのと同じアルゴリズムに従います。

\lstinputlisting[caption=\Optimizing GCC 4.4.5 (IDA),style=customasmMIPS]{patterns/185_64bit_in_32_env/shifting/MIPS_O3_IDA.lst}

% TODO add explanation

}

\EN{\subsection{Win32 PE}
\label{win32_pe}
\myindex{Windows!Win32}

\acs{PE} is an executable file format used in Windows.
The difference between .exe, .dll and .sys is that .exe and .sys usually do not have exports, only imports.

\myindex{OEP}

A \ac{DLL}, just like any other PE-file, has an entry point (\ac{OEP}) (the function DllMain() is located there)
but this function usually does nothing.
.sys is usually a device driver.
As of drivers, Windows requires the checksum to be present in the PE file and for it to be correct
\footnote{For example, Hiew(\myref{Hiew}) can calculate it}.

\myindex{Windows!Windows Vista}
Starting at Windows Vista, a driver's files must also be signed with a digital signature. It will fail to load otherwise.

\myindex{MS-DOS}
Every PE file begins with tiny DOS program that prints a
message like \q{This program cannot be run in DOS mode.}---if you run this program in DOS or Windows 3.1 (\ac{OS}-es which are not aware of the PE format),
this message will be printed.

\subsubsection{Terminology}

\myindex{VA}
\myindex{Base address}
\myindex{RVA}
\myindex{Windows!IAT}
\myindex{Windows!INT}

\begin{itemize}
\item Module---a separate file, .exe or .dll.

\item Process---a program loaded into memory and currently running.  Commonly consists of one .exe file and bunch of .dll files.

\item Process memory---the memory a process works with.  Each process has its own.
There usually are loaded modules, memory of the stack, \gls{heap}(s), etc.

\item \ac{VA}---an address which is to be used in program while runtime.

\item Base address (of module)---the address within the process memory at which the module is to be loaded.
\ac{OS} loader may change it, if the base address is already occupied by another module just loaded before.

\item \ac{RVA}---the \ac{VA}-address minus the base address.

Many addresses in PE-file tables use \ac{RVA}-addresses.

%\item
%Data directory --- ...

\item \ac{IAT}---an array of addresses of imported symbols \footnote{\PietrekPE}.
Sometimes, the \TT{IMAGE\_DIRECTORY\_ENTRY\_IAT} data directory points at the \ac{IAT}.
\label{IDA_idata}
It is worth noting that \ac{IDA} (as of 6.1) may allocate a pseudo-section named \TT{.idata} for
\ac{IAT}, even if the \ac{IAT} is a part of another section!

\item \ac{INT}---an array of names of symbols to be imported\footnote{\PietrekPE}.
\end{itemize}

\subsubsection{Base address}

The problem is that several module authors can prepare DLL files for others to use and it is not possible
to reach an agreement which addresses is to be assigned to whose modules.

So that is why if two necessary DLLs for a process have the same base address,
one of them will be loaded at this base address, and the other---at some other free space in process memory,
and each virtual addresses in the second DLL will be corrected.

\par With \ac{MSVC} the linker often generates the .exe files with a base address of \TT{0x400000}
\footnote{The origin of this address choice is described here: \href{http://go.yurichev.com/17041}{MSDN}},
and with the code section starting at \TT{0x401000}.
This means that the \ac{RVA} of the start of the code section is \TT{0x1000}.

DLLs are often generated by MSVC's linker with a base address of \TT{0x10000000}
\footnote{This can be changed by the /BASE linker option}.

\myindex{ASLR}

There is also another reason to load modules at various base addresses, in this case random ones.
It is \ac{ASLR}\footnote{\href{http://go.yurichev.com/17140}{wikipedia}}.

\myindex{Shellcode}

A shellcode trying to get executed on a compromised system must call system functions, hence, know their addresses.

In older \ac{OS} (in \gls{Windows NT} line: before Windows Vista),
system DLL (like kernel32.dll, user32.dll) were always loaded at known addresses,
and if we also recall
that their versions rarely changed, the addresses of functions were
fixed and shellcode could call them directly.

In order to avoid this, the \ac{ASLR}
method loads your program and all modules it needs at random base addresses, different every time.

\ac{ASLR} support is denoted in a PE file by setting the flag
\par \TT{IMAGE\_DLL\_CHARACTERISTICS\_DYNAMIC\_BASE} \InSqBrackets{see \Russinovich}.

\subsubsection{Subsystem}

There is also a \emph{subsystem} field, usually it is:

\myindex{Native API}

\begin{itemize}
\item native\footnote{Meaning, the module use Native API instead of Win32} (.sys-driver),

\item console (console application) or

\item \ac{GUI} (non-console).
\end{itemize}

\subsubsection{OS version}

A PE file also specifies the minimal Windows version it needs in order to be loadable.

The table of version numbers stored in the PE file and corresponding Windows codenames is
here\footnote{\href{http://go.yurichev.com/17044}{wikipedia}}.

\myindex{Windows!Windows NT4}
\myindex{Windows!Windows 2000}
For example, \ac{MSVC} 2005 compiles .exe files for running on Windows NT4 (version 4.00), but \ac{MSVC} 2008 does not
(the generated files have a version of 5.00, at least Windows 2000 is needed to run them).

\myindex{Windows!Windows XP}

\ac{MSVC} 2012 generates .exe files of version 6.00 by default,
targeting at least Windows Vista.
However, by changing the compiler's options\footnote{\href{http://go.yurichev.com/17045}{MSDN}},
it is possible to force it to compile for Windows XP.

\subsubsection{Sections}

Division in sections, as it seems, is present in all executable file formats.

It is devised in order to separate code from data, and data---from constant data.

\begin{itemize}
\item Either the \emph{IMAGE\_SCN\_CNT\_CODE} or \emph{IMAGE\_SCN\_MEM\_EXECUTE} flags will be set on the code section---this is executable code.

\item On data section---\emph{IMAGE\_SCN\_CNT\_INITIALIZED\_DATA},\\
\emph{IMAGE\_SCN\_MEM\_READ} and \emph{IMAGE\_SCN\_MEM\_WRITE} flags.

\item On an empty section with uninitialized data---\\
\emph{IMAGE\_SCN\_CNT\_UNINITIALIZED\_DATA}, \emph{IMAGE\_SCN\_MEM\_READ} \\
        and \emph{IMAGE\_SCN\_MEM\_WRITE}.

\item On a constant data section (one that's protected from writing), the flags \\
\emph{IMAGE\_SCN\_CNT\_INITIALIZED\_DATA} and \emph{IMAGE\_SCN\_MEM\_READ} can be set, \\
but not \emph{IMAGE\_SCN\_MEM\_WRITE}.
A process going to crash if it tries to write to this section.
\end{itemize}

\myindex{TLS}
\myindex{BSS}
Each section in PE-file may have a name, however, it is not very important.
Often (but not always) the code section is named \TT{.text},
the data section---\TT{.data}, the constant data section --- \TT{.rdata} \emph{(readable data)}.
Other popular section names are:

\myindex{MIPS}
\begin{itemize}
\item \TT{.idata}---imports section.
\ac{IDA} may create a pseudo-section named like this: \myref{IDA_idata}.
\item \TT{.edata}---exports section (rare)
\item \TT{.pdata}---section holding all information about exceptions in Windows NT for MIPS, \ac{IA64} and x64: \myref{SEH_win64}
\item \TT{.reloc}---relocs section
\item \TT{.bss}---uninitialized data (\ac{BSS})
\item \TT{.tls}---thread local storage (\ac{TLS})
\item \TT{.rsrc}---resources
\item \TT{.CRT}---may present in binary files compiled by ancient MSVC versions
\end{itemize}

PE file packers/encryptors often garble section names or replace the names with their own.

\ac{MSVC} allows you to declare data in arbitrarily named section
\footnote{\href{http://go.yurichev.com/17047}{MSDN}}.

Some compilers and linkers can add a section with debugging symbols and
other debugging information (MinGW for instance).
\myindex{Windows!PDB}
However it is not so in latest versions of \ac{MSVC} (separate \gls{PDB} files are used there for this purpose).\\
\\
That is how a PE section is described in the file:

\begin{lstlisting}[style=customc]
typedef struct _IMAGE_SECTION_HEADER {
  BYTE  Name[IMAGE_SIZEOF_SHORT_NAME];
  union {
    DWORD PhysicalAddress;
    DWORD VirtualSize;
  } Misc;
  DWORD VirtualAddress;
  DWORD SizeOfRawData;
  DWORD PointerToRawData;
  DWORD PointerToRelocations;
  DWORD PointerToLinenumbers;
  WORD  NumberOfRelocations;
  WORD  NumberOfLinenumbers;
  DWORD Characteristics;
} IMAGE_SECTION_HEADER, *PIMAGE_SECTION_HEADER;
\end{lstlisting}
\footnote{\href{http://go.yurichev.com/17048}{MSDN}}

\myindex{Hiew}
A word about terminology: \emph{PointerToRawData} is called \q{Offset} in Hiew
and \emph{VirtualAddress} is called \q{RVA} there.

\subsubsection{Data section}

Data section in file can be smaller than in memory.
For example, some variables can be initialized, some are not.
Compiler and linker will collect them all into one section, but the first part of it is initialized and allocated in file,
while another is absent in file (of course, to make it smaller).
\emph{VirtualSize} will be equal to the size of section in memory, and \emph{SizeOfRawData} --- to
size of section in file.

IDA can show the border between initialized and not initialized parts like that:

\begin{lstlisting}[style=customasmx86]
...

.data:10017FFA                 db    0
.data:10017FFB                 db    0
.data:10017FFC                 db    0
.data:10017FFD                 db    0
.data:10017FFE                 db    0
.data:10017FFF                 db    0
.data:10018000                 db    ? ;
.data:10018001                 db    ? ;
.data:10018002                 db    ? ;
.data:10018003                 db    ? ;
.data:10018004                 db    ? ;
.data:10018005                 db    ? ;

...
\end{lstlisting}

\subsubsection{Relocations (relocs)}
\label{subsec:relocs}

\ac{AKA} FIXUP-s (at least in Hiew).

They are also present in almost all executable file formats
\footnote{Even in .exe files for MS-DOS}.
Exceptions are shared dynamic libraries compiled with \ac{PIC}, or any other \ac{PIC}-code.

What are they for?

Obviously, modules can be loaded on various base addresses, but how to deal with global variables, for example?
They must be accessed by address.  One solution is \PICcode{} (\myref{sec:PIC}).
But it is not always convenient.

That is why a relocations table is present.
There the addresses of points that must be corrected are enumerated,
in case of loading at a different base address.

% TODO тут бы пример с HIEW или objdump..
For example, there is a global variable at address \TT{0x410000} and this is how it is accessed:

\begin{lstlisting}[style=customasmx86]
A1 00 00 41 00         mov         eax,[000410000]
\end{lstlisting}

The base address of the module is \TT{0x400000}, the \ac{RVA} of the global variable is \TT{0x10000}.

If the module is loaded at base address \TT{0x500000}, the real address of the global variable must be \TT{0x510000}.

\myindex{x86!\Instructions!MOV}

As we can see, the address of variable is encoded in the instruction \TT{MOV}, after the byte \TT{0xA1}.

That is why the address of the 4 bytes after \TT{0xA1}, is written in the relocs table.

If the module is loaded at a different base address, the \ac{OS} loader enumerates all addresses in the table,

finds each 32-bit word the address points to, subtracts the original base address from it
(we get the \ac{RVA} here), and adds the new base address to it.

If a module is loaded at its original base address, nothing happens.

All global variables can be treated like that.

Relocs may have various types, however, in Windows for x86 processors, the type is usually \\
\emph{IMAGE\_REL\_BASED\_HIGHLOW}.

\myindex{Hiew}

By the way, relocs are darkened in Hiew, for example: \figref{fig:scanf_ex3_hiew_1}.

\myindex{\olly}
\olly underlines the places in memory to which relocs are to be applied, for example: \figref{fig:switch_lot_olly3}.

\subsubsection{Exports and imports}

\label{PE_exports_imports}
As we all know, any executable program must use the \ac{OS}'s services and other DLL-libraries somehow.

It can be said that functions from one module (usually DLL) must be connected somehow to the points of their
calls in other modules (.exe-file or another DLL).

For this, each DLL has an \q{exports} table, which consists of functions plus their addresses in a module.

And every .exe file or DLL has \q{imports}, a table of functions it needs for execution including
list of DLL filenames.

After loading the main .exe-file, the \ac{OS} loader processes imports table:
it loads the additional DLL-files, finds function names
among the DLL exports and writes their addresses down in the \ac{IAT} of the main .exe-module.

\myindex{Windows!Win32!Ordinal}

As we can see, during loading the loader must compare a lot of function names, but string comparison is not a very
fast procedure, so there is a support for \q{ordinals} or \q{hints},
which are function numbers stored in the table, instead of their names.

That is how they can be located faster when loading a DLL.
Ordinals are always present in the \q{export} table.

\myindex{MFC}
For example, a program using the \ac{MFC} library usually loads mfc*.dll by ordinals,
and in such programs there are no \ac{MFC} function names in \ac{INT}.

% TODO example!
When loading such programs in \IDA, it will ask for a path to the mfc*.dll files
in order to determine the function names.

If you don't tell \IDA the path to these DLLs, there will be \emph{mfc80\_123} instead of function names.

\myparagraph{Imports section}

Often a separate section is allocated for the imports table and everything related to it (with name like \TT{.idata}),
however, this is not a strict rule.

Imports are also a confusing subject because of the terminological mess. Let's try to collect all information in one place.

\begin{figure}[H]
\centering
\myincludegraphics{OS/PE/unnamed0.png}
\caption{
A scheme that unites all PE-file structures related to imports}
\end{figure}

The main structure is the array \emph{IMAGE\_IMPORT\_DESCRIPTOR}.
Each element for each DLL being imported.

Each element holds the \ac{RVA} address of the text string (DLL name) (\emph{Name}).

\emph{OriginalFirstThunk} is the \ac{RVA} address of the \ac{INT} table.
This is an array of \ac{RVA} addresses, each of which points to a text string with a function name.
Each string is prefixed by a 16-bit integer
(\q{hint})---\q{ordinal} of function.

While loading, if it is possible to find a function by ordinal,
then the strings comparison will not occur. The array is terminated by zero.

There is also a pointer to the \ac{IAT} table named \emph{FirstThunk}, it is just the \ac{RVA} address
of the place where the loader writes the addresses of the resolved functions.

The points where the loader writes addresses are marked by \IDA like this: \emph{\_\_imp\_CreateFileA}, etc.

There are at least two ways to use the addresses written by the loader.

\myindex{x86!\Instructions!CALL}
\begin{itemize}
\item The code will have instructions like \emph{call \_\_imp\_CreateFileA},
and since the field with the address of the imported function is a global variable in some sense,
the address of the \emph{call} instruction (plus 1 or 2) is to be added to the relocs table,
for the case when the module is loaded at a different base address.

But, obviously, this may enlarge relocs table significantly.

Because there are might be a lot of calls to imported functions in the module.

Furthermore, large relocs table slows down the process of loading modules.

\myindex{x86!\Instructions!JMP}
\myindex{thunk-functions}
\item For each imported function, there is only one jump allocated, using the \JMP instruction
plus a reloc to it.
Such points are also called \q{thunks}.

All calls to the imported functions are just \CALL instructions to the corresponding \q{thunk}.
In this case, additional relocs are not necessary because these \CALL{}-s
have relative addresses and do not need to be corrected.
\end{itemize}

These two methods can be combined.

Possible, the linker creates individual \q{thunk}s if there are too many calls to the function,
but not done by default. \\
\\
By the way, the array of function addresses to which FirstThunk is pointing is not necessary to be located in the \ac{IAT} section.
For example, the author of these lines once wrote the PE\_add\_import\footnote{\href{http://go.yurichev.com/17049}{yurichev.com}}
utility for adding imports to an existing .exe-file.

Some time earlier, in the previous versions of the utility,
at the place of the function you want to substitute with a call to another DLL,
my utility wrote the following code:

\begin{lstlisting}[style=customasmx86]
MOV EAX, [yourdll.dll!function]
JMP EAX
\end{lstlisting}

FirstThunk points to the first instruction. In other words, when loading yourdll.dll,
the loader writes the address of the \emph{function} function right in the code.

It also worth noting that a code section is usually write-protected, so my utility adds the \\
\emph{IMAGE\_SCN\_MEM\_WRITE}
flag for code section. Otherwise, the program to crash while loading with error code
5 (access denied). \\
\\
One might ask: what if I supply a program with a set of DLL files which is not supposed to change (including addresses of all DLL functions),
is it possible to speed up the loading process?

Yes, it is possible to write the addresses of the functions to be imported into the FirstThunk arrays in advance.
The \emph{Timestamp} field is present in the \\
\emph{IMAGE\_IMPORT\_DESCRIPTOR} structure.

If a value is present there, then the loader compares this value with the date-time of the DLL file.

If the values are equal, then the loader does not do anything, and the loading of the process can be faster.
This is called \q{old-style binding}
\footnote{\href{http://go.yurichev.com/17050}{MSDN}. There is also the \q{new-style binding}.}.
\myindex{BIND.EXE}

The BIND.EXE utility in Windows SDK is for this.
For speeding up the loading of your program, Matt Pietrek in \PietrekPEURL, suggests to do the binding shortly after your program
installation on the computer of the end user. \\
\\
PE-files packers/encryptors may also compress/encrypt imports table.

In this case, the Windows loader, of course, will not load all necessary DLLs.
\myindex{Windows!Win32!LoadLibrary}
\myindex{Windows!Win32!GetProcAddress}

Therefore, the packer/encryptor does this on its own, with the help of
\emph{LoadLibrary()} and the \emph{GetProcAddress()} functions.

That is why these two functions are often present in \ac{IAT} in packed files.\\
\\
In the standard DLLs from the Windows installation, \ac{IAT} often is located right at the beginning of the PE file.
Supposedly, it is made so for optimization.

While loading, the .exe file is not loaded into memory as a whole (recall huge install programs which are
started suspiciously fast), it is \q{mapped}, and loaded into memory in parts as they are accessed.

Probably, Microsoft developers decided it will be faster.

\subsubsection{Resources}

\label{PEresources}

Resources in a PE file are just a set of icons, pictures, text strings, dialog descriptions.

Perhaps they were separated from the main code, so all these things could be multilingual,
and it would be simpler to pick text or picture for the language that is currently set in the \ac{OS}. \\
\\
As a side effect, they can be edited easily and saved back to the executable file, even if one does not have special knowledge,
by using the ResHack editor, for example (\myref{ResHack}).

\subsubsection{.NET}

\myindex{.NET}

.NET programs are not compiled into machine code but into a special bytecode.
\myindex{OEP}
Strictly speaking, there is bytecode instead of the usual x86 code
in the .exe file, however, the entry point (\ac{OEP}) points to this tiny fragment of x86 code:

\begin{lstlisting}[style=customasmx86]
jmp         mscoree.dll!_CorExeMain
\end{lstlisting}

The .NET loader is located in mscoree.dll, which processes the PE file.
\myindex{Windows!Windows XP}

It was so in all pre-Windows XP \ac{OS}es. Starting from XP, the \ac{OS} loader is able to detect the .NET file
and run it without executing that \JMP instruction
\footnote{\href{http://go.yurichev.com/17051}{MSDN}}.

\myindex{TLS}
\subsubsection{TLS}

This section holds initialized data for the \ac{TLS}(\myref{TLS}) (if needed).
When a new thread start, its \ac{TLS} data is initialized using the data from this section. \\
\\
\myindex{TLS!Callbacks}

Aside from that, the PE file specification also provides initialization of the
\ac{TLS} section, the so-called TLS callbacks.

If they are present, they are to be called before the control is passed to the main entry point (\ac{OEP}).

This is used widely in the PE file packers/encryptors.

\subsubsection{Tools}

\myindex{objdump}
\myindex{Cygwin}
\myindex{Hiew}
\label{ResHack}

\begin{itemize}
\item objdump (present in cygwin) for dumping all PE-file structures.

\item Hiew(\myref{Hiew}) as editor.

\item pefile---Python-library for PE-file processing \footnote{\url{http://go.yurichev.com/17052}}.

\item ResHack \acs{AKA} Resource Hacker---resources editor\footnote{\url{http://go.yurichev.com/17052}}.

\item PE\_add\_import\footnote{\url{http://go.yurichev.com/17049}}---
simple tool for adding symbol(s) to PE executable import table.

\item PE\_patcher\footnote{\href{http://go.yurichev.com/17054}{yurichev.com}}---simple tool for patching PE executables.

\item PE\_search\_str\_refs\footnote{\href{http://go.yurichev.com/17055}{yurichev.com}}---simple tool for searching for a function in PE executables which use some text string.
\end{itemize}

\subsubsection{Further reading}

% FIXME: bibliography per chapter or section
\begin{itemize}
\item Daniel Pistelli---The .NET File Format \footnote{\url{http://go.yurichev.com/17056}}
\end{itemize}

}
\RU{\mysection{Пример вычисления адреса сети}

Как мы знаем, TCP/IP-адрес (IPv4) состоит из четырех чисел в пределах $0 \ldots 255$, т.е. 4 байта.

4 байта легко помещаются в 32-битную переменную, так что адрес хоста в IPv4, сетевая маска или адрес сети
могут быть 32-битными числами.

С точки зрения пользователя, маска сети определяется четырьмя числами в формате вроде \\
255.255.255.0,
но сетевые инженеры (сисадмины) используют более компактную нотацию (\ac{CIDR}),
вроде  \q{/8}, \q{/16}, итд.

Эта нотация просто определяет количество бит в сетевой маске, начиная с \ac{MSB}.

\small
\begin{center}
\begin{tabular}{ | l | l | l | l | l | l | }
\hline
\HeaderColor Маска & 
\HeaderColor Хосты & 
\HeaderColor Свободно &
\HeaderColor Сетевая маска &
\HeaderColor В шест.виде &
\HeaderColor \\
\hline
/30  & 4        & 2        & 255.255.255.252  & 0xfffffffc  & \\
\hline
/29  & 8        & 6        & 255.255.255.248  & 0xfffffff8  & \\
\hline
/28  & 16       & 14       & 255.255.255.240  & 0xfffffff0  & \\
\hline
/27  & 32       & 30       & 255.255.255.224  & 0xffffffe0  & \\
\hline
/26  & 64       & 62       & 255.255.255.192  & 0xffffffc0  & \\
\hline
/24  & 256      & 254      & 255.255.255.0    & 0xffffff00  & сеть класса C \\
\hline
/23  & 512      & 510      & 255.255.254.0    & 0xfffffe00  & \\
\hline
/22  & 1024     & 1022     & 255.255.252.0    & 0xfffffc00  & \\
\hline
/21  & 2048     & 2046     & 255.255.248.0    & 0xfffff800  & \\
\hline
/20  & 4096     & 4094     & 255.255.240.0    & 0xfffff000  & \\
\hline
/19  & 8192     & 8190     & 255.255.224.0    & 0xffffe000  & \\
\hline
/18  & 16384    & 16382    & 255.255.192.0    & 0xffffc000  & \\
\hline
/17  & 32768    & 32766    & 255.255.128.0    & 0xffff8000  & \\
\hline
/16  & 65536    & 65534    & 255.255.0.0      & 0xffff0000  & сеть класса B \\
\hline
/8   & 16777216 & 16777214 & 255.0.0.0        & 0xff000000  & сеть класса A \\
\hline
\end{tabular}
\end{center}
\normalsize

Вот простой пример, вычисляющий адрес сети используя сетевую маску и адрес хоста.

\lstinputlisting[style=customc]{\CURPATH/netmask.c}

\subsection{calc\_network\_address()}

Функция \TT{calc\_network\_address()} самая простая: 

она просто умножает (логически, используя \AND) адрес хоста на сетевую маску, в итоге давая адрес
сети.

\lstinputlisting[caption=\Optimizing MSVC 2012 /Ob0,numbers=left,style=customasmx86]{\CURPATH/calc_network_address_MSVC_2012_Ox.asm}

На строке 22 мы видим самую важную инструкцию \AND --- так вычисляется адрес сети.

\subsection{form\_IP()}

Функция \TT{form\_IP()} просто собирает все 4 байта в одно 32-битное значение.

Вот как это обычно происходит:

\begin{itemize}
\item Выделите переменную для возвращаемого значения. Обнулите её.

\item 
Возьмите четвертый (самый младший) байт, сложите его (логически, инструкцией \OR) с возвращаемым
значением. Оно содержит теперь 4-й байт.

\item Возьмите третий байт, сдвиньте его на 8 бит влево.
Получится значение в виде \TT{0x0000bb00}, где \TT{bb} это третий байт.
Сложите итоговое значение (логически, инструкцией \OR) с возвращаемым значением.
Возвращаемое значение пока что содержит \TT{0x000000aa}, так что логическое сложение
в итоге выдаст значение вида \TT{0x0000bbaa}.

\item 
Возьмите второй байт, сдвиньте его на 16 бит влево.
Вы получите значение вида \TT{0x00cc0000}, где \TT{cc} это второй байт.
Сложите (логически) результат и возвращаемое значение.
Выходное значение содержит пока что \TT{0x0000bbaa}, так что логическое сложение
в итоге выдаст значение вида \TT{0x00ccbbaa}.

\item 
Возьмите первый байт, сдвиньте его на 24 бита влево.
Вы получите значение вида \TT{0xdd000000}, где \TT{dd} это первый байт.
Сложите (логически) результат и выходное значение.
Выходное значение содержит пока что \TT{0x00ccbbaa}, так что сложение выдаст в итоге значение
вида \TT{0xddccbbaa}.

\end{itemize}

И вот как работает неоптимизирующий MSVC 2012:

\lstinputlisting[caption=\NonOptimizing MSVC 2012,style=customasmx86]{\CURPATH/form_IP_MSVC_2012_RU.asm}

Хотя, порядок операций другой, но, конечно, порядок роли не играет.

\Optimizing MSVC 2012 делает то же самое, но немного иначе:

\lstinputlisting[caption=\Optimizing MSVC 2012 /Ob0,style=customasmx86]{\CURPATH/form_IP_MSVC_2012_Ox_RU.asm}

Можно сказать, что каждый байт записывается в младшие 8 бит возвращаемого значения,
и затем возвращаемое значение сдвигается на один байт влево на каждом шаге.

Повторять 4 раза, для каждого байта.

\par
Вот и всё! 
К сожалению, наверное, нет способа делать это иначе.
Не существует более-менее популярных \ac{CPU} или \ac{ISA}, где имеется инструкция для сборки значения из бит или байт.
Обычно всё это делает сдвигами бит и логическим сложением (OR).

\subsection{print\_as\_IP()}

\TT{print\_as\_IP()} делает наоборот: расщепляет 32-битное значение на 4 байта.

Расщепление работает немного проще: просто сдвигайте входное значение на 24, 16, 8 или 0 бит,
берите биты с нулевого по седьмой (младший байт), вот и всё:

\lstinputlisting[caption=\NonOptimizing MSVC 2012,style=customasmx86]{\CURPATH/print_as_IP_MSVC_2012_RU.asm}

\Optimizing MSVC 2012 делает почти всё то же самое, только без ненужных перезагрузок входного значения:

\lstinputlisting[caption=\Optimizing MSVC 2012 /Ob0,style=customasmx86]{\CURPATH/print_as_IP_MSVC_2012_Ox.asm}

\subsection{form\_netmask() и set\_bit()}

\TT{form\_netmask()} делает сетевую маску из \ac{CIDR}-нотации.

Конечно, было бы куда эффективнее использовать здесь какую-то уже готовую таблицу, но мы рассматриваем
это именно так, сознательно, для демонстрации битовых сдвигов.
Мы также сделаем отдельную функцию \TT{set\_bit()}. 

Не очень хорошая идея выделять отдельную функцию для такой примитивной операции, но так будет проще понять,
как это всё работает.

\lstinputlisting[caption=\Optimizing MSVC 2012 /Ob0,style=customasmx86]{\CURPATH/form_netmask_MSVC_2012_Ox.asm}

\TT{set\_bit()} примитивна: просто сдвигает единицу на нужное количество бит, затем складывает (логически) с
входным значением \q{input}.
\TT{form\_netmask()} имеет цикл: он выставит столько бит (начиная с \ac{MSB}), 
сколько передано в аргументе \TT{netmask\_bits}.

\subsection{Итог}

Вот и всё!
Мы запускаем и видим:

\begin{lstlisting}
netmask=255.255.255.0
network address=10.1.2.0
netmask=255.0.0.0
network address=10.0.0.0
netmask=255.255.255.128
network address=10.1.2.0
netmask=255.255.255.192
network address=10.1.2.64
\end{lstlisting}
}
\DE{\mysection{\Stack}
\label{sec:stack}
\myindex{\Stack}

Der Stack ist eine der fundamentalen Datenstrukturen in der Informatik.
\footnote{\href{http://go.yurichev.com/17119}{wikipedia.org/wiki/Call\_Stack}}.
\ac{AKA} \ac{LIFO}.

Technisch betrachtet ist es ein Stapelspeicher innerhalb des Prozessspeichers der zusammen mit den \ESP (x86), \RSP (x64) oder dem \ac{SP} (ARM) Register als ein Zeiger in diesem Speicherblock fungiert.

\myindex{ARM!\Instructions!PUSH}
\myindex{ARM!\Instructions!POP}
\myindex{x86!\Instructions!PUSH}
\myindex{x86!\Instructions!POP}

Die häufigsten Stack-Zugriffsinstruktionen sind die \PUSH- und \POP-Instruktionen (in beidem x86 und ARM Thumb-Modus). \PUSH subtrahiert vom \ESP/\RSP/\ac{SP} 4 Byte im 32-Bit Modus (oder 8 im 64-Bit Modus) und schreibt dann den Inhalt des Zeigers an die Adresse auf die von \ESP/\RSP/\ac{SP} gezeigt wird.

\POP ist die umgekehrte Operation: Die Daten des Zeigers für die Speicherregion auf die von \ac{SP}
gezeigt wird werden ausgelesen und die Inhalte in den Instruktionsoperanden geschreiben (oft ist das ein Register). Dann werden 4 (beziehungsweise 8) Byte zum \gls{stack pointer} addiert.

Nach der Stackallokation, zeigt der \gls{stack pointer} auf den Boden des Stacks.
\PUSH verringert den \gls{stack pointer} und \POP erhöht ihn.
Der Boden des Stacks ist eigentlich der Anfang der Speicherregion die für den Stack reserviert wurde.
Das wirkt zunächst seltsam, aber so funktioniert es.

ARM unterstützt beides, aufsteigende und absteigende Stacks.

\myindex{ARM!\Instructions!STMFD}
\myindex{ARM!\Instructions!LDMFD}
\myindex{ARM!\Instructions!STMED}
\myindex{ARM!\Instructions!LDMED}
\myindex{ARM!\Instructions!STMFA}
\myindex{ARM!\Instructions!LDMFA}
\myindex{ARM!\Instructions!STMEA}
\myindex{ARM!\Instructions!LDMEA}

Zum Beispiel die \ac{STMFD}/\ac{LDMFD} und \ac{STMED}/\ac{LDMED} Instruktionen sind alle dafür gedacht mit einem absteigendem Stack zu arbeiten ( wächst nach unten, fängt mit hohen Adressen an und entwickelt sich zu niedrigeren Adressen). Die \ac{STMFA}/\ac{LDMFA} und \ac{STMEA}/\ac{LDMEA} Instruktionen sind dazu gedacht mit einem aufsteigendem Stack zu arbeiten (wächst nach oben und fängt mit niedrigeren Adressen an und wächst nach oben).

% It might be worth mentioning that STMED and STMEA write first,
% and then move the pointer, and that LDMED and LDMEA move the pointer first, and then read.
% In other words, ARM not only lets the stack grow in a non-standard direction,
% but also in a non-standard order.
% Maybe this can be in the glossary, which would explain why E stands for "empty".

\subsection{Warum wächst der Stack nach unten?}
\label{stack_grow_backwards}

Intuitiv, würden man annehmen das der Stack nach oben wächst z.B Richtung höherer Adressen, so wie bei jeder anderen Datenstruktur.

Der Grund das der Stack rückwärts wächst ist wohl historisch bedingt. Als Computer so groß waren das sie einen ganzen Raum beansprucht haben war es einfach Speicher in zwei Sektionen zu unterteilen, einen Teil für den \gls{heap} und einen Teil für den Stack. Sicher war zu dieser Zeit nicht bekannt wie groß der \gls{heap} und der Stack wachsen würden, während der Programm Laufzeit, also war die Lösung die einfachste mögliche.

\input{patterns/02_stack/stack_and_heap}

In \RitchieThompsonUNIX können wir folgendes lesen:

\begin{framed}
\begin{quotation}
Der user-core eines Programm Images wird in drei logische Segmente unterteilt. Das Programm-Text Segment beginnt bei 0 im virtuellen Adress Speicher. Während der Ausführung wird das Segment als schreibgeschützt markiert und eine einzelne Kopie des Segments wird unter allen Prozessen geteilt die das Programm ausführen. An der ersten 8K grenze über dem Programm Text Segment im Virtuellen Speicher, fängt der ``nonshared'' Bereich an, der nach Bedarf von Syscalls erweitert werden kann. Beginnend bei der höchsten Adresse im Virtuellen Speicher ist das Stack Segment, das Automatisch nach unten wächst während der Hardware Stackpointer sich ändert.
\end{quotation}
\end{framed}

Das erinnert daran wie manche Schüler Notizen zu  zwei Vorträgen in einem Notebook dokumentieren:
Notizen für den ersten Vortrag werden normal notiert, und Notizen zur zum zweiten Vortrag werden 
ans Ende des Notizbuches geschrieben, indem man das Notizbuch umdreht. Die Notizen treffen sich irgendwann
im Notizbuch aufgrund des fehlenden Freien Platzes.

% I think if we want to expand on this analogy,
% one might remember that the line number increases as as you go down a page.
% So when you decrease the address when pushing to the stack, visually,
% the stack does grow upwards.
% Of course, the problem is that in most human languages,
% just as with computers,
% we write downwards, so this direction is what makes buffer overflows so messy.

\subsection{Für was wird der Stack benutzt?}

% subsections
\EN{\input{patterns/02_stack/01_saving_ret_addr_EN}}
\RU{\input{patterns/02_stack/01_saving_ret_addr_RU}}
\DE{\input{patterns/02_stack/01_saving_ret_addr_DE}}
\FR{\input{patterns/02_stack/01_saving_ret_addr_FR}}
\PTBR{\input{patterns/02_stack/01_saving_ret_addr_PTBR}}
\IT{\input{patterns/02_stack/01_saving_ret_addr_IT}}
\PL{\input{patterns/02_stack/01_saving_ret_addr_PL}}
\JA{\input{patterns/02_stack/01_saving_ret_addr_JA}}

\EN{\input{patterns/02_stack/02_args_passing_EN}}
\RU{\input{patterns/02_stack/02_args_passing_RU}}
\PTBR{\input{patterns/02_stack/02_args_passing_PTBR}}
\DE{\input{patterns/02_stack/02_args_passing_DE}}
\IT{\input{patterns/02_stack/02_args_passing_IT}}
\FR{\input{patterns/02_stack/02_args_passing_FR}}
\JA{\input{patterns/02_stack/02_args_passing_JA}}
\PL{\input{patterns/02_stack/02_args_passing_PL}}


\EN{\input{patterns/02_stack/03_local_vars_EN}}
\RU{\input{patterns/02_stack/03_local_vars_RU}}
\DE{\input{patterns/02_stack/03_local_vars_DE}}
\PTBR{\input{patterns/02_stack/03_local_vars_PTBR}}
\EN{\input{patterns/02_stack/04_alloca/main_EN}}
\FR{\input{patterns/02_stack/04_alloca/main_FR}}
\RU{\input{patterns/02_stack/04_alloca/main_RU}}
\PTBR{\input{patterns/02_stack/04_alloca/main_PTBR}}
\IT{\input{patterns/02_stack/04_alloca/main_IT}}
\DE{\input{patterns/02_stack/04_alloca/main_DE}}
\PL{\input{patterns/02_stack/04_alloca/main_PL}}
\JA{\input{patterns/02_stack/04_alloca/main_JA}}

\subsubsection{(Windows) SEH}
\myindex{Windows!Structured Exception Handling}

\ifdefined\RUSSIAN
В стеке хранятся записи \ac{SEH} для функции (если они присутствуют).
Читайте больше о нем здесь: (\myref{sec:SEH}).
\fi % RUSSIAN

\ifdefined\ENGLISH
\ac{SEH} records are also stored on the stack (if they are present).
Read more about it: (\myref{sec:SEH}).
\fi % ENGLISH

\ifdefined\BRAZILIAN
\ac{SEH} também são guardados na pilha (se estiverem presentes).
\PTBRph{}: (\myref{sec:SEH}).
\fi % BRAZILIAN

\ifdefined\ITALIAN
I record \ac{SEH}, se presenti, sono anch'essi memorizzati nello stack.
Maggiori informazioni qui: (\myref{sec:SEH}).
\fi % ITALIAN

\ifdefined\FRENCH
Les enregistrements \ac{SEH} sont aussi stockés dans la pile (s'ils sont présents).
Lire à ce propos: (\myref{sec:SEH}).
\fi % FRENCH


\ifdefined\POLISH
Na stosie są przechowywane wpisy \ac{SEH} dla funkcji (jeśli są one obecne).
Więcej o tym tutaj: (\myref{sec:SEH}).
\fi % POLISH

\ifdefined\JAPANESE
\ac{SEH}レコードはスタックにも格納されます(存在する場合)。
それについてもっと読む:(\myref{sec:SEH})
\fi % JAPANESE

\ifdefined\ENGLISH
\subsubsection{Buffer overflow protection}

More about it here~(\myref{subsec:bufferoverflow}).
\fi

\ifdefined\RUSSIAN
\subsubsection{Защита от переполнений буфера}

Здесь больше об этом~(\myref{subsec:bufferoverflow}).
\fi

\ifdefined\BRAZILIAN
\subsubsection{Proteção contra estouro de buffer}

Mais sobre aqui~(\myref{subsec:bufferoverflow}).
\fi

\ifdefined\ITALIAN
\subsubsection{Protezione da buffer overflow}

Maggiori informazioni qui~(\myref{subsec:bufferoverflow}).
\fi

\ifdefined\FRENCH
\subsubsection{Protection contre les débordements de tampon}

Lire à ce propos~(\myref{subsec:bufferoverflow}).
\fi


\ifdefined\POLISH
\subsubsection{Metody zabiezpieczenia przed przepełnieniem stosu}

Więcej o tym tutaj~(\myref{subsec:bufferoverflow}).
\fi

\ifdefined\JAPANESE
\subsubsection{バッファオーバーフロー保護}

詳細はこちら~(\myref{subsec:bufferoverflow})
\fi

\subsubsection{Automatisches deallokieren der Daten auf dem Stack}

Vielleicht ist der Grund warum man lokale Variablen und SEH Einträge auf dem Stack speichert, weil sie beim 
verlassen der Funktion automatisch aufgeräumt werden. Man braucht dabei nur eine Instruktion um die Position
des Stackpointers zu korrigieren (oftmals ist es die \ADD Instruktion). Funktions Argumente, könnte man sagen 
werden auch am Ende der Funktion deallokiert. Im Kontrast dazu, alles was auf dem \emph{heap} gespeichert wird muss
explizit deallokiert werden. 

% sections
\EN{\input{patterns/02_stack/07_layout_EN}}
\RU{\input{patterns/02_stack/07_layout_RU}}
\DE{\input{patterns/02_stack/07_layout_DE}}
\PTBR{\input{patterns/02_stack/07_layout_PTBR}}
\EN{\input{patterns/02_stack/08_noise/main_EN}}
\FR{\input{patterns/02_stack/08_noise/main_FR}}
\RU{\input{patterns/02_stack/08_noise/main_RU}}
\IT{\input{patterns/02_stack/08_noise/main_IT}}
\DE{\input{patterns/02_stack/08_noise/main_DE}}
\PL{\input{patterns/02_stack/08_noise/main_PL}}
\JA{\input{patterns/02_stack/08_noise/main_JA}}

\input{patterns/02_stack/exercises}
}
\FR{\subsection{Convertir une valeur 32-bit en 64-bit}
\label{subsec:sign_extending_32_to_64}

\lstinputlisting[style=customc]{patterns/185_64bit_in_32_env/conversion/4.c}

\subsubsection{x86}

\lstinputlisting[caption=MSVC 2012 \Optimizing,style=customasmx86]{patterns/185_64bit_in_32_env/conversion/MSVC2012_Ox.asm}

Ici, nous nous heurtons à la nécessité d'étendre une valeur 32-bit signée en une
64-bit signée.
Les valeurs non signées sont converties directement: tous les bits de la partie haute
doivent être mis à 0.
Mais ce n'est pas approprié pour les types de donnée signée: le signe doit être copié
dans la partie haute du nombre résultant.
\myindex{x86!\Instructions!CDQ}

L'instruction \INS{CDQ} fait cela ici, elle prend sa valeur d'entrée dans \EAX{},
étend le signe sur 64-bit et laisse le résultat dans la paire de registres \EDX{}:\EAX{}.
En d'autres mots, \INS{CDQ} prend le signe du nombre dans \EAX{} (en prenant le bit
le plus significatif dans \EAX{}), et suivant sa valeur, met tous les 32 bits de
\EDX{} à 0 ou 1.
Cette opération est quelque peu similaire à l'instruction \MOVSX{}.

\subsubsection{ARM}

\lstinputlisting[caption=\OptimizingKeilVI (\ARMMode),style=customasmARM]{patterns/185_64bit_in_32_env/conversion/Keil_ARM_O3.s}

Keil pour ARM est différent: il décale simplement arithmétiquement de 31 bits vers
la droite la valeur en entrée.
Comme nous le savons, le bit de signe est le \ac{MSB}, et le décalage arithmétique
copie le bit de signe dans les bits \q{apparus}.
Donc après \q{ASR r1,r0,\#31}, \Reg{1} contient 0xFFFFFFFF si la valeur d'entrée
était négative et 0 sinon.
\Reg{1} contient la partie haute de la valeur 64-bit résultante.
En d'autres mots, ce code copie juste le \ac{MSB} (bit de signe) de la valeur d'entrée
dans \Reg{0} dans tous les bits de la partie haute 32-bit de la valeur 64-bit résultante.

\subsubsection{MIPS}

GCC pour MIPS fait la même chose que Keil a fait pour le mode ARM:

\lstinputlisting[caption=GCC 4.4.5 \Optimizing (IDA),style=customasmMIPS]{patterns/185_64bit_in_32_env/conversion/MIPS_O3_IDA.lst}
}
\JA{\subsection{32ビット値から64ビット値への変換}
\label{subsec:sign_extending_32_to_64}

\lstinputlisting[style=customc]{patterns/185_64bit_in_32_env/conversion/4.c}

\subsubsection{x86}

\lstinputlisting[caption=\Optimizing MSVC 2012,style=customasmx86]{patterns/185_64bit_in_32_env/conversion/MSVC2012_Ox.asm}

ここでも、32ビットの符号付き値を64ビットの符号付き値に拡張する必要があります。 
符号なしの値は単純に変換されます:上位部分のすべてのビットは0に設定する必要があります。
ただし、符号付きデータ型には適していません:符号は結果の数値の上位部分にコピーする必要があります。
\myindex{x86!\Instructions!CDQ}

\INS{CDQ}命令はここでそれを行います。\EAX{}でその入力値を取り、それを64ビットに拡張しそして
\EDX{}:\EAX{}レジスタペアに残します。
つまり、\INS{CDQ}は(EAXの最上位ビットを取得することによって)\EAX{}から番号記号を取得し、
それに応じて\EDX{}の32ビットすべてを0または1に設定します。
その動作は、\MOVSX{}命令とやや似ています。

\subsubsection{ARM}

\lstinputlisting[caption=\OptimizingKeilVI (\ARMMode),style=customasmARM]{patterns/185_64bit_in_32_env/conversion/Keil_ARM_O3.s}

ARM用Keilは異なります。入力値を算術的に右に31ビットシフトします。 
知っての通り、符号ビットは\ac{MSB}で、算術シフトは符号ビットを\q{出現した}ビットにコピーします。 
したがって、\q{ASR r1,r0,\#31}の後、入力値が負の場合は\Reg{1}に0xFFFFFFFFが含まれ、それ以外の場合は0が含まれます。 
\Reg{1}には、結果の64ビット値の上位部分が含まれています。 
つまり、このコードは\Reg{0} の入力値から結果の64ビット値の上位32ビット部分のすべてのビットに\ac{MSB}(符号ビット)をコピーするだけです。

\subsubsection{MIPS}

MIPS向けのGCCは、KeilがARMモードで行ったのと同じことを行います。

\lstinputlisting[caption=\Optimizing GCC 4.4.5 (IDA),style=customasmMIPS]{patterns/185_64bit_in_32_env/conversion/MIPS_O3_IDA.lst}
}


