\mysection{Saut conditionnels}
\label{sec:Jcc}
\myindex{\CLanguageElements!if}

% sections
\subsection{\RU{Простой пример}\EN{Simple example}\DE{einfaches Beispiel}
\FR{Exemple simple}\IT{Esempio semplice}\JA{シンプルな例}
}

\lstinputlisting[style=customc]{patterns/07_jcc/simple/ex.c}

% subsections
\EN{\subsubsection{x86}

\myparagraph{x86 + MSVC}

Here is how the \TT{f\_signed()} function looks like:

\lstinputlisting[caption=\NonOptimizing MSVC 2010,style=customasmx86]{patterns/07_jcc/simple/signed_MSVC.asm}

\myindex{x86!\Instructions!JLE}

The first instruction, \JLE, stands for \emph{Jump if Less or Equal}. 
In other words, if the second operand is 
larger or equal to the first one, the control flow will be passed to the address or label specified in the instruction.
If this condition does not trigger because the second operand is smaller than the first one, the control flow would not be altered and the first \printf would be executed.
\myindex{x86!\Instructions!JNE}
The second check is \JNE: \emph{Jump if Not Equal}.
The control flow will not change if the operands are equal.

\myindex{x86!\Instructions!JGE}
The third check is \JGE: \emph{Jump if Greater or Equal}---jump if the first operand is larger than 
the second or if they are equal.
So, if all three conditional jumps are triggered, none of the \printf calls would be executed whatsoever. 
This is impossible without special intervention.
Now let's take a look at the \TT{f\_unsigned()} function.
The \TT{f\_unsigned()} function is the same as \TT{f\_signed()}, with the exception that the \JBE and \JAE instructions
are used instead of \JLE and \JGE, as follows:

\lstinputlisting[caption=GCC,style=customasmx86]{patterns/07_jcc/simple/unsigned_MSVC.asm}

\myindex{x86!\Instructions!JBE}
\myindex{x86!\Instructions!JAE}

As already mentioned, the branch instructions are different:
\JBE---\emph{Jump if Below or Equal} and \JAE---\emph{Jump if Above or Equal}.
These instructions (\INS{JA}/\JAE/\JB/\JBE) differ from \JG/\JGE/\JL/\JLE in the fact that they work with unsigned numbers.

\myindex{x86!\Instructions!JA}
\myindex{x86!\Instructions!JB}
\myindex{x86!\Instructions!JG}
\myindex{x86!\Instructions!JL}
\myindex{Signed numbers}

See also the section about signed number representations~(\myref{sec:signednumbers}).
That is why if we see \JG/\JL in use instead of \INS{JA}/\JB or vice-versa, 
we can be almost sure that the variables are signed or unsigned, respectively.
Here is also the \main function, where there is nothing much new to us:

\lstinputlisting[caption=\main,style=customasmx86]{patterns/07_jcc/simple/main_MSVC.asm}

\input{patterns/07_jcc/simple/olly_EN.tex}

\clearpage
\myparagraph{x86 + MSVC + Hiew}
\myindex{Hiew}

We can try to patch the executable file in a way 
that the \TT{f\_unsigned()} function would always print \q{a==b}, 
no matter the input values.
Here is how it looks in Hiew:

\begin{figure}[H]
\centering
\myincludegraphics{patterns/07_jcc/simple/hiew_unsigned1.png}
\caption{Hiew: \TT{f\_unsigned()} function}
\label{fig:jcc_hiew_1}
\end{figure}

Essentially, we have to accomplish three tasks:
\begin{itemize}
\item force the first jump to always trigger;
\item force the second jump to never trigger;
\item force the third jump to always trigger.
\end{itemize}

Thus we can direct the code flow to always pass through the second \printf, and output \q{a==b}.

Three instructions (or bytes) has to be patched:

\begin{itemize}
\item The first jump becomes \JMP, but the \gls{jump offset} would remain the same.

\item 
The second jump might be triggered sometimes, but in any case it will jump to the next
instruction, because, we set the \gls{jump offset} to 0.

In these instructions the \gls{jump offset} is added to the address for the next instruction.
So if the offset is 0,
the jump will transfer the control to the next instruction.

\item 
The third jump we replace with \JMP just as we do with the first one, so it will always trigger.

\end{itemize}

\clearpage
Here is the modified code:

\begin{figure}[H]
\centering
\myincludegraphics{patterns/07_jcc/simple/hiew_unsigned2.png}
\caption{Hiew: let's modify the \TT{f\_unsigned()} function}
\label{fig:jcc_hiew_2}
\end{figure}

If we miss to change any of these jumps, then several \printf calls may execute, while we want to execute only one.

\myparagraph{\NonOptimizing GCC}

\myindex{puts() instead of printf()}
\NonOptimizing GCC 4.4.1 
produces almost the same code, but with \puts~(\myref{puts}) instead of \printf.

\myparagraph{\Optimizing GCC}

An observant reader may ask, why execute \CMP several times, 
if the flags has the same values after each execution?

Perhaps optimizing MSVC cannot do this, but optimizing GCC 4.8.1 can go deeper:

\lstinputlisting[caption=GCC 4.8.1 f\_signed(),style=customasmx86]{patterns/07_jcc/simple/GCC_O3_signed.asm}

% should be here instead of 'switch' section?
We also see \TT{JMP puts} here instead of \TT{CALL puts / RETN}.

This kind of trick will have explained later: \myref{JMP_instead_of_RET}.

This type of x86 code 
is somewhat rare.
MSVC 2012 as it seems, can't generate such code.
On the other hand, assembly language programmers are fully aware of the fact that \TT{Jcc} 
instructions can be stacked.

So if you see such stacking somewhere, it is highly probable that the code was hand-written.

The \TT{f\_unsigned()} function is not that 
\ae{}sthetically short:

\lstinputlisting[caption=GCC 4.8.1 f\_unsigned(),style=customasmx86]{patterns/07_jcc/simple/GCC_O3_unsigned_EN.asm}

Nevertheless, there are two \TT{CMP} instructions instead of three.

So optimization algorithms of GCC 4.8.1 are probably not perfect yet. 
}
\RU{\subsubsection{x86}

\myparagraph{x86 + MSVC}

Имеем в итоге функцию \TT{f\_signed()}:

\lstinputlisting[caption=\NonOptimizing MSVC 2010,style=customasmx86]{patterns/07_jcc/simple/signed_MSVC.asm}

\myindex{x86!\Instructions!JLE}
Первая инструкция \JLE значит \emph{Jump if Less or Equal}. 
Если второй операнд больше первого или равен ему, произойдет переход туда, где будет следующая проверка.

А если это условие не срабатывает (то есть второй операнд меньше первого), то перехода не будет, 
и сработает первый \printf.

\myindex{x86!\Instructions!JNE}
Вторая проверка это \JNE: \emph{Jump if Not Equal}.
Переход не произойдет, если операнды равны.

\myindex{x86!\Instructions!JGE}
Третья проверка \JGE: \emph{Jump if Greater or Equal} --- переход 
если первый операнд больше второго или равен ему.
Кстати, если все три условных перехода сработают, ни один \printf не вызовется. 
Но без внешнего вмешательства это невозможно.

Функция \TT{f\_unsigned()} точно такая же, за тем исключением, что используются инструкции 
\JBE и \JAE вместо \JLE и \JGE:

\lstinputlisting[caption=GCC,style=customasmx86]{patterns/07_jcc/simple/unsigned_MSVC.asm}

\myindex{x86!\Instructions!JBE}
\myindex{x86!\Instructions!JAE}
Здесь всё то же самое, только инструкции условных переходов немного другие:

\JBE --- \emph{Jump if Below or Equal} и \JAE --- \emph{Jump if Above or Equal}.
Эти инструкции (\INS{JA}/\JAE/\JB/\JBE) 
отличаются от \JG/\JGE/\JL/\JLE тем, что работают с беззнаковыми переменными.

\myindex{x86!\Instructions!JA}
\myindex{x86!\Instructions!JB}
\myindex{x86!\Instructions!JG}
\myindex{x86!\Instructions!JL}
\myindex{Signed numbers}
Отступление: смотрите также секцию о представлении знака в числах ~(\myref{sec:signednumbers}).
Таким образом, увидев где используется \JG/\JL вместо \INS{JA}/\JB и наоборот, 
можно сказать почти уверенно насчет того, 
является ли тип переменной знаковым (signed) или беззнаковым (unsigned).

Далее функция \main, где ничего нового для нас нет:

\lstinputlisting[caption=\main,style=customasmx86]{patterns/07_jcc/simple/main_MSVC.asm}

\input{patterns/07_jcc/simple/olly_RU.tex}

\clearpage
\myparagraph{x86 + MSVC + Hiew}
\myindex{Hiew}

Можем попробовать модифицировать исполняемый файл так, чтобы функция \TT{f\_unsigned()} всегда показывала \q{a==b},
при любых входящих значениях.
Вот как она выглядит в Hiew:

\begin{figure}[H]
\centering
\myincludegraphics{patterns/07_jcc/simple/hiew_unsigned1.png}
\caption{Hiew: функция \TT{f\_unsigned()}}
\label{fig:jcc_hiew_1}
\end{figure}

Собственно, задач три:
\begin{itemize}
\item заставить первый переход срабатывать всегда;
\item заставить второй переход не срабатывать никогда;
\item заставить третий переход срабатывать всегда.
\end{itemize}

Так мы направим путь исполнения кода (code flow) во второй \printf,
и он всегда будет срабатывать и выводить на консоль \q{a==b}.

Для этого нужно изменить три инструкции (или байта):

\begin{itemize}
\item Первый переход теперь будет \JMP, но смещение перехода 
(\gls{jump offset}) останется прежним.

\item Второй переход может быть и будет срабатывать иногда, но в любом случае он будет совершать переход
только на следующую инструкцию, потому что мы выставляем смещение перехода (\gls{jump offset}) в 0.

В этих инструкциях смещение перехода просто прибавляется к адресу следующей инструкции.

Когда смещение 0, переход будет на следующую инструкцию.

\item Третий переход конвертируем в \JMP точно так же, как и первый, он будет срабатывать всегда.

\end{itemize}

\clearpage
Что и делаем:

\begin{figure}[H]
\centering
\myincludegraphics{patterns/07_jcc/simple/hiew_unsigned2.png}
\caption{Hiew: модифицируем функцию \TT{f\_unsigned()}}
\label{fig:jcc_hiew_2}
\end{figure}

Если забыть про какой-то из переходов, то тогда будет срабатывать несколько вызовов \printf, 
а нам ведь нужно чтобы исполнялся только один.

\myparagraph{\NonOptimizing GCC}

\myindex{puts() вместо printf()}
\NonOptimizing GCC 4.4.1 производит почти такой же код, за исключением \puts~(\myref{puts}) вместо \printf.

\myparagraph{\Optimizing GCC}

Наблюдательный читатель может спросить, зачем исполнять \CMP так много раз,
если флаги всегда одни и те же?
По-видимому, оптимизирующий MSVC не может этого делать, но GCC 4.8.1 делает больше оптимизаций:

\lstinputlisting[caption=GCC 4.8.1 f\_signed(),style=customasmx86]{patterns/07_jcc/simple/GCC_O3_signed.asm}

% should be here instead of 'switch' section?
Мы здесь также видим \TT{JMP puts} вместо \TT{CALL puts / RETN}.
Этот прием описан немного позже: \myref{JMP_instead_of_RET}.

Нужно сказать, что x86-код такого типа редок.
MSVC 2012, как видно, не может генерировать подобное.
С другой стороны, программисты на ассемблере прекрасно осведомлены о том,
что инструкции \TT{Jcc} можно располагать последовательно.

Так что если вы видите это где-то, имеется немалая вероятность, что этот фрагмент кода был написан вручную.

Функция \TT{f\_unsigned()} получилась не настолько эстетически короткой:

\lstinputlisting[caption=GCC 4.8.1 f\_unsigned(),style=customasmx86]{patterns/07_jcc/simple/GCC_O3_unsigned_RU.asm}

Тем не менее, здесь 2 инструкции \TT{CMP} вместо трех.

Так что, алгоритмы оптимизации GCC 4.8.1, наверное, ещё пока не идеальны.
}
\DE{\subsubsection{x86}

\myparagraph{x86 + MSVC}

Die Funktions \TT{f\_signed()} sieht folgendermaßen aus:

\lstinputlisting[caption=\NonOptimizing MSVC 2010,style=customasmx86]{patterns/07_jcc/simple/signed_MSVC.asm}

\myindex{x86!\Instructions!JLE}
Der erste Befehl, \JLE steht für \emph{Jump if Less or Equal}.
Mit anderen Worten, wenn der zweite Operand größer gleich dem ersten ist, wird der Control Flow an die angegebene
Adresse bzw. das angegebene Label übergeben.
Wenn die Bedingung falsch ist, weil der zweite Operand kleiner ist als der erste, wird der Control Flow nicht verändert
und das erste \printf wird ausgeführt.

Tmyindex{x86!\Instructions!JNE}
Der zweite Check ist \JNE: \emph{Jump if Not Equal}.
Der Control Flow wird nicht verändert, wenn die Operanden gleich sind.

\myindex{x86!\Instructions!JGE}
Der dritte Check ist \JGE: \emph{Jump if Greater or Equal}---springe, falls der erste Operand größer gleich dem zweiten
ist.
Wenn also alle drei bedingten Sprünge ausgeführt werden, wird also kein Aufruf von \printf ausgeführt.
Dies ist ohne manuellen Eingriff unmöglich.
Werfen wir nun einen Blick auf die Funktion \TT{f\_unsigned()}.
Diese Funktion macht prinzipiell das gleiche wie \TT{f\_signed()} mit der Ausnahme, dass die Befehle \JBE und \JAE
anstelle von \JLE und \JGE wie folgt verwendet werden:

\lstinputlisting[caption=GCC,style=customasmx86]{patterns/07_jcc/simple/unsigned_MSVC.asm}

\myindex{x86!\Instructions!JBE}
\myindex{x86!\Instructions!JAE}
Wie bereits erwähnt unterscheiden sich die Verzweigungsbefehle:
\JBE---\emph{Jump if Below or Equal} und \JAE---\emph{Jump if Above or Equal}.
Diese Befehle (\INS{JA}/\JAE/\JB/\JBE) unterscheiden sich von \JG/\JGE/\JL/\JLE dadurch, dass sie mit vorzeichenlosen Zahlen
arbeiten.

\myindex{x86!\Instructions!JA}
\myindex{x86!\Instructions!JB}
\myindex{x86!\Instructions!JG}
\myindex{x86!\Instructions!JL}
\myindex{Signed numbers}
Siehe hierzu auch den Abschnitt über Darstellung vorzeichenbehafteter Zahlen~(\myref{sec:signednumbers}).
Das ist der Grund warum wir, wenn wir \JG/\JL anstelle von \INS{JA}/\JB und umgekehrt finden, fast mit Gewissheit sagen
können, dass die Variablen vorzeichenbehaftet bzw. vorzeichenlos sind.
Hier befindet sich auch die Funktion \main, welche für uns nichts Neues bereithält:

\lstinputlisting[caption=\main,style=customasmx86]{patterns/07_jcc/simple/main_MSVC.asm}

\input{patterns/07_jcc/simple/olly_EN.tex}

\clearpage
\myparagraph{x86 + MSVC + Hiew}
\myindex{Hiew}
Wir können versuchen, die Executable so zu verändern, dass die Funktion \TT{f\_unsigned()} stets \q{a==b} ausgibt, egal
was wir eingben.
So sieht das ganze in Hiew aus:
\begin{figure}[H]
\centering
\myincludegraphics{patterns/07_jcc/simple/hiew_unsigned1.png}
\caption{Hiew: Funktion \TT{f\_unsigned()}}
\label{fig:jcc_hiew_1}
\end{figure}
Grundsätzlich haben wir drei Dinge zu erzwingen:
\begin{itemize}
  \item den ersten Sprung stets ausführen;
  \item den zweiten Sprung nie ausführen;
  \item den dritten Sprung stets ausführen.
\end{itemize}

Dadurch können wir den Code Flow so manupulieren, dass das zweite \printf immer ausgeführt wird und \q{a==b} ausgibt.
Drei Befehle (oder Bytes) müssen verändert werden:
\begin{itemize}
\item Der erste Sprung wird zu \JMP verändert, aber der \gls{jump offset} bleibt gleich.

 
\item Der zweite Sprung könnte manchmal ausgeführt werden, wird aber in jedem Fall zum nächsten Befehl springen, denn
wir setzen den \gls{jump offset} auf 0.
Bei diesen Befehlen wird der \gls{jump offset} zu der Adresse der nächsten Befehls addiert. Wenn der Offset 0 ist, wird
die Ausführung also beim nächsten Befehl fortgesetzt.
\item 
Den dritten Sprung ersetzen wie genau wie den ersten durch \JMP, damit er stets ausgeführt wird.

\end{itemize}

\clearpage
Hier ist der veränderte Code:

\begin{figure}[H]
\centering
\myincludegraphics{patterns/07_jcc/simple/hiew_unsigned2.png}
\caption{Hiew: Veränderte Funktion \TT{f\_unsigned()}}
\label{fig:jcc_hiew_2}
\end{figure}
Wenn wir es verpassen, einen dieser Sprünge zu verändern, könnten mehrere Aufrufe von \printf ausgeführt werden; wir
wollen aber nur genau einen Aufruf ausführen.

\myparagraph{\NonOptimizing GCC}

\myindex{puts() anstelle von printf()}
\NonOptimizing GCC 4.4.1 
erzeugt fast identischen Code, aber mit \puts~(\myref{puts}) anstelle von \printf.

\myparagraph{\Optimizing GCC}
Der aufmerksame Leser fragt sich vielleicht, warum \CMP mehrmals ausgeführt wird, wenn doch die Flags nach jeder
Ausführung dieselben Werte haben. 

Vielleicht kann der optimierende MSVC dies nicht leisten, aber der optimierende GCC 4.8.1 gräbt tiefer:

\lstinputlisting[caption=GCC 4.8.1 f\_signed(),style=customasmx86]{patterns/07_jcc/simple/GCC_O3_signed.asm}

% should be here instead of 'switch' section?
Wir finden auch hier \TT{JMP puts} anstelle von \TT{CALL puts / RETN}.

Dieser Trick wird später erklärt:\myref{JMP_instead_of_RET}.

Diese Sorte x86 Code ist trotzdem selten. MSVC 2012 kann wie es scheint solchen Code nicht erzeugen.
Andererseits sind Assemblerprogrammierer sich natürlich der Tatsache bewusst, dass \TT{Jcc} Befehle gestackt werden
können.
Wenn man solche gestackten Befehle findet, ist es sehr wahrscheinlich, dass der entsprechende Code von Hand geschrieben
wurde. 
Die Funktion \TT{f\_unsigned()} ist nicht so ästhetisch:


\lstinputlisting[caption=GCC 4.8.1 f\_unsigned(),style=customasmx86]{patterns/07_jcc/simple/GCC_O3_unsigned_DE.asm}
Trotzdem werden hier immerhin nur zwei statt drei \TT{CMP} Befehle verwendet.

Die Optimierungsalgorithmen von GCC 4.8.1 sind möglicherweise noch nicht so ausgereift.
}
\FR{\subsubsection{x86}

\myparagraph{x86 + MSVC}

Voici à quoi ressemble la fonction  \TT{f\_signed()}:

\lstinputlisting[caption=MSVC 2010 \NonOptimizing,style=customasmx86]{patterns/07_jcc/simple/signed_MSVC.asm}

\myindex{x86!\Instructions!JLE}

La première instruction, \JLE, représente \emph{Jump if Less or Equal} (saut si inférieur ou égal).
En d'autres mots, si le deuxième opérande est plus grand ou égal au premier,
le flux d'exécution est passé à l'adresse ou au label spécifié dans l'instruction.
Si la condition ne déclenche pas le saut, car le second opérande est plus petit que
le premier, le flux d'exécution ne sera pas altéré et le premier \printf sera
exécuté.
\myindex{x86!\Instructions!JNE}
Le second test est \JNE: \emph{Jump if Not Equal} (saut si non égal).
Le flux d'exécution ne changera pas si les opérandes sont égaux.

\myindex{x86!\Instructions!JGE}
Le troisième test est \JGE: \emph{Jump if Greater or Equal}---saute si le premier
opérande est supérieur ou égal au deuxième.
Donc, si les trois sauts conditionnels sont effectués, aucun des appels à \printf
ne sera exécuté.
Ceci est impossible sans intervention spéciale.
Regardons maintenant la fonction \TT{f\_unsigned()}.
La fonction \TT{f\_unsigned()} est la même que \TT{f\_signed()}, à la différence
que les instructions \JBE et \JAE sont utilisées à la place de \JLE et \JGE, comme
suit:

\lstinputlisting[caption=GCC,style=customasmx86]{patterns/07_jcc/simple/unsigned_MSVC.asm}

\myindex{x86!\Instructions!JBE}
\myindex{x86!\Instructions!JAE}

Comme déjà mentionné, les instructions de branchement sont différentes:
\JBE---\emph{Jump if Below or Equal} (saut si inférieur ou égal) et \JAE---\emph{Jump if Above or Equal}
(saut si supérieur ou égal).
Ces instructions (\INS{JA}/\JAE/\JB/\JBE) diffèrent de \JG/\JGE/\JL/\JLE par le fait qu'elles
travaillent avec des nombres non signés.

\myindex{x86!\Instructions!JA}
\myindex{x86!\Instructions!JB}
\myindex{x86!\Instructions!JG}
\myindex{x86!\Instructions!JL}
\myindex{Signed numbers}

Voir aussi la section sur la représentation des nombres signés~(\myref{sec:signednumbers}).
C'est pourquoi si nous voyons que \JG/\JL sont utilisés à la place de \INS{JA}/\JB ou
vice-versa, nous pouvons être presque sûr que les variables sont signées ou non
signées, respectivement.
Voici la fonction \main, où presque rien n'est nouveau pour nous:

\lstinputlisting[caption=\main,style=customasmx86]{patterns/07_jcc/simple/main_MSVC.asm}

\input{patterns/07_jcc/simple/olly_FR.tex}

\clearpage
\myparagraph{x86 + MSVC + Hiew}
\myindex{Hiew}

Nous pouvons essayer de patcher l'exécutable afin que la fonction \TT{f\_unsigned()}
affiche toujours \q{a==b}, quelque soient les valeurs en entrée.
Voici à quoi ça ressemble dans Hiew:

\begin{figure}[H]
\centering
\myincludegraphics{patterns/07_jcc/simple/hiew_unsigned1.png}
\caption{Hiew: fonction \TT{f\_unsigned()}}
\label{fig:jcc_hiew_1}
\end{figure}

Essentiellement, nous devons accomplir ces trois choses:
\begin{itemize}
\item forcer le premier saut à toujours être effectué;
\item forcer le second saut à n'être jamais effectué;
\item forcer le troisième saut à toujours être effectué.
\end{itemize}

Nous devons donc diriger le déroulement du code pour toujours effectuer le second \printf,
et afficher \q{a==b}.

Trois instructions (ou octets) doivent être modifiées:

\begin{itemize}
\item Le premier saut devient un \JMP, mais l'\glslink{jump offset}{offset} reste
le même.

\item
Le second saut peut être parfois suivi, mais dans chaque cas il sautera à l'instruction
suivante, car nous avons mis l'\glslink{jump offset}{offset} à 0.

Dans cette instruction, l'\glslink{jump offset}{offset} est ajouté à l'adresse
de l'instruction suivante. Donc si l'offset est 0, le saut va transférer l'exécution
à l'instruction suivante.

\item
Le troisième saut est remplacé par \JMP comme nous l'avons fait pour le premier,
il sera donc toujours effectué.

\end{itemize}

\clearpage
Voici le code modifié:

\begin{figure}[H]
\centering
\myincludegraphics{patterns/07_jcc/simple/hiew_unsigned2.png}
\caption{Hiew: modifions la fonction \TT{f\_unsigned()}}
\label{fig:jcc_hiew_2}
\end{figure}

Si nous oublions de modifier une de ces sauts conditionnels, plusieurs appels à \printf
pourraient être faits, alors que nous voulons qu'un seul soit exécuté.

\myparagraph{GCC \NonOptimizing}

\myindex{puts() instead of printf()}
GCC 4.4.1 \NonOptimizing produit presque le même code, mais avec \puts~(\myref{puts})
à la place de \printf.

\myparagraph{GCC \Optimizing}

Le lecteur attentif pourrait demander pourquoi exécuter \CMP plusieurs fois, si
les flags ont les mêmes valeurs après l'exécution ?

Peut-être que l'optimiseur de de MSVC ne peut pas faire cela, mais celui de GCC
4.8.1 peut aller plus loin:

\lstinputlisting[caption=GCC 4.8.1 f\_signed(),style=customasmx86]{patterns/07_jcc/simple/GCC_O3_signed.asm}

% should be here instead of 'switch' section?
Nous voyons ici \TT{JMP puts} au lieu de \TT{CALL puts / RETN}.

Ce genre de truc sera expliqué plus loin: \myref{JMP_instead_of_RET}.

Ce genre de code x86 est plutôt rare.
Il semble que MSVC 2012 ne puisse pas générer un tel code.
D'un autre côté, les programmeurs en langage d'assemblage sont parfaitement conscients
du fait que les instructions \TT{Jcc} peuvent être empilées.

Donc si vous voyez ce genre d'empilement, il est très probable que le code a été
écrit à la main.

La fonction \TT{f\_unsigned()} n'est pas si esthétiquement courte:

\lstinputlisting[caption=GCC 4.8.1 f\_unsigned(),style=customasmx86]{patterns/07_jcc/simple/GCC_O3_unsigned_FR.asm}

Néanmoins, il y a deux instructions \TT{CMP} au lieu de trois.

Donc les algorithmes d'optimisation de GCC 4.8.1 ne sont probablement pas encore parfaits.

}
\IT{\subsubsection{x86}

\myparagraph{x86 + MSVC}

La funzione \TT{f\_signed()} appare così:

\lstinputlisting[caption=\NonOptimizing MSVC 2010,style=customasmx86]{patterns/07_jcc/simple/signed_MSVC.asm}

\myindex{x86!\Instructions!JLE}

La prima istruzione, \JLE, sta per \emph{Jump if Less or Equal} (\emph{salta se è minore o uguale}). 
In altre parole, se il secondo operando è
maggiore o uguale al primo, il flusso di controllo sarà pasato all'indirizzo o alla label specificata nell'istruzione. 
Se questa condizione non è soddisfatta, poiché il secondo operando è più piccolo del primo, il flusso non viene alterato e la prima \printf sarà eseguita.

\myindex{x86!\Instructions!JNE}
Il secondo controllo è \JNE: \emph{Jump if Not Equal}.
Il flusso non cambia se i due operandi sono uguali.

\myindex{x86!\Instructions!JGE}
Il terzo controllo è \JGE: \emph{Jump if Greater or Equal}---salta se il primo operando è maggiore del secondo, o se sono uguali.

Quindi, se tutti i tre salti condizionali vengono innescati, nessuna delle chiamate a \printf sarà eseguita.
Ciò è chiaramente impossibile, almeno senza un intervento speciale.

Diamo ora un'occhiata alla funzione \TT{f\_unsigned()}.
La funzione \TT{f\_unsigned()} è uguale a \TT{f\_signed()}, con l'eccezione che le istruzioni \JBE e \JAE
sono utilizzate al posto di \JLE e \JGE:

\lstinputlisting[caption=GCC,style=customasmx86]{patterns/07_jcc/simple/unsigned_MSVC.asm}

\myindex{x86!\Instructions!JBE}
\myindex{x86!\Instructions!JAE}

Come già detto, le istruzioni di salto (branch instructions) sono diverse:
\JBE---\emph{Jump if Below or Equal} e \JAE---\emph{Jump if Above or Equal}.
Queste istruzioni (\INS{JA}/\JAE/\JB/\JBE) differiscono da \JG/\JGE/\JL/\JLE in quanto operano con numeri senza segno (unsigned).

\myindex{x86!\Instructions!JA}
\myindex{x86!\Instructions!JB}
\myindex{x86!\Instructions!JG}
\myindex{x86!\Instructions!JL}
\myindex{Signed numbers}

Vedi anche la sezione sulle rappresentazioni di numeri con segno (signed) ~(\myref{sec:signednumbers}).

Questo è il motivo per cui se vediamo usare \JG/\JL al posto di \INS{JA}/\JB, o viceversa, 
possiamo essere quasi certi che le variabili sono rispettivamente di tipo signed o unsigned.

Di seguito è riportata anche la funzione \main, dove non c'è niente di nuovo:

\lstinputlisting[caption=\main,style=customasmx86]{patterns/07_jcc/simple/main_MSVC.asm}

\input{patterns/07_jcc/simple/olly_IT}

\clearpage
\myparagraph{x86 + MSVC + Hiew}
\myindex{Hiew}

Possiamo provare a patchare l'eseguibile (applicare una patch) in maniera tale che la funzione \TT{f\_unsigned()} stampi sempre \q{a==b}, 
a prescindere dai valori in input.
In Hiew appare così:

\begin{figure}[H]
\centering
\myincludegraphics{patterns/07_jcc/simple/hiew_unsigned1.png}
\caption{Hiew: \TT{f\_unsigned()} function}
\label{fig:jcc_hiew_1}
\end{figure}

Essenzialmente, per ottenere il risultato desiderato, dobbiamo:
\begin{itemize}
\item forzare il primo jump in modo che sia sempre seguito;
\item forzare il secondo jump a non essere mai seguito;
\item forzare il terzo jump ad essere sempre seguito.
\end{itemize}

Possiamo così diriggere il flusso di esecuzione in modo tale da farlo sempre passare attraverso la seconda \printf, dando in output \q{a==b}.

Devono essere corrette (patchate) tre istruzioni (o byte):

\begin{itemize}
\item Il primo jump diventa \JMP, ma il \gls{jump offset} resta invariato.

\item 
Il secondo jump potrebbe essere innescato in alcune occasioni, ma in ogni caso salterebbe alla prossima istruzione, poiché settiamo il \gls{jump offset} a 0.

In queste istruzioni il \gls{jump offset} viene sommato all'indirizzo della prossima istruzione.
Quindi se l'offset è 0, il jump trasferirà il controllo all'istruzione successiva,

\item 
Possiamo sostituire il terzo jump con \JMP allo stesso modo del primo, in modo che sia sempre innescato.

\end{itemize}

\clearpage
Ecco il codice modificato:

\begin{figure}[H]
\centering
\myincludegraphics{patterns/07_jcc/simple/hiew_unsigned2.png}
\caption{Hiew: funzione \TT{f\_unsigned()} modificata}
\label{fig:jcc_hiew_2}
\end{figure}

Se ci dimentichiamo di cambiare uno di questi jump, potrebbero essere eseguite diverse chiamate a \printf, ma noi vogliamo eseguirne solo una.

\myparagraph{\NonOptimizing GCC}

\myindex{puts() instead of printf()}
\NonOptimizing GCC 4.4.1 
procude pressoché lo stesso codice, ma usa \puts~(\myref{puts}) invece di \printf.

\myparagraph{\Optimizing GCC}

Un lettore attento potrebbe domandare: perchè eseguire \CMP più volte se i flag hanno gli stessi valori dopo ogni esecuzione?

Forse MSVC con ottimizzazioni non è in grado di applicare questa ottimizzazione, al contrario di GCC 4.8.1:

\lstinputlisting[caption=GCC 4.8.1 f\_signed(),style=customasmx86]{patterns/07_jcc/simple/GCC_O3_signed.asm}

% should be here instead of 'switch' section?
Notiamo anche l'uso di \TT{JMP puts} al posto di \TT{CALL puts / RETN}.
Questo trucco sarà spiegato più avanti: \myref{JMP_instead_of_RET}.

Questo tipo di codice x86 è piuttosto raro.
MSVC 2012 apparentemente non è in grado di generarne di simile.
Dall'altro lato, i programmatori assembly sanno perfettamente che le istruzioni \TT{Jcc} possono essere disposte in fila.

Se vedete codice con una disposizione simile, è molto probabile che sia stato scritto a mano.

La funzione \TT{f\_unsigned()} non è esteticamente corta allo stesso modo:

\lstinputlisting[caption=GCC 4.8.1 f\_unsigned(),style=customasmx86]{patterns/07_jcc/simple/GCC_O3_unsigned_EN.asm}

Ciò nonostante, ci sono due istruzioni \TT{CMP} invece di tre.
Gli algoritmi di ottimizzazione di GCC 4.8.1 probabilmente non sono ancora perfetti. 
}
\JA{\subsubsection{x86}

\myparagraph{x86 + MSVC}

以下は、\TT{f\_signed()} 関数がどうなっているかを示しています。

\lstinputlisting[caption=\NonOptimizing MSVC 2010,style=customasmx86]{patterns/07_jcc/simple/signed_MSVC.asm}

\myindex{x86!\Instructions!JLE}

最初の命令 \JLE は、\emph{Jump if Less or Equal}の場合はJumpを表します。 
言い換えれば、第2オペランドが第1オペランドより大きいか等しい場合、
制御フローは命令で指定されたアドレスまたはラベルに移ります。 
第2オペランドが最初のオペランドより小さいためにこの条件がトリガされない場合、制御フローは変更されず、最初の \printf が実行されます。 
\myindex{x86!\Instructions!JNE}
2番目のチェックは、 \JNE :\emph{Jump if Not Equal}です。 
オペランドが等しい場合、制御フローは変更されません。

\myindex{x86!\Instructions!JGE}
3番目のチェックは、最初のオペランドが2番目のオペランドより大きい場合、または等しい場合は \JGE :\emph{Jump if Greater or Equal}です。 
したがって、3つの条件ジャンプがすべてトリガされた場合、\printf の呼び出しはまったく実行されません。 
これは特別な介入なしには不可能です。 
\TT{f\_unsigned()}関数を見てみましょう。 
\TT{f\_unsigned()}関数は、次のように、 \JLE および \JGE の代わりにJBEおよびJAE命令が使用される点を除いて、
\TT{f\_signed()}と同じです。

\lstinputlisting[caption=GCC,style=customasmx86]{patterns/07_jcc/simple/unsigned_MSVC.asm}

\myindex{x86!\Instructions!JBE}
\myindex{x86!\Instructions!JAE}

すでに説明したように、分岐命令は異なります。
\JBE---\emph{Jump if Below or Equal} and \JAE---\emph{Jump if Above or Equal}
これらの命令(\INS{JA}/\JAE/\JB/\JBE)は、 \JG/\JGE/\JL/\JLE とは、符号なしの数字で動作する点が異なります。

\myindex{x86!\Instructions!JA}
\myindex{x86!\Instructions!JB}
\myindex{x86!\Instructions!JG}
\myindex{x86!\Instructions!JL}
\myindex{Signed numbers}

また、符号付き数値表現についてのセクションも参照してください(\myref{sec:signednumbers})。 
\INS{JA}/\JB の代わりに \JG/\JL が使用されている場合や、その逆の場合は、
変数がそれぞれ符号付きか、または符号なしなのかがほぼはっきりします。
ここには、もう何も新しくない、 \main 関数もあります。

\lstinputlisting[caption=\main,style=customasmx86]{patterns/07_jcc/simple/main_MSVC.asm}

\input{patterns/07_jcc/simple/olly_JA.tex}

\clearpage
\myparagraph{x86 + MSVC + Hiew}
\myindex{Hiew}

入力値にかかわらず、\TT{f\_unsigned()}関数が常に \q{a==b}を出力するように、
実行可能ファイルにパッチを当てることができます。 
ここで、Hiewでどのように見えるか見てみましょう。

\begin{figure}[H]
\centering
\myincludegraphics{patterns/07_jcc/simple/hiew_unsigned1.png}
\caption{Hiew: \TT{f\_unsigned()} 関数}
\label{fig:jcc_hiew_1}
\end{figure}

本質的には、次の3つのタスクを実行する必要があります。
\begin{itemize}
\item 最初のジャンプが常に起動しなければならない
\item 2番目のジャンプが決して起動してはならない
\item 3番目のジャンプが常にト起動しなければならない
\end{itemize}

したがって、コードフローは常に2番目の \printf を通過し、 \q{a==b}を出力するように指示できます。 

3つの命令(またはバイト)をパッチする必要があります。

\begin{itemize}
\item 最初のジャンプはJMPになりますが、\gls{jump offset}は同じままです。

\item 
2回目のジャンプがトリガされることもありますが、いずれにしても次の命令にジャンプします。

なぜなら、\gls{jump offset}を0に設定しているからです。これらの命令では、ジャンプオフセットが次の命令のアドレスに追加されます。 
オフセットが0の場合、
ジャンプは制御を次の命令に移します。

\item 
私たちが最初のものと同様に \JMP を置き換える3番目のジャンプは、常に起動します。

\end{itemize}

\clearpage
変更されたコードは次のとおりです。

\begin{figure}[H]
\centering
\myincludegraphics{patterns/07_jcc/simple/hiew_unsigned2.png}
\caption{Hiew: let's modify the \TT{f\_unsigned()} function}
\label{fig:jcc_hiew_2}
\end{figure}

これらのジャンプのいずれかを変更することができなければ、\printf 呼び出しを1回だけ実行したいのですが、何回か実行することになるでしょう。

\myparagraph{\NonOptimizing GCC}

\myindex{puts() instead of printf()}
\NonOptimizing GCC 4.4.1 
はほとんど同じコードを生成しますが、 \printf ではなく \puts~(\myref{puts}) が生成されます。

\myparagraph{\Optimizing GCC}

実行される度にフラグが同じ値を持つ場合、
鋭い読者はなぜ \CMP が何度も実行されるのかと尋ねるかもしれません。

おそらく、最適化されたMSVCではこうはできませんが、GCC 4.8.1の最適化はより深刻です。

\lstinputlisting[caption=GCC 4.8.1 f\_signed(),style=customasmx86]{patterns/07_jcc/simple/GCC_O3_signed.asm}

% should be here instead of 'switch' section?
また、\TT{CALL puts / RETN}の代わりにここに\TT{JMPを入れて}います。

この種のトリックは後で説明します:\myref{JMP_instead_of_RET}

この種のx86コードは、まれです。
MSVC 2012のように、そのようなコードを生成することはできません。 
一方、アセンブリ言語プログラマは、\TT{Jcc}命令を積み重ねることができるという
事実を十分に認識しています。

だから、どこかでそのような積み重ねを見ると、コードは手書きの可能性が高いです。

\TT{f\_unsigned()}関数は
巧妙に短いものではありません:

\lstinputlisting[caption=GCC 4.8.1 f\_unsigned(),style=customasmx86]{patterns/07_jcc/simple/GCC_O3_unsigned_JA.asm}

それにもかかわらず、3つではなく2つの\TT{CMP}命令があります。

したがって、GCC 4.8.1の最適化アルゴリズムはまだ完璧ではないでしょう。
}

\subsubsection{ARM}

% subsubsections here
\EN{\myparagraph{32-bit ARM}
\label{subsec:jcc_ARM}

\mysubparagraph{\OptimizingKeilVI (\ARMMode)}

\lstinputlisting[caption=\OptimizingKeilVI (\ARMMode),style=customasmARM]{patterns/07_jcc/simple/ARM/ARM_O3_signed.asm}

\myindex{ARM!Condition codes}
% FIXME \ref -> which instructions?

Many instructions in ARM mode could be executed only when specific flags are set.
E.g. this is often used when comparing numbers.

\myindex{ARM!\Instructions!ADD}
\myindex{ARM!\Instructions!ADDAL}

For instance, the \ADD instruction is in fact named \TT{ADDAL} internally, where \TT{AL} stands for
\emph{Always}, i.e., execute always.
The predicates are encoded in 4 high bits of the 32-bit ARM instructions (\emph{condition field}).
\myindex{ARM!\Instructions!B}
The \TT{B} instruction for unconditional jumping is in fact conditional and encoded just like any other
conditional jump, but has \TT{AL} in the \emph{condition field}, and it implies \emph{execute ALways}, 
ignoring flags.

\myindex{ARM!\Instructions!ADR}
\myindex{ARM!\Instructions!ADRcc}
\myindex{ARM!\Instructions!CMP}

The \TT{ADRGT} instruction works just like \TT{ADR} but executes only in case the previous \CMP
instruction founds one of the numbers greater than the another, while comparing the two (\emph{Greater Than}).

\myindex{ARM!\Instructions!BL}
\myindex{ARM!\Instructions!BLcc}

The next \TT{BLGT} instruction behaves exactly as \TT{BL} 
and is triggered only if the result of the comparison has been (\emph{Greater Than}). 
\TT{ADRGT} writes a pointer to the string \TT{a>b\textbackslash{}n} into \Reg{0} and \TT{BLGT} calls \printf.
Therefore, instructions suffixed with \TT{-GT} are to execute only in case the value in \Reg{0} (which is $a$) is bigger than the value in \Reg{4} (which is $b$).

\myindex{ARM!\Instructions!ADRcc}
\myindex{ARM!\Instructions!BLcc}

Moving forward we see the \TT{ADREQ} and \TT{BLEQ} instructions.
They behave just like \TT{ADR} and \TT{BL}, but are to be executed only if operands were equal to each
other during the last comparison.
Another \CMP is located before them (because the \printf execution may have tampered the flags).

\myindex{ARM!\Instructions!LDMccFD}
\myindex{ARM!\Instructions!LDMFD}

Then we see \TT{LDMGEFD}, this instruction works just like \TT{LDMFD}\footnote{\ac{LDMFD}},
but is triggered only when one of the values is greater or equal than the other (\emph{Greater or Equal}).
The \TT{LDMGEFD SP!, \{R4-R6,PC\}} instruction acts like a function epilogue, but it will be triggered only if $a>=b$, and only then the function execution will finish.

\myindex{Function epilogue}

But if that condition is not satisfied, i.e., $a<b$, then the control flow will continue to the next \\
\TT{\q{LDMFD SP!, \{R4-R6,LR\}}} instruction, which is one more function epilogue. This instruction restores not only the \TT{R4-R6} registers state, but also \ac{LR} instead of \ac{PC}, thus, it does not return from the function.
The last two instructions call \printf with the string <<a<b\textbackslash{}n>> as a sole argument.
We already examined an unconditional jump to the \printf function instead of function return in <<\PrintfSeveralArgumentsSectionName>> section~(\myref{ARM_B_to_printf}).

\myindex{ARM!\Instructions!ADRcc}
\myindex{ARM!\Instructions!BLcc}
\myindex{ARM!\Instructions!LDMccFD}
\TT{f\_unsigned} is similar, only the \TT{ADRHI}, \TT{BLHI}, and \TT{LDMCSFD} instructions are used there, these predicates (\emph{HI = Unsigned higher, CS = Carry Set (greater than or equal)}) are analogous to those examined before, but for unsigned values.

There is not much new in the \main function for us:

\lstinputlisting[caption=\main,style=customasmARM]{patterns/07_jcc/simple/ARM/ARM_O3_main.asm}

That is how you can get rid of conditional jumps in ARM mode.

\myindex{RISC pipeline}
Why is this so good? Read here: \myref{branch_predictors}.

\myindex{x86!\Instructions!CMOVcc}

There is no such feature in x86, except the \TT{CMOVcc} instruction, it is the same as \MOV,
but triggered only when specific flags are set, usually set by \CMP.

\mysubparagraph{\OptimizingKeilVI (\ThumbMode)}

\lstinputlisting[caption=\OptimizingKeilVI (\ThumbMode),style=customasmARM]{patterns/07_jcc/simple/ARM/ARM_thumb_signed.asm}

\myindex{ARM!\Instructions!BLE}
\myindex{ARM!\Instructions!BNE}
\myindex{ARM!\Instructions!BGE}
\myindex{ARM!\Instructions!BLS}
\myindex{ARM!\Instructions!BCS}
\myindex{ARM!\Instructions!B}
\myindex{ARM!\ThumbMode}

Only \TT{B} instructions in Thumb mode may be supplemented by \emph{condition codes}, so the Thumb code 
looks more ordinary.

\TT{BLE} is a normal conditional jump \emph{Less than or Equal}, 
\TT{BNE}---\emph{Not Equal}, 
\TT{BGE}---\emph{Greater than or Equal}.

\TT{f\_unsigned} is similar, only other instructions are used while dealing 
with unsigned values: \TT{BLS} 
(\emph{Unsigned lower or same}) and \TT{BCS} (\emph{Carry Set (Greater than or equal)}).
}
\RU{\myparagraph{32-битный ARM}
\label{subsec:jcc_ARM}

\mysubparagraph{\OptimizingKeilVI (\ARMMode)}

\lstinputlisting[caption=\OptimizingKeilVI (\ARMMode),style=customasmARM]{patterns/07_jcc/simple/ARM/ARM_O3_signed.asm}

\myindex{ARM!Condition codes}
% FIXME \ref -> which instructions?
Многие инструкции в режиме ARM могут быть исполнены только при некоторых выставленных флагах.

Это нередко используется для сравнения чисел.

\myindex{ARM!\Instructions!ADD}
\myindex{ARM!\Instructions!ADDAL}
К примеру, инструкция \ADD на самом деле называется \TT{ADDAL} внутри, \TT{AL} означает \emph{Always}, то есть, исполнять всегда.
Предикаты кодируются в 4-х старших битах инструкции 32-битных ARM-инструкций (\emph{condition field}).
\myindex{ARM!\Instructions!B}
Инструкция безусловного перехода \TT{B} на самом деле условная и кодируется так же, 
как и прочие инструкции условных переходов, но имеет \TT{AL} в \emph{condition field}, 
то есть исполняется всегда (\emph{execute ALways}), игнорируя флаги.

\myindex{ARM!\Instructions!ADR}
\myindex{ARM!\Instructions!ADRcc}
\myindex{ARM!\Instructions!CMP}
Инструкция \TT{ADRGT} работает так же, как и \TT{ADR}, но исполняется только в случае,
если предыдущая инструкция \CMP,
сравнивая два числа, обнаруживает, что одно из них больше второго (\emph{Greater Than}).

\myindex{ARM!\Instructions!BL}
\myindex{ARM!\Instructions!BLcc}
Следующая инструкция \TT{BLGT} ведет себя так же, как и \TT{BL} и сработает, только если 
результат сравнения ``больше чем'' (\emph{Greater Than}).
\TT{ADRGT} записывает в \Reg{0} указатель на строку \TT{a>b\textbackslash{}n}, а \TT{BLGT} вызывает \printf.
Следовательно, эти инструкции с суффиксом \TT{-GT} исполнятся только в том случае, если значение
в \Reg{0} (там $a$) было больше, чем значение в \Reg{4} (там $b$).

\myindex{ARM!\Instructions!ADRcc}
\myindex{ARM!\Instructions!BLcc}
Далее мы увидим инструкции \TT{ADREQ} и \TT{BLEQ}.
Они работают так же, как и \TT{ADR} и \TT{BL}, но исполнятся только если значения при последнем сравнении были равны.
Перед ними расположен ещё один \CMP, потому что вызов \printf мог испортить состояние флагов.

\myindex{ARM!\Instructions!LDMccFD}
\myindex{ARM!\Instructions!LDMFD}
Далее мы увидим \TT{LDMGEFD}. Эта инструкция работает так же, как и \TT{LDMFD}\footnote{\ac{LDMFD}}, 
но сработает только если в результате сравнения одно из значений было больше или равно второму (\emph{Greater or Equal}).
Смысл инструкции \TT{LDMGEFD SP!, \{R4-R6,PC\}} 
в том, что это как бы эпилог функции, но он сработает только если $a>=b$, только тогда работа 
функции закончится.

\myindex{Function epilogue}
Но если это не так, то есть $a<b$, то исполнение дойдет до следующей инструкции 
\TT{LDMFD SP!, \{R4-R6,LR\}}. Это ещё один эпилог функции. Эта инструкция восстанавливает состояние регистров
\TT{R4-R6}, но и \ac{LR} вместо \ac{PC}, таким образом, пока что, не делая возврата из функции.

Последние две инструкции вызывают \printf 
со строкой <<a<b\textbackslash{}n>> в качестве единственного аргумента.
Безусловный переход на \printf вместо возврата из функции мы уже рассматривали в секции
 <<\PrintfSeveralArgumentsSectionName>>~(\myref{ARM_B_to_printf}).

\myindex{ARM!\Instructions!ADRcc}
\myindex{ARM!\Instructions!BLcc}
\myindex{ARM!\Instructions!LDMccFD}
Функция \TT{f\_unsigned} точно такая же, но там используются инструкции \TT{ADRHI}, \TT{BLHI}, и \TT{LDMCSFD}. Эти предикаты
(\emph{HI = Unsigned higher, CS = Carry Set (greater than or equal)})
аналогичны рассмотренным, но служат для работы с беззнаковыми значениями.

В функции \main ничего нового для нас нет:

\lstinputlisting[caption=\main,style=customasmARM]{patterns/07_jcc/simple/ARM/ARM_O3_main.asm}

Так, в режиме ARM можно обойтись без условных переходов.

\myindex{Конвейер RISC}
Почему это хорошо? Читайте здесь: \myref{branch_predictors}.

\myindex{x86!\Instructions!CMOVcc}
В x86 нет аналогичной возможности, если не считать инструкцию \TT{CMOVcc}, это то же что и \MOV, 
но она срабатывает только при определенных выставленных флагах, обычно выставленных при помощи \CMP во время сравнения.

\mysubparagraph{\OptimizingKeilVI (\ThumbMode)}

\lstinputlisting[caption=\OptimizingKeilVI (\ThumbMode),style=customasmARM]{patterns/07_jcc/simple/ARM/ARM_thumb_signed.asm}

\myindex{ARM!\Instructions!BLE}
\myindex{ARM!\Instructions!BNE}
\myindex{ARM!\Instructions!BGE}
\myindex{ARM!\Instructions!BLS}
\myindex{ARM!\Instructions!BCS}
\myindex{ARM!\Instructions!B}
\myindex{ARM!\ThumbMode}
В режиме Thumb только инструкции \TT{B} могут быть дополнены условием исполнения (\emph{condition code}), 
так что код для режима Thumb выглядит привычнее.

\TT{BLE} это обычный переход с условием \emph{Less than or Equal}, 
\TT{BNE} --- \emph{Not Equal}, 
\TT{BGE} --- \emph{Greater than or Equal}.

Функция \TT{f\_unsigned} точно такая же, но для работы с беззнаковыми величинами 
там используются инструкции \TT{BLS} 
(\emph{Unsigned lower or same}) и \TT{BCS} (\emph{Carry Set (Greater than or equal)}).
}
\DE{\myparagraph{32-bit ARM}
\label{subsec:jcc_ARM}

\mysubparagraph{\OptimizingKeilVI (\ARMMode)}

\lstinputlisting[caption=\OptimizingKeilVI (\ARMMode),style=customasmARM]{patterns/07_jcc/simple/ARM/ARM_O3_signed.asm}

\myindex{ARM!Condition codes}
% FIXME \ref -> which instructions?
Viele Befehle im ARM mode können nur ausgeführt werden, wenn spezielle Flags gesetzt sind.
Dies ist beispielsweise oft beim Vergleich von Zahlen der Fall.

\myindex{ARM!\Instructions!ADD}
\myindex{ARM!\Instructions!ADDAL}
Der \ADD Befehl zum Beispiel heißt hier intern \TT{ADDAL}, wobei \TT{AL} für \emph{Always} (dt. immer) steht, d.h. er wird
immer ausgeführt.
Die Prädikate werden in den 4 höchstwertigsten Bits des 32-Bit-ARM-Befehls kodiert, dem \emph{condition field}.

\myindex{ARM!\Instructions!B}
Der Befehl \TT{B} für einen unbedingten Sprung ist tatsächlich doch bedingt und genau wie jeder andere bedingte Sprung
kodiert, nut dass er \TT{AL} im \emph{condition field} hat und dadurch die Flags ignoriert und immer ausgeführt wird.

\myindex{ARM!\Instructions!ADR}
\myindex{ARM!\Instructions!ADRcc}
\myindex{ARM!\Instructions!CMP}
Der Befehl \TT{ADRGT} arbeitet wie \TT{ADR}, wird aber nur ausgeführt, wenn das vorangehende \CMP ergeben hat, dass eine
der beiden Eingabezahlen größer war als die andere. 

\myindex{ARM!\Instructions!BL}
\myindex{ARM!\Instructions!BLcc}
% ToBeUpdated
Der folgende \TT{BLGT} Befehl verhält sich genau wie \TT{BL} und wird nur dann ausgeführt, wenn das Ergebnis des
Vergleichs das gleiche war (d.h. größer als).
\TT{ADRGT} schreibt einen Pointer auf den String \TT{a>b\textbackslash{}n} nach \Reg{0} und \TT{BLGT} ruft \printf auf.
Das heißt, Befehl mit dem Suffix \TT{-GT} werden nur ausgeführt, wenn der Wert in \Reg{0} (das ist $a$) größer ist als
der Wert in \Reg{4} (das ist $b$).

\myindex{ARM!\Instructions!ADRcc}
\myindex{ARM!\Instructions!BLcc}
Im Folgenden finden wir die Befehle \TT{ADREQ} und \TT{BLEQ}.
Sie verhalten sich wie \TT{ADR} und \TT{BL}, werden aber nur ausgeführt, wenn die beiden Operanden des letzten
Vergleichs gleich waren.
Ein weiteres \CMP befindet sich davor (denn die Ausführung von \printf könnte die Flags verändert haben).

\myindex{ARM!\Instructions!LDMccFD}
\myindex{ARM!\Instructions!LDMFD}
Dann finden wir \TT{LDMGEFD}; dieser Befehl arbeitet genau wie \TT{LDMFD}\footnote{\ac{LDMFD}}, wird aber nur
ausgeführt, wenn einer der Werte größer gleich dem anderen ist. 
Der Befehl \TT{LDMGEFD SP!, \{R4-R6,PC\}} fungiert als Funktionsepilog, wird aber nur ausgeführt, ewnn $a>=b$ und nur
dann wird die Funktionsausführung beendet.
\myindex{Function epilogue}
Wenn aber diese Bedingung nicht erfüllt ist, d.h. $a<b$, wird der Control Flow zum nächsten \\
\TT{\q{LDMFD SP!, \{R4-R6,LR\}}} springen, der ebenfalls einen Funktionsepilog darstellt. Dieser Befehl stellt nicht nur
den Zustand der \TT{R4-R6} Register wieder her, sondern auch \ac{LR} anstatt \ac{PC}, dadurch gibt er nichts aus der
Funktion zurück.
Die letzten beiden Befehle rufen \printf mit dem String <<a<b\textbackslash{}n>> als einzigem Argument auf.
Wir haben bereits einen unbedingten Sprung zur Funktion \printf anstelle einer Funktionsrückgabe im Abschnitt
<<\PrintfSeveralArgumentsSectionName>>~(\myref{ARM_B_to_printf}) untersucht.

\myindex{ARM!\Instructions!ADRcc}
\myindex{ARM!\Instructions!BLcc}
\myindex{ARM!\Instructions!LDMccFD}
\TT{f\_unsigned} ist ähnlich, nur die Befehle \TT{ADRHI}, \TT{BLHI} und \TT{LDMCSFD} werden hier verwendet.
Deren Prädikaten (\emph{HI = Unsigned higher, CS = Carry Set (greater than or equal)}) sind analog zu den eben
betrachteten, nur eben für vorzeichenlose Werte. 

In der Funktion \main finden wir nicht viel Neues:

\lstinputlisting[caption=\main,style=customasmARM]{patterns/07_jcc/simple/ARM/ARM_O3_main.asm}
Auf diese Weise kann man bedingte Sprünge im ARM mode entfernen.


\myindex{RISC pipeline}
Für eine Begründung warum dies vorteilhaft ist, siehe: \myref{branch_predictors}.

\myindex{x86!\Instructions!CMOVcc}
In x86 gibt es kein solches Feature, außer dem \TT{CMOVcc} Befehl, der genau wie \MOV funktioniert, aber nur ausgeführt
wird, wenn spezielle Flags - normalerweise durch \CMP - gesetzt sind.


\mysubparagraph{\OptimizingKeilVI (\ThumbMode)}

\lstinputlisting[caption=\OptimizingKeilVI (\ThumbMode),style=customasmARM]{patterns/07_jcc/simple/ARM/ARM_thumb_signed.asm}

\myindex{ARM!\Instructions!BLE}
\myindex{ARM!\Instructions!BNE}
\myindex{ARM!\Instructions!BGE}
\myindex{ARM!\Instructions!BLS}
\myindex{ARM!\Instructions!BCS}
\myindex{ARM!\Instructions!B}
\myindex{ARM!\ThumbMode}
Nur der \TT{B} Befehl im Thumb mode kann mit condition codes versehen werden, sodass der Thumb Code gewöhnlicher
aussieht.


\TT{BLE} ist ein normaler bedingter Sprung \emph{Less than or Equal}, 
\TT{BNE}---\emph{Not Equal}, 
\TT{BGE}---\emph{Greater than or Equal}.

\TT{f\_unsigned} ist ähnlich, nur dass andere Befehle verwendet werden, wenn mit vorzeichenlosen Werten umgegangen wird:
\TT{BLS} (\emph{Unsigned lower or same}) und \TT{BCS} (\emph{Carry Set (Greater than or equal)}).
}
\FR{\myparagraph{ARM 32-bit}
\label{subsec:jcc_ARM}

\mysubparagraph{\OptimizingKeilVI (\ARMMode)}

\lstinputlisting[caption=\OptimizingKeilVI (\ARMMode),style=customasmARM]{patterns/07_jcc/simple/ARM/ARM_O3_signed.asm}

\myindex{ARM!Condition codes}
% FIXME \ref -> which instructions?

Beaucoup d'instructions en mode ARM ne peuvent être exécutées que lorsque certains
flags sont mis.
E.g, ceci est souvent utilisé lorsque l'on compare les nombres

\myindex{ARM!\Instructions!ADD}
\myindex{ARM!\Instructions!ADDAL}

Par exemple, l'instruction \ADD est en fait appelée \TT{ADDAL} en interne, où \TT{AL}
signifie \emph{Always}, i.e., toujours exécuter.
Les prédicats sont encodés dans les 4 bits du haut des instructions ARM 32-bit. (\emph{condition field}).
\myindex{ARM!\Instructions!B}
L'instruction de saut inconditionnel \TT{B} est en fait conditionnelle et encodée
comme toutes les autres instructions de saut conditionnel, mais a \TT{AL} dans le
\emph{champ de condition}, et \emph{s'exécute toujours} (ALways), ignorants les flags.

\myindex{ARM!\Instructions!ADR}
\myindex{ARM!\Instructions!ADRcc}
\myindex{ARM!\Instructions!CMP}

L'instruction \TT{ADRGT} fonctionne comme \TT{ADR} mais ne s'exécute que dans le
cas où l'instruction \CMP précédente a trouvé un des nombres plus grand que l'autre,
en comparant les deux (\emph{Greater Than}).

\myindex{ARM!\Instructions!BL}
\myindex{ARM!\Instructions!BLcc}

L'instruction \TT{BLGT} se comporte exactement comme \TT{BL} et n'est effectuée
que si le résultat de la comparaison était \emph{Greater Than} (plus grand).
\TT{ADRGT} écrit un pointeur sur la chaîne \TT{a>b\textbackslash{}n} dans \Reg{0}
et \TT{BLGT} appelle \printf.
Donc, les instructions suffixées par \TT{-GT} ne sont exécutées que si la valeur
dans \Reg{0} (qui est $a$) est plus grande que la valeur dans \Reg{4} (qui est $b$).

\myindex{ARM!\Instructions!ADRcc}
\myindex{ARM!\Instructions!BLcc}

En avançant, nous voyons les instructions \TT{ADREQ} et \TT{BLEQ}.
Elles se comportent comme \TT{ADR} et \TT{BL}, mais ne sont exécutées que si les
opérandes étaient égaux lors de la dernière comparaison.
Un autre \CMP se trouve avant elles (car l'exécution de \printf pourrait avoir
modifiée les flags).

\myindex{ARM!\Instructions!LDMccFD}
\myindex{ARM!\Instructions!LDMFD}

Ensuite nous voyons \TT{LDMGEFD}, cette instruction fonctionne comme \TT{LDMFD}\footnote{\ac{LDMFD}},
mais n'est exécutée que si l'une des valeurs est supérieure ou égale à l'autre
(\emph{Greater or Equal}).\\
L'instruction \TT{LDMGEFD SP!, \{R4-R6,PC\}} se comporte comme une fonction épilogue,
mais elle ne sera exécutée que si $a>=b$, et seulement lorsque l'exécution de la
fonction se terminera.

\myindex{Function epilogue}

Mais si cette condition n'est pas satisfaite, i.e., $a<b$, alors le flux d'exécution
continue à l'instruction suivante, \TT{\q{LDMFD SP!, \{R4-R6,LR\}}}, qui est aussi
un épilogue de la fonction. Cette instruction ne restaure pas seulement l'état des
registres \TT{R4-R6}, mais aussi \ac{LR} au lieu de \ac{PC}, donc il ne retourne
pas de la fonction.
Les deux dernières instructions appellent \printf avec la chaîne <<a<b\textbackslash{}n>>
comme unique argument.
Nous avons déjà examiné un saut inconditionnel à la fonction \printf au lieu
d'un appel avec retour dans <<\PrintfSeveralArgumentsSectionName>> section~(\myref{ARM_B_to_printf}).

\myindex{ARM!\Instructions!ADRcc}
\myindex{ARM!\Instructions!BLcc}
\myindex{ARM!\Instructions!LDMccFD}
\TT{f\_unsigned} est très similaire, à part les instructions \TT{ADRHI}, \TT{BLHI},
et \TT{LDMCSFD} utilisées ici, ces prédicats (\emph{HI = Unsigned higher, CS = Carry
Set (greater than or equal)}) sont analogues à ceux examinés avant, mais pour des
valeurs non signées.

Il n'y a pas grand chose de nouveau pour nous dans la fonction \main:

\lstinputlisting[caption=\main,style=customasmARM]{patterns/07_jcc/simple/ARM/ARM_O3_main.asm}

C'est ainsi que vous pouvez vous débarrasser des sauts conditionnels en mode ARM.

\myindex{RISC pipeline}
Pourquoi est-ce que c'est si utile? Lire ici: \myref{branch_predictors}.

\myindex{x86!\Instructions!CMOVcc}

Il n'y a pas de telle caractéristique en x86, exceptée l'instruction \TT{CMOVcc},
qui est comme un \MOV, mais effectuée seulement lorsque certains flags sont mis,
en général mis par \CMP.

\mysubparagraph{\OptimizingKeilVI (\ThumbMode)}

\lstinputlisting[caption=\OptimizingKeilVI (\ThumbMode),style=customasmARM]{patterns/07_jcc/simple/ARM/ARM_thumb_signed.asm}

\myindex{ARM!\Instructions!BLE}
\myindex{ARM!\Instructions!BNE}
\myindex{ARM!\Instructions!BGE}
\myindex{ARM!\Instructions!BLS}
\myindex{ARM!\Instructions!BCS}
\myindex{ARM!\Instructions!B}
\myindex{ARM!\ThumbMode}

En mode Thumb, seules les instructions \TT{B} peuvent être complétées par un
\emph{condition codes}, (code de condition) donc le code Thumb paraît plus ordinaire.

\TT{BLE} est un saut conditionnel normal \emph{Less than or Equal} (inférieur ou égal),
\TT{BNE}---\emph{Not Equal} (non égal),
\TT{BGE}---\emph{Greater than or Equal} (supérieur ou égal).

\TT{f\_unsigned} est similaire, seules d'autres instructions sont utilisées
pour travailler avec des valeurs non-signées: \TT{BLS}
(\emph{Unsigned lower or same} non signée, inférieur ou égal) et \TT{BCS} (\emph{Carry
Set (Greater than or equal)} supérieur ou égal).
}
\IT{\myparagraph{32-bit ARM}
\label{subsec:jcc_ARM}

\mysubparagraph{\OptimizingKeilVI (\ARMMode)}

\lstinputlisting[caption=\OptimizingKeilVI (\ARMMode),style=customasmARM]{patterns/07_jcc/simple/ARM/ARM_O3_signed.asm}

\myindex{ARM!Condition codes}
% FIXME \ref -> which instructions?

In modalità ARM molte istruzioni possono essere eseguite solo quando specifici flag sono settati.
Es. sono spesso usate quando si confrontano numeri.

\myindex{ARM!\Instructions!ADD}
\myindex{ARM!\Instructions!ADDAL}

Ad esempio, l'istruzione \ADD è infatti chiamata internamente is in fact named \TT{ADDAL}, il suffisso \TT{AL} sta per
\emph{Always}, ad indicare che viene eseguita sempre.
I predicati sono codificati nei 4 bit alti dell'istruzione ARM a 32-bit (\emph{condition field}).
\myindex{ARM!\Instructions!B}
L'istruzione \TT{B} per effettuare un salto non condizionale è in realtà condizionale ed è codificata proprio come ogni altro
jump condizionale, ma ha \TT{AL} (\emph{execute ALways}) nel \emph{condition field}, e ciò implica che venga sempre eseguito, ignorando i flag.

\myindex{ARM!\Instructions!ADR}
\myindex{ARM!\Instructions!ADRcc}
\myindex{ARM!\Instructions!CMP}

L'istruzione \TT{ADRGT} funziona come \TT{ADR}, ma viene eseguita soltanto nel caso in cui la precedente istruzione \CMP
trovi uno dei due numeri a confronto più grande dell'altro, (\emph{Greater Than}).

\myindex{ARM!\Instructions!BL}
\myindex{ARM!\Instructions!BLcc}

La successiva istruzione \TT{BLGT} si comporta esattamente come \TT{BL} 
ed il salto viene innescato solo se il risultato del confronto è (\emph{Greater Than}). 
\TT{ADRGT} scrive un putatore alla stringa \TT{a>b\textbackslash{}n} nel registro \Reg{0} e \TT{BLGT} chiama \printf.
Le istruzioni aventi il suffisso \TT{-GT} in questo caso sono quindi eseguite solo se il valore in \Reg{0} (ovvero $a$) è maggiore del valore 
in \Reg{4} (ovvero $b$).

\myindex{ARM!\Instructions!ADRcc}
\myindex{ARM!\Instructions!BLcc}

Andando avanti vediamo le istruzioni \TT{ADREQ} e \TT{BLEQ} instructions.
Si comporano come \TT{ADR} e \TT{BL}, ma vengono eseguite solo se gli operandi erano uguali al momento dell'ultimo confronto.
Un altra \CMP si trova subito prima di loro (poiché l'esecuzione di \printf potrebbe aver alterato i flag).

\myindex{ARM!\Instructions!LDMccFD}
\myindex{ARM!\Instructions!LDMFD}

Ancora più avanti vediamo \TT{LDMGEFD}, questa istruzione funziona come \TT{LDMFD}\footnote{\ac{LDMFD}},
ma viene eseguita solo quando uno dei valori e maggiore di o uguale all'altro (\emph{Greater or Equal}).
L'istruzione \TT{LDMGEFD SP!, \{R4-R6,PC\}} si comporta come un epilogo di funzione, ma viene eseguita solo se $a>=b$, e solo in tal caso avrà termine l'esecuzione della funzione.

\myindex{Function epilogue}

Nel caso in cui questa condizione non venga soddisfatta, ovvero se $a<b$, il flusso continuerà alla successiva istruzione \\
\TT{\q{LDMFD SP!, \{R4-R6,LR\}}} , un altro epilogo di funzione. Questa istruzione non ripristina soltanto lo stato dei registri \TT{R4-R6} , ma anche \ac{LR} invece di \ac{PC}, non ritornando così dalla funzione.
Le due ultime istruzioni chiamano \printf con la stringa <<a<b\textbackslash{}n>> come unico argomento.
Abbiamo già visto un salto diretto non condizionale alla funzione \printf senza altro codice di uscita/ritorno dalla funzione nella sezione <<\PrintfSeveralArgumentsSectionName>> section~(\myref{ARM_B_to_printf}).

\myindex{ARM!\Instructions!ADRcc}
\myindex{ARM!\Instructions!BLcc}
\myindex{ARM!\Instructions!LDMccFD}
\TT{f\_unsigned} è simile, e vengono utilizzate le funzioni \TT{ADRHI}, \TT{BLHI}, e \TT{LDMCSFD}. Questi predicati (\emph{HI = Unsigned higher, CS = Carry Set (maggiore di o uguale a)}) sono analoghi a quelli visti in precedenza, e operano su valori di tipo unsigned.

Nella funzione \main non c'è nulla di nuovo:

\lstinputlisting[caption=\main,style=customasmARM]{patterns/07_jcc/simple/ARM/ARM_O3_main.asm}

In questo modo ci si può sbarazzare dei salti condizionali in modalità ARM.
\myindex{RISC pipeline}
Perchè è bene? Leggi qui: \myref{branch_predictors}.

\myindex{x86!\Instructions!CMOVcc}

Non esiste una funzionalità simile in x86, eccetto per l'istruzione \TT{CMOVcc} , che è uguale a \MOV ma viene eseguita solo
se specifici flag sono settati, solitamente da \CMP.

\mysubparagraph{\OptimizingKeilVI (\ThumbMode)}

\lstinputlisting[caption=\OptimizingKeilVI (\ThumbMode),style=customasmARM]{patterns/07_jcc/simple/ARM/ARM_thumb_signed.asm}

\myindex{ARM!\Instructions!BLE}
\myindex{ARM!\Instructions!BNE}
\myindex{ARM!\Instructions!BGE}
\myindex{ARM!\Instructions!BLS}
\myindex{ARM!\Instructions!BCS}
\myindex{ARM!\Instructions!B}
\myindex{ARM!\ThumbMode}

Solo le istruzioni \TT{B} in modalità Thumb possono essere supplementate da \emph{condition codes}, pertanto il codice Thumb 
ha un aspetto più ordinario.

\TT{BLE} è un normale jump condizionale \emph{Less than or Equal}, 
\TT{BNE}---\emph{Not Equal}, 
\TT{BGE}---\emph{Greater than or Equal}.

\TT{f\_unsigned} è simile, con la differenza che vengono usate altre istruzioni per operare con valori
di tipo unsigned: \TT{BLS} 
(\emph{Unsigned lower or same}) e \TT{BCS} (\emph{Carry Set (Greater than or equal)}).
}
\JA{\myparagraph{32-bit ARM}
\label{subsec:jcc_ARM}

\mysubparagraph{\OptimizingKeilVI (\ARMMode)}

\lstinputlisting[caption=\OptimizingKeilVI (\ARMMode),style=customasmARM]{patterns/07_jcc/simple/ARM/ARM_O3_signed.asm}

\myindex{ARM!Condition codes}
% FIXME \ref -> which instructions?

ARMモードの多くの命令は、特定のフラグがセットされている場合にのみ実行できます。
例えば、これは数字を比較するときによく使用されます。

\myindex{ARM!\Instructions!ADD}
\myindex{ARM!\Instructions!ADDAL}

例えば、 \ADD 命令は実際には内部で\TT{ADDAL}と名付けられ、
ALは\emph{常に}、すなわち常に実行する。
述語は、32ビットARM命令の4つの上位ビット(\emph{条件フィールド})でエンコードされます。
\myindex{ARM!\Instructions!B}
無条件ジャンプの\TT{B}命令は、実際には条件付きで他の条件ジャンプと同様にエンコードされますが、
条件フィールドには\TT{AL}があり、フラグを無視して\emph{常に実行}することを意味します。

\myindex{ARM!\Instructions!ADR}
\myindex{ARM!\Instructions!ADRcc}
\myindex{ARM!\Instructions!CMP}

\TT{ADRGT}命令は\TT{ADR}と同じように動作しますが、前の \CMP 命令が2つ(\emph{大きい方})を比較しながら、
他の命令より大きな数値の1つを検出した場合にのみ実行されます。

\myindex{ARM!\Instructions!BL}
\myindex{ARM!\Instructions!BLcc}

次の\TT{BLGT}命令は\TT{BL}と同じように動作し、
比較の結果が(より大きい)場合にのみ実行されます。 
\TT{ADRGT}は文字列\TT{a>b\textbackslash{}n}へのポインタを\Reg{0}に書き込み、\TT{BLGT}は \printf を呼び出します。
したがって、\TT{-GT}の後に続く命令は、\Reg{0}($a$)の値が\Reg{4}($b$)の値より大きい場合にのみ実行されます。

\myindex{ARM!\Instructions!ADRcc}
\myindex{ARM!\Instructions!BLcc}

\TT{ADREQ}命令と\TT{BLEQ}命令が順方向に進みます。
それらは\TT{ADR}と\TT{BL}のように動作しますが、最後の比較時にオペランドが等しい場合にのみ実行されます。 
\printf の実行によってフラグが改ざんされた可能性があるため、別の \CMP がその前に配置されます。

\myindex{ARM!\Instructions!LDMccFD}
\myindex{ARM!\Instructions!LDMFD}

次に、\TT{LDMGEFD}を参照してください。この命令は\TT{LDMFD}\footnote{\ac{LDMFD}}のように機能しますが、
一方の値が他方の値より大きいか等しい場合にのみ実行されます。 
\TT{LDMGEFD SP!, \{R4-R6,PC\}}命令は関数エピローグのように動作しますが、$a>=b$の場合にのみトリガされ、その後に関数の実行が終了します。

\myindex{Function epilogue}

しかし、その条件が満たされない場合、すなわち$a<b$の場合、制御フローは次の
\TT{\q{LDMFD SP!, \{R4-R6,LR\}}}命令に続き、これはもう1つの関数エピローグです。この命令は、\TT{R4-R6}だけでなく\ac{PC}の代わりに\ac{LR}も登録されているため、関数からは戻りません。
最後の2つの命令は、文字列<<a<b\textbackslash{}n>>を唯一の引数として \printf を呼び出します。  
\printf セクション(\myref{ARM_B_to_printf})の関数の戻り値ではなく、 \printf 関数への無条件ジャンプを調べました。

\myindex{ARM!\Instructions!ADRcc}
\myindex{ARM!\Instructions!BLcc}
\myindex{ARM!\Instructions!LDMccFD}
\TT{f\_unsigned}は類似しており、\TT{ADRHI}、\TT{BLHI}、および\TT{LDMCSFD}命令のみが使用されています。これらの述部(\emph{HI = Unsigned higher, CS = Carry Set (greater than or equal)})は、前に説明したものと類似しています。

\main 関数にはそんなに新しい点はありません。

\lstinputlisting[caption=\main,style=customasmARM]{patterns/07_jcc/simple/ARM/ARM_O3_main.asm}

これは、ARMモードでの条件付きジャンプを取り除く方法です。

\myindex{RISC pipeline}
なぜこれがよいのでしょう? 以下を読んでください:\myref{branch_predictors}

\myindex{x86!\Instructions!CMOVcc}

x86では、\TT{CMOVcc}命令以外は \MOV と同じですが、
通常は \CMP によって設定された特定のフラグが設定されている場合にのみ実行されます。

\mysubparagraph{\OptimizingKeilVI (\ThumbMode)}

\lstinputlisting[caption=\OptimizingKeilVI (\ThumbMode),style=customasmARM]{patterns/07_jcc/simple/ARM/ARM_thumb_signed.asm}

\myindex{ARM!\Instructions!BLE}
\myindex{ARM!\Instructions!BNE}
\myindex{ARM!\Instructions!BGE}
\myindex{ARM!\Instructions!BLS}
\myindex{ARM!\Instructions!BCS}
\myindex{ARM!\Instructions!B}
\myindex{ARM!\ThumbMode}

Thumbモードの\TT{B}命令だけが\emph{条件コード}で補完されるため、
Thumbコードはより一般的に見えます。

\TT{BLE}は通常の条件ジャンプであり、\emph{Less than or Equal}の意味です。 
\TT{BNE}は\emph{Not Equal}の意味です。
\TT{BGE}は\emph{Greater than or Equal}の意味です。

\TT{f\_unsigned}は似ていますが、符号なしの値を扱う際には、
\TT{BLS}(\emph{Unsigned lower or same})および\TT{BCS}(\emph{Carry Set (Greater than or equal)})命令しか使用されません。
}

\EN{\subsubsection{ARM64}

\myparagraph{\Optimizing GCC (Linaro) 4.9}

\myindex{Fused multiply–add}
\myindex{ARM!\Instructions!MADD}
Everything here is simple.
\TT{MADD} is just an instruction doing fused multiply/add (similar to the \TT{MLA} we already saw).
All 3 arguments are passed in the 32-bit parts of X-registers.
Indeed, the argument types are 32-bit \emph{int}'s.
The result is returned in \TT{W0}.

\lstinputlisting[caption=\Optimizing GCC (Linaro) 4.9,style=customasmARM]{patterns/05_passing_arguments/ARM/ARM64_O3_EN.s}

Let's also extend all data types to 64-bit \TT{uint64\_t} and test:

\lstinputlisting[style=customc]{patterns/05_passing_arguments/ex64.c}

\begin{lstlisting}[style=customasmARM]
f:
	madd	x0, x0, x1, x2
	ret
main:
	mov	x1, 13396
	adrp	x0, .LC8
	stp	x29, x30, [sp, -16]!
	movk	x1, 0x27d0, lsl 16
	add	x0, x0, :lo12:.LC8
	movk	x1, 0x122, lsl 32
	add	x29, sp, 0
	movk	x1, 0x58be, lsl 48
	bl	printf
	mov	w0, 0
	ldp	x29, x30, [sp], 16
	ret

.LC8:
	.string	"%lld\n"
\end{lstlisting}

The \ttf{} function is the same, only the whole 64-bit X-registers are now used.
Long 64-bit values are loaded into the registers by parts, this is also described here: \myref{ARM_big_constants_loading}.

\myparagraph{\NonOptimizing GCC (Linaro) 4.9}

The non-optimizing compiler is more redundant:

\begin{lstlisting}[style=customasmARM]
f:
	sub	sp, sp, #16
	str	w0, [sp,12]
	str	w1, [sp,8]
	str	w2, [sp,4]
	ldr	w1, [sp,12]
	ldr	w0, [sp,8]
	mul	w1, w1, w0
	ldr	w0, [sp,4]
	add	w0, w1, w0
	add	sp, sp, 16
	ret
\end{lstlisting}

The code saves its input arguments in the local stack, 
in case someone (or something) in this function needs using the \TT{W0...W2} 
registers. This prevents overwriting the original
function arguments, which may be needed again in the future.

This is called \emph{Register Save Area.} (\ARMPCS).
The callee, however, is not obliged to save them.
This is somewhat similar to \q{Shadow Space}: \myref{shadow_space}.

Why did the optimizing GCC 4.9 drop this argument saving code?
Because it did some additional optimizing work and concluded
that the function arguments will not be needed in the future 
and also that the registers \TT{W0...W2} will not be used.

\myindex{ARM!\Instructions!MUL}
\myindex{ARM!\Instructions!ADD}

We also see a \TT{MUL}/\TT{ADD} instruction pair instead of single a \TT{MADD}.
}
\RU{\subsubsection{ARM64}

\myparagraph{GCC}

Компилируем пример в GCC 4.8.1 для ARM64:

\lstinputlisting[numbers=left,label=hw_ARM64_GCC,caption=\NonOptimizing GCC 4.8.1 + objdump,style=customasmARM]{patterns/01_helloworld/ARM/hw.lst}

В ARM64 нет режима Thumb и Thumb-2, только ARM, так что тут только 32-битные инструкции.

Регистров тут в 2 раза больше: \myref{ARM64_GPRs}.
64-битные регистры теперь имеют префикс 
\TT{X-}, а их 32-битные части --- \TT{W-}.

\myindex{ARM!\Instructions!STP}
Инструкция \TT{STP} (\emph{Store Pair}) 
сохраняет в стеке сразу два регистра: \RegX{29} и \RegX{30}.
Конечно, эта инструкция может сохранять эту пару где угодно в памяти, но здесь указан регистр \ac{SP}, так что
пара сохраняется именно в стеке.

Регистры в ARM64 64-битные, каждый имеет длину в 8 байт, так что для хранения двух регистров нужно именно 16 байт.

Восклицательный знак (``!'') после операнда означает, что сначала от \ac{SP} будет отнято 16 и только затем
значения из пары регистров будут записаны в стек.

Это называется \emph{pre-index}.
Больше о разнице между \emph{post-index} и \emph{pre-index} 
описано здесь: \myref{ARM_postindex_vs_preindex}.

Таким образом, в терминах более знакомого всем процессора x86, первая инструкция~--- это просто аналог 
пары инструкций \TT{PUSH X29} и \TT{PUSH X30}.
\RegX{29} в ARM64 используется как \ac{FP}, а \RegX{30} 
как \ac{LR}, поэтому они сохраняются в прологе функции и
восстанавливаются в эпилоге.

Вторая инструкция копирует \ac{SP} в \RegX{29} (или \ac{FP}).
Это нужно для установки стекового фрейма функции.

\label{pointers_ADRP_and_ADD}
\myindex{ARM!\Instructions!ADRP/ADD pair}
Инструкции \TT{ADRP} и \ADD нужны для формирования адреса строки \q{Hello!} в регистре \RegX{0}, 
ведь первый аргумент функции передается через этот регистр.
Но в ARM нет инструкций, при помощи которых можно записать в регистр длинное число 
(потому что сама длина инструкции ограничена 4-я байтами. Больше об этом здесь: \myref{ARM_big_constants_loading}).
Так что нужно использовать несколько инструкций.
Первая инструкция (\TT{ADRP}) записывает в \RegX{0} адрес 4-килобайтной страницы где находится строка, 
а вторая (\ADD) просто прибавляет к этому адресу остаток.
Читайте больше об этом: \myref{ARM64_relocs}.

\TT{0x400000 + 0x648 = 0x400648}, и мы видим, что в секции данных \TT{.rodata} по этому адресу как раз находится наша
Си-строка \q{Hello!}.

\myindex{ARM!\Instructions!BL}
Затем при помощи инструкции \TT{BL} вызывается \puts. Это уже рассматривалось ранее: \myref{puts}.

Инструкция \MOV записывает 0 в \RegW{0}. 
\RegW{0} это младшие 32 бита 64-битного регистра \RegX{0}:

\input{ARM_X0_register}

А результат функции возвращается через \RegX{0}, и \main возвращает 0, 
так что вот так готовится возвращаемый результат.

Почему именно 32-битная часть?
Потому что в ARM64, как и в x86-64, тип \Tint оставили 32-битным, для лучшей совместимости.

Следовательно, раз уж функция возвращает 32-битный \Tint, то нужно заполнить только 32 младших бита регистра \RegX{0}.

Для того, чтобы удостовериться в этом, немного отредактируем этот пример и перекомпилируем его.%

Теперь \main возвращает 64-битное значение:

\begin{lstlisting}[caption=\main возвращающая значение типа \TT{uint64\_t},style=customc]
#include <stdio.h>
#include <stdint.h>

uint64_t main()
{
        printf ("Hello!\n");
        return 0;
}
\end{lstlisting}

Результат точно такой же, только \MOV в той строке теперь выглядит так:

\begin{lstlisting}[caption=\NonOptimizing GCC 4.8.1 + objdump]
  4005a4:       d2800000        mov     x0, #0x0      // #0
\end{lstlisting}

\myindex{ARM!\Instructions!LDP}
Далее при помощи инструкции \INS{LDP} (\emph{Load Pair}) восстанавливаются регистры \RegX{29} и \RegX{30}.

Восклицательного знака после инструкции нет. Это означает, что сначала значения достаются из стека, и только потом \ac{SP} увеличивается на 16.

Это называется \emph{post-index}.

\myindex{ARM!\Instructions!RET}
В ARM64 есть новая инструкция: \RET. 
Она работает так же как и \INS{BX LR}, но там добавлен специальный бит,
подсказывающий процессору, что это именно выход из функции, а не просто переход, чтобы процессор
мог более оптимально исполнять эту инструкцию.

Из-за простоты этой функции оптимизирующий GCC генерирует точно такой же код.

}
\DE{\subsubsection{ARM64}

\myparagraph{\Optimizing GCC (Linaro) 4.9}

\lstinputlisting[style=customasmARM]{patterns/12_FPU/3_comparison/ARM/ARM64_GCC_O3_DE.lst}
Der ARM64 \ac{ISA} verfügt über FPU-Befehle, die der Einfachheit halber die Flags der CPU \ac{APSR} anstelle von
\ac{FPSCR} setzen.
Die \ac{FPU} ist hier kein separates Gerät mehr (zumindest logisch).

\myindex{ARM!\Instructions!FCMPE}
Wir finden hier \INS{FCMPE}. Er vergleicht die beiden über \RegD{0} und \RegD{1} übergebenen Werte (dabei handelt es
sich um das erste und zweite Argument der Funktion) und setzt \ac{APSR} die Flags (N, Z, C, V).

\myindex{ARM!\Instructions!FCSEL}
\INS{FCSEL} (\emph{Floating Conditional Select}) kopiert den Wert von \RegD{0} oder \RegD{1} nach \RegD{0}, abhängig von
der Bedingung (\GTT{GT}---Greater Than), und verwendet wiederum Flags im \ac{APSR} Register anstatt derer von
\ac{FPSCR}.

Dies ist im Vergleich zum Befehlssatz alter CPUs deutlich praktischer.

Falls die Bedingung wahr ist (\GTT{GT}), dann wird der Wert von \RegD{0} nach \RegD{0} kopiert (d.h. es geschieht
nichts).
Falls die Bedingung falsch ist, wird der Wert von \RegD{1} nach \RegD{0} kopiert.

\myparagraph{\NonOptimizing GCC (Linaro) 4.9}

\lstinputlisting[style=customasmARM]{patterns/12_FPU/3_comparison/ARM/ARM64_GCC_DE.lst}
Der nicht optimierende GCC ist weniger kompakt.

Zunächst speichert die Funktion ihre Eingabewerte auf dem lokalen Stack (\emph{Register Save Area}), danach lädt der Code
die Werte erneut in die Register \RegX{0}/\RegX{1} und kopiert sie schließlich nach \RegD{0}/\RegD{1}, um sie mittels
\INS{FCMPE} zu vergleichen.
Eine Menge redundanter Code, aber so arbeitet ein nicht optimierender Compiler nun einmal.
\INS{FCMPE} vergleich die Werte und setzt die \ac{APSR} Flags.
Zu diesem Zeitpunkt entscheidet sich der Compiler noch nicht für den praktischeren \INS{FCSEL} Befehl und arbeitet
stattdessen mit herkömmlichen Methoden:
er verwendet den \INS{BLE} Befehl (\emph{Branch if Less than or Equal}).
Im ersten Fall ($a>b$) wird der Wert von $a$ nach \RegX{0} geladen. 
Im anderen Fall ($a<=b$) wird der Wert von $b$ nach \RegX{0} geladen.
Schließlich wird der Wert aus \RegX{0} nach \RegD{0} kopiert, denn der Rückgabewert muss sich in diesem Register
befinden.


\mysubparagraph{\Exercise}
Dem Leser bleibt als Übung, den vorstehenden Code zu optimieren, indem manuell die redundanten Instruktionen entfernt
werden ohne dabei neue einzuführen (wie etwa \INS{FCSEL}).

\myparagraph{\Optimizing GCC (Linaro) 4.9---float}
Wir wollen nun dieses Beispiel umschreiben, indem wir \Tfloat anstelle von \Tdouble verwenden.

\begin{lstlisting}[style=customc]
float f_max (float a, float b)
{
	if (a>b)
		return a;

	return b;
};
\end{lstlisting}

\lstinputlisting[style=customasmARM]{patterns/12_FPU/3_comparison/ARM/ARM64_GCC_O3_float_DE.lst}
Es ist der gleiche Code, aber hier werden die S-Register anstelle der D-Register verwendet.
Das ist darauf zurückzuführen, dass der \Tfloat Typ in 32-Bit-S-Registern übergeben wird (welche in Wirklichkeit nichts
anderes als die niederen Teile der 64-Bit-D-Register sind).
}
\FR{\subsubsection{ARM64}

\myparagraph{GCC (Linaro) 4.9 \Optimizing}

\myindex{Multiplication-addition fusionnées} % TODO FIXME verify
\myindex{ARM!\Instructions!MADD}
Tout ce qu'il y a ici est simple.
\TT{MADD} est juste une instruction qui effectue une multiplication/addition fusionnées
(similaire à l'instruction \TT{MLA} que nous avons déjà vue).
Tous les 3 arguments sont passés dans la partie 32-bit de X-registres.
Effectivement, le type des arguments est \emph{int} 32-bit.
Le résultat est renvoyé dans \TT{W0}.

\lstinputlisting[caption=GCC (Linaro) 4.9 \Optimizing,style=customasmARM]{patterns/05_passing_arguments/ARM/ARM64_O3_FR.s}

Étendons le type de toutes les données à 64-bit \TT{uint64\_t} et testons:

\lstinputlisting[style=customc]{patterns/05_passing_arguments/ex64.c}

\begin{lstlisting}[style=customasmARM]
f:
	madd	x0, x0, x1, x2
	ret
main:
	mov	x1, 13396
	adrp	x0, .LC8
	stp	x29, x30, [sp, -16]!
	movk	x1, 0x27d0, lsl 16
	add	x0, x0, :lo12:.LC8
	movk	x1, 0x122, lsl 32
	add	x29, sp, 0
	movk	x1, 0x58be, lsl 48
	bl	printf
	mov	w0, 0
	ldp	x29, x30, [sp], 16
	ret

.LC8:
	.string	"%lld\n"
\end{lstlisting}

La fonction \ttf{} est la même, seulement les X-registres 64-bit sont utilisés entièrement
maintenant.
Les valeurs longues sur 64-bit sont chargées dans les registres par partie, c'est
également décrit ici: \myref{ARM_big_constants_loading}.

\myparagraph{GCC (Linaro) 4.9 \NonOptimizing}

Le code sans optimisation est plus redondant:

\begin{lstlisting}[style=customasmARM]
f:
	sub	sp, sp, #16
	str	w0, [sp,12]
	str	w1, [sp,8]
	str	w2, [sp,4]
	ldr	w1, [sp,12]
	ldr	w0, [sp,8]
	mul	w1, w1, w0
	ldr	w0, [sp,4]
	add	w0, w1, w0
	add	sp, sp, 16
	ret
\end{lstlisting}

Le code sauve ses arguments en entrée dans la pile locale, dans le cas où quelqu'un
(ou quelque chose) dans cette fonction aurait besoin d'utiliser les registres \TT{W0...W2}.
Cela évite d'écraser les arguments originels de la fonction, qui pourraient être
de nouveau utilisés par la suite.

Cela est appelé \emph{Zone de sauvegarde de registre.} (\ARMPCS).
L'appelée, toutefois, n'est pas obligée de les sauvegarder.
C'est un peu similaire au \q{Shadow Space}: \myref{shadow_space}.

Pourquoi est-ce que GCC 4.9 avec l'option d'optimisation supprime ce code de sauvegarde?
Parce qu'il a fait plus d'optimisation et en a conclu que les arguments de la fonction
n'allaient pas être utilisés par la suite et donc que les registres \TT{W0...W2}
ne vont pas être utilisés.

\myindex{ARM!\Instructions!MUL}
\myindex{ARM!\Instructions!ADD}

Nous avons donc une paire d'instructions \TT{MUL}/\TT{ADD} au lieu d'un seul \TT{MADD}.
}
\IT{\myparagraph{ARM64: \Optimizing GCC (Linaro) 4.9}

\lstinputlisting[caption=f\_signed(),style=customasmARM]{patterns/07_jcc/simple/ARM/ARM64_GCC_O3_signed_IT.lst}

\lstinputlisting[caption=f\_unsigned(),style=customasmARM]{patterns/07_jcc/simple/ARM/ARM64_GCC_O3_unsigned_IT.lst}

I commenti nel codice sono stati inseriti dall'autore di questo libro.
E' impressionante notare come il compilatore non si sia reso conto che alcuni condizioni sono del tutto impossibili, e per questo motivo
si trovano delle parti con codice "morto" (dead code), che non può mai essere eseguito.

\mysubparagraph{\Exercise}

Prova ad ottimizzare manualmente queste funzioni per ottenere una versione più compatta, rimuovendo istruzioni ridondanti e senza aggiungerne di nuove.

}
\JA{\myparagraph{ARM64: \Optimizing GCC (Linaro) 4.9}

\lstinputlisting[caption=f\_signed(),style=customasmARM]{patterns/07_jcc/simple/ARM/ARM64_GCC_O3_signed_JA.lst}

\lstinputlisting[caption=f\_unsigned(),style=customasmARM]{patterns/07_jcc/simple/ARM/ARM64_GCC_O3_unsigned_JA.lst}

コメントはこの本の著者によって追加されました。 
目立ったことは、コンパイラはいくつかの条件がまったく不可能であることを認識していないため、
決して実行できない場所ではデッドコードがあることです。

\mysubparagraph{\Exercise}

これらの機能をサイズが少なくなるように手動で最適化し、新しい命令を追加せずに冗長な命令を削除してください。
}

\EN{\subsubsection{MIPS}
% FIXME better start at non-optimizing version?

The function uses a lot of S- registers which must be preserved, so that's why its 
values are saved in the function prologue and restored in the epilogue.

\lstinputlisting[caption=\Optimizing GCC 4.4.5 (IDA),style=customasmMIPS]{patterns/13_arrays/1_simple/MIPS_O3_IDA_EN.lst}

Something interesting: there are two loops and the first one doesn't need $i$, it needs only 
$i*2$ (increased by 2 at each iteration) and also the address in memory (increased by 4 at each iteration).

So here we see two variables, one (in \$V0) increasing by 2 each time, and another (in \$V1) --- by 4.

The second loop is where \printf is called and it reports the value of $i$ to the user, 
so there is a variable
which is increased by 1 each time (in \$S0) and also a memory address (in \$S1) increased by 4 each time.

That reminds us of loop optimizations: \myref{loop_iterators}.

Their goal is to get rid of multiplications.

}
\RU{\subsubsection{MIPS}
% FIXME better start at non-optimizing version?
Функция использует много S-регистров, которые должны быть сохранены. Вот почему их значения сохраняются
в прологе функции и восстанавливаются в эпилоге.

\lstinputlisting[caption=\Optimizing GCC 4.4.5 (IDA),style=customasmMIPS]{patterns/13_arrays/1_simple/MIPS_O3_IDA_RU.lst}

Интересная вещь: здесь два цикла и в первом не нужна переменная $i$, а нужна только переменная
$i*2$ (скачущая через 2 на каждой итерации) и ещё адрес в памяти (скачущий через 4 на каждой итерации).

Так что мы видим здесь две переменных: одна (в \$V0) увеличивается на 2 каждый раз, и вторая (в \$V1) --- на 4.

Второй цикл содержит вызов \printf. Он должен показывать значение $i$ пользователю,
поэтому здесь есть переменная, увеличивающаяся на 1 каждый раз (в \$S0), а также адрес в памяти (в \$S1) 
увеличивающийся на 4 каждый раз.

Это напоминает нам оптимизацию циклов: \myref{loop_iterators}.
Цель оптимизации в том, чтобы избавиться от операций умножения.

}
\DE{\subsubsection{MIPS}

\lstinputlisting[caption=\Optimizing GCC 4.4.5 (IDA),style=customasmMIPS]{patterns/08_switch/1_few/MIPS_O3_IDA_DE.lst}

\myindex{MIPS!\Instructions!JR}
Die Funktion endet stets mit einem Aufruf von \puts, weshalb wir hier einen Sprung zu \puts (\INS{JR}: \q{Jump
Register}) anstelle von \q{jump and link} finden.
Dieses Feature haben wir bereit in \myref{JMP_instead_of_RET} besprochen.

\myindex{MIPS!Load delay slot}
Wir finden auch oft \INS{NOP} Befehle nach \INS{LW} Befehlen.
Dies ist \q{load delay slot}: ein anderer \emph{delay slot} in MIPS.
\myindex{MIPS!\Instructions!LW}
Ein Befehl neben \INS{LW} kann in dem Moment ausgeführt werden, in dem \INS{LW} Werte aus dem Speicher lädt.
Der nächste Befehl muss aber nicht das Ergebnis von \INS{LW} verwenden.
Moderne MIPS CPUs haben die Eigenschaft abwarten zu können, ob der folgende Befehl das Ergebnis von \INS{LW} verwendet,
sodass dieses Vorgehen überholt wirkt, aber GCC fügt für ältere MIPS CPUs immer noch NOPs hinzu.
Im Allgemeinen können diese aber ignoriert werden.}
\FR{\subsection{MIPS}

\lstinputlisting[caption=GCC 4.4.5 \Optimizing (IDA),style=customasmMIPS]{\CURPATH/MIPS_O3_IDA_FR.lst}

Les registres préfixés avec S- sont aussi appelés  \q{saved temporaries} (sauvé temporairement),
donc la valeur de \$S0 est sauvée dans la pile locale et restaurée à la fin.
}
\IT{\subsubsection{MIPS}

Una caratteristica distintiva di MIPS è l'assenza dei flag.
Apparentemente è una scelta fatta per semplificare l'analisi della dipendenza dai dati.

\myindex{x86!\Instructions!SETcc}
\myindex{MIPS!\Instructions!SLT}
\myindex{MIPS!\Instructions!SLTU}

Esistono istruzioni simili a \INS{SETcc} in x86: \INS{SLT} (\q{Set on Less Than}: versione signed) e 
\INS{SLTU} (versione unsigned).
Queste istruzioni settano il valore del registro di destinazione a 1 se la condizione è vera, a 0 se è falsa.

\myindex{MIPS!\Instructions!BEQ}
\myindex{MIPS!\Instructions!BNE}

Il registro di destinazione viene quindi controllato usando \INS{BEQ} (\q{Branch on Equal}) oppure \INS{BNE} (\q{Branch on Not Equal}) 
ed in base al caso si può verificare un salto.
Questa coppia di istruzioni è usata in MIPS per eseguire confronti e conseguenti branch.
Iniziamo con la versione signed della nostra funzione:

\lstinputlisting[caption=\NonOptimizing GCC 4.4.5 (IDA),style=customasmMIPS]{patterns/07_jcc/simple/O0_MIPS_signed_IDA_IT.lst}

\INS{SLT REG0, REG0, REG1} è stata ridotta da Ida nella sua forma breve:\\
\INS{SLT REG0, REG1}.
\myindex{MIPS!\Pseudoinstructions!BEQZ}

Notiamo anche la pseudo istruzione \INS{BEQZ} (\q{Branch if Equal to Zero}),\\
che è in realtà \INS{BEQ REG, \$ZERO, LABEL}.

\myindex{MIPS!\Instructions!SLTU}

La versione unsigned è uguale, \INS{SLTU} (versione unsigned, da cui la \q{U} nel nome) è usata al posto di \INS{SLT}:

\lstinputlisting[caption=\NonOptimizing GCC 4.4.5 (IDA),style=customasmMIPS]{patterns/07_jcc/simple/O0_MIPS_unsigned_IDA.lst}

}
\JA{\subsubsection{MIPS}

1つの特徴的なMIPS機能は、フラグが存在しないことです。 
明らかに、データ依存性の分析を簡素化するために行われました。

\myindex{x86!\Instructions!SETcc}
\myindex{MIPS!\Instructions!SLT}
\myindex{MIPS!\Instructions!SLTU}

x86には\INS{SETcc}に似た命令があります。\INS{SLT}(\q{Set on Less Than}:符号付きバージョン)と\INS{SLTU}(符号なしバージョン)です。 
これらの命令は、条件が真であれば宛先レジスタの値を1に設定し、そうでない場合は0に設定します。

\myindex{MIPS!\Instructions!BEQ}
\myindex{MIPS!\Instructions!BNE}

宛先レジスタは、\INS{BEQ} (\q{Branch on Equal}) または \INS{BNE} (\q{Branch on Not Equal})を使用してチェックされ、
ジャンプが発生することがあります。 
したがって、この命令ペアは比較および分岐のためにMIPSで使用されなければなりません。
最初に関数の符号付きバージョンから始めましょう。

\lstinputlisting[caption=\NonOptimizing GCC 4.4.5 (IDA),style=customasmMIPS]{patterns/07_jcc/simple/O0_MIPS_signed_IDA_JA.lst}

\INS{SLT REG0, REG0, REG1}は、IDAによって短縮形式
\INS{SLT REG0, REG1}に縮小されます。
\myindex{MIPS!\Pseudoinstructions!BEQZ}

実際には\INS{BEQ REG, \$ZERO, LABEL}の
\INS{BEQZ}擬似命令もあります(\q{Branch if Equal to Zero})。

\myindex{MIPS!\Instructions!SLTU}

符号なしバージョンはまったく同じですが、\INS{SLT}の代わりに\INS{SLTU}(符号なしバージョン、したがって\q{U}という名前)が使用されます。

\lstinputlisting[caption=\NonOptimizing GCC 4.4.5 (IDA),style=customasmMIPS]{patterns/07_jcc/simple/O0_MIPS_unsigned_IDA.lst}
}


\EN{\subsection{Win32 PE}
\label{win32_pe}
\myindex{Windows!Win32}

\acs{PE} is an executable file format used in Windows.
The difference between .exe, .dll and .sys is that .exe and .sys usually do not have exports, only imports.

\myindex{OEP}

A \ac{DLL}, just like any other PE-file, has an entry point (\ac{OEP}) (the function DllMain() is located there)
but this function usually does nothing.
.sys is usually a device driver.
As of drivers, Windows requires the checksum to be present in the PE file and for it to be correct
\footnote{For example, Hiew(\myref{Hiew}) can calculate it}.

\myindex{Windows!Windows Vista}
Starting at Windows Vista, a driver's files must also be signed with a digital signature. It will fail to load otherwise.

\myindex{MS-DOS}
Every PE file begins with tiny DOS program that prints a
message like \q{This program cannot be run in DOS mode.}---if you run this program in DOS or Windows 3.1 (\ac{OS}-es which are not aware of the PE format),
this message will be printed.

\subsubsection{Terminology}

\myindex{VA}
\myindex{Base address}
\myindex{RVA}
\myindex{Windows!IAT}
\myindex{Windows!INT}

\begin{itemize}
\item Module---a separate file, .exe or .dll.

\item Process---a program loaded into memory and currently running.  Commonly consists of one .exe file and bunch of .dll files.

\item Process memory---the memory a process works with.  Each process has its own.
There usually are loaded modules, memory of the stack, \gls{heap}(s), etc.

\item \ac{VA}---an address which is to be used in program while runtime.

\item Base address (of module)---the address within the process memory at which the module is to be loaded.
\ac{OS} loader may change it, if the base address is already occupied by another module just loaded before.

\item \ac{RVA}---the \ac{VA}-address minus the base address.

Many addresses in PE-file tables use \ac{RVA}-addresses.

%\item
%Data directory --- ...

\item \ac{IAT}---an array of addresses of imported symbols \footnote{\PietrekPE}.
Sometimes, the \TT{IMAGE\_DIRECTORY\_ENTRY\_IAT} data directory points at the \ac{IAT}.
\label{IDA_idata}
It is worth noting that \ac{IDA} (as of 6.1) may allocate a pseudo-section named \TT{.idata} for
\ac{IAT}, even if the \ac{IAT} is a part of another section!

\item \ac{INT}---an array of names of symbols to be imported\footnote{\PietrekPE}.
\end{itemize}

\subsubsection{Base address}

The problem is that several module authors can prepare DLL files for others to use and it is not possible
to reach an agreement which addresses is to be assigned to whose modules.

So that is why if two necessary DLLs for a process have the same base address,
one of them will be loaded at this base address, and the other---at some other free space in process memory,
and each virtual addresses in the second DLL will be corrected.

\par With \ac{MSVC} the linker often generates the .exe files with a base address of \TT{0x400000}
\footnote{The origin of this address choice is described here: \href{http://go.yurichev.com/17041}{MSDN}},
and with the code section starting at \TT{0x401000}.
This means that the \ac{RVA} of the start of the code section is \TT{0x1000}.

DLLs are often generated by MSVC's linker with a base address of \TT{0x10000000}
\footnote{This can be changed by the /BASE linker option}.

\myindex{ASLR}

There is also another reason to load modules at various base addresses, in this case random ones.
It is \ac{ASLR}\footnote{\href{http://go.yurichev.com/17140}{wikipedia}}.

\myindex{Shellcode}

A shellcode trying to get executed on a compromised system must call system functions, hence, know their addresses.

In older \ac{OS} (in \gls{Windows NT} line: before Windows Vista),
system DLL (like kernel32.dll, user32.dll) were always loaded at known addresses,
and if we also recall
that their versions rarely changed, the addresses of functions were
fixed and shellcode could call them directly.

In order to avoid this, the \ac{ASLR}
method loads your program and all modules it needs at random base addresses, different every time.

\ac{ASLR} support is denoted in a PE file by setting the flag
\par \TT{IMAGE\_DLL\_CHARACTERISTICS\_DYNAMIC\_BASE} \InSqBrackets{see \Russinovich}.

\subsubsection{Subsystem}

There is also a \emph{subsystem} field, usually it is:

\myindex{Native API}

\begin{itemize}
\item native\footnote{Meaning, the module use Native API instead of Win32} (.sys-driver),

\item console (console application) or

\item \ac{GUI} (non-console).
\end{itemize}

\subsubsection{OS version}

A PE file also specifies the minimal Windows version it needs in order to be loadable.

The table of version numbers stored in the PE file and corresponding Windows codenames is
here\footnote{\href{http://go.yurichev.com/17044}{wikipedia}}.

\myindex{Windows!Windows NT4}
\myindex{Windows!Windows 2000}
For example, \ac{MSVC} 2005 compiles .exe files for running on Windows NT4 (version 4.00), but \ac{MSVC} 2008 does not
(the generated files have a version of 5.00, at least Windows 2000 is needed to run them).

\myindex{Windows!Windows XP}

\ac{MSVC} 2012 generates .exe files of version 6.00 by default,
targeting at least Windows Vista.
However, by changing the compiler's options\footnote{\href{http://go.yurichev.com/17045}{MSDN}},
it is possible to force it to compile for Windows XP.

\subsubsection{Sections}

Division in sections, as it seems, is present in all executable file formats.

It is devised in order to separate code from data, and data---from constant data.

\begin{itemize}
\item Either the \emph{IMAGE\_SCN\_CNT\_CODE} or \emph{IMAGE\_SCN\_MEM\_EXECUTE} flags will be set on the code section---this is executable code.

\item On data section---\emph{IMAGE\_SCN\_CNT\_INITIALIZED\_DATA},\\
\emph{IMAGE\_SCN\_MEM\_READ} and \emph{IMAGE\_SCN\_MEM\_WRITE} flags.

\item On an empty section with uninitialized data---\\
\emph{IMAGE\_SCN\_CNT\_UNINITIALIZED\_DATA}, \emph{IMAGE\_SCN\_MEM\_READ} \\
        and \emph{IMAGE\_SCN\_MEM\_WRITE}.

\item On a constant data section (one that's protected from writing), the flags \\
\emph{IMAGE\_SCN\_CNT\_INITIALIZED\_DATA} and \emph{IMAGE\_SCN\_MEM\_READ} can be set, \\
but not \emph{IMAGE\_SCN\_MEM\_WRITE}.
A process going to crash if it tries to write to this section.
\end{itemize}

\myindex{TLS}
\myindex{BSS}
Each section in PE-file may have a name, however, it is not very important.
Often (but not always) the code section is named \TT{.text},
the data section---\TT{.data}, the constant data section --- \TT{.rdata} \emph{(readable data)}.
Other popular section names are:

\myindex{MIPS}
\begin{itemize}
\item \TT{.idata}---imports section.
\ac{IDA} may create a pseudo-section named like this: \myref{IDA_idata}.
\item \TT{.edata}---exports section (rare)
\item \TT{.pdata}---section holding all information about exceptions in Windows NT for MIPS, \ac{IA64} and x64: \myref{SEH_win64}
\item \TT{.reloc}---relocs section
\item \TT{.bss}---uninitialized data (\ac{BSS})
\item \TT{.tls}---thread local storage (\ac{TLS})
\item \TT{.rsrc}---resources
\item \TT{.CRT}---may present in binary files compiled by ancient MSVC versions
\end{itemize}

PE file packers/encryptors often garble section names or replace the names with their own.

\ac{MSVC} allows you to declare data in arbitrarily named section
\footnote{\href{http://go.yurichev.com/17047}{MSDN}}.

Some compilers and linkers can add a section with debugging symbols and
other debugging information (MinGW for instance).
\myindex{Windows!PDB}
However it is not so in latest versions of \ac{MSVC} (separate \gls{PDB} files are used there for this purpose).\\
\\
That is how a PE section is described in the file:

\begin{lstlisting}[style=customc]
typedef struct _IMAGE_SECTION_HEADER {
  BYTE  Name[IMAGE_SIZEOF_SHORT_NAME];
  union {
    DWORD PhysicalAddress;
    DWORD VirtualSize;
  } Misc;
  DWORD VirtualAddress;
  DWORD SizeOfRawData;
  DWORD PointerToRawData;
  DWORD PointerToRelocations;
  DWORD PointerToLinenumbers;
  WORD  NumberOfRelocations;
  WORD  NumberOfLinenumbers;
  DWORD Characteristics;
} IMAGE_SECTION_HEADER, *PIMAGE_SECTION_HEADER;
\end{lstlisting}
\footnote{\href{http://go.yurichev.com/17048}{MSDN}}

\myindex{Hiew}
A word about terminology: \emph{PointerToRawData} is called \q{Offset} in Hiew
and \emph{VirtualAddress} is called \q{RVA} there.

\subsubsection{Data section}

Data section in file can be smaller than in memory.
For example, some variables can be initialized, some are not.
Compiler and linker will collect them all into one section, but the first part of it is initialized and allocated in file,
while another is absent in file (of course, to make it smaller).
\emph{VirtualSize} will be equal to the size of section in memory, and \emph{SizeOfRawData} --- to
size of section in file.

IDA can show the border between initialized and not initialized parts like that:

\begin{lstlisting}[style=customasmx86]
...

.data:10017FFA                 db    0
.data:10017FFB                 db    0
.data:10017FFC                 db    0
.data:10017FFD                 db    0
.data:10017FFE                 db    0
.data:10017FFF                 db    0
.data:10018000                 db    ? ;
.data:10018001                 db    ? ;
.data:10018002                 db    ? ;
.data:10018003                 db    ? ;
.data:10018004                 db    ? ;
.data:10018005                 db    ? ;

...
\end{lstlisting}

\subsubsection{Relocations (relocs)}
\label{subsec:relocs}

\ac{AKA} FIXUP-s (at least in Hiew).

They are also present in almost all executable file formats
\footnote{Even in .exe files for MS-DOS}.
Exceptions are shared dynamic libraries compiled with \ac{PIC}, or any other \ac{PIC}-code.

What are they for?

Obviously, modules can be loaded on various base addresses, but how to deal with global variables, for example?
They must be accessed by address.  One solution is \PICcode{} (\myref{sec:PIC}).
But it is not always convenient.

That is why a relocations table is present.
There the addresses of points that must be corrected are enumerated,
in case of loading at a different base address.

% TODO тут бы пример с HIEW или objdump..
For example, there is a global variable at address \TT{0x410000} and this is how it is accessed:

\begin{lstlisting}[style=customasmx86]
A1 00 00 41 00         mov         eax,[000410000]
\end{lstlisting}

The base address of the module is \TT{0x400000}, the \ac{RVA} of the global variable is \TT{0x10000}.

If the module is loaded at base address \TT{0x500000}, the real address of the global variable must be \TT{0x510000}.

\myindex{x86!\Instructions!MOV}

As we can see, the address of variable is encoded in the instruction \TT{MOV}, after the byte \TT{0xA1}.

That is why the address of the 4 bytes after \TT{0xA1}, is written in the relocs table.

If the module is loaded at a different base address, the \ac{OS} loader enumerates all addresses in the table,

finds each 32-bit word the address points to, subtracts the original base address from it
(we get the \ac{RVA} here), and adds the new base address to it.

If a module is loaded at its original base address, nothing happens.

All global variables can be treated like that.

Relocs may have various types, however, in Windows for x86 processors, the type is usually \\
\emph{IMAGE\_REL\_BASED\_HIGHLOW}.

\myindex{Hiew}

By the way, relocs are darkened in Hiew, for example: \figref{fig:scanf_ex3_hiew_1}.

\myindex{\olly}
\olly underlines the places in memory to which relocs are to be applied, for example: \figref{fig:switch_lot_olly3}.

\subsubsection{Exports and imports}

\label{PE_exports_imports}
As we all know, any executable program must use the \ac{OS}'s services and other DLL-libraries somehow.

It can be said that functions from one module (usually DLL) must be connected somehow to the points of their
calls in other modules (.exe-file or another DLL).

For this, each DLL has an \q{exports} table, which consists of functions plus their addresses in a module.

And every .exe file or DLL has \q{imports}, a table of functions it needs for execution including
list of DLL filenames.

After loading the main .exe-file, the \ac{OS} loader processes imports table:
it loads the additional DLL-files, finds function names
among the DLL exports and writes their addresses down in the \ac{IAT} of the main .exe-module.

\myindex{Windows!Win32!Ordinal}

As we can see, during loading the loader must compare a lot of function names, but string comparison is not a very
fast procedure, so there is a support for \q{ordinals} or \q{hints},
which are function numbers stored in the table, instead of their names.

That is how they can be located faster when loading a DLL.
Ordinals are always present in the \q{export} table.

\myindex{MFC}
For example, a program using the \ac{MFC} library usually loads mfc*.dll by ordinals,
and in such programs there are no \ac{MFC} function names in \ac{INT}.

% TODO example!
When loading such programs in \IDA, it will ask for a path to the mfc*.dll files
in order to determine the function names.

If you don't tell \IDA the path to these DLLs, there will be \emph{mfc80\_123} instead of function names.

\myparagraph{Imports section}

Often a separate section is allocated for the imports table and everything related to it (with name like \TT{.idata}),
however, this is not a strict rule.

Imports are also a confusing subject because of the terminological mess. Let's try to collect all information in one place.

\begin{figure}[H]
\centering
\myincludegraphics{OS/PE/unnamed0.png}
\caption{
A scheme that unites all PE-file structures related to imports}
\end{figure}

The main structure is the array \emph{IMAGE\_IMPORT\_DESCRIPTOR}.
Each element for each DLL being imported.

Each element holds the \ac{RVA} address of the text string (DLL name) (\emph{Name}).

\emph{OriginalFirstThunk} is the \ac{RVA} address of the \ac{INT} table.
This is an array of \ac{RVA} addresses, each of which points to a text string with a function name.
Each string is prefixed by a 16-bit integer
(\q{hint})---\q{ordinal} of function.

While loading, if it is possible to find a function by ordinal,
then the strings comparison will not occur. The array is terminated by zero.

There is also a pointer to the \ac{IAT} table named \emph{FirstThunk}, it is just the \ac{RVA} address
of the place where the loader writes the addresses of the resolved functions.

The points where the loader writes addresses are marked by \IDA like this: \emph{\_\_imp\_CreateFileA}, etc.

There are at least two ways to use the addresses written by the loader.

\myindex{x86!\Instructions!CALL}
\begin{itemize}
\item The code will have instructions like \emph{call \_\_imp\_CreateFileA},
and since the field with the address of the imported function is a global variable in some sense,
the address of the \emph{call} instruction (plus 1 or 2) is to be added to the relocs table,
for the case when the module is loaded at a different base address.

But, obviously, this may enlarge relocs table significantly.

Because there are might be a lot of calls to imported functions in the module.

Furthermore, large relocs table slows down the process of loading modules.

\myindex{x86!\Instructions!JMP}
\myindex{thunk-functions}
\item For each imported function, there is only one jump allocated, using the \JMP instruction
plus a reloc to it.
Such points are also called \q{thunks}.

All calls to the imported functions are just \CALL instructions to the corresponding \q{thunk}.
In this case, additional relocs are not necessary because these \CALL{}-s
have relative addresses and do not need to be corrected.
\end{itemize}

These two methods can be combined.

Possible, the linker creates individual \q{thunk}s if there are too many calls to the function,
but not done by default. \\
\\
By the way, the array of function addresses to which FirstThunk is pointing is not necessary to be located in the \ac{IAT} section.
For example, the author of these lines once wrote the PE\_add\_import\footnote{\href{http://go.yurichev.com/17049}{yurichev.com}}
utility for adding imports to an existing .exe-file.

Some time earlier, in the previous versions of the utility,
at the place of the function you want to substitute with a call to another DLL,
my utility wrote the following code:

\begin{lstlisting}[style=customasmx86]
MOV EAX, [yourdll.dll!function]
JMP EAX
\end{lstlisting}

FirstThunk points to the first instruction. In other words, when loading yourdll.dll,
the loader writes the address of the \emph{function} function right in the code.

It also worth noting that a code section is usually write-protected, so my utility adds the \\
\emph{IMAGE\_SCN\_MEM\_WRITE}
flag for code section. Otherwise, the program to crash while loading with error code
5 (access denied). \\
\\
One might ask: what if I supply a program with a set of DLL files which is not supposed to change (including addresses of all DLL functions),
is it possible to speed up the loading process?

Yes, it is possible to write the addresses of the functions to be imported into the FirstThunk arrays in advance.
The \emph{Timestamp} field is present in the \\
\emph{IMAGE\_IMPORT\_DESCRIPTOR} structure.

If a value is present there, then the loader compares this value with the date-time of the DLL file.

If the values are equal, then the loader does not do anything, and the loading of the process can be faster.
This is called \q{old-style binding}
\footnote{\href{http://go.yurichev.com/17050}{MSDN}. There is also the \q{new-style binding}.}.
\myindex{BIND.EXE}

The BIND.EXE utility in Windows SDK is for this.
For speeding up the loading of your program, Matt Pietrek in \PietrekPEURL, suggests to do the binding shortly after your program
installation on the computer of the end user. \\
\\
PE-files packers/encryptors may also compress/encrypt imports table.

In this case, the Windows loader, of course, will not load all necessary DLLs.
\myindex{Windows!Win32!LoadLibrary}
\myindex{Windows!Win32!GetProcAddress}

Therefore, the packer/encryptor does this on its own, with the help of
\emph{LoadLibrary()} and the \emph{GetProcAddress()} functions.

That is why these two functions are often present in \ac{IAT} in packed files.\\
\\
In the standard DLLs from the Windows installation, \ac{IAT} often is located right at the beginning of the PE file.
Supposedly, it is made so for optimization.

While loading, the .exe file is not loaded into memory as a whole (recall huge install programs which are
started suspiciously fast), it is \q{mapped}, and loaded into memory in parts as they are accessed.

Probably, Microsoft developers decided it will be faster.

\subsubsection{Resources}

\label{PEresources}

Resources in a PE file are just a set of icons, pictures, text strings, dialog descriptions.

Perhaps they were separated from the main code, so all these things could be multilingual,
and it would be simpler to pick text or picture for the language that is currently set in the \ac{OS}. \\
\\
As a side effect, they can be edited easily and saved back to the executable file, even if one does not have special knowledge,
by using the ResHack editor, for example (\myref{ResHack}).

\subsubsection{.NET}

\myindex{.NET}

.NET programs are not compiled into machine code but into a special bytecode.
\myindex{OEP}
Strictly speaking, there is bytecode instead of the usual x86 code
in the .exe file, however, the entry point (\ac{OEP}) points to this tiny fragment of x86 code:

\begin{lstlisting}[style=customasmx86]
jmp         mscoree.dll!_CorExeMain
\end{lstlisting}

The .NET loader is located in mscoree.dll, which processes the PE file.
\myindex{Windows!Windows XP}

It was so in all pre-Windows XP \ac{OS}es. Starting from XP, the \ac{OS} loader is able to detect the .NET file
and run it without executing that \JMP instruction
\footnote{\href{http://go.yurichev.com/17051}{MSDN}}.

\myindex{TLS}
\subsubsection{TLS}

This section holds initialized data for the \ac{TLS}(\myref{TLS}) (if needed).
When a new thread start, its \ac{TLS} data is initialized using the data from this section. \\
\\
\myindex{TLS!Callbacks}

Aside from that, the PE file specification also provides initialization of the
\ac{TLS} section, the so-called TLS callbacks.

If they are present, they are to be called before the control is passed to the main entry point (\ac{OEP}).

This is used widely in the PE file packers/encryptors.

\subsubsection{Tools}

\myindex{objdump}
\myindex{Cygwin}
\myindex{Hiew}
\label{ResHack}

\begin{itemize}
\item objdump (present in cygwin) for dumping all PE-file structures.

\item Hiew(\myref{Hiew}) as editor.

\item pefile---Python-library for PE-file processing \footnote{\url{http://go.yurichev.com/17052}}.

\item ResHack \acs{AKA} Resource Hacker---resources editor\footnote{\url{http://go.yurichev.com/17052}}.

\item PE\_add\_import\footnote{\url{http://go.yurichev.com/17049}}---
simple tool for adding symbol(s) to PE executable import table.

\item PE\_patcher\footnote{\href{http://go.yurichev.com/17054}{yurichev.com}}---simple tool for patching PE executables.

\item PE\_search\_str\_refs\footnote{\href{http://go.yurichev.com/17055}{yurichev.com}}---simple tool for searching for a function in PE executables which use some text string.
\end{itemize}

\subsubsection{Further reading}

% FIXME: bibliography per chapter or section
\begin{itemize}
\item Daniel Pistelli---The .NET File Format \footnote{\url{http://go.yurichev.com/17056}}
\end{itemize}

}
\RU{\mysection{Пример вычисления адреса сети}

Как мы знаем, TCP/IP-адрес (IPv4) состоит из четырех чисел в пределах $0 \ldots 255$, т.е. 4 байта.

4 байта легко помещаются в 32-битную переменную, так что адрес хоста в IPv4, сетевая маска или адрес сети
могут быть 32-битными числами.

С точки зрения пользователя, маска сети определяется четырьмя числами в формате вроде \\
255.255.255.0,
но сетевые инженеры (сисадмины) используют более компактную нотацию (\ac{CIDR}),
вроде  \q{/8}, \q{/16}, итд.

Эта нотация просто определяет количество бит в сетевой маске, начиная с \ac{MSB}.

\small
\begin{center}
\begin{tabular}{ | l | l | l | l | l | l | }
\hline
\HeaderColor Маска & 
\HeaderColor Хосты & 
\HeaderColor Свободно &
\HeaderColor Сетевая маска &
\HeaderColor В шест.виде &
\HeaderColor \\
\hline
/30  & 4        & 2        & 255.255.255.252  & 0xfffffffc  & \\
\hline
/29  & 8        & 6        & 255.255.255.248  & 0xfffffff8  & \\
\hline
/28  & 16       & 14       & 255.255.255.240  & 0xfffffff0  & \\
\hline
/27  & 32       & 30       & 255.255.255.224  & 0xffffffe0  & \\
\hline
/26  & 64       & 62       & 255.255.255.192  & 0xffffffc0  & \\
\hline
/24  & 256      & 254      & 255.255.255.0    & 0xffffff00  & сеть класса C \\
\hline
/23  & 512      & 510      & 255.255.254.0    & 0xfffffe00  & \\
\hline
/22  & 1024     & 1022     & 255.255.252.0    & 0xfffffc00  & \\
\hline
/21  & 2048     & 2046     & 255.255.248.0    & 0xfffff800  & \\
\hline
/20  & 4096     & 4094     & 255.255.240.0    & 0xfffff000  & \\
\hline
/19  & 8192     & 8190     & 255.255.224.0    & 0xffffe000  & \\
\hline
/18  & 16384    & 16382    & 255.255.192.0    & 0xffffc000  & \\
\hline
/17  & 32768    & 32766    & 255.255.128.0    & 0xffff8000  & \\
\hline
/16  & 65536    & 65534    & 255.255.0.0      & 0xffff0000  & сеть класса B \\
\hline
/8   & 16777216 & 16777214 & 255.0.0.0        & 0xff000000  & сеть класса A \\
\hline
\end{tabular}
\end{center}
\normalsize

Вот простой пример, вычисляющий адрес сети используя сетевую маску и адрес хоста.

\lstinputlisting[style=customc]{\CURPATH/netmask.c}

\subsection{calc\_network\_address()}

Функция \TT{calc\_network\_address()} самая простая: 

она просто умножает (логически, используя \AND) адрес хоста на сетевую маску, в итоге давая адрес
сети.

\lstinputlisting[caption=\Optimizing MSVC 2012 /Ob0,numbers=left,style=customasmx86]{\CURPATH/calc_network_address_MSVC_2012_Ox.asm}

На строке 22 мы видим самую важную инструкцию \AND --- так вычисляется адрес сети.

\subsection{form\_IP()}

Функция \TT{form\_IP()} просто собирает все 4 байта в одно 32-битное значение.

Вот как это обычно происходит:

\begin{itemize}
\item Выделите переменную для возвращаемого значения. Обнулите её.

\item 
Возьмите четвертый (самый младший) байт, сложите его (логически, инструкцией \OR) с возвращаемым
значением. Оно содержит теперь 4-й байт.

\item Возьмите третий байт, сдвиньте его на 8 бит влево.
Получится значение в виде \TT{0x0000bb00}, где \TT{bb} это третий байт.
Сложите итоговое значение (логически, инструкцией \OR) с возвращаемым значением.
Возвращаемое значение пока что содержит \TT{0x000000aa}, так что логическое сложение
в итоге выдаст значение вида \TT{0x0000bbaa}.

\item 
Возьмите второй байт, сдвиньте его на 16 бит влево.
Вы получите значение вида \TT{0x00cc0000}, где \TT{cc} это второй байт.
Сложите (логически) результат и возвращаемое значение.
Выходное значение содержит пока что \TT{0x0000bbaa}, так что логическое сложение
в итоге выдаст значение вида \TT{0x00ccbbaa}.

\item 
Возьмите первый байт, сдвиньте его на 24 бита влево.
Вы получите значение вида \TT{0xdd000000}, где \TT{dd} это первый байт.
Сложите (логически) результат и выходное значение.
Выходное значение содержит пока что \TT{0x00ccbbaa}, так что сложение выдаст в итоге значение
вида \TT{0xddccbbaa}.

\end{itemize}

И вот как работает неоптимизирующий MSVC 2012:

\lstinputlisting[caption=\NonOptimizing MSVC 2012,style=customasmx86]{\CURPATH/form_IP_MSVC_2012_RU.asm}

Хотя, порядок операций другой, но, конечно, порядок роли не играет.

\Optimizing MSVC 2012 делает то же самое, но немного иначе:

\lstinputlisting[caption=\Optimizing MSVC 2012 /Ob0,style=customasmx86]{\CURPATH/form_IP_MSVC_2012_Ox_RU.asm}

Можно сказать, что каждый байт записывается в младшие 8 бит возвращаемого значения,
и затем возвращаемое значение сдвигается на один байт влево на каждом шаге.

Повторять 4 раза, для каждого байта.

\par
Вот и всё! 
К сожалению, наверное, нет способа делать это иначе.
Не существует более-менее популярных \ac{CPU} или \ac{ISA}, где имеется инструкция для сборки значения из бит или байт.
Обычно всё это делает сдвигами бит и логическим сложением (OR).

\subsection{print\_as\_IP()}

\TT{print\_as\_IP()} делает наоборот: расщепляет 32-битное значение на 4 байта.

Расщепление работает немного проще: просто сдвигайте входное значение на 24, 16, 8 или 0 бит,
берите биты с нулевого по седьмой (младший байт), вот и всё:

\lstinputlisting[caption=\NonOptimizing MSVC 2012,style=customasmx86]{\CURPATH/print_as_IP_MSVC_2012_RU.asm}

\Optimizing MSVC 2012 делает почти всё то же самое, только без ненужных перезагрузок входного значения:

\lstinputlisting[caption=\Optimizing MSVC 2012 /Ob0,style=customasmx86]{\CURPATH/print_as_IP_MSVC_2012_Ox.asm}

\subsection{form\_netmask() и set\_bit()}

\TT{form\_netmask()} делает сетевую маску из \ac{CIDR}-нотации.

Конечно, было бы куда эффективнее использовать здесь какую-то уже готовую таблицу, но мы рассматриваем
это именно так, сознательно, для демонстрации битовых сдвигов.
Мы также сделаем отдельную функцию \TT{set\_bit()}. 

Не очень хорошая идея выделять отдельную функцию для такой примитивной операции, но так будет проще понять,
как это всё работает.

\lstinputlisting[caption=\Optimizing MSVC 2012 /Ob0,style=customasmx86]{\CURPATH/form_netmask_MSVC_2012_Ox.asm}

\TT{set\_bit()} примитивна: просто сдвигает единицу на нужное количество бит, затем складывает (логически) с
входным значением \q{input}.
\TT{form\_netmask()} имеет цикл: он выставит столько бит (начиная с \ac{MSB}), 
сколько передано в аргументе \TT{netmask\_bits}.

\subsection{Итог}

Вот и всё!
Мы запускаем и видим:

\begin{lstlisting}
netmask=255.255.255.0
network address=10.1.2.0
netmask=255.0.0.0
network address=10.0.0.0
netmask=255.255.255.128
network address=10.1.2.0
netmask=255.255.255.192
network address=10.1.2.64
\end{lstlisting}
}
\DE{\subsection{Betrag berechnen}
\label{sec:abs}

Eine einfache Funktion:

\lstinputlisting[style=customc]{abs.c}

\subsubsection{\Optimizing MSVC}

Normalerweise wird folgender Code erzeugt:

\lstinputlisting[caption=\Optimizing MSVC 2012 x64,style=customasmx86]{patterns/07_jcc/abs/abs_MSVC2012_Ox_x64_DE.asm}

GCC 4.9 macht ungefähr das gleiche.

\subsubsection{\OptimizingKeilVI: \ThumbMode}

\lstinputlisting[caption=\OptimizingKeilVI: \ThumbMode,style=customasmARM]{patterns/07_jcc/abs/abs_Keil_thumb_O3_DE.s}

\myindex{ARM!\Instructions!RSB}
ARM fehlt ein Befehl zur Negation, sodass der Keil Compiler den \q{Reverse
Subtract} Befehl verwendet, der mit umgekehrten Operanden subtrahiert.

\subsubsection{\OptimizingKeilVI: \ARMMode}
Es ist im ARM mode möglich, einigen Befehlen condition codes hinzuzufügen und genau das tut der Keil Compiler:

\lstinputlisting[caption=\OptimizingKeilVI: \ARMMode,style=customasmARM]{patterns/07_jcc/abs/abs_Keil_ARM_O3_DE.s}
Jetzt sind keine bedingten Sprünge mehr übrig und das ist vorteilhaft: \myref{branch_predictors}.

\subsubsection{\NonOptimizing GCC 4.9 (ARM64)}

\myindex{ARM!\Instructions!XOR}

ARM64 kennt den Befehl \INS{NEG} zum Negieren:

\lstinputlisting[caption=\Optimizing GCC 4.9 (ARM64),style=customasmARM]{patterns/07_jcc/abs/abs_GCC49_ARM64_O0_DE.s}

\subsubsection{MIPS}

\lstinputlisting[caption=\Optimizing GCC 4.4.5 (IDA),style=customasmMIPS]{patterns/07_jcc/abs/MIPS_O3_IDA_DE.lst}

\myindex{MIPS!\Instructions!BLTZ}
Hier finden wir einen neuen Befehl: \INS{BLTZ} (\q{Branch if Less Than Zero}).
\myindex{MIPS!\Instructions!SUBU}
\myindex{MIPS!\Pseudoinstructions!NEGU}
Es gibt zusätzlich noch den \INS{NEGU} Pseudo-Befehl, der eine Subtraktion von Null durchführt. Der Suffix \q{U} bei
\INS{SUBU} und \INS{NEGU} zeigt an, dass keine Exception für den Fall eines Integer Overflows geworfen wird.


\subsubsection{Verzweigungslose Version?}
Man kann auch eine verzweigungslose Version dieses Codes erzeugen. Dies werden wir später betrachten:
\myref{chap:branchless_abs}. 
}
\FR{\subsubsection{MIPS}

Une place est allouée sur la pile locale pour la variable $x$, et elle doit être appelée par $\$sp+24$.
\myindex{MIPS!\Instructions!LW}

Son adresse est passée à \scanf, et la valeur entrée par l'utilisateur est chargée en utilisant
l'instruction \INS{LW} (\q{Load Word}), puis passée à \printf.

\lstinputlisting[caption=GCC 4.4.5 \Optimizing (\assemblyOutput),style=customasmMIPS]{patterns/04_scanf/1_simple/MIPS/ex1.O3_FR.s}

IDA affiche la disposition de la pile comme suit:

\lstinputlisting[caption=GCC 4.4.5 \Optimizing (IDA),style=customasmMIPS]{patterns/04_scanf/1_simple/MIPS/ex1.O3.IDA_FR.lst}

% TODO non-optimized version?
}
\IT{\subsection{Calcolo del valore assoluto}
\label{sec:abs}

Una funzione semplice:

\lstinputlisting[style=customc]{abs.c}

\subsubsection{\Optimizing MSVC}

Il codice solitamente generato è questo:

\lstinputlisting[caption=\Optimizing MSVC 2012 x64,style=customasmx86]{patterns/07_jcc/abs/abs_MSVC2012_Ox_x64_EN.asm}

GCC 4.9 fa più o meno lo stesso.

\subsubsection{\OptimizingKeilVI: \ThumbMode}

\lstinputlisting[caption=\OptimizingKeilVI: \ThumbMode,style=customasmARM]{patterns/07_jcc/abs/abs_Keil_thumb_O3_EN.s}

\myindex{ARM!\Instructions!RSB}

In ARM manca l'istruzione di negazione, quindi il compilatore Keil usa l'istruzione \q{Reverse Subtract}, che semplicemente sottrae gli operandi in modo inverso.

\subsubsection{\OptimizingKeilVI: \ARMMode}

In modalità ARM è possibile agguingere condition codes ad alcune istruzioni, e questo è ciò che fa il compilatore keil:

\lstinputlisting[caption=\OptimizingKeilVI: \ARMMode,style=customasmARM]{patterns/07_jcc/abs/abs_Keil_ARM_O3_EN.s}

Adesso non ci sono più jump condizionali, e ciò è bene: \myref{branch_predictors}.

\subsubsection{\NonOptimizing GCC 4.9 (ARM64)}

\myindex{ARM!\Instructions!XOR}

ARM64 ha un'istruzione \INS{NEG} per la negazione:

\lstinputlisting[caption=\Optimizing GCC 4.9 (ARM64),style=customasmARM]{patterns/07_jcc/abs/abs_GCC49_ARM64_O0_EN.s}

\subsubsection{MIPS}

\lstinputlisting[caption=\Optimizing GCC 4.4.5 (IDA),style=customasmMIPS]{patterns/07_jcc/abs/MIPS_O3_IDA_EN.lst}

\myindex{MIPS!\Instructions!BLTZ}
Qui vediamo una nuova istruzione: \INS{BLTZ} (\q{Branch if Less Than Zero}).
\myindex{MIPS!\Instructions!SUBU}
\myindex{MIPS!\Pseudoinstructions!NEGU}

C'è anche la pseudoistruzione \INS{NEGU} , che semplicemente fa la sottrazione da zero.
Il suffisso \q{U} suffix in entrambe \INS{SUBU} e \INS{NEGU} implica che non verrà sollevata nessuna eccezione in caso di integer overflow.

\subsubsection{Branchless version?}

You could have also a branchless version of this code. This we will review later: \myref{chap:branchless_abs}.
}
\JA{\subsection{絶対値の計算}
\label{sec:abs}

簡単な関数の例。

\lstinputlisting[style=customc]{abs.c}

\subsubsection{\Optimizing MSVC}

これは普通、どのようにコードが生成されるのかを示したものです。

\lstinputlisting[caption=\Optimizing MSVC 2012 x64,style=customasmx86]{patterns/07_jcc/abs/abs_MSVC2012_Ox_x64_JA.asm}

GCC 4.9はほとんど同じです。

\subsubsection{\OptimizingKeilVI: \ThumbMode}

\lstinputlisting[caption=\OptimizingKeilVI: \ThumbMode,style=customasmARM]{patterns/07_jcc/abs/abs_Keil_thumb_O3_JA.s}

\myindex{ARM!\Instructions!RSB}

ARMにはネゲート命令がないため、Keilコンパイラは\q{逆引き命令}を使用します。これは逆のオペランドで減算するだけです。

\subsubsection{\OptimizingKeilVI: \ARMMode}

ARMモードでは、いくつかの命令に条件コードを追加することができます。そのため、Keilコンパイラは次のように処理します。

\lstinputlisting[caption=\OptimizingKeilVI: \ARMMode,style=customasmARM]{patterns/07_jcc/abs/abs_Keil_ARM_O3_JA.s}

今度は条件付きジャンプはありません。これは良いですね。:\myref{branch_predictors}

\subsubsection{\NonOptimizing GCC 4.9 (ARM64)}

\myindex{ARM!\Instructions!XOR}

ARM64には、否定するための命令\INS{NEG}があります。

\lstinputlisting[caption=\Optimizing GCC 4.9 (ARM64),style=customasmARM]{patterns/07_jcc/abs/abs_GCC49_ARM64_O0_JA.s}

\subsubsection{MIPS}

\lstinputlisting[caption=\Optimizing GCC 4.4.5 (IDA),style=customasmMIPS]{patterns/07_jcc/abs/MIPS_O3_IDA_JA.lst}

\myindex{MIPS!\Instructions!BLTZ}
ここでは\INS{BLTZ}(\q{Branch if Less Than Zero})という新しい命令があります。
\myindex{MIPS!\Instructions!SUBU}
\myindex{MIPS!\Pseudoinstructions!NEGU}

\INS{NEGU}擬似命令もあります。これはゼロからの減算だけです。 
\INS{SUBU}と\INS{NEGU}の両方の \q{U}接尾辞は、整数オーバーフローの場合に発生する例外がないことを意味します。

\subsubsection{Branchless version?}

このコードを分岐がないバージョンにすることもできます。 これについては、後述の\myref{chap:branchless_abs}を参照してください。
}

\EN{\subsection{Win32 PE}
\label{win32_pe}
\myindex{Windows!Win32}

\acs{PE} is an executable file format used in Windows.
The difference between .exe, .dll and .sys is that .exe and .sys usually do not have exports, only imports.

\myindex{OEP}

A \ac{DLL}, just like any other PE-file, has an entry point (\ac{OEP}) (the function DllMain() is located there)
but this function usually does nothing.
.sys is usually a device driver.
As of drivers, Windows requires the checksum to be present in the PE file and for it to be correct
\footnote{For example, Hiew(\myref{Hiew}) can calculate it}.

\myindex{Windows!Windows Vista}
Starting at Windows Vista, a driver's files must also be signed with a digital signature. It will fail to load otherwise.

\myindex{MS-DOS}
Every PE file begins with tiny DOS program that prints a
message like \q{This program cannot be run in DOS mode.}---if you run this program in DOS or Windows 3.1 (\ac{OS}-es which are not aware of the PE format),
this message will be printed.

\subsubsection{Terminology}

\myindex{VA}
\myindex{Base address}
\myindex{RVA}
\myindex{Windows!IAT}
\myindex{Windows!INT}

\begin{itemize}
\item Module---a separate file, .exe or .dll.

\item Process---a program loaded into memory and currently running.  Commonly consists of one .exe file and bunch of .dll files.

\item Process memory---the memory a process works with.  Each process has its own.
There usually are loaded modules, memory of the stack, \gls{heap}(s), etc.

\item \ac{VA}---an address which is to be used in program while runtime.

\item Base address (of module)---the address within the process memory at which the module is to be loaded.
\ac{OS} loader may change it, if the base address is already occupied by another module just loaded before.

\item \ac{RVA}---the \ac{VA}-address minus the base address.

Many addresses in PE-file tables use \ac{RVA}-addresses.

%\item
%Data directory --- ...

\item \ac{IAT}---an array of addresses of imported symbols \footnote{\PietrekPE}.
Sometimes, the \TT{IMAGE\_DIRECTORY\_ENTRY\_IAT} data directory points at the \ac{IAT}.
\label{IDA_idata}
It is worth noting that \ac{IDA} (as of 6.1) may allocate a pseudo-section named \TT{.idata} for
\ac{IAT}, even if the \ac{IAT} is a part of another section!

\item \ac{INT}---an array of names of symbols to be imported\footnote{\PietrekPE}.
\end{itemize}

\subsubsection{Base address}

The problem is that several module authors can prepare DLL files for others to use and it is not possible
to reach an agreement which addresses is to be assigned to whose modules.

So that is why if two necessary DLLs for a process have the same base address,
one of them will be loaded at this base address, and the other---at some other free space in process memory,
and each virtual addresses in the second DLL will be corrected.

\par With \ac{MSVC} the linker often generates the .exe files with a base address of \TT{0x400000}
\footnote{The origin of this address choice is described here: \href{http://go.yurichev.com/17041}{MSDN}},
and with the code section starting at \TT{0x401000}.
This means that the \ac{RVA} of the start of the code section is \TT{0x1000}.

DLLs are often generated by MSVC's linker with a base address of \TT{0x10000000}
\footnote{This can be changed by the /BASE linker option}.

\myindex{ASLR}

There is also another reason to load modules at various base addresses, in this case random ones.
It is \ac{ASLR}\footnote{\href{http://go.yurichev.com/17140}{wikipedia}}.

\myindex{Shellcode}

A shellcode trying to get executed on a compromised system must call system functions, hence, know their addresses.

In older \ac{OS} (in \gls{Windows NT} line: before Windows Vista),
system DLL (like kernel32.dll, user32.dll) were always loaded at known addresses,
and if we also recall
that their versions rarely changed, the addresses of functions were
fixed and shellcode could call them directly.

In order to avoid this, the \ac{ASLR}
method loads your program and all modules it needs at random base addresses, different every time.

\ac{ASLR} support is denoted in a PE file by setting the flag
\par \TT{IMAGE\_DLL\_CHARACTERISTICS\_DYNAMIC\_BASE} \InSqBrackets{see \Russinovich}.

\subsubsection{Subsystem}

There is also a \emph{subsystem} field, usually it is:

\myindex{Native API}

\begin{itemize}
\item native\footnote{Meaning, the module use Native API instead of Win32} (.sys-driver),

\item console (console application) or

\item \ac{GUI} (non-console).
\end{itemize}

\subsubsection{OS version}

A PE file also specifies the minimal Windows version it needs in order to be loadable.

The table of version numbers stored in the PE file and corresponding Windows codenames is
here\footnote{\href{http://go.yurichev.com/17044}{wikipedia}}.

\myindex{Windows!Windows NT4}
\myindex{Windows!Windows 2000}
For example, \ac{MSVC} 2005 compiles .exe files for running on Windows NT4 (version 4.00), but \ac{MSVC} 2008 does not
(the generated files have a version of 5.00, at least Windows 2000 is needed to run them).

\myindex{Windows!Windows XP}

\ac{MSVC} 2012 generates .exe files of version 6.00 by default,
targeting at least Windows Vista.
However, by changing the compiler's options\footnote{\href{http://go.yurichev.com/17045}{MSDN}},
it is possible to force it to compile for Windows XP.

\subsubsection{Sections}

Division in sections, as it seems, is present in all executable file formats.

It is devised in order to separate code from data, and data---from constant data.

\begin{itemize}
\item Either the \emph{IMAGE\_SCN\_CNT\_CODE} or \emph{IMAGE\_SCN\_MEM\_EXECUTE} flags will be set on the code section---this is executable code.

\item On data section---\emph{IMAGE\_SCN\_CNT\_INITIALIZED\_DATA},\\
\emph{IMAGE\_SCN\_MEM\_READ} and \emph{IMAGE\_SCN\_MEM\_WRITE} flags.

\item On an empty section with uninitialized data---\\
\emph{IMAGE\_SCN\_CNT\_UNINITIALIZED\_DATA}, \emph{IMAGE\_SCN\_MEM\_READ} \\
        and \emph{IMAGE\_SCN\_MEM\_WRITE}.

\item On a constant data section (one that's protected from writing), the flags \\
\emph{IMAGE\_SCN\_CNT\_INITIALIZED\_DATA} and \emph{IMAGE\_SCN\_MEM\_READ} can be set, \\
but not \emph{IMAGE\_SCN\_MEM\_WRITE}.
A process going to crash if it tries to write to this section.
\end{itemize}

\myindex{TLS}
\myindex{BSS}
Each section in PE-file may have a name, however, it is not very important.
Often (but not always) the code section is named \TT{.text},
the data section---\TT{.data}, the constant data section --- \TT{.rdata} \emph{(readable data)}.
Other popular section names are:

\myindex{MIPS}
\begin{itemize}
\item \TT{.idata}---imports section.
\ac{IDA} may create a pseudo-section named like this: \myref{IDA_idata}.
\item \TT{.edata}---exports section (rare)
\item \TT{.pdata}---section holding all information about exceptions in Windows NT for MIPS, \ac{IA64} and x64: \myref{SEH_win64}
\item \TT{.reloc}---relocs section
\item \TT{.bss}---uninitialized data (\ac{BSS})
\item \TT{.tls}---thread local storage (\ac{TLS})
\item \TT{.rsrc}---resources
\item \TT{.CRT}---may present in binary files compiled by ancient MSVC versions
\end{itemize}

PE file packers/encryptors often garble section names or replace the names with their own.

\ac{MSVC} allows you to declare data in arbitrarily named section
\footnote{\href{http://go.yurichev.com/17047}{MSDN}}.

Some compilers and linkers can add a section with debugging symbols and
other debugging information (MinGW for instance).
\myindex{Windows!PDB}
However it is not so in latest versions of \ac{MSVC} (separate \gls{PDB} files are used there for this purpose).\\
\\
That is how a PE section is described in the file:

\begin{lstlisting}[style=customc]
typedef struct _IMAGE_SECTION_HEADER {
  BYTE  Name[IMAGE_SIZEOF_SHORT_NAME];
  union {
    DWORD PhysicalAddress;
    DWORD VirtualSize;
  } Misc;
  DWORD VirtualAddress;
  DWORD SizeOfRawData;
  DWORD PointerToRawData;
  DWORD PointerToRelocations;
  DWORD PointerToLinenumbers;
  WORD  NumberOfRelocations;
  WORD  NumberOfLinenumbers;
  DWORD Characteristics;
} IMAGE_SECTION_HEADER, *PIMAGE_SECTION_HEADER;
\end{lstlisting}
\footnote{\href{http://go.yurichev.com/17048}{MSDN}}

\myindex{Hiew}
A word about terminology: \emph{PointerToRawData} is called \q{Offset} in Hiew
and \emph{VirtualAddress} is called \q{RVA} there.

\subsubsection{Data section}

Data section in file can be smaller than in memory.
For example, some variables can be initialized, some are not.
Compiler and linker will collect them all into one section, but the first part of it is initialized and allocated in file,
while another is absent in file (of course, to make it smaller).
\emph{VirtualSize} will be equal to the size of section in memory, and \emph{SizeOfRawData} --- to
size of section in file.

IDA can show the border between initialized and not initialized parts like that:

\begin{lstlisting}[style=customasmx86]
...

.data:10017FFA                 db    0
.data:10017FFB                 db    0
.data:10017FFC                 db    0
.data:10017FFD                 db    0
.data:10017FFE                 db    0
.data:10017FFF                 db    0
.data:10018000                 db    ? ;
.data:10018001                 db    ? ;
.data:10018002                 db    ? ;
.data:10018003                 db    ? ;
.data:10018004                 db    ? ;
.data:10018005                 db    ? ;

...
\end{lstlisting}

\subsubsection{Relocations (relocs)}
\label{subsec:relocs}

\ac{AKA} FIXUP-s (at least in Hiew).

They are also present in almost all executable file formats
\footnote{Even in .exe files for MS-DOS}.
Exceptions are shared dynamic libraries compiled with \ac{PIC}, or any other \ac{PIC}-code.

What are they for?

Obviously, modules can be loaded on various base addresses, but how to deal with global variables, for example?
They must be accessed by address.  One solution is \PICcode{} (\myref{sec:PIC}).
But it is not always convenient.

That is why a relocations table is present.
There the addresses of points that must be corrected are enumerated,
in case of loading at a different base address.

% TODO тут бы пример с HIEW или objdump..
For example, there is a global variable at address \TT{0x410000} and this is how it is accessed:

\begin{lstlisting}[style=customasmx86]
A1 00 00 41 00         mov         eax,[000410000]
\end{lstlisting}

The base address of the module is \TT{0x400000}, the \ac{RVA} of the global variable is \TT{0x10000}.

If the module is loaded at base address \TT{0x500000}, the real address of the global variable must be \TT{0x510000}.

\myindex{x86!\Instructions!MOV}

As we can see, the address of variable is encoded in the instruction \TT{MOV}, after the byte \TT{0xA1}.

That is why the address of the 4 bytes after \TT{0xA1}, is written in the relocs table.

If the module is loaded at a different base address, the \ac{OS} loader enumerates all addresses in the table,

finds each 32-bit word the address points to, subtracts the original base address from it
(we get the \ac{RVA} here), and adds the new base address to it.

If a module is loaded at its original base address, nothing happens.

All global variables can be treated like that.

Relocs may have various types, however, in Windows for x86 processors, the type is usually \\
\emph{IMAGE\_REL\_BASED\_HIGHLOW}.

\myindex{Hiew}

By the way, relocs are darkened in Hiew, for example: \figref{fig:scanf_ex3_hiew_1}.

\myindex{\olly}
\olly underlines the places in memory to which relocs are to be applied, for example: \figref{fig:switch_lot_olly3}.

\subsubsection{Exports and imports}

\label{PE_exports_imports}
As we all know, any executable program must use the \ac{OS}'s services and other DLL-libraries somehow.

It can be said that functions from one module (usually DLL) must be connected somehow to the points of their
calls in other modules (.exe-file or another DLL).

For this, each DLL has an \q{exports} table, which consists of functions plus their addresses in a module.

And every .exe file or DLL has \q{imports}, a table of functions it needs for execution including
list of DLL filenames.

After loading the main .exe-file, the \ac{OS} loader processes imports table:
it loads the additional DLL-files, finds function names
among the DLL exports and writes their addresses down in the \ac{IAT} of the main .exe-module.

\myindex{Windows!Win32!Ordinal}

As we can see, during loading the loader must compare a lot of function names, but string comparison is not a very
fast procedure, so there is a support for \q{ordinals} or \q{hints},
which are function numbers stored in the table, instead of their names.

That is how they can be located faster when loading a DLL.
Ordinals are always present in the \q{export} table.

\myindex{MFC}
For example, a program using the \ac{MFC} library usually loads mfc*.dll by ordinals,
and in such programs there are no \ac{MFC} function names in \ac{INT}.

% TODO example!
When loading such programs in \IDA, it will ask for a path to the mfc*.dll files
in order to determine the function names.

If you don't tell \IDA the path to these DLLs, there will be \emph{mfc80\_123} instead of function names.

\myparagraph{Imports section}

Often a separate section is allocated for the imports table and everything related to it (with name like \TT{.idata}),
however, this is not a strict rule.

Imports are also a confusing subject because of the terminological mess. Let's try to collect all information in one place.

\begin{figure}[H]
\centering
\myincludegraphics{OS/PE/unnamed0.png}
\caption{
A scheme that unites all PE-file structures related to imports}
\end{figure}

The main structure is the array \emph{IMAGE\_IMPORT\_DESCRIPTOR}.
Each element for each DLL being imported.

Each element holds the \ac{RVA} address of the text string (DLL name) (\emph{Name}).

\emph{OriginalFirstThunk} is the \ac{RVA} address of the \ac{INT} table.
This is an array of \ac{RVA} addresses, each of which points to a text string with a function name.
Each string is prefixed by a 16-bit integer
(\q{hint})---\q{ordinal} of function.

While loading, if it is possible to find a function by ordinal,
then the strings comparison will not occur. The array is terminated by zero.

There is also a pointer to the \ac{IAT} table named \emph{FirstThunk}, it is just the \ac{RVA} address
of the place where the loader writes the addresses of the resolved functions.

The points where the loader writes addresses are marked by \IDA like this: \emph{\_\_imp\_CreateFileA}, etc.

There are at least two ways to use the addresses written by the loader.

\myindex{x86!\Instructions!CALL}
\begin{itemize}
\item The code will have instructions like \emph{call \_\_imp\_CreateFileA},
and since the field with the address of the imported function is a global variable in some sense,
the address of the \emph{call} instruction (plus 1 or 2) is to be added to the relocs table,
for the case when the module is loaded at a different base address.

But, obviously, this may enlarge relocs table significantly.

Because there are might be a lot of calls to imported functions in the module.

Furthermore, large relocs table slows down the process of loading modules.

\myindex{x86!\Instructions!JMP}
\myindex{thunk-functions}
\item For each imported function, there is only one jump allocated, using the \JMP instruction
plus a reloc to it.
Such points are also called \q{thunks}.

All calls to the imported functions are just \CALL instructions to the corresponding \q{thunk}.
In this case, additional relocs are not necessary because these \CALL{}-s
have relative addresses and do not need to be corrected.
\end{itemize}

These two methods can be combined.

Possible, the linker creates individual \q{thunk}s if there are too many calls to the function,
but not done by default. \\
\\
By the way, the array of function addresses to which FirstThunk is pointing is not necessary to be located in the \ac{IAT} section.
For example, the author of these lines once wrote the PE\_add\_import\footnote{\href{http://go.yurichev.com/17049}{yurichev.com}}
utility for adding imports to an existing .exe-file.

Some time earlier, in the previous versions of the utility,
at the place of the function you want to substitute with a call to another DLL,
my utility wrote the following code:

\begin{lstlisting}[style=customasmx86]
MOV EAX, [yourdll.dll!function]
JMP EAX
\end{lstlisting}

FirstThunk points to the first instruction. In other words, when loading yourdll.dll,
the loader writes the address of the \emph{function} function right in the code.

It also worth noting that a code section is usually write-protected, so my utility adds the \\
\emph{IMAGE\_SCN\_MEM\_WRITE}
flag for code section. Otherwise, the program to crash while loading with error code
5 (access denied). \\
\\
One might ask: what if I supply a program with a set of DLL files which is not supposed to change (including addresses of all DLL functions),
is it possible to speed up the loading process?

Yes, it is possible to write the addresses of the functions to be imported into the FirstThunk arrays in advance.
The \emph{Timestamp} field is present in the \\
\emph{IMAGE\_IMPORT\_DESCRIPTOR} structure.

If a value is present there, then the loader compares this value with the date-time of the DLL file.

If the values are equal, then the loader does not do anything, and the loading of the process can be faster.
This is called \q{old-style binding}
\footnote{\href{http://go.yurichev.com/17050}{MSDN}. There is also the \q{new-style binding}.}.
\myindex{BIND.EXE}

The BIND.EXE utility in Windows SDK is for this.
For speeding up the loading of your program, Matt Pietrek in \PietrekPEURL, suggests to do the binding shortly after your program
installation on the computer of the end user. \\
\\
PE-files packers/encryptors may also compress/encrypt imports table.

In this case, the Windows loader, of course, will not load all necessary DLLs.
\myindex{Windows!Win32!LoadLibrary}
\myindex{Windows!Win32!GetProcAddress}

Therefore, the packer/encryptor does this on its own, with the help of
\emph{LoadLibrary()} and the \emph{GetProcAddress()} functions.

That is why these two functions are often present in \ac{IAT} in packed files.\\
\\
In the standard DLLs from the Windows installation, \ac{IAT} often is located right at the beginning of the PE file.
Supposedly, it is made so for optimization.

While loading, the .exe file is not loaded into memory as a whole (recall huge install programs which are
started suspiciously fast), it is \q{mapped}, and loaded into memory in parts as they are accessed.

Probably, Microsoft developers decided it will be faster.

\subsubsection{Resources}

\label{PEresources}

Resources in a PE file are just a set of icons, pictures, text strings, dialog descriptions.

Perhaps they were separated from the main code, so all these things could be multilingual,
and it would be simpler to pick text or picture for the language that is currently set in the \ac{OS}. \\
\\
As a side effect, they can be edited easily and saved back to the executable file, even if one does not have special knowledge,
by using the ResHack editor, for example (\myref{ResHack}).

\subsubsection{.NET}

\myindex{.NET}

.NET programs are not compiled into machine code but into a special bytecode.
\myindex{OEP}
Strictly speaking, there is bytecode instead of the usual x86 code
in the .exe file, however, the entry point (\ac{OEP}) points to this tiny fragment of x86 code:

\begin{lstlisting}[style=customasmx86]
jmp         mscoree.dll!_CorExeMain
\end{lstlisting}

The .NET loader is located in mscoree.dll, which processes the PE file.
\myindex{Windows!Windows XP}

It was so in all pre-Windows XP \ac{OS}es. Starting from XP, the \ac{OS} loader is able to detect the .NET file
and run it without executing that \JMP instruction
\footnote{\href{http://go.yurichev.com/17051}{MSDN}}.

\myindex{TLS}
\subsubsection{TLS}

This section holds initialized data for the \ac{TLS}(\myref{TLS}) (if needed).
When a new thread start, its \ac{TLS} data is initialized using the data from this section. \\
\\
\myindex{TLS!Callbacks}

Aside from that, the PE file specification also provides initialization of the
\ac{TLS} section, the so-called TLS callbacks.

If they are present, they are to be called before the control is passed to the main entry point (\ac{OEP}).

This is used widely in the PE file packers/encryptors.

\subsubsection{Tools}

\myindex{objdump}
\myindex{Cygwin}
\myindex{Hiew}
\label{ResHack}

\begin{itemize}
\item objdump (present in cygwin) for dumping all PE-file structures.

\item Hiew(\myref{Hiew}) as editor.

\item pefile---Python-library for PE-file processing \footnote{\url{http://go.yurichev.com/17052}}.

\item ResHack \acs{AKA} Resource Hacker---resources editor\footnote{\url{http://go.yurichev.com/17052}}.

\item PE\_add\_import\footnote{\url{http://go.yurichev.com/17049}}---
simple tool for adding symbol(s) to PE executable import table.

\item PE\_patcher\footnote{\href{http://go.yurichev.com/17054}{yurichev.com}}---simple tool for patching PE executables.

\item PE\_search\_str\_refs\footnote{\href{http://go.yurichev.com/17055}{yurichev.com}}---simple tool for searching for a function in PE executables which use some text string.
\end{itemize}

\subsubsection{Further reading}

% FIXME: bibliography per chapter or section
\begin{itemize}
\item Daniel Pistelli---The .NET File Format \footnote{\url{http://go.yurichev.com/17056}}
\end{itemize}

}
\RU{\mysection{Пример вычисления адреса сети}

Как мы знаем, TCP/IP-адрес (IPv4) состоит из четырех чисел в пределах $0 \ldots 255$, т.е. 4 байта.

4 байта легко помещаются в 32-битную переменную, так что адрес хоста в IPv4, сетевая маска или адрес сети
могут быть 32-битными числами.

С точки зрения пользователя, маска сети определяется четырьмя числами в формате вроде \\
255.255.255.0,
но сетевые инженеры (сисадмины) используют более компактную нотацию (\ac{CIDR}),
вроде  \q{/8}, \q{/16}, итд.

Эта нотация просто определяет количество бит в сетевой маске, начиная с \ac{MSB}.

\small
\begin{center}
\begin{tabular}{ | l | l | l | l | l | l | }
\hline
\HeaderColor Маска & 
\HeaderColor Хосты & 
\HeaderColor Свободно &
\HeaderColor Сетевая маска &
\HeaderColor В шест.виде &
\HeaderColor \\
\hline
/30  & 4        & 2        & 255.255.255.252  & 0xfffffffc  & \\
\hline
/29  & 8        & 6        & 255.255.255.248  & 0xfffffff8  & \\
\hline
/28  & 16       & 14       & 255.255.255.240  & 0xfffffff0  & \\
\hline
/27  & 32       & 30       & 255.255.255.224  & 0xffffffe0  & \\
\hline
/26  & 64       & 62       & 255.255.255.192  & 0xffffffc0  & \\
\hline
/24  & 256      & 254      & 255.255.255.0    & 0xffffff00  & сеть класса C \\
\hline
/23  & 512      & 510      & 255.255.254.0    & 0xfffffe00  & \\
\hline
/22  & 1024     & 1022     & 255.255.252.0    & 0xfffffc00  & \\
\hline
/21  & 2048     & 2046     & 255.255.248.0    & 0xfffff800  & \\
\hline
/20  & 4096     & 4094     & 255.255.240.0    & 0xfffff000  & \\
\hline
/19  & 8192     & 8190     & 255.255.224.0    & 0xffffe000  & \\
\hline
/18  & 16384    & 16382    & 255.255.192.0    & 0xffffc000  & \\
\hline
/17  & 32768    & 32766    & 255.255.128.0    & 0xffff8000  & \\
\hline
/16  & 65536    & 65534    & 255.255.0.0      & 0xffff0000  & сеть класса B \\
\hline
/8   & 16777216 & 16777214 & 255.0.0.0        & 0xff000000  & сеть класса A \\
\hline
\end{tabular}
\end{center}
\normalsize

Вот простой пример, вычисляющий адрес сети используя сетевую маску и адрес хоста.

\lstinputlisting[style=customc]{\CURPATH/netmask.c}

\subsection{calc\_network\_address()}

Функция \TT{calc\_network\_address()} самая простая: 

она просто умножает (логически, используя \AND) адрес хоста на сетевую маску, в итоге давая адрес
сети.

\lstinputlisting[caption=\Optimizing MSVC 2012 /Ob0,numbers=left,style=customasmx86]{\CURPATH/calc_network_address_MSVC_2012_Ox.asm}

На строке 22 мы видим самую важную инструкцию \AND --- так вычисляется адрес сети.

\subsection{form\_IP()}

Функция \TT{form\_IP()} просто собирает все 4 байта в одно 32-битное значение.

Вот как это обычно происходит:

\begin{itemize}
\item Выделите переменную для возвращаемого значения. Обнулите её.

\item 
Возьмите четвертый (самый младший) байт, сложите его (логически, инструкцией \OR) с возвращаемым
значением. Оно содержит теперь 4-й байт.

\item Возьмите третий байт, сдвиньте его на 8 бит влево.
Получится значение в виде \TT{0x0000bb00}, где \TT{bb} это третий байт.
Сложите итоговое значение (логически, инструкцией \OR) с возвращаемым значением.
Возвращаемое значение пока что содержит \TT{0x000000aa}, так что логическое сложение
в итоге выдаст значение вида \TT{0x0000bbaa}.

\item 
Возьмите второй байт, сдвиньте его на 16 бит влево.
Вы получите значение вида \TT{0x00cc0000}, где \TT{cc} это второй байт.
Сложите (логически) результат и возвращаемое значение.
Выходное значение содержит пока что \TT{0x0000bbaa}, так что логическое сложение
в итоге выдаст значение вида \TT{0x00ccbbaa}.

\item 
Возьмите первый байт, сдвиньте его на 24 бита влево.
Вы получите значение вида \TT{0xdd000000}, где \TT{dd} это первый байт.
Сложите (логически) результат и выходное значение.
Выходное значение содержит пока что \TT{0x00ccbbaa}, так что сложение выдаст в итоге значение
вида \TT{0xddccbbaa}.

\end{itemize}

И вот как работает неоптимизирующий MSVC 2012:

\lstinputlisting[caption=\NonOptimizing MSVC 2012,style=customasmx86]{\CURPATH/form_IP_MSVC_2012_RU.asm}

Хотя, порядок операций другой, но, конечно, порядок роли не играет.

\Optimizing MSVC 2012 делает то же самое, но немного иначе:

\lstinputlisting[caption=\Optimizing MSVC 2012 /Ob0,style=customasmx86]{\CURPATH/form_IP_MSVC_2012_Ox_RU.asm}

Можно сказать, что каждый байт записывается в младшие 8 бит возвращаемого значения,
и затем возвращаемое значение сдвигается на один байт влево на каждом шаге.

Повторять 4 раза, для каждого байта.

\par
Вот и всё! 
К сожалению, наверное, нет способа делать это иначе.
Не существует более-менее популярных \ac{CPU} или \ac{ISA}, где имеется инструкция для сборки значения из бит или байт.
Обычно всё это делает сдвигами бит и логическим сложением (OR).

\subsection{print\_as\_IP()}

\TT{print\_as\_IP()} делает наоборот: расщепляет 32-битное значение на 4 байта.

Расщепление работает немного проще: просто сдвигайте входное значение на 24, 16, 8 или 0 бит,
берите биты с нулевого по седьмой (младший байт), вот и всё:

\lstinputlisting[caption=\NonOptimizing MSVC 2012,style=customasmx86]{\CURPATH/print_as_IP_MSVC_2012_RU.asm}

\Optimizing MSVC 2012 делает почти всё то же самое, только без ненужных перезагрузок входного значения:

\lstinputlisting[caption=\Optimizing MSVC 2012 /Ob0,style=customasmx86]{\CURPATH/print_as_IP_MSVC_2012_Ox.asm}

\subsection{form\_netmask() и set\_bit()}

\TT{form\_netmask()} делает сетевую маску из \ac{CIDR}-нотации.

Конечно, было бы куда эффективнее использовать здесь какую-то уже готовую таблицу, но мы рассматриваем
это именно так, сознательно, для демонстрации битовых сдвигов.
Мы также сделаем отдельную функцию \TT{set\_bit()}. 

Не очень хорошая идея выделять отдельную функцию для такой примитивной операции, но так будет проще понять,
как это всё работает.

\lstinputlisting[caption=\Optimizing MSVC 2012 /Ob0,style=customasmx86]{\CURPATH/form_netmask_MSVC_2012_Ox.asm}

\TT{set\_bit()} примитивна: просто сдвигает единицу на нужное количество бит, затем складывает (логически) с
входным значением \q{input}.
\TT{form\_netmask()} имеет цикл: он выставит столько бит (начиная с \ac{MSB}), 
сколько передано в аргументе \TT{netmask\_bits}.

\subsection{Итог}

Вот и всё!
Мы запускаем и видим:

\begin{lstlisting}
netmask=255.255.255.0
network address=10.1.2.0
netmask=255.0.0.0
network address=10.0.0.0
netmask=255.255.255.128
network address=10.1.2.0
netmask=255.255.255.192
network address=10.1.2.64
\end{lstlisting}
}
\DE{\mysection{\Stack}
\label{sec:stack}
\myindex{\Stack}

Der Stack ist eine der fundamentalen Datenstrukturen in der Informatik.
\footnote{\href{http://go.yurichev.com/17119}{wikipedia.org/wiki/Call\_Stack}}.
\ac{AKA} \ac{LIFO}.

Technisch betrachtet ist es ein Stapelspeicher innerhalb des Prozessspeichers der zusammen mit den \ESP (x86), \RSP (x64) oder dem \ac{SP} (ARM) Register als ein Zeiger in diesem Speicherblock fungiert.

\myindex{ARM!\Instructions!PUSH}
\myindex{ARM!\Instructions!POP}
\myindex{x86!\Instructions!PUSH}
\myindex{x86!\Instructions!POP}

Die häufigsten Stack-Zugriffsinstruktionen sind die \PUSH- und \POP-Instruktionen (in beidem x86 und ARM Thumb-Modus). \PUSH subtrahiert vom \ESP/\RSP/\ac{SP} 4 Byte im 32-Bit Modus (oder 8 im 64-Bit Modus) und schreibt dann den Inhalt des Zeigers an die Adresse auf die von \ESP/\RSP/\ac{SP} gezeigt wird.

\POP ist die umgekehrte Operation: Die Daten des Zeigers für die Speicherregion auf die von \ac{SP}
gezeigt wird werden ausgelesen und die Inhalte in den Instruktionsoperanden geschreiben (oft ist das ein Register). Dann werden 4 (beziehungsweise 8) Byte zum \gls{stack pointer} addiert.

Nach der Stackallokation, zeigt der \gls{stack pointer} auf den Boden des Stacks.
\PUSH verringert den \gls{stack pointer} und \POP erhöht ihn.
Der Boden des Stacks ist eigentlich der Anfang der Speicherregion die für den Stack reserviert wurde.
Das wirkt zunächst seltsam, aber so funktioniert es.

ARM unterstützt beides, aufsteigende und absteigende Stacks.

\myindex{ARM!\Instructions!STMFD}
\myindex{ARM!\Instructions!LDMFD}
\myindex{ARM!\Instructions!STMED}
\myindex{ARM!\Instructions!LDMED}
\myindex{ARM!\Instructions!STMFA}
\myindex{ARM!\Instructions!LDMFA}
\myindex{ARM!\Instructions!STMEA}
\myindex{ARM!\Instructions!LDMEA}

Zum Beispiel die \ac{STMFD}/\ac{LDMFD} und \ac{STMED}/\ac{LDMED} Instruktionen sind alle dafür gedacht mit einem absteigendem Stack zu arbeiten ( wächst nach unten, fängt mit hohen Adressen an und entwickelt sich zu niedrigeren Adressen). Die \ac{STMFA}/\ac{LDMFA} und \ac{STMEA}/\ac{LDMEA} Instruktionen sind dazu gedacht mit einem aufsteigendem Stack zu arbeiten (wächst nach oben und fängt mit niedrigeren Adressen an und wächst nach oben).

% It might be worth mentioning that STMED and STMEA write first,
% and then move the pointer, and that LDMED and LDMEA move the pointer first, and then read.
% In other words, ARM not only lets the stack grow in a non-standard direction,
% but also in a non-standard order.
% Maybe this can be in the glossary, which would explain why E stands for "empty".

\subsection{Warum wächst der Stack nach unten?}
\label{stack_grow_backwards}

Intuitiv, würden man annehmen das der Stack nach oben wächst z.B Richtung höherer Adressen, so wie bei jeder anderen Datenstruktur.

Der Grund das der Stack rückwärts wächst ist wohl historisch bedingt. Als Computer so groß waren das sie einen ganzen Raum beansprucht haben war es einfach Speicher in zwei Sektionen zu unterteilen, einen Teil für den \gls{heap} und einen Teil für den Stack. Sicher war zu dieser Zeit nicht bekannt wie groß der \gls{heap} und der Stack wachsen würden, während der Programm Laufzeit, also war die Lösung die einfachste mögliche.

\input{patterns/02_stack/stack_and_heap}

In \RitchieThompsonUNIX können wir folgendes lesen:

\begin{framed}
\begin{quotation}
Der user-core eines Programm Images wird in drei logische Segmente unterteilt. Das Programm-Text Segment beginnt bei 0 im virtuellen Adress Speicher. Während der Ausführung wird das Segment als schreibgeschützt markiert und eine einzelne Kopie des Segments wird unter allen Prozessen geteilt die das Programm ausführen. An der ersten 8K grenze über dem Programm Text Segment im Virtuellen Speicher, fängt der ``nonshared'' Bereich an, der nach Bedarf von Syscalls erweitert werden kann. Beginnend bei der höchsten Adresse im Virtuellen Speicher ist das Stack Segment, das Automatisch nach unten wächst während der Hardware Stackpointer sich ändert.
\end{quotation}
\end{framed}

Das erinnert daran wie manche Schüler Notizen zu  zwei Vorträgen in einem Notebook dokumentieren:
Notizen für den ersten Vortrag werden normal notiert, und Notizen zur zum zweiten Vortrag werden 
ans Ende des Notizbuches geschrieben, indem man das Notizbuch umdreht. Die Notizen treffen sich irgendwann
im Notizbuch aufgrund des fehlenden Freien Platzes.

% I think if we want to expand on this analogy,
% one might remember that the line number increases as as you go down a page.
% So when you decrease the address when pushing to the stack, visually,
% the stack does grow upwards.
% Of course, the problem is that in most human languages,
% just as with computers,
% we write downwards, so this direction is what makes buffer overflows so messy.

\subsection{Für was wird der Stack benutzt?}

% subsections
\EN{\input{patterns/02_stack/01_saving_ret_addr_EN}}
\RU{\input{patterns/02_stack/01_saving_ret_addr_RU}}
\DE{\input{patterns/02_stack/01_saving_ret_addr_DE}}
\FR{\input{patterns/02_stack/01_saving_ret_addr_FR}}
\PTBR{\input{patterns/02_stack/01_saving_ret_addr_PTBR}}
\IT{\input{patterns/02_stack/01_saving_ret_addr_IT}}
\PL{\input{patterns/02_stack/01_saving_ret_addr_PL}}
\JA{\input{patterns/02_stack/01_saving_ret_addr_JA}}

\EN{\input{patterns/02_stack/02_args_passing_EN}}
\RU{\input{patterns/02_stack/02_args_passing_RU}}
\PTBR{\input{patterns/02_stack/02_args_passing_PTBR}}
\DE{\input{patterns/02_stack/02_args_passing_DE}}
\IT{\input{patterns/02_stack/02_args_passing_IT}}
\FR{\input{patterns/02_stack/02_args_passing_FR}}
\JA{\input{patterns/02_stack/02_args_passing_JA}}
\PL{\input{patterns/02_stack/02_args_passing_PL}}


\EN{\input{patterns/02_stack/03_local_vars_EN}}
\RU{\input{patterns/02_stack/03_local_vars_RU}}
\DE{\input{patterns/02_stack/03_local_vars_DE}}
\PTBR{\input{patterns/02_stack/03_local_vars_PTBR}}
\EN{\input{patterns/02_stack/04_alloca/main_EN}}
\FR{\input{patterns/02_stack/04_alloca/main_FR}}
\RU{\input{patterns/02_stack/04_alloca/main_RU}}
\PTBR{\input{patterns/02_stack/04_alloca/main_PTBR}}
\IT{\input{patterns/02_stack/04_alloca/main_IT}}
\DE{\input{patterns/02_stack/04_alloca/main_DE}}
\PL{\input{patterns/02_stack/04_alloca/main_PL}}
\JA{\input{patterns/02_stack/04_alloca/main_JA}}

\subsubsection{(Windows) SEH}
\myindex{Windows!Structured Exception Handling}

\ifdefined\RUSSIAN
В стеке хранятся записи \ac{SEH} для функции (если они присутствуют).
Читайте больше о нем здесь: (\myref{sec:SEH}).
\fi % RUSSIAN

\ifdefined\ENGLISH
\ac{SEH} records are also stored on the stack (if they are present).
Read more about it: (\myref{sec:SEH}).
\fi % ENGLISH

\ifdefined\BRAZILIAN
\ac{SEH} também são guardados na pilha (se estiverem presentes).
\PTBRph{}: (\myref{sec:SEH}).
\fi % BRAZILIAN

\ifdefined\ITALIAN
I record \ac{SEH}, se presenti, sono anch'essi memorizzati nello stack.
Maggiori informazioni qui: (\myref{sec:SEH}).
\fi % ITALIAN

\ifdefined\FRENCH
Les enregistrements \ac{SEH} sont aussi stockés dans la pile (s'ils sont présents).
Lire à ce propos: (\myref{sec:SEH}).
\fi % FRENCH


\ifdefined\POLISH
Na stosie są przechowywane wpisy \ac{SEH} dla funkcji (jeśli są one obecne).
Więcej o tym tutaj: (\myref{sec:SEH}).
\fi % POLISH

\ifdefined\JAPANESE
\ac{SEH}レコードはスタックにも格納されます(存在する場合)。
それについてもっと読む:(\myref{sec:SEH})
\fi % JAPANESE

\ifdefined\ENGLISH
\subsubsection{Buffer overflow protection}

More about it here~(\myref{subsec:bufferoverflow}).
\fi

\ifdefined\RUSSIAN
\subsubsection{Защита от переполнений буфера}

Здесь больше об этом~(\myref{subsec:bufferoverflow}).
\fi

\ifdefined\BRAZILIAN
\subsubsection{Proteção contra estouro de buffer}

Mais sobre aqui~(\myref{subsec:bufferoverflow}).
\fi

\ifdefined\ITALIAN
\subsubsection{Protezione da buffer overflow}

Maggiori informazioni qui~(\myref{subsec:bufferoverflow}).
\fi

\ifdefined\FRENCH
\subsubsection{Protection contre les débordements de tampon}

Lire à ce propos~(\myref{subsec:bufferoverflow}).
\fi


\ifdefined\POLISH
\subsubsection{Metody zabiezpieczenia przed przepełnieniem stosu}

Więcej o tym tutaj~(\myref{subsec:bufferoverflow}).
\fi

\ifdefined\JAPANESE
\subsubsection{バッファオーバーフロー保護}

詳細はこちら~(\myref{subsec:bufferoverflow})
\fi

\subsubsection{Automatisches deallokieren der Daten auf dem Stack}

Vielleicht ist der Grund warum man lokale Variablen und SEH Einträge auf dem Stack speichert, weil sie beim 
verlassen der Funktion automatisch aufgeräumt werden. Man braucht dabei nur eine Instruktion um die Position
des Stackpointers zu korrigieren (oftmals ist es die \ADD Instruktion). Funktions Argumente, könnte man sagen 
werden auch am Ende der Funktion deallokiert. Im Kontrast dazu, alles was auf dem \emph{heap} gespeichert wird muss
explizit deallokiert werden. 

% sections
\EN{\input{patterns/02_stack/07_layout_EN}}
\RU{\input{patterns/02_stack/07_layout_RU}}
\DE{\input{patterns/02_stack/07_layout_DE}}
\PTBR{\input{patterns/02_stack/07_layout_PTBR}}
\EN{\input{patterns/02_stack/08_noise/main_EN}}
\FR{\input{patterns/02_stack/08_noise/main_FR}}
\RU{\input{patterns/02_stack/08_noise/main_RU}}
\IT{\input{patterns/02_stack/08_noise/main_IT}}
\DE{\input{patterns/02_stack/08_noise/main_DE}}
\PL{\input{patterns/02_stack/08_noise/main_PL}}
\JA{\input{patterns/02_stack/08_noise/main_JA}}

\input{patterns/02_stack/exercises}
}
\FR{\subsection{Opérateur conditionnel ternaire}
\label{chap:cond}

L'opérateur conditionnel ternaire en \CCpp a la syntaxe suivante:

\begin{lstlisting}
expression ? expression : expression
\end{lstlisting}

Voici un exemple:

\lstinputlisting[style=customc]{patterns/07_jcc/cond_operator/cond.c}

\subsubsection{x86}

Les vieux compilateurs et ceux sans optimisation génèrent du code assembleur comme
si des instructions \TT{if/else} avaient été utilisées:

\lstinputlisting[caption=MSVC 2008 \NonOptimizing,style=customasmx86]{patterns/07_jcc/cond_operator/MSVC2008_FR.asm}

\lstinputlisting[caption=MSVC 2008 \Optimizing ,style=customasmx86]{patterns/07_jcc/cond_operator/MSVC2008_Ox_FR.asm}

Les nouveaux compilateurs sont plus concis:

\lstinputlisting[caption=MSVC 2012 x64 \Optimizing,style=customasmx86]{patterns/07_jcc/cond_operator/MSVC2012_Ox_x64_FR.asm}

\myindex{x86!\Instructions!CMOVcc}
GCC 4.8 \Optimizing pour x86 utilise également l'instruction \TT{CMOVcc}, tandis
que GCC 4.8 \NonOptimizing utilise des sauts conditionnels.

\subsubsection{ARM}

\myindex{x86!\Instructions!ADRcc}
Keil \Optimizing pour le mode ARM utilise les instructions conditionnelles \TT{ADRcc}:

\lstinputlisting[label=cond_Keil_ARM_O3,caption=\OptimizingKeilVI (\ARMMode),style=customasmARM]{patterns/07_jcc/cond_operator/Keil_ARM_O3_FR.s}

Sans intervention manuelle, les deux instructions \TT{ADREQ} et \TT{ADRNE} ne peuvent
être exécutées lors de la même exécution.

Keil \Optimizing pour le mode Thumb à besoin d'utiliser des instructions de saut
conditionnel, puisqu'il n'y a pas d'instruction qui supporte le flag conditionnel.

\lstinputlisting[caption=\OptimizingKeilVI (\ThumbMode),style=customasmARM]{patterns/07_jcc/cond_operator/Keil_thumb_O3_FR.s}

\subsubsection{ARM64}

GCC (Linaro) 4.9 \Optimizing pour ARM64 utilise aussi des sauts conditionnels:

\lstinputlisting[label=cond_ARM64,caption=GCC (Linaro) 4.9 \Optimizing,style=customasmARM]{patterns/07_jcc/cond_operator/ARM64_GCC_O3_FR.s}

C'est parce qu'ARM64 n'a pas d'instruction de chargement simple avec le flag conditionnel
comme \TT{ADRcc} en ARM 32-bit ou \INS{CMOVcc} en x86.

\myindex{ARM!\Instructions!CSEL}
Il a toutefois l'instruction \q{Conditional SELect} (\TT{CSEL})\InSqBrackets{\ARMSixFourRef p390, C5.5},
mais GCC 4.9 ne semble pas assez malin pour l'utiliser dans un tel morceau de code.

\subsubsection{MIPS}

Malheureusement, GCC 4.4.5 pour MIPS n'est pas très malin non plus:

\lstinputlisting[caption=GCC 4.4.5 \Optimizing (\assemblyOutput),style=customasmMIPS]{patterns/07_jcc/cond_operator/MIPS_O3_FR.s}

\subsubsection{Récrivons-le à l'aide d'un\TT{if/else}}

\lstinputlisting[style=customc]{patterns/07_jcc/cond_operator/cond2.c}

\myindex{x86!\Instructions!CMOVcc}

Curieusement, GCC 4.8 avec l'optimisation a pû utiliser \TT{CMOVcc} dans ce cas:

\lstinputlisting[caption=GCC 4.8 \Optimizing,style=customasmx86]{patterns/07_jcc/cond_operator/cond2_GCC_O3_FR.s}

Keil avec optimisation génère un code identique à \lstref{cond_Keil_ARM_O3}.

Mais MSVC 2012 avec optimisation n'est pas (encore) si bon.

\subsubsection{\Conclusion{}}

Pourquoi est-ce que les compilateurs qui optimisent essayent de se débarrasser des
sauts conditionnels? Voir à ce propos: \myref{branch_predictors}.
}
\IT{\subsection{Operatore ternario}
\label{chap:cond}

L'operatore ternario in \CCpp ha la seguente sintassi:

\begin{lstlisting}
expression ? expression : expression
\end{lstlisting}

Ecco un semplice esempio:

\lstinputlisting[style=customc]{patterns/07_jcc/cond_operator/cond.c}

\subsubsection{x86}

I vecchi compilatori e quelli non ottimizzanti generano codice assembly come se fosse stata usata una coppia \TT{if/else}:

\lstinputlisting[caption=\NonOptimizing MSVC 2008,style=customasmx86]{patterns/07_jcc/cond_operator/MSVC2008_EN.asm}

\lstinputlisting[caption=\Optimizing MSVC 2008,style=customasmx86]{patterns/07_jcc/cond_operator/MSVC2008_Ox_EN.asm}

I nuovi compilatori sono più concisi:

\lstinputlisting[caption=\Optimizing MSVC 2012 x64,style=customasmx86]{patterns/07_jcc/cond_operator/MSVC2012_Ox_x64_EN.asm}

\myindex{x86!\Instructions!CMOVcc}
\Optimizing GCC 4.8 per x86 usa anche l'istruzione \TT{CMOVcc}, mentre la versione non-optimizing usa jump condizionali.

\subsubsection{ARM}

\myindex{x86!\Instructions!ADRcc}
Anche Optimizing Keil per ARM mode usa le istruzioni condizionali \TT{ADRcc}:

\lstinputlisting[label=cond_Keil_ARM_O3,caption=\OptimizingKeilVI (\ARMMode),style=customasmARM]{patterns/07_jcc/cond_operator/Keil_ARM_O3_EN.s}

Senza alcun intervento manuale, le due istruzioni \TT{ADREQ} e \TT{ADRNE} non possono essere eseguite nella stessa istanza.

\Optimizing Keil per Thumb mode deve usare i jump condizionali, in quanto non esistono istruzioni di caricamento che supportano i flag condizionali:

\lstinputlisting[caption=\OptimizingKeilVI (\ThumbMode),style=customasmARM]{patterns/07_jcc/cond_operator/Keil_thumb_O3_EN.s}

\subsubsection{ARM64}

\Optimizing GCC (Linaro) 4.9 per ARM64 usa anch'esso i jump condizionali:

\lstinputlisting[label=cond_ARM64,caption=\Optimizing GCC (Linaro) 4.9,style=customasmARM]{patterns/07_jcc/cond_operator/ARM64_GCC_O3_EN.s}

Ciò avviene perchè ARM64 non ha una semplice istruzione di caricament con flag condizionali, come \TT{ADRcc} in ARM mode a 32-bit o \INS{CMOVcc} in x86.

\myindex{ARM!\Instructions!CSEL}
Tuttavia ha l'istruzione \q{Conditional SELect} (\TT{CSEL})\InSqBrackets{\ARMSixFourRef p390, C5.5},
ma GCC 4.9 non sembra essere abbastanza intelligente da usarla in un simile pezzo di codice.

\subsubsection{MIPS}

Sfortunatamente, anche GCC 4.4.5 per MIPS non è molto intelligente:

\lstinputlisting[caption=\Optimizing GCC 4.4.5 (\assemblyOutput),style=customasmMIPS]{patterns/07_jcc/cond_operator/MIPS_O3_EN.s}

\subsubsection{Riscriviamolo come un \TT{if/else}}

\lstinputlisting[style=customc]{patterns/07_jcc/cond_operator/cond2.c}

\myindex{x86!\Instructions!CMOVcc}

E' interessante notare che GCC 4.8 ottimizzante per x86 è stato in grado di usare \TT{CMOVcc} in questo caso:

\lstinputlisting[caption=\Optimizing GCC 4.8,style=customasmx86]{patterns/07_jcc/cond_operator/cond2_GCC_O3_EN.s}

Keil ottimizzante in ARM mode genera codice identico a \lstref{cond_Keil_ARM_O3}.

Ma MSVC 2012 ottimizzante non è (ancora) così in gamba.

% Do not translate, this is macro:
\subsubsection{\Conclusion{}}

Perchè i compilatori ottimizzanti cercano di sbarazzarsi dei jump condizionali? Leggi qui: \myref{branch_predictors}.
}
\JA{\subsection{三項条件演算子}
\label{chap:cond}

\CCpp の三項条件演算子の構文は次のとおりです。

\begin{lstlisting}
expression ? expression : expression
\end{lstlisting}

次に例を示します。

\lstinputlisting[style=customc]{patterns/07_jcc/cond_operator/cond.c}

\subsubsection{x86}

古いコンパイラと最適化していないコンパイラは、\TT{if/else}文が使用されたかのようにアセンブリコードを生成します。

\lstinputlisting[caption=\NonOptimizing MSVC 2008,style=customasmx86]{patterns/07_jcc/cond_operator/MSVC2008_JA.asm}

\lstinputlisting[caption=\Optimizing MSVC 2008,style=customasmx86]{patterns/07_jcc/cond_operator/MSVC2008_Ox_JA.asm}

新しいコンパイラはより簡潔です。

\lstinputlisting[caption=\Optimizing MSVC 2012 x64,style=customasmx86]{patterns/07_jcc/cond_operator/MSVC2012_Ox_x64_JA.asm}

\myindex{x86!\Instructions!CMOVcc}
x86用の\Optimizing GCC 4.8も\TT{CMOVcc}命令を使用し、非最適化GCC 4.8は条件付きジャンプを使用します。

\subsubsection{ARM}

\myindex{x86!\Instructions!ADRcc}
ARMモード用の \Optimizing Keilでは、条件付き命令\TT{ADRcc}を使います。

\lstinputlisting[label=cond_Keil_ARM_O3,caption=\OptimizingKeilVI (\ARMMode),style=customasmARM]{patterns/07_jcc/cond_operator/Keil_ARM_O3_JA.s}

手動で介入しなければ、2つの命令\TT{ADREQ} と \TT{ADRNE}を同じときにで実行することはできません。

Thumbモードでは、 \Optimizing Keilは、条件付きフラグをサポートするロード命令がないため、
条件付きジャンプ命令を使用する必要があります。

\lstinputlisting[caption=\OptimizingKeilVI (\ThumbMode),style=customasmARM]{patterns/07_jcc/cond_operator/Keil_thumb_O3_JA.s}

\subsubsection{ARM64}

ARM64の \Optimizing GCC(Linaro)4.9でも、条件付きジャンプが使用されます。

\lstinputlisting[label=cond_ARM64,caption=\Optimizing GCC (Linaro) 4.9,style=customasmARM]{patterns/07_jcc/cond_operator/ARM64_GCC_O3_JA.s}

これは、ARM64には32ビットARMモードの\TT{ADRcc}やx86の\INS{CMOVcc}などの
条件フラグを伴った単純なロード命令がないためです。

\myindex{ARM!\Instructions!CSEL}
しかし、\q{Conditional SELect}命令(\TT{CSEL})\InSqBrackets{\ARMSixFourRef p390, C5.5}を使用していますが、
GCC 4.9ではこのようなコードの中で使用するには十分スマートではないようです。

\subsubsection{MIPS}

残念ながら、MIPS用のGCC 4.4.5はそれほどスマートではありません。

\lstinputlisting[caption=\Optimizing GCC 4.4.5 (\assemblyOutput),style=customasmMIPS]{patterns/07_jcc/cond_operator/MIPS_O3_JA.s}

\subsubsection{\TT{if/else}の方法で書き直しましょう}

\lstinputlisting[style=customc]{patterns/07_jcc/cond_operator/cond2.c}

\myindex{x86!\Instructions!CMOVcc}

興味深いことに、x86用のGCC 4.8の最適化は、この場合に\TT{CMOVcc}を使用することもできました。

\lstinputlisting[caption=\Optimizing GCC 4.8,style=customasmx86]{patterns/07_jcc/cond_operator/cond2_GCC_O3_JA.s}

ARMモードの \Optimizing Keilでは、\lstref{cond_Keil_ARM_O3}と同じコードが生成されます。

しかし、MSVC 2012の最適化は(まだ)あまり良くありません。

\subsubsection{\Conclusion{}}

コンパイラを最適化するとどうして条件付きジャンプを取り除こうとするのでしょうか?以下を読んでください:\myref{branch_predictors}
}

\EN{\subsection{Getting minimal and maximal values}

\subsubsection{32-bit}

\lstinputlisting[style=customc]{patterns/07_jcc/minmax/minmax.c}

\lstinputlisting[caption=\NonOptimizing MSVC 2013,style=customasmx86]{patterns/07_jcc/minmax/minmax_MSVC_2013_EN.asm}

\myindex{x86!\Instructions!Jcc}

These two functions differ only in the conditional jump instruction: 
\INS{JGE} (\q{Jump if Greater or Equal}) is used in the first one
and \INS{JLE} (\q{Jump if Less or Equal}) in the second.

\myindex{\CompilerAnomaly}
\label{MSVC_double_JMP_anomaly}

There is one unneeded \JMP instruction in each function, which MSVC presumably left by mistake.

\myparagraph{Branchless}

ARM for Thumb mode reminds us of x86 code:

\lstinputlisting[caption=\OptimizingKeilVI (\ThumbMode),style=customasmARM]{patterns/07_jcc/minmax/minmax_Keil_Thumb_O3_EN.s}

\myindex{ARM!\Instructions!Bcc}

The functions differ in the branching instruction: \INS{BGT} and \INS{BLT}.
It's possible to use conditional suffixes in ARM mode, so the code is shorter.

\myindex{ARM!\Instructions!MOVcc}
\INS{MOVcc} is to be executed only if the condition is met:

\lstinputlisting[caption=\OptimizingKeilVI (\ARMMode),style=customasmARM]{patterns/07_jcc/minmax/minmax_Keil_ARM_O3_EN.s}

\myindex{x86!\Instructions!CMOVcc}
\Optimizing GCC 4.8.1 and optimizing MSVC 2013 can use \INS{CMOVcc} instruction, which is analogous to \INS{MOVcc} in ARM:

\lstinputlisting[caption=\Optimizing MSVC 2013,style=customasmx86]{patterns/07_jcc/minmax/minmax_GCC481_O3_EN.s}

\subsubsection{64-bit}

\lstinputlisting[style=customc]{patterns/07_jcc/minmax/minmax64.c}

There is some unneeded value shuffling, but the code is comprehensible:

\lstinputlisting[caption=\NonOptimizing GCC 4.9.1 ARM64,style=customasmARM]{patterns/07_jcc/minmax/minmax64_GCC_49_ARM64_O0.s}

\myparagraph{Branchless}

No need to load function arguments from the stack, as they are already in the registers:

\lstinputlisting[caption=\Optimizing GCC 4.9.1 x64,style=customasmx86]{patterns/07_jcc/minmax/minmax64_GCC_49_x64_O3_EN.s}

MSVC 2013 does almost the same.

\myindex{ARM!\Instructions!CSEL}

ARM64 has the \INS{CSEL} instruction, which works just as \INS{MOVcc} in ARM or \INS{CMOVcc} in x86, just the name is different:
\q{Conditional SELect}.

\lstinputlisting[caption=\Optimizing GCC 4.9.1 ARM64,style=customasmARM]{patterns/07_jcc/minmax/minmax64_GCC_49_ARM64_O3_EN.s}

\subsubsection{MIPS}

Unfortunately, GCC 4.4.5 for MIPS is not that good:

\lstinputlisting[caption=\Optimizing GCC 4.4.5 (IDA),style=customasmMIPS]{patterns/07_jcc/minmax/minmax_MIPS_O3_IDA_EN.lst}

Do not forget about the \emph{branch delay slots}: the first \INS{MOVE} is executed \emph{before} \INS{BEQZ}, 
the second \INS{MOVE} is executed only if the branch hasn't been taken.

}
\RU{\subsection{Поиск минимального и максимального значения}

\subsubsection{32-bit}

\lstinputlisting[style=customc]{patterns/07_jcc/minmax/minmax.c}

\lstinputlisting[caption=\NonOptimizing MSVC 2013,style=customasmx86]{patterns/07_jcc/minmax/minmax_MSVC_2013_RU.asm}

\myindex{x86!\Instructions!Jcc}
Эти две функции отличаются друг от друга только инструкцией условного перехода:
\INS{JGE} (\q{Jump if Greater or Equal}~--- переход если больше или равно) используется в первой
и \INS{JLE} (\q{Jump if Less or Equal}~--- переход если меньше или равно) во второй.

\myindex{\CompilerAnomaly}
\label{MSVC_double_JMP_anomaly}
Здесь есть ненужная инструкция \JMP в каждой функции, которую MSVC, наверное, оставил по ошибке.

\myparagraph{Без переходов}

ARM в режиме Thumb напоминает нам x86-код:

\lstinputlisting[caption=\OptimizingKeilVI (\ThumbMode),style=customasmARM]{patterns/07_jcc/minmax/minmax_Keil_Thumb_O3_RU.s}

\myindex{ARM!\Instructions!Bcc}
Функции отличаются только инструкцией перехода: \INS{BGT} и \INS{BLT}.
А в режиме ARM можно использовать условные суффиксы, так что код более плотный.
\INS{MOVcc} будет исполнена только если условие верно:

\myindex{ARM!\Instructions!MOVcc}

\lstinputlisting[caption=\OptimizingKeilVI (\ARMMode),style=customasmARM]{patterns/07_jcc/minmax/minmax_Keil_ARM_O3_RU.s}

\myindex{x86!\Instructions!CMOVcc}
\Optimizing GCC 4.8.1 и оптимизирующий MSVC 2013 
могут использовать инструкцию \INS{CMOVcc}, которая аналогична \INS{MOVcc} в ARM:

\lstinputlisting[caption=\Optimizing MSVC 2013,style=customasmx86]{patterns/07_jcc/minmax/minmax_GCC481_O3_RU.s}

\subsubsection{64-bit}

\lstinputlisting[style=customc]{patterns/07_jcc/minmax/minmax64.c}

Тут есть ненужные перетасовки значений, но код в целом понятен:

\lstinputlisting[caption=\NonOptimizing GCC 4.9.1 ARM64,style=customasmARM]{patterns/07_jcc/minmax/minmax64_GCC_49_ARM64_O0.s}

\myparagraph{Без переходов}

Нет нужды загружать аргументы функции из стека, они уже в регистрах:

\lstinputlisting[caption=\Optimizing GCC 4.9.1 x64,style=customasmx86]{patterns/07_jcc/minmax/minmax64_GCC_49_x64_O3_RU.s}

MSVC 2013 делает то же самое.

\myindex{ARM!\Instructions!CSEL}
В ARM64 есть инструкция \INS{CSEL}, которая работает точно также, как и \INS{MOVcc} в ARM и \INS{CMOVcc} в x86,
но название другое: \q{Conditional SELect}.

\lstinputlisting[caption=\Optimizing GCC 4.9.1 ARM64,style=customasmARM]{patterns/07_jcc/minmax/minmax64_GCC_49_ARM64_O3_RU.s}

\subsubsection{MIPS}

А GCC 4.4.5 для MIPS не так хорош, к сожалению:

\lstinputlisting[caption=\Optimizing GCC 4.4.5 (IDA),style=customasmMIPS]{patterns/07_jcc/minmax/minmax_MIPS_O3_IDA_RU.lst}

Не забывайте о \emph{branch delay slots}: первая \INS{MOVE} исполняется \emph{перед} \INS{BEQZ},
вторая \INS{MOVE} исполняется только если переход не произошел.

}
\DE{\subsection{Minimale und maximale Werte berechnen}

\subsubsection{32-bit}

\lstinputlisting[style=customc]{patterns/07_jcc/minmax/minmax.c}

\lstinputlisting[caption=\NonOptimizing MSVC 2013,style=customasmx86]{patterns/07_jcc/minmax/minmax_MSVC_2013_DE.asm}

\myindex{x86!\Instructions!Jcc}
Diese beiden Funktionen unterscheiden sich nur hinsichtliche der bedingten Sprungbefehle:
\INS{JGE} (\q{Jump if Greater or Equal}) wird in der ersten verwendet
und \INS{JLE} (\q{Jump if Less or Equal}) in der zweiten.

\myindex{\CompilerAnomaly}
\label{MSVC_double_JMP_anomaly}
Hier gibt es jeweils einen unnötigen \JMP Befehl pro Funtion, den MSVC wahrscheinlich fehlerhafterweise dort belassen
hat.

\myparagraph{Verzweigungslos}

ARM im Thumb mode erinnert uns an den x86 Code:

\lstinputlisting[caption=\OptimizingKeilVI
(\ThumbMode),style=customasmARM]{patterns/07_jcc/minmax/minmax_Keil_Thumb_O3_DE.s}

\myindex{ARM!\Instructions!Bcc}
Die Funktionen unterscheiden sich in den Verzweigebefehlen: \INS{BGT} und \INS{BLT}.
Es ist möglich im ARM mode conditional codes zu verwenden, sodass der Code kürzer ist.

\myindex{ARM!\Instructions!MOVcc}
\INS{MOVcc} wird nur ausgeführt, wenn die Bedingung erfüllt (d.h. wahr) ist:

\lstinputlisting[caption=\OptimizingKeilVI
(\ARMMode),style=customasmARM]{patterns/07_jcc/minmax/minmax_Keil_ARM_O3_DE.s}

\myindex{x86!\Instructions!CMOVcc}
\Optimizing GCC 4.8.1 und der optimierende MSVC 2013 können den \INS{CMOVcc} Befehl verwenden, der analog zu
\INS{MOVcc} in ARM funktioniert:

\lstinputlisting[caption=\Optimizing MSVC 2013,style=customasmx86]{patterns/07_jcc/minmax/minmax_GCC481_O3_DE.s}

\subsubsection{64-bit}

\lstinputlisting[style=customc]{patterns/07_jcc/minmax/minmax64.c}
Hier findet ein unnötiges Verschieben von Variablen statt, aber der Code ist verständlich:

\lstinputlisting[caption=\NonOptimizing GCC 4.9.1 ARM64,style=customasmARM]{patterns/07_jcc/minmax/minmax64_GCC_49_ARM64_O0.s}

\myparagraph{Verzweigungslos}
Die Funtionsargumente müssen nicht vom Stack geladen werden, da sie sich bereits in den Registern befinden:

\lstinputlisting[caption=\Optimizing GCC 4.9.1
x64,style=customasmx86]{patterns/07_jcc/minmax/minmax64_GCC_49_x64_O3_DE.s}

MSVC 2013 tut beinahe das gleiche:

\myindex{ARM!\Instructions!CSEL}

ARM64 verfügt über den \INS{CSEL} Befehl, der genau wie \INS{MOVcc} in ARM oder \INS{CMOVcc} in x86 arbeitet; er hat
lediglich einen anderen Namen:
\q{Conditional SELect}.

\lstinputlisting[caption=\Optimizing GCC 4.9.1
ARM64,style=customasmARM]{patterns/07_jcc/minmax/minmax64_GCC_49_ARM64_O3_DE.s}

\subsubsection{MIPS}

Leider ist GCC 4.4.5 für MIPS nicht so gut:

\lstinputlisting[caption=\Optimizing GCC 4.4.5
(IDA),style=customasmMIPS]{patterns/07_jcc/minmax/minmax_MIPS_O3_IDA_DE.lst} 
Vergessen Sie nicht die \emph{branch delay slots}: der erste \INS{MOVE} wird \emph{vor} \INS{BEQZ} ausgeführt, der zweite
\INS{MOVE} wird nur dann ausgeführt, wenn die Verzweigung nicht genommen wird.


}
\FR{\subsection{Trouver les valeurs minimale et maximale}

\subsubsection{32-bit}

\lstinputlisting[style=customc]{patterns/07_jcc/minmax/minmax.c}

\lstinputlisting[caption=MSVC 2013 \NonOptimizing,style=customasmx86]{patterns/07_jcc/minmax/minmax_MSVC_2013_FR.asm}

\myindex{x86!\Instructions!Jcc}

Ces deux fonctions ne diffèrent que de l'instruction de saut conditionnel:
\INS{JGE} (\q{Jump if Greater or Equal} saut si supérieur ou égal) est utilisée
dans la première et \INS{JLE} (\q{Jump if Less or Equal} saut si inférieur ou égal)
dans la seconde.

\myindex{\CompilerAnomaly}
\label{MSVC_double_JMP_anomaly}

Il y a une instruction \JMP en trop dans chaque fonction, que MSVC a probablement
mise par erreur.

\myparagraph{Sans branchement}

Le mode Thumb d'ARM nous rappelle le code x86:

\lstinputlisting[caption=\OptimizingKeilVI (\ThumbMode),style=customasmARM]{patterns/07_jcc/minmax/minmax_Keil_Thumb_O3_FR.s}

\myindex{ARM!\Instructions!Bcc}

Les fonctions diffèrent au niveau de l'instruction de branchement: \INS{BGT} et \INS{BLT}.
Il est possible d'utiliser le suffixe conditionnel en mode ARM, donc le code est plus
court.

\myindex{ARM!\Instructions!MOVcc}
\INS{MOVcc} n'est exécutée que si la condition est remplie:

\lstinputlisting[caption=\OptimizingKeilVI (\ARMMode),style=customasmARM]{patterns/07_jcc/minmax/minmax_Keil_ARM_O3_FR.s}

\myindex{x86!\Instructions!CMOVcc}
GCC 4.8.1 \Optimizing et MSVC 2013 \Optimizing peuvent utiliser l'instruction \INS{CMOVcc},
qui est analogue à \INS{MOVcc} en ARM:

\lstinputlisting[caption=MSVC 2013 \Optimizing,style=customasmx86]{patterns/07_jcc/minmax/minmax_GCC481_O3_FR.s}

\subsubsection{64-bit}

\lstinputlisting[style=customc]{patterns/07_jcc/minmax/minmax64.c}

Il y a beaucoup de code inutile qui embrouille, mais il est compréhensible:

\lstinputlisting[caption=GCC 4.9.1 ARM64 \NonOptimizing,style=customasmARM]{patterns/07_jcc/minmax/minmax64_GCC_49_ARM64_O0.s}

\myparagraph{Sans branchement}

Il n'y a pas besoin de lire les arguments dans la pile, puisqu'ils sont déjà dans
les registres:

\lstinputlisting[caption=GCC 4.9.1 x64 \Optimizing,style=customasmx86]{patterns/07_jcc/minmax/minmax64_GCC_49_x64_O3_FR.s}

MSVC 2013 fait presque la même chose.

\myindex{ARM!\Instructions!CSEL}

ARM64 possède l'instruction \INS{CSEL}, qui fonctionne comme \INS{MOVcc} en ARM ou
\INS{CMOVcc} en x86, seul le nom diffère:
\q{Conditional SELect}.

\lstinputlisting[caption=GCC 4.9.1 ARM64 \Optimizing,style=customasmARM]{patterns/07_jcc/minmax/minmax64_GCC_49_ARM64_O3_FR.s}

\subsubsection{MIPS}

Malheureusement, GCC 4.4.5 pour MIPS n'est pas si performant:

\lstinputlisting[caption=GCC 4.4.5 \Optimizing (IDA),style=customasmMIPS]{patterns/07_jcc/minmax/minmax_MIPS_O3_IDA_FR.lst}

N'oubliez pas le slot de délai de branchement (\emph{branch delay slots}): le premier
\INS{MOVE} est exécuté \emph{avant} \INS{BEQZ}, le second \INS{MOVE} n'est exécuté
que si la branche n'a pas été prise.

}
\JA{\subsection{最小値と最大値の取得}

\subsubsection{32-bit}

\lstinputlisting[style=customc]{patterns/07_jcc/minmax/minmax.c}

\lstinputlisting[caption=\NonOptimizing MSVC 2013,style=customasmx86]{patterns/07_jcc/minmax/minmax_MSVC_2013_JA.asm}

\myindex{x86!\Instructions!Jcc}

これらの2つの機能は条件ジャンプ命令でのみ異なります。
最初の命令では\INS{JGE} (\q{Jump if Greater or Equal})が使用され、
2番目の場合は\INS{JLE} (\q{Jump if Less or Equal})が使用されます。

\myindex{\CompilerAnomaly}
\label{MSVC_double_JMP_anomaly}

各関数には不必要な \JMP 命令が1つありますが、おそらく誤って残っています。

\myparagraph{分岐}

ThumbモードのARMは、x86コードを思い起こします。

\lstinputlisting[caption=\OptimizingKeilVI (\ThumbMode),style=customasmARM]{patterns/07_jcc/minmax/minmax_Keil_Thumb_O3_JA.s}

\myindex{ARM!\Instructions!Bcc}

関数は分岐命令が異なります。\INS{BGT} と \INS{BLT}です。
ARMモードでは条件付きの接尾辞を使用することができるため、コードは短くなります。

\myindex{ARM!\Instructions!MOVcc}
\INS{MOVcc}は、条件が満たされた場合にのみ実行されます。

\lstinputlisting[caption=\OptimizingKeilVI (\ARMMode),style=customasmARM]{patterns/07_jcc/minmax/minmax_Keil_ARM_O3_JA.s}

\myindex{x86!\Instructions!CMOVcc}
\Optimizing GCC 4.8.1とMSVC 2013の最適化では、ARMの\INS{CMOVcc}に似た\INS{MOVcc}命令を使用できます。

\lstinputlisting[caption=\Optimizing MSVC 2013,style=customasmx86]{patterns/07_jcc/minmax/minmax_GCC481_O3_JA.s}

\subsubsection{64-bit}

\lstinputlisting[style=customc]{patterns/07_jcc/minmax/minmax64.c}

いくつかの不要な値のシャッフルがありますが、コードは理解できます。

\lstinputlisting[caption=\NonOptimizing GCC 4.9.1 ARM64,style=customasmARM]{patterns/07_jcc/minmax/minmax64_GCC_49_ARM64_O0.s}

\myparagraph{分岐なし}

スタックから関数の引数をロードする必要はありません。レジスタにすでに入っています。

\lstinputlisting[caption=\Optimizing GCC 4.9.1 x64,style=customasmx86]{patterns/07_jcc/minmax/minmax64_GCC_49_x64_O3_JA.s}

MSVC 2013はほぼ同じです。

\myindex{ARM!\Instructions!CSEL}

ARM64にはARMの\INS{MOVcc}またはx86の\INS{CMOVcc}と同じように機能する\INS{CSEL}命令がありますが、
その名前は\q{Conditional SELect}とは異なります。

\lstinputlisting[caption=\Optimizing GCC 4.9.1 ARM64,style=customasmARM]{patterns/07_jcc/minmax/minmax64_GCC_49_ARM64_O3_JA.s}

\subsubsection{MIPS}

残念ながら、MIPS用のGCC 4.4.5はあまり良くありません。

\lstinputlisting[caption=\Optimizing GCC 4.4.5 (IDA),style=customasmMIPS]{patterns/07_jcc/minmax/minmax_MIPS_O3_IDA_JA.lst}

\emph{分岐遅延スロット}を忘れないでください。最初の\INS{MOVE}は\INS{BEQZ}の\emph{前に}実行され、
2番目の\INS{MOVE}は分岐が実行されなかった場合にのみ実行されます。
}


\subsection{\Conclusion{}}

\subsubsection{x86}

Voici le squelette générique d'un saut conditionnel:

\begin{lstlisting}[caption=x86,style=customasmx86]
CMP registre, registre/valeur
Jcc true ; cc=condition code, code de condition
false:
... le code qui sera exécuté si le résultat de la comparaison est faux (false) ...
JMP exit 
true:
... le code qui sera exécuté si le résultat de la comparaison est vrai (true) ...
exit:
\end{lstlisting}

\subsubsection{ARM}

\begin{lstlisting}[caption=ARM,style=customasmARM]
CMP registre, registre/valeur
Bcc true ; cc=condition code
false:
... le code qui sera exécuté si le résultat de la comparaison est faux (false) ...
JMP exit 
true:
... le code qui sera exécuté si le résultat de la comparaison est vrai (true) ...
exit:
\end{lstlisting}

\subsubsection{MIPS}

\begin{lstlisting}[caption=Check si zéro (Branch if EQual Zero),style=customasmMIPS]
BEQZ REG, label
...
\end{lstlisting}

\begin{lstlisting}[caption=Check si plus petit que zéro (Branch if Less Than Zero) en utilisant une pseudo instruction,style=customasmMIPS]
BLTZ REG, label
...
\end{lstlisting}

\begin{lstlisting}[caption=Check si les valeurs sont égales (Branch if EQual),style=customasmMIPS]
BEQ REG1, REG2, label
...
\end{lstlisting}

\begin{lstlisting}[caption=Check si les valeurs ne sont pas égales (Branch if Not Equal),style=customasmMIPS]
BNE REG1, REG2, label
...
\end{lstlisting}

\begin{lstlisting}[caption=Check REG2 plus petit que REG3 (signé),style=customasmMIPS]
SLT REG1, REG2, REG3
BEQ REG1, label
...
\end{lstlisting}

\begin{lstlisting}[caption=Check REG2 plus petit que REG3 (non signé),style=customasmMIPS]
SLTU REG1, REG2, REG3
BEQ REG1, label
...
\end{lstlisting}

\subsubsection{Sans branchement}

\myindex{ARM!\Instructions!MOVcc}
\myindex{x86!\Instructions!CMOVcc}
\myindex{ARM!\Instructions!CSEL}
Si le corps d'instruction conditionnelle est très petit, l'instruction de déplacement
conditionnel peut être utilisée:
\INS{MOVcc} en ARM (en mode ARM), \INS{CSEL} en ARM64, \INS{CMOVcc} en x86.

\myparagraph{ARM}

Il est possible d'utiliser les suffixes conditionnels pour certaines instructions
ARM:

\begin{lstlisting}[caption=ARM (\ARMMode),style=customasmARM]
CMP registre, registre/valeur
instr1_cc ; cette instruction sera exécutée si le code conditionnel est vrai (true)
instr2_cc ; cette autre instruction sera exécutée si cet autre code conditionnel est vrai (true)
... etc.
\end{lstlisting}

Bien sûr, il n'y a pas de limite au nombre d'instructions avec un suffixe de code
conditionnel, tant que les flags du CPU ne sont pas modifiés par l'une d'entre elles.
% FIXME: list of such instructions or \myref{} to it

\myindex{ARM!\Instructions!IT}

Le mode Thumb possède l'instruction \INS{IT}, permettant d'ajouter le suffixe conditionnel
pour les quatre instructions suivantes.
Lire à ce propos: \myref{ARM_Thumb_IT}.

\begin{lstlisting}[caption=ARM (\ThumbMode),style=customasmARM]
CMP registre, registre/valeur
ITEEE EQ ; met ces suffixes: if-then-else-else-else
instr1   ; instruction exécutée si la condition est vraie
instr2   ; instruction exécutée si la condition est fausse
instr3   ; instruction exécutée si la condition est fausse
instr4   ; instruction exécutée si la condition est fausse
\end{lstlisting}

\subsection{\Exercise}

(ARM64) Essayez de récrire le code pour \lstref{cond_ARM64} en supprimant toutes
les instructions de saut conditionnel et en utilisant l'instruction \TT{CSEL}.

