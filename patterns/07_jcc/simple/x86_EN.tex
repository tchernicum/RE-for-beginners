\subsubsection{x86}

\myparagraph{x86 + MSVC}

Here is how the \TT{f\_signed()} function looks like:

\lstinputlisting[caption=\NonOptimizing MSVC 2010,style=customasmx86]{patterns/07_jcc/simple/signed_MSVC.asm}

\myindex{x86!\Instructions!JLE}

The first instruction, \JLE, stands for \emph{Jump if Less or Equal}. 
In other words, if the second operand is 
larger or equal to the first one, the control flow will be passed to the address or label specified in the instruction.
If this condition does not trigger because the second operand is smaller than the first one, the control flow would not be altered and the first \printf would be executed.
\myindex{x86!\Instructions!JNE}
The second check is \JNE: \emph{Jump if Not Equal}.
The control flow will not change if the operands are equal.

\myindex{x86!\Instructions!JGE}
The third check is \JGE: \emph{Jump if Greater or Equal}---jump if the first operand is larger than 
the second or if they are equal.
So, if all three conditional jumps are triggered, none of the \printf calls would be executed whatsoever. 
This is impossible without special intervention.
Now let's take a look at the \TT{f\_unsigned()} function.
The \TT{f\_unsigned()} function is the same as \TT{f\_signed()}, with the exception that the \JBE and \JAE instructions
are used instead of \JLE and \JGE, as follows:

\lstinputlisting[caption=GCC,style=customasmx86]{patterns/07_jcc/simple/unsigned_MSVC.asm}

\myindex{x86!\Instructions!JBE}
\myindex{x86!\Instructions!JAE}

As already mentioned, the branch instructions are different:
\JBE---\emph{Jump if Below or Equal} and \JAE---\emph{Jump if Above or Equal}.
These instructions (\INS{JA}/\JAE/\JB/\JBE) differ from \JG/\JGE/\JL/\JLE in the fact that they work with unsigned numbers.

\myindex{x86!\Instructions!JA}
\myindex{x86!\Instructions!JB}
\myindex{x86!\Instructions!JG}
\myindex{x86!\Instructions!JL}
\myindex{Signed numbers}

See also the section about signed number representations~(\myref{sec:signednumbers}).
That is why if we see \JG/\JL in use instead of \INS{JA}/\JB or vice-versa, 
we can be almost sure that the variables are signed or unsigned, respectively.
Here is also the \main function, where there is nothing much new to us:

\lstinputlisting[caption=\main,style=customasmx86]{patterns/07_jcc/simple/main_MSVC.asm}

\clearpage
\mysubparagraph{\olly}
\myindex{\olly}

Let's try this example in \olly.
The input value of the function (2) is loaded into \EAX: 

\begin{figure}[H]
\centering
\myincludegraphics{patterns/08_switch/2_lot/olly1.png}
\caption{\olly: function's input value is loaded in \EAX}
\label{fig:switch_lot_olly1}
\end{figure}

\clearpage
The input value is checked, is it bigger than 4? 
If not, the \q{default} jump is not taken:
\begin{figure}[H]
\centering
\myincludegraphics{patterns/08_switch/2_lot/olly2.png}
\caption{\olly: 2 is no bigger than 4: no jump is taken}
\label{fig:switch_lot_olly2}
\end{figure}

\clearpage
Here we see a jumptable:

\begin{figure}[H]
\centering
\myincludegraphics{patterns/08_switch/2_lot/olly3.png}
\caption{\olly: calculating destination address using jumptable}
\label{fig:switch_lot_olly3}
\end{figure}

Here we've clicked \q{Follow in Dump} $\rightarrow$ \q{Address constant}, so now we see the \emph{jumptable} in the data window.
These are 5 32-bit values\footnote{They are underlined by \olly because
these are also FIXUPs: \myref{subsec:relocs}, we are going to come back to them later}.
\ECX is now 2, so the third element (can be indexed as 2\footnote{About indexing, see also: \ref{arrays_at_one}}) of the table is to be used.
It's also possible to click \q{Follow in Dump} $\rightarrow$ 
\q{Memory address} and \olly will show the element addressed by the \JMP instruction. 
That's \TT{0x010B103A}.

\clearpage
After the jump we are at \TT{0x010B103A}: the code printing \q{two} will now be executed:

\begin{figure}[H]
\centering
\myincludegraphics{patterns/08_switch/2_lot/olly4.png}
\caption{\olly: now we at the \emph{case:} label}
\label{fig:switch_lot_olly4}
\end{figure}


\clearpage
\myparagraph{x86 + MSVC + Hiew}
\myindex{Hiew}

We can try to patch the executable file in a way 
that the \TT{f\_unsigned()} function would always print \q{a==b}, 
no matter the input values.
Here is how it looks in Hiew:

\begin{figure}[H]
\centering
\myincludegraphics{patterns/07_jcc/simple/hiew_unsigned1.png}
\caption{Hiew: \TT{f\_unsigned()} function}
\label{fig:jcc_hiew_1}
\end{figure}

Essentially, we have to accomplish three tasks:
\begin{itemize}
\item force the first jump to always trigger;
\item force the second jump to never trigger;
\item force the third jump to always trigger.
\end{itemize}

Thus we can direct the code flow to always pass through the second \printf, and output \q{a==b}.

Three instructions (or bytes) has to be patched:

\begin{itemize}
\item The first jump becomes \JMP, but the \gls{jump offset} would remain the same.

\item 
The second jump might be triggered sometimes, but in any case it will jump to the next
instruction, because, we set the \gls{jump offset} to 0.

In these instructions the \gls{jump offset} is added to the address for the next instruction.
So if the offset is 0,
the jump will transfer the control to the next instruction.

\item 
The third jump we replace with \JMP just as we do with the first one, so it will always trigger.

\end{itemize}

\clearpage
Here is the modified code:

\begin{figure}[H]
\centering
\myincludegraphics{patterns/07_jcc/simple/hiew_unsigned2.png}
\caption{Hiew: let's modify the \TT{f\_unsigned()} function}
\label{fig:jcc_hiew_2}
\end{figure}

If we miss to change any of these jumps, then several \printf calls may execute, while we want to execute only one.

\myparagraph{\NonOptimizing GCC}

\myindex{puts() instead of printf()}
\NonOptimizing GCC 4.4.1 
produces almost the same code, but with \puts~(\myref{puts}) instead of \printf.

\myparagraph{\Optimizing GCC}

An observant reader may ask, why execute \CMP several times, 
if the flags has the same values after each execution?

Perhaps optimizing MSVC cannot do this, but optimizing GCC 4.8.1 can go deeper:

\lstinputlisting[caption=GCC 4.8.1 f\_signed(),style=customasmx86]{patterns/07_jcc/simple/GCC_O3_signed.asm}

% should be here instead of 'switch' section?
We also see \TT{JMP puts} here instead of \TT{CALL puts / RETN}.

This kind of trick will have explained later: \myref{JMP_instead_of_RET}.

This type of x86 code 
is somewhat rare.
MSVC 2012 as it seems, can't generate such code.
On the other hand, assembly language programmers are fully aware of the fact that \TT{Jcc} 
instructions can be stacked.

So if you see such stacking somewhere, it is highly probable that the code was hand-written.

The \TT{f\_unsigned()} function is not that 
\ae{}sthetically short:

\lstinputlisting[caption=GCC 4.8.1 f\_unsigned(),style=customasmx86]{patterns/07_jcc/simple/GCC_O3_unsigned_EN.asm}

Nevertheless, there are two \TT{CMP} instructions instead of three.

So optimization algorithms of GCC 4.8.1 are probably not perfect yet. 
