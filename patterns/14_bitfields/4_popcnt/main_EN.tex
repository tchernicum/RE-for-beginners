\subsection{Counting bits set to 1}

Here is a simple example of a function that calculates the number of bits set in the input value.

This operation is also called \q{population count}\footnote{modern x86 CPUs (supporting SSE4) even have a POPCNT instruction for it}.

\lstinputlisting[style=customc]{patterns/14_bitfields/4_popcnt/shifts.c}

In this loop, the iteration count value $i$ is counting from 0 to 31, 
so the $1 \ll i$ statement is counting from 1 to \TT{0x80000000}.
Describing this operation in natural language, we would say \emph{shift 1 by n bits left}.
In other words, $1 \ll i$ statement consequently produces all possible bit positions in a 32-bit number.
The freed bit at right is always cleared.

\label{2n_numbers_table}
Here is a table of all possible $1 \ll i$ 
for $i=0 \ldots 31$:

%\small
\begin{center}
\begin{tabular}{ | l | l | l | l | }
\hline
\HeaderColor \CCpp expression & 
\HeaderColor Power of two & 
\HeaderColor Decimal form & 
\HeaderColor Hexadecimal form \\
\hline
$1 \ll 0$ & $2^{0}$ & 1 & 1 \\
\hline
$1 \ll 1$ & $2^{1}$ & 2 & 2 \\
\hline
$1 \ll 2$ & $2^{2}$ & 4 & 4 \\
\hline
$1 \ll 3$ & $2^{3}$ & 8 & 8 \\
\hline
$1 \ll 4$ & $2^{4}$ & 16 & 0x10 \\
\hline
$1 \ll 5$ & $2^{5}$ & 32 & 0x20 \\
\hline
$1 \ll 6$ & $2^{6}$ & 64 & 0x40 \\
\hline
$1 \ll 7$ & $2^{7}$ & 128 & 0x80 \\
\hline
$1 \ll 8$ & $2^{8}$ & 256 & 0x100 \\
\hline
$1 \ll 9$ & $2^{9}$ & 512 & 0x200 \\
\hline
$1 \ll 10$ & $2^{10}$ & 1024 & 0x400 \\
\hline
$1 \ll 11$ & $2^{11}$ & 2048 & 0x800 \\
\hline
$1 \ll 12$ & $2^{12}$ & 4096 & 0x1000 \\
\hline
$1 \ll 13$ & $2^{13}$ & 8192 & 0x2000 \\
\hline
$1 \ll 14$ & $2^{14}$ & 16384 & 0x4000 \\
\hline
$1 \ll 15$ & $2^{15}$ & 32768 & 0x8000 \\
\hline
$1 \ll 16$ & $2^{16}$ & 65536 & 0x10000 \\
\hline
$1 \ll 17$ & $2^{17}$ & 131072 & 0x20000 \\
\hline
$1 \ll 18$ & $2^{18}$ & 262144 & 0x40000 \\
\hline
$1 \ll 19$ & $2^{19}$ & 524288 & 0x80000 \\
\hline
$1 \ll 20$ & $2^{20}$ & 1048576 & 0x100000 \\
\hline
$1 \ll 21$ & $2^{21}$ & 2097152 & 0x200000 \\
\hline
$1 \ll 22$ & $2^{22}$ & 4194304 & 0x400000 \\
\hline
$1 \ll 23$ & $2^{23}$ & 8388608 & 0x800000 \\
\hline
$1 \ll 24$ & $2^{24}$ & 16777216 & 0x1000000 \\
\hline
$1 \ll 25$ & $2^{25}$ & 33554432 & 0x2000000 \\
\hline
$1 \ll 26$ & $2^{26}$ & 67108864 & 0x4000000 \\
\hline
$1 \ll 27$ & $2^{27}$ & 134217728 & 0x8000000 \\
\hline
$1 \ll 28$ & $2^{28}$ & 268435456 & 0x10000000 \\
\hline
$1 \ll 29$ & $2^{29}$ & 536870912 & 0x20000000 \\
\hline
$1 \ll 30$ & $2^{30}$ & 1073741824 & 0x40000000 \\
\hline
$1 \ll 31$ & $2^{31}$ & 2147483648 & 0x80000000 \\
\hline
\end{tabular}
\end{center}
%\normalsize

These constant numbers (bit masks) very often appear in code and a practicing reverse engineer 
must be able to spot them quickly.

Decimal numbers below 65536 and hexadecimal ones are very easy to memorize.
While decimal numbers above 65536 are, probably, not worth memorizing.

These constants are very often used for mapping flags to specific bits.
For example, here is excerpt from \TT{ssl\_private.h} 
from Apache 2.4.6 source code:

\begin{lstlisting}[style=customc]
/**
 * Define the SSL options
 */
#define SSL_OPT_NONE           (0)
#define SSL_OPT_RELSET         (1<<0)
#define SSL_OPT_STDENVVARS     (1<<1)
#define SSL_OPT_EXPORTCERTDATA (1<<3)
#define SSL_OPT_FAKEBASICAUTH  (1<<4)
#define SSL_OPT_STRICTREQUIRE  (1<<5)
#define SSL_OPT_OPTRENEGOTIATE (1<<6)
#define SSL_OPT_LEGACYDNFORMAT (1<<7)
\end{lstlisting}

Let's get back to our example.

The \TT{IS\_SET} macro checks bit presence in $a$.
\myindex{x86!\Instructions!AND}

The \TT{IS\_SET} macro is in fact the logical AND operation (\emph{AND}) 
and it returns 0 if the specific bit is absent there,
or the bit mask, if the bit is present.
\emph{The if()} operator in \CCpp triggers if the expression in it is not zero, it might be even 123456, that is why
it always works correctly.

% subsections
\subsubsection{x86}

\myparagraph{x86 + MSVC}

Here is how the \TT{f\_signed()} function looks like:

\lstinputlisting[caption=\NonOptimizing MSVC 2010,style=customasmx86]{patterns/07_jcc/simple/signed_MSVC.asm}

\myindex{x86!\Instructions!JLE}

The first instruction, \JLE, stands for \emph{Jump if Less or Equal}. 
In other words, if the second operand is 
larger or equal to the first one, the control flow will be passed to the address or label specified in the instruction.
If this condition does not trigger because the second operand is smaller than the first one, the control flow would not be altered and the first \printf would be executed.
\myindex{x86!\Instructions!JNE}
The second check is \JNE: \emph{Jump if Not Equal}.
The control flow will not change if the operands are equal.

\myindex{x86!\Instructions!JGE}
The third check is \JGE: \emph{Jump if Greater or Equal}---jump if the first operand is larger than 
the second or if they are equal.
So, if all three conditional jumps are triggered, none of the \printf calls would be executed whatsoever. 
This is impossible without special intervention.
Now let's take a look at the \TT{f\_unsigned()} function.
The \TT{f\_unsigned()} function is the same as \TT{f\_signed()}, with the exception that the \JBE and \JAE instructions
are used instead of \JLE and \JGE, as follows:

\lstinputlisting[caption=GCC,style=customasmx86]{patterns/07_jcc/simple/unsigned_MSVC.asm}

\myindex{x86!\Instructions!JBE}
\myindex{x86!\Instructions!JAE}

As already mentioned, the branch instructions are different:
\JBE---\emph{Jump if Below or Equal} and \JAE---\emph{Jump if Above or Equal}.
These instructions (\INS{JA}/\JAE/\JB/\JBE) differ from \JG/\JGE/\JL/\JLE in the fact that they work with unsigned numbers.

\myindex{x86!\Instructions!JA}
\myindex{x86!\Instructions!JB}
\myindex{x86!\Instructions!JG}
\myindex{x86!\Instructions!JL}
\myindex{Signed numbers}

See also the section about signed number representations~(\myref{sec:signednumbers}).
That is why if we see \JG/\JL in use instead of \INS{JA}/\JB or vice-versa, 
we can be almost sure that the variables are signed or unsigned, respectively.
Here is also the \main function, where there is nothing much new to us:

\lstinputlisting[caption=\main,style=customasmx86]{patterns/07_jcc/simple/main_MSVC.asm}

\clearpage
\mysubparagraph{\olly}
\myindex{\olly}

Let's try this example in \olly.
The input value of the function (2) is loaded into \EAX: 

\begin{figure}[H]
\centering
\myincludegraphics{patterns/08_switch/2_lot/olly1.png}
\caption{\olly: function's input value is loaded in \EAX}
\label{fig:switch_lot_olly1}
\end{figure}

\clearpage
The input value is checked, is it bigger than 4? 
If not, the \q{default} jump is not taken:
\begin{figure}[H]
\centering
\myincludegraphics{patterns/08_switch/2_lot/olly2.png}
\caption{\olly: 2 is no bigger than 4: no jump is taken}
\label{fig:switch_lot_olly2}
\end{figure}

\clearpage
Here we see a jumptable:

\begin{figure}[H]
\centering
\myincludegraphics{patterns/08_switch/2_lot/olly3.png}
\caption{\olly: calculating destination address using jumptable}
\label{fig:switch_lot_olly3}
\end{figure}

Here we've clicked \q{Follow in Dump} $\rightarrow$ \q{Address constant}, so now we see the \emph{jumptable} in the data window.
These are 5 32-bit values\footnote{They are underlined by \olly because
these are also FIXUPs: \myref{subsec:relocs}, we are going to come back to them later}.
\ECX is now 2, so the third element (can be indexed as 2\footnote{About indexing, see also: \ref{arrays_at_one}}) of the table is to be used.
It's also possible to click \q{Follow in Dump} $\rightarrow$ 
\q{Memory address} and \olly will show the element addressed by the \JMP instruction. 
That's \TT{0x010B103A}.

\clearpage
After the jump we are at \TT{0x010B103A}: the code printing \q{two} will now be executed:

\begin{figure}[H]
\centering
\myincludegraphics{patterns/08_switch/2_lot/olly4.png}
\caption{\olly: now we at the \emph{case:} label}
\label{fig:switch_lot_olly4}
\end{figure}


\clearpage
\myparagraph{x86 + MSVC + Hiew}
\myindex{Hiew}

We can try to patch the executable file in a way 
that the \TT{f\_unsigned()} function would always print \q{a==b}, 
no matter the input values.
Here is how it looks in Hiew:

\begin{figure}[H]
\centering
\myincludegraphics{patterns/07_jcc/simple/hiew_unsigned1.png}
\caption{Hiew: \TT{f\_unsigned()} function}
\label{fig:jcc_hiew_1}
\end{figure}

Essentially, we have to accomplish three tasks:
\begin{itemize}
\item force the first jump to always trigger;
\item force the second jump to never trigger;
\item force the third jump to always trigger.
\end{itemize}

Thus we can direct the code flow to always pass through the second \printf, and output \q{a==b}.

Three instructions (or bytes) has to be patched:

\begin{itemize}
\item The first jump becomes \JMP, but the \gls{jump offset} would remain the same.

\item 
The second jump might be triggered sometimes, but in any case it will jump to the next
instruction, because, we set the \gls{jump offset} to 0.

In these instructions the \gls{jump offset} is added to the address for the next instruction.
So if the offset is 0,
the jump will transfer the control to the next instruction.

\item 
The third jump we replace with \JMP just as we do with the first one, so it will always trigger.

\end{itemize}

\clearpage
Here is the modified code:

\begin{figure}[H]
\centering
\myincludegraphics{patterns/07_jcc/simple/hiew_unsigned2.png}
\caption{Hiew: let's modify the \TT{f\_unsigned()} function}
\label{fig:jcc_hiew_2}
\end{figure}

If we miss to change any of these jumps, then several \printf calls may execute, while we want to execute only one.

\myparagraph{\NonOptimizing GCC}

\myindex{puts() instead of printf()}
\NonOptimizing GCC 4.4.1 
produces almost the same code, but with \puts~(\myref{puts}) instead of \printf.

\myparagraph{\Optimizing GCC}

An observant reader may ask, why execute \CMP several times, 
if the flags has the same values after each execution?

Perhaps optimizing MSVC cannot do this, but optimizing GCC 4.8.1 can go deeper:

\lstinputlisting[caption=GCC 4.8.1 f\_signed(),style=customasmx86]{patterns/07_jcc/simple/GCC_O3_signed.asm}

% should be here instead of 'switch' section?
We also see \TT{JMP puts} here instead of \TT{CALL puts / RETN}.

This kind of trick will have explained later: \myref{JMP_instead_of_RET}.

This type of x86 code 
is somewhat rare.
MSVC 2012 as it seems, can't generate such code.
On the other hand, assembly language programmers are fully aware of the fact that \TT{Jcc} 
instructions can be stacked.

So if you see such stacking somewhere, it is highly probable that the code was hand-written.

The \TT{f\_unsigned()} function is not that 
\ae{}sthetically short:

\lstinputlisting[caption=GCC 4.8.1 f\_unsigned(),style=customasmx86]{patterns/07_jcc/simple/GCC_O3_unsigned_EN.asm}

Nevertheless, there are two \TT{CMP} instructions instead of three.

So optimization algorithms of GCC 4.8.1 are probably not perfect yet. 

\subsubsection{x64}
\label{subsec:popcnt}

Let's modify the example slightly to extend it to 64-bit:

\lstinputlisting[label=popcnt_x64_example,style=customc]{patterns/14_bitfields/4_popcnt/shifts64.c}

\myparagraph{\NonOptimizing GCC 4.8.2}

So far so easy.

\lstinputlisting[caption=\NonOptimizing GCC 4.8.2,style=customasmx86]{patterns/14_bitfields/4_popcnt/shifts64_GCC_O0_EN.s}

\myparagraph{\Optimizing GCC 4.8.2}

\lstinputlisting[caption=\Optimizing GCC 4.8.2,numbers=left,label=shifts64_GCC_O3,style=customasmx86]{patterns/14_bitfields/4_popcnt/shifts64_GCC_O3_EN.s}

This code is terser, but has a quirk.

In all examples that we see so far, we were incrementing the \q{rt} value after comparing a specific bit,
but the code here increments \q{rt} before (line 6), writing the new value into register \EDX .
Thus, if the last bit is 1, the \CMOVNE\footnote{Conditional MOVe if Not Equal} instruction
(which is a synonym for \CMOVNZ\footnote{Conditional MOVe if Not Zero}) \emph{commits} 
the new value of \q{rt}
by moving \EDX (\q{proposed rt value}) into \EAX (\q{current rt} to be returned at the end).

Hence, the incrementing is performed at each step of loop, i.e., 64 times, without any relation to the input value.

The advantage of this code is that it contain only one conditional jump (at the end of the loop) instead of 
two jumps (skipping the \q{rt} value increment and at the end of loop).
And that might work faster on the modern CPUs with branch predictors: \myref{branch_predictors}.

\label{FATRET}
\myindex{x86!\Instructions!FATRET}
The last instruction is \INS{REP RET} (opcode \TT{F3 C3}) 
which is also called \INS{FATRET} by MSVC.
This is somewhat optimized version of \RET, 
which is recommended by AMD to be placed at the end of function, if \RET goes right after conditional jump: 
\InSqBrackets{\AMDOptimization p.15}
\footnote{More information on it: \url{http://go.yurichev.com/17328}}.

\myparagraph{\Optimizing MSVC 2010}

\lstinputlisting[caption=\Optimizing MSVC 2010,style=customasmx86]{patterns/14_bitfields/4_popcnt/MSVC_2010_x64_Ox_EN.asm}

\myindex{x86!\Instructions!ROL}
Here the \ROL instruction is used instead of 
\SHL, which is in fact \q{rotate left} 
instead of \q{shift left},
but in this example it works just as \TT{SHL}.

You can read more about the rotate instruction here: \myref{ROL_ROR}.

\Reg{8} here is counting from 64 to 0.
It's just like an inverted $i$.

Here is a table of some registers during the execution:

\begin{center}
\begin{tabular}{ | l | l | }
\hline
\HeaderColor RDX & \HeaderColor R8 \\
\hline
0x0000000000000001 & 64 \\
\hline
0x0000000000000002 & 63 \\
\hline
0x0000000000000004 & 62 \\
\hline
0x0000000000000008 & 61 \\
\hline
... & ... \\
\hline
0x4000000000000000 & 2 \\
\hline
0x8000000000000000 & 1 \\
\hline
\end{tabular}
\end{center}

\myindex{x86!\Instructions!FATRET}
At the end we see the \INS{FATRET} instruction, which was explained here: \myref{FATRET}.

\myparagraph{\Optimizing MSVC 2012}

\lstinputlisting[caption=\Optimizing MSVC 2012,style=customasmx86]{patterns/14_bitfields/4_popcnt/MSVC_2012_x64_Ox_EN.asm}

\myindex{\CompilerAnomaly}
\label{MSVC2012_anomaly}
\Optimizing MSVC 2012 does almost the same job as 
optimizing MSVC 2010, but somehow, it generates two identical loop bodies and the loop count is now 32 instead of 64.

To be honest, it's not possible to say why. Some optimization trick? Maybe it's better for the loop body to be slightly 
longer?

Anyway, such code is relevant here to show that sometimes the compiler output may be really weird and 
illogical, but perfectly working.


\subsubsection{ARM + \OptimizingXcodeIV (\ARMMode)}

\lstinputlisting[caption=\OptimizingXcodeIV (\ARMMode),label=ARM_leaf_example4,style=customasmARM]{patterns/14_bitfields/4_popcnt/ARM_Xcode_O3_EN.lst}

\myindex{ARM!\Instructions!TST}
\TST is the same thing as \TEST in x86.

\myindex{ARM!Optional operators!LSL}
\myindex{ARM!Optional operators!LSR}
\myindex{ARM!Optional operators!ASR}
\myindex{ARM!Optional operators!ROR}
\myindex{ARM!Optional operators!RRX}
\myindex{ARM!\Instructions!MOV}
\myindex{ARM!\Instructions!TST}
\myindex{ARM!\Instructions!CMP}
\myindex{ARM!\Instructions!ADD}
\myindex{ARM!\Instructions!SUB}
\myindex{ARM!\Instructions!RSB}
As was noted before~(\myref{shifts_in_ARM_mode}),
there are no separate shifting instructions in ARM mode.
However, there are modifiers 
LSL (\emph{Logical Shift Left}), 
LSR (\emph{Logical Shift Right}), 
ASR (\emph{Arithmetic Shift Right}), 
ROR (\emph{Rotate Right}) and
RRX (\emph{Rotate Right with Extend}), which may be added to such instructions as \MOV, \TST,
\CMP, \ADD, \SUB, \RSB\footnote{\DataProcessingInstructionsFootNote}.

These modificators define how to shift the second operand and by how many bits.

\myindex{ARM!\Instructions!TST}
\myindex{ARM!Optional operators!LSL}
Thus the \TT{\q{TST R1, R2,LSL R3}} instruction works here as $R1 \land (R2 \ll R3)$.

\subsubsection{ARM + \OptimizingXcodeIV (\ThumbTwoMode)}

\myindex{ARM!\Instructions!LSL.W}
\myindex{ARM!\Instructions!LSL}
Almost the same, but here are two \INS{LSL.W}/\TST instructions are used instead of a single \TST, because in Thumb mode it is not
possible to define \LSL modifier directly in \TST.

\begin{lstlisting}[label=ARM_leaf_example5,style=customasmARM]
                MOV             R1, R0
                MOVS            R0, #0
                MOV.W           R9, #1
                MOVS            R3, #0
loc_2F7A
                LSL.W           R2, R9, R3
                TST             R2, R1
                ADD.W           R3, R3, #1
                IT NE
                ADDNE           R0, #1
                CMP             R3, #32
                BNE             loc_2F7A
                BX              LR
\end{lstlisting}

\subsubsection{ARM64 + \Optimizing GCC 4.9}

Let's take the 64-bit example which has been already used: \myref{popcnt_x64_example}.

\lstinputlisting[caption=\Optimizing GCC (Linaro) 4.8,style=customasmARM]{patterns/14_bitfields/4_popcnt/ARM64_GCC_O3_EN.s}

The result is very similar to what GCC generates for x64: \myref{shifts64_GCC_O3}.

\myindex{ARM!\Instructions!CSEL}
The \CSEL instruction is \q{Conditional SELect}. 
It just chooses one variable of two depending on the flags set by \TST and copies the value into \RegW{2}, which holds the \q{rt} variable.

\subsubsection{ARM64 + \NonOptimizing GCC 4.9}

And again, we'll work on the 64-bit example which was already used: \myref{popcnt_x64_example}.
The code is more verbose, as usual.

\lstinputlisting[caption=\NonOptimizing GCC (Linaro) 4.8,style=customasmARM]{patterns/14_bitfields/4_popcnt/ARM64_GCC_O0_EN.s}


\subsubsection{MIPS}
% FIXME better start at non-optimizing version?

The function uses a lot of S- registers which must be preserved, so that's why its 
values are saved in the function prologue and restored in the epilogue.

\lstinputlisting[caption=\Optimizing GCC 4.4.5 (IDA),style=customasmMIPS]{patterns/13_arrays/1_simple/MIPS_O3_IDA_EN.lst}

Something interesting: there are two loops and the first one doesn't need $i$, it needs only 
$i*2$ (increased by 2 at each iteration) and also the address in memory (increased by 4 at each iteration).

So here we see two variables, one (in \$V0) increasing by 2 each time, and another (in \$V1) --- by 4.

The second loop is where \printf is called and it reports the value of $i$ to the user, 
so there is a variable
which is increased by 1 each time (in \$S0) and also a memory address (in \$S1) increased by 4 each time.

That reminds us of loop optimizations: \myref{loop_iterators}.

Their goal is to get rid of multiplications.


