\clearpage
\mysubparagraph{\olly}
\myindex{\olly}

例を \olly にロードしてみましょう。
入力値は\TT{0x12345678}です。

$i=1$の場合に、$i$がどう \ECX にロードされるかを見ていきます。

\begin{figure}[H]
\centering
\myincludegraphics{patterns/14_bitfields/4_popcnt/olly1_1.png}
\caption{\olly: $i=1$、 $i$ が \ECX にロードされる}
\label{fig:shifts_olly1_1}
\end{figure}

\EDX は1です。 \SHL はたった今実行されます。

\clearpage
\SHL が実行されました。

\begin{figure}[H]
\centering
\myincludegraphics{patterns/14_bitfields/4_popcnt/olly1_2.png}
\caption{\olly: $i=1$、 \EDX=$1 \ll 1=2$}
\label{fig:shifts_olly1_2}
\end{figure}

\EDX は$1 \ll 1$(または2)を含みます。これはビットマスクです。

\clearpage
\AND は \ZF を1にセットし、入力値(\TT{0x12345678})を2でANDして結果が0になることを意味します。

\begin{figure}[H]
\centering
\myincludegraphics{patterns/14_bitfields/4_popcnt/olly1_3.png}
\caption{\olly: $i=1$、
入力値にそのようなビットはありますか?いいえ (\ZF=1)}
\label{fig:shifts_olly1_3}
\end{figure}

従って、入力値には対応するビットはありません。

カウンタを\glslink{increment}{インクリメント}するコード片は実行されません。
\JZ 命令はバイパスします。

\clearpage
もう少しトレースしてみましょう。$i$は4です。
\SHL がここで実行されます。

\begin{figure}[H]
\centering
\myincludegraphics{patterns/14_bitfields/4_popcnt/olly4_1.png}
\caption{\olly: $i=4$、 $i$ は \ECX にロードされる}
\label{fig:shifts_olly4_1}
\end{figure}

\clearpage
\EDX=$1 \ll 4$ (または \TT{0x10} もしくは 16): 

\begin{figure}[H]
\centering
\myincludegraphics{patterns/14_bitfields/4_popcnt/olly4_2.png}
\caption{\olly: $i=4$、 \EDX=$1 \ll 4=0x10$}
\label{fig:shifts_olly4_2}
\end{figure}

これは別のビットマスクです。

\clearpage
\AND が実行されます。

\begin{figure}[H]
\centering
\myincludegraphics{patterns/14_bitfields/4_popcnt/olly4_3.png}
\caption{\olly: $i=4$、 
入力値にそのようなビットはありますか?はい(\ZF=0)}
\label{fig:shifts_olly4_3}
\end{figure}

\ZF は0です。このビットは入力値にあるからです。
実際、\TT{0x12345678 \& 0x10 = 0x10} です。

ジャンプは実行されず、ビットカウンタは\glslink{increment}{インクリメント}します

関数は13をリターンします。
これは、\TT{0x12345678}に設定されたビットの総数です。
