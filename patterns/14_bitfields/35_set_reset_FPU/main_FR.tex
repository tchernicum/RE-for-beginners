\subsection{Mettre et effacer des bits spécifiques: exemple avec le \ac{FPU}}

\myindex{IEEE 754}

Voici comment les bits sont organisés dans le type \Tfloat au format IEEE 754:

\bigskip
% a hack used here! http://tex.stackexchange.com/questions/73524/bytefield-package
\begin{center}
\begin{bytefield}{32}
	\bitheader[endianness=big]{0,22,23,30,31} \\
	\bitbox{1}{S} &
	\bitbox{8}{%
		\RU{экспонента}%
		\EN{exponent}%
		\ES{exponente}%
		\PTBRph{}%
		\DEph{}\PLph{}%
		\IT{esponente}%
		\FR{exposant}%
		\JA{指数}
	} &
	\bitbox{23}{%
		\RU{мантисса}%
		\EN{mantissa or fraction}%
		\ES{mantisa o fracci\'on}%
		\PTBRph{}%
		\DEph{}
		\PLph{}%
		\IT{mantissa}%
		\FR{mantisse ou fraction}%
		\JA{対数または比}
	}
\end{bytefield}
\end{center}

\begin{center}
( S --- %
	\RU{знак}%
	\EN{sign}%
	\ES{signo}%
	\PTBRph{}%
	\DEph{}\PLph{}%
	\IT{segno}%
	\FR{signe}%
	\JA{記号}
)
\end{center}


Le signe du nombre est dans le \ac{MSB}.
Est-ce qu'il est possible de changer le signe d'un nombre en virgule flottante sans
aucune instruction FPU?

\lstinputlisting[style=customc]{patterns/14_bitfields/35_set_reset_FPU/abs.c}

Nous avons besoin de cette ruse en \CCpp pour copier vers/depuis des valeurs \Tfloat
sans conversion effective.
Donc il y a trois fonctions: my\_abs() supprime \ac{MSB}; \TT{set\_sign()} met \ac{MSB}
et negate() l'inverse.

\XOR peut être utilisé pour inverser un bit: \myref{XOR_property}.

\subsubsection{x86}

Le code est assez simple.

\lstinputlisting[caption=MSVC 2012 \Optimizing,style=customasmx86]{patterns/14_bitfields/35_set_reset_FPU/abs_MSVC2012_Ox.asm}

Une valeur en entrée de type \Tfloat est prise sur la pile, mais traitée comme une
valeur entière.

\AND et \OR supprime et mette le bit désiré.
\XOR l'inverse.

Enfin, la valeur modifiée est chargée dans \TT{ST0},car les nombres en virgule flottante
sont renvoyés dans ce registre.

Maintenant essayons l'optimisation de MSVC 2012 pour x64:

\lstinputlisting[caption=MSVC 2012 x64 \Optimizing,style=customasmx86]{patterns/14_bitfields/35_set_reset_FPU/abs_MSVC2012_x64_Ox.asm}

\myindex{x86!\Instructions!BTR}
\myindex{x86!\Instructions!BTS}
\myindex{x86!\Instructions!BTC}

La valeur en entrée est passée dans \TT{XMM0}, puis elle est copiée sur la pile locale
et nous voyons des nouvelles instructions: \BTR, \BTS, \BTC.

Ces instructions sont utilisées pour effacer (\BTR), mettre (\BTS) et inverser (ou
faire le complément: \BTC) de bits spécifiques.
Le bit d'index 31 est le \ac{MSB}, en comptant depuis 0.

Enfin, le résultat est copié dans \TT{XMM0}, car les valeurs en virgule flottante
sont renvoyées dans \TT{XMM0} en environnement Win64.

\subsubsection{MIPS}

GCC 4.4.5 pour MIPS fait essentiellement la même chose:

\lstinputlisting[caption=GCC 4.4.5 \Optimizing (IDA),style=customasmMIPS]{patterns/14_bitfields/35_set_reset_FPU/MIPS_O3_IDA_FR.lst}

\myindex{MIPS!\Instructions!LUI}

Une seule instruction \LUI est utilisée pour charger 0x80000000 dans un registre,
car \LUI efface les 16 bits bas et ils sont à zéro dans la constante, donc un \LUI
sans \ORI ultérieur est suffisant.

\subsubsection{ARM}

\myparagraph{\OptimizingKeilVI (\ARMMode)}

\lstinputlisting[caption=\OptimizingKeilVI (\ARMMode),style=customasmARM]{patterns/14_bitfields/35_set_reset_FPU/abs_Keil_ARM_O3_FR.s}

Jusqu'ici tout va bien.
\myindex{ARM!\Instructions!BIC}
\myindex{ARM!\Instructions!EOR}

ARM a l'instruction \BIC, qui efface explicitement un (des) bit(s) spécifique(s).
\EOR est le nom de l'instruction ARM pour \XOR (\q{Exclusive OR / OU exclusif}).

\myparagraph{\OptimizingKeilVI (\ThumbMode)}

\lstinputlisting[caption=\OptimizingKeilVI (\ThumbMode),style=customasmx86]{patterns/14_bitfields/35_set_reset_FPU/abs_Keil_thumb_O3.s}

En ARM, le mode Thumb offre des instructions 16-bit et peu de données peuvent y être
encodées, donc ici une paire d'instructions \INS{MOVS}/\INS{LSLS} est utilisée pour
former la constante 0x80000000.
Ça fonctionne comme ceci: $1<<31 = 0x80000000$.

\myindex{ARM!\Instructions!LSLS}
\myindex{ARM!\Instructions!LSRS}

Le code de \TT{my\_abs} est bizarre et fonctionne pratiquement comme cette expression:
$(i<<1)>>1$.
Cette déclaration semble vide de sens.
Mais néanmoins, lorsque $input<<1$ est exécuté, le \ac{MSB} (bit de signe) est simplement
supprimé.
Puis lorsque la déclaration suivante $result>>1$ est exécutée, tous les bits sont
à nouveau à leur place, mais le \ac{MSB} vaut zéro, car tous les \q{nouveaux} bits
apparaissant lors d'une opération de décalage sont toujours zéro.
C'est ainsi que la paire d'instructions \LSLS/\LSRS efface le \ac{MSB}.

\myparagraph{\Optimizing GCC 4.6.3 (Raspberry Pi, \ARMMode)}

\lstinputlisting[caption=\Optimizing GCC 4.6.3 for Raspberry Pi (\ARMMode),style=customasmARM]{patterns/14_bitfields/35_set_reset_FPU/raspberry_GCC_O3_ARM_mode_FR.lst}

Lançons Linux pour Raspberry Pi dans QEMU et ça émule un FPU ARM, dons les S-registres
sont utilisés pour les nombres en virgule flottante au lieu des R-registres.

\myindex{ARM!\Instructions!FMRS}

L'instruction \FMRS copie des données des \ac{GPR} vers le FPU et retour.

\TT{my\_abs()} et \TT{set\_sign()} ressemblent a ce que l'on attend, mais negate()?
Pourquoi est-ce qu'il y a \ADD au lieu de \XOR?

\myindex{ARM!\Instructions!XOR}
\myindex{ARM!\Instructions!ADD}
C'est dur à croire, mais l'instruction
\INS{ADD register, 0x80000000} fonctionne tout comme \\
\INS{XOR register, 0x80000000}.
Tout d'abord, quel est notre but?
Le but est de changer le \ac{MSB}, donc oublions l'opération \XOR.
Des mathématiques niveau scolaire, nous nous rappelons qu'ajouter une valeur comme
1000 à une autre valeur n'affecte jamais les 3 derniers chiffres.
Par exemple: $1234567 + 10000 = 1244567$ (les 4 derniers chiffres ne sont jamais
affectés).

Mais ici nous opérons en base décimale et\\
0x80000000 est 0b100000000000000000000000000000000, i.e., seulement le bit le plus
haut est mis.

Ajouter 0x80000000 à n'importe quelle valeur n'affecte jamais les 31 bits les plus
bas, mais affecte seulement le \ac{MSB}.
Ajouter 1 à 0 donne 1.

Ajouter 1 à 1 donne 0b10 au format binaire, mais le bit d'indice 32 (en comptant
à partir de zéro) est abandonné, car notre registre est large de 32 bit, donc le
résultat est 0.
C'est pourquoi \XOR peut être remplacé par \ADD ici.

Il est difficile de dire pourquoi GCC a décidé de faire ça, mais ça fonctionne correctement.
