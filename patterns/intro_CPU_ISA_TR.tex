\subsection{CPU'ya Giriş}

\ac{CPU} makine kodundan oluşan programları çalıştıran donanımdır. 

\textbf{Ufak bir Sözlük:}

\begin{description}
\item[Instruction]: İlkel bir \ac{CPU} komutu.
Basit bir örnek olarak: register (yazmaçlar) arasında değer taşıma, bellek ile çalışma, ilkel aritmetik operasyonlar verilebilir.
Genellikle, her \ac{CPU}'nun kendine has instruction listesi vardır (\ac{ISA}).

\item[Machine code]: \ac{CPU}'nun direkt olarak işleyebildiği kod.
Her instruction genellikle bir kaç byte ile gösterilir.
\item[Assembly language]: Programcının hayatını kolaylaştırmak için mnemonic kodlar ve bazı eklentiler (çeşitli microlar gibi) içeren bir dildir.
\item[CPU register]: Her \ac{CPU} sabit sayıda genel amaçlı register'dan (yazmaç) oluşur (\ac{GPR}).
x86'da $\approx 8$, x86-64'de $\approx 16$ ve ARM'da $\approx 16$ adet vardır.
Register'ları anlamanın en kolay yolu onları belirli bir tipi olmayan geçici değişkenler olarak düşünmektir.
Yüksek seviyeli bir \ac{PL} kullandığınızı ve yalnızca 32-bit (ya da 64-bit) değişkenler kullanabildiğinizi hayal edin.
Buna rağmen bir çok şeyi yapabilirdiniz!
\end{description}

% TODO1 add about linker: "компоновщик" и "редактор связей" в русскоязычной лит-ре

% A note on the experiments in this area (like the LISP machines http://en.wikipedia.org/wiki/Lisp_machine
% might be useful
Makine kodu ile bir \ac{PL} arasında neden fark olduğunu merak edebilirsiniz. Cevap insan beyninin ve \ac{CPU}'ların benzer çalışmadığında yatıyor.---%
İnsanlar için \CCpp, Java, or Python, gibi yüksek-seviyeli \ac{PL}'leri algılamaları daha kolay olmasına karşın \ac{CPU}'ler daha alçak-seviyeli soyutlamaları algılayabiliyor.
Muhtemelen \ac{CPU}'nun algılayabileceği yüksek-seviyeli bir dil icat etmek mümkün ama o zaman kullanılacak \ac{CPU}'ler günümüzdekilerden çok daha karmaşık olurlardı. 
Benzer olarak insanların düşük seviyeli bir dil olan assembly'de program yazması can sıkıcı hatalar ve buglar olmadan pek mümkün değil.
Yüksek seviyeli bir \ac{PL}'den assembly'e çevirme işlemini yapan programlara  \emph{compiler} denir.
\footnote{Old-school Russian literatöründe bu terim \q{translator} olarak kullanılır.}.

\myindex{ARM!\ARMMode}%
\myindex{ARM!\ThumbMode}%
\myindex{ARM!\ThumbTwoMode}%

\subsubsection{Farklı \ac{ISA}'ler hakkında bir kaç söz}
x86 \ac{ISA} daima değişken uzunluktaki instruction'lara sahipti bu yüzden 64-bit mimari geldiğinde, x64 uzantısı \ac{ISA}'yı çok fazla etkilemedi. Aslında, x86 \ac{ISA} hala daha ilk 16-bit 8086 CPU'ya ait instructionları içermekte ve bugünün işlemcilerinde bir çoğu kullanılabilir.
ARM, sabit uzunluklu incstruction'lara sahip bir \ac{RISC} \ac{CPU}'dur ki bunun geçmişte faydaları da olmuştur.
İlk başlarda ARM instructionları sabit olarak 4byte uzunluktaydı.%
\footnote{
Sabit uzunluklu instruction'ların faydası bir sonraki (veya bir önceki) instruction'ın yerinin tespiti çok daha kolay olmaktadır. Bu özellik switch() operator~(\myref{sec:SwitchARMLot})  bölümünde tartışılacak.
}.
Şu an bu \q{ARM mode} olarak geçiyor.
Sonraları bunun başta sandıkları gibi tutumlu bir yöntem olmadığını anladılar.
Aslında, en çok kullanılan \ac{CPU} instruction'ları \footnote{e.g. MOV/PUSH/CALL/Jcc} gerçek dünya uygulamalarında daha az bilgi ile kodlanabilmektedir.
Bu yüzden her instruction'un 2 byte yer kapladığı Thumb adında yeni bir \ac{ISA} eklediler
Şu an bu \q{Thumb mode} olarak geçiyor.
Malesef \emph{tüm} ARM instruction'ları 2 byte ile temsil edilemiyordu bu yüzden Thumb instruction set biraz sınırlı kaldı.
Tabi bir programda hem ARM modunun hem de Thumb modunun bir arada kullanılabileceğini belirtmek gerekir.
ARM'ın yaratıcıları Thumb'ın genişletilebileceğini düşündüler ve ARMv7'de kullanılan Thumb-2'yi oluşturdular. 
Thumb-2'de de 2-byte uzunluğundaki instruction'lar kullanılıyor ancak bazı instructionlar 4-byte uzunluğunda olabilmektedir.
Thumb-2'nin ARM ve Thumb modlarının karışımı olduğu şeklinde genel bir yanlış kanı mevcut ancak bu yanlış bir algıdır.
Yanlış kanının aksine Thumb-2 ARM moduyla yarışmak için tüm işlemcileri destekleyecek şekilde genişletildi ki Thumb-2 amacına ulaşarak \idevices uygulamaların çoğunluğu Thumb-2 instruction set ile derlendi.
are compiled for the Thumb-2 instruction set. (Tabi hedefe ulaşmada XCode'un bunu varsayılan olarak yapmasının payı büyük)
Sonra 64-bit geldi. Bu \ac{ISA} 4-byte instructionlara sahipti ve Thumb modun bunun için herhangi bir değişiklik yapmasına gerek yoktu.
Ne yazık ki 64-bit \ac{ISA}'yı etkiledi ve bu şu an kullanılan üç instruction set'inin var olmasına neden oldu:instruction sets: ARM mod, Thumb mod (Thumb-2 dahil) ve ARM64.
Bu \ac{ISA}'ler kısmen kesişirler ama bir tanesinin farklı versiyonları demektense farklı \ac{ISA}'ler oldukları söylenebilir.
Buna rağmen biz bu kitapta üç ARM \ac{ISA} için de kod parçaları kullanacağız.

\myindex{PowerPC}%
\myindex{MIPS}%
\myindex{Alpha AXP}%
Ayrıca piyasada sabit uzunluklu 32-bit instruction'lara sahip başka MIPS, PowerPC ve Alpha AXP gibi bir çok \ac{RISC} \ac{ISA}'ler mevcut. 