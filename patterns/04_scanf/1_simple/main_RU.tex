\subsection{Простой пример}

\lstinputlisting[style=customc]{patterns/04_scanf/1_simple/ex1.c}

Использовать \scanf в наши времена для того, чтобы спросить у пользователя что-то --- не самая хорошая идея.
Но так мы проиллюстрируем передачу указателя на переменную типа \Tint.

\subsubsection{Об указателях}
\myindex{\CLanguageElements!\Pointers}

Это одна из фундаментальных вещей в программировании.
Часто большой массив, структуру или объект передавать в другую функцию путем копирования данных невыгодно, а передать адрес массива, структуры или объекта куда проще.
Например, если вы собираетесь вывести в консоль текстовую строку, достаточно только передать её адрес в ядро \ac{OS}.

К тому же, если вызываемая функция (\gls{callee}) должна изменить что-то в этом большом массиве или структуре, то возвращать её полностью так же абсурдно.
Так что самое простое, что можно сделать, это передать в функцию-\gls{callee} адрес массива или структуры, и пусть \gls{callee} что-то там изменит.

Указатель в \CCpp --- это просто адрес какого-либо места в памяти.

\myindex{x86-64}
В x86 адрес представляется в виде 32-битного числа (т.е. занимает 4 байта), а в x86-64 как 64-битное число (занимает 8 байт).
Кстати, отсюда негодование некоторых людей, связанное с переходом на x86-64 --- на этой архитектуре все указатели занимают в 2 раза больше места, в том числе и в ``дорогой'' кэш-памяти.

% TODO ... а делать разные версии memcpy для разных типов - абсурд
\myindex{\CStandardLibrary!memcpy()}
При некотором упорстве можно работать только с безтиповыми указателями (\TT{void*}), например, стандартная функция Си \TT{memcpy()},
копирующая блок из одного места памяти в другое принимает на вход 2 указателя типа \TT{void*}, потому что нельзя заранее предугадать, какого типа блок вы собираетесь копировать.
Для копирования тип данных не важен, важен только размер блока.

Также указатели широко используются, когда функции нужно вернуть более одного значения
(мы ещё вернемся к этому в будущем
~(\myref{label_pointers})
).

Функция \emph{scanf()}---это как раз такой случай.

Помимо того, что этой функции нужно показать, сколько значений было прочитано успешно, ей ещё и нужно вернуть сами значения.

Тип указателя в \CCpp нужен только для проверки типов на стадии компиляции.

Внутри, в скомпилированном коде, никакой информации о типах указателей нет вообще.
% TODO это сильно затрудняет декомпиляцию

\EN{\subsubsection{x86}

\myparagraph{MSVC}

Here is what we get after compiling with MSVC 2010:

\lstinputlisting[style=customasmx86]{patterns/04_scanf/1_simple/ex1_MSVC_EN.asm}

\TT{x} is a local variable.

According to the \CCpp standard it must be visible only in this function and not from any other external scope. 
Traditionally, local variables are stored on the stack. 
There are probably other ways to allocate them, but in x86 that is the way it is.

\myindex{x86!\Instructions!PUSH}
The goal of the instruction following the function prologue, \TT{PUSH ECX}, is not to save the \ECX state 
(notice the absence of corresponding \TT{POP ECX} at the function's end).

In fact it allocates 4 bytes on the stack for storing the \TT{x} variable.

\label{stack_frame}
\myindex{\Stack!Stack frame}
\myindex{x86!\Registers!EBP}
\TT{x} is to be accessed with the assistance of the \TT{\_x\$} macro (it equals to -4) and the \EBP register pointing to the current frame.

Over the span of the function's execution, \EBP is pointing to the current \gls{stack frame}
making it possible to access local variables and function arguments via \TT{EBP+offset}.

\myindex{x86!\Registers!ESP}
It is also possible to use \ESP for the same purpose, although that is not very convenient since it changes frequently.
The value of the \EBP could be perceived as a \emph{frozen state} of the value in \ESP at the start of the function's execution.

% FIXME1 это уже было в 02_stack?
Here is a typical \gls{stack frame} layout in 32-bit environment:

\begin{center}
\begin{tabular}{ | l | l | }
\hline
\dots & \dots \\
\hline
EBP-8 & local variable \#2, \MarkedInIDAAs{} \TT{var\_8} \\
\hline
EBP-4 & local variable \#1, \MarkedInIDAAs{} \TT{var\_4} \\
\hline
EBP & saved value of \EBP \\
\hline
EBP+4 & return address \\
\hline
EBP+8 & \argument \#1, \MarkedInIDAAs{} \TT{arg\_0} \\
\hline
EBP+0xC & \argument \#2, \MarkedInIDAAs{} \TT{arg\_4} \\
\hline
EBP+0x10 & \argument \#3, \MarkedInIDAAs{} \TT{arg\_8} \\
\hline
\dots & \dots \\
\hline
\end{tabular}
\end{center}

The \scanf function in our example has two arguments.

The first one is a pointer to the string containing \TT{\%d} and the second is the address of the \TT{x} variable.

\myindex{x86!\Instructions!LEA}
First, the \TT{x} variable's address is loaded into the \EAX register by the \\
\TT{lea eax, DWORD PTR \_x\$[ebp]} instruction.

\LEA stands for \emph{load effective address}, and is often used for forming an address ~(\myref{sec:LEA}).

We could say that in this case \LEA simply stores the sum of the \EBP register value and the \TT{\_x\$} macro in the \EAX register.

This is the same as \INS{lea eax, [ebp-4]}.

So, 4 is being subtracted from the \EBP register value and the result is loaded in the \EAX register.
Next the \EAX register value is pushed into the stack and \scanf is being called.

\printf is being called after that with its first argument --- a pointer to the string:
\TT{You entered \%d...\textbackslash{}n}.

The second argument is prepared with: \TT{mov ecx, [ebp-4]}.
The instruction stores the \TT{x} variable value and not its address, in the \ECX register.

Next the value in the \ECX is stored on the stack and the last \printf is being called.

\EN{\input{patterns/04_scanf/1_simple/olly_EN}}
\RU{\input{patterns/04_scanf/1_simple/olly_RU}}
\IT{\input{patterns/04_scanf/1_simple/olly_IT}}
\DE{\input{patterns/04_scanf/1_simple/olly_DE}}
\FR{\input{patterns/04_scanf/1_simple/olly_FR}}
\JA{\input{patterns/04_scanf/1_simple/olly_JA}}


\myparagraph{GCC}

Let's try to compile this code in GCC 4.4.1 under Linux:

\lstinputlisting[style=customasmx86]{patterns/04_scanf/1_simple/ex1_GCC.asm}

\myindex{puts() instead of printf()}
GCC replaced the \printf call with call to \puts. The reason for this was explained in ~(\myref{puts}).

% TODO: rewrite
%\RU{Почему \scanf переименовали в \TT{\_\_\_isoc99\_scanf}, я честно говоря, пока не знаю.}
%\EN{Why \scanf is renamed to \TT{\_\_\_isoc99\_scanf}, I do not know yet.}
% 
% Apparently it has to do with the ISO c99 standard compliance. By default GCC allows specifying a standard to adhere to.
% For example if you compile with -std=c89 the outputted assmebly file will contain scanf and not __isoc99__scanf. I guess current GCC version adhares to c99 by default.
% According to my understanding the two implementations differ in the set of suported modifyers (See printf man page)

As in the MSVC example---the arguments are placed on the stack using the \MOV instruction.

\myparagraph{By the way}

This simple example is a demonstration of the fact that compiler translates
list of expressions in \CCpp-block into sequential list of instructions.
There are nothing between expressions in \CCpp, and so in resulting machine code, 
there are nothing between, control flow slips from one expression to the next one.

}
\RU{\subsubsection{x86}

\myparagraph{MSVC}

Что получаем на ассемблере, компилируя в MSVC 2010:

\lstinputlisting[style=customasmx86]{patterns/04_scanf/1_simple/ex1_MSVC_RU.asm}

Переменная \TT{x} является локальной.

По стандарту \CCpp она доступна только из этой же функции и нигде более. 
Так получилось, что локальные переменные располагаются в стеке. 
Может быть, можно было бы использовать и другие варианты, но в x86 это традиционно так.

\myindex{x86!\Instructions!PUSH}
Следующая после пролога инструкция \TT{PUSH ECX} не ставит своей целью сохранить 
значение регистра \ECX. 
(Заметьте отсутствие соответствующей инструкции \TT{POP ECX} в конце функции).

Она на самом деле выделяет в стеке 4 байта для хранения \TT{x} в будущем.

\label{stack_frame}
\myindex{\Stack!Стековый фрейм}
\myindex{x86!\Registers!EBP}
Доступ к \TT{x} будет осуществляться при помощи объявленного макроса \TT{\_x\$} (он равен -4) и регистра \EBP указывающего на текущий фрейм.

Во всё время исполнения функции \EBP указывает на текущий \glslink{stack frame}{фрейм} и через \TT{EBP+смещение}
можно получить доступ как к локальным переменным функции, так и аргументам функции.

\myindex{x86!\Registers!ESP}
Можно было бы использовать \ESP, но он во время исполнения функции часто меняется, а это не удобно. 
Так что можно сказать, что \EBP это \emph{замороженное состояние} \ESP на момент начала исполнения функции.

% FIXME1 это уже было в 02_stack?
Разметка типичного стекового \glslink{stack frame}{фрейма} в 32-битной среде:

\begin{center}
\begin{tabular}{ | l | l | }
\hline
\dots & \dots \\
\hline
EBP-8 & локальная переменная \#2, \MarkedInIDAAs{} \TT{var\_8} \\
\hline
EBP-4 & локальная переменная \#1, \MarkedInIDAAs{} \TT{var\_4} \\
\hline
EBP & сохраненное значение \EBP \\
\hline
EBP+4 & адрес возврата \\
\hline
EBP+8 & \argument \#1, \MarkedInIDAAs{} \TT{arg\_0} \\
\hline
EBP+0xC & \argument \#2, \MarkedInIDAAs{} \TT{arg\_4} \\
\hline
EBP+0x10 & \argument \#3, \MarkedInIDAAs{} \TT{arg\_8} \\
\hline
\dots & \dots \\
\hline
\end{tabular}
\end{center}

У функции \scanf в нашем примере два аргумента.

Первый~--- указатель на строку, содержащую \TT{\%d} и второй~--- адрес переменной \TT{x}.

\myindex{x86!\Instructions!LEA}
Вначале адрес \TT{x} помещается в регистр \EAX при помощи инструкции \TT{lea eax, DWORD PTR \_x\$[ebp]}.

Инструкция \LEA означает \emph{load effective address}, и часто используется для формирования адреса чего-либо ~(\myref{sec:LEA}).

Можно сказать, что в данном случае \LEA просто помещает в \EAX результат суммы значения в регистре \EBP и макроса \TT{\_x\$}.

Это тоже что и \INS{lea eax, [ebp-4]}.

Итак, от значения \EBP отнимается 4 и помещается в \EAX.
Далее значение \EAX заталкивается в стек и вызывается \scanf.

После этого вызывается \printf. Первый аргумент вызова строка:
\TT{You entered \%d...\textbackslash{}n}.

Второй аргумент: \INS{mov ecx, [ebp-4]}.
Эта инструкция помещает в \ECX не адрес переменной \TT{x}, а её значение.

Далее значение \ECX заталкивается в стек и вызывается \printf.

\EN{\input{patterns/04_scanf/1_simple/olly_EN}}
\RU{\input{patterns/04_scanf/1_simple/olly_RU}}
\IT{\input{patterns/04_scanf/1_simple/olly_IT}}
\DE{\input{patterns/04_scanf/1_simple/olly_DE}}
\FR{\input{patterns/04_scanf/1_simple/olly_FR}}
\JA{\input{patterns/04_scanf/1_simple/olly_JA}}


\myparagraph{GCC}

Попробуем тоже самое скомпилировать в Linux при помощи GCC 4.4.1:

\lstinputlisting[style=customasmx86]{patterns/04_scanf/1_simple/ex1_GCC.asm}

\myindex{puts() вместо printf()}
GCC заменил первый вызов \printf на \puts. Почему это было сделано, 
уже было описано ранее~(\myref{puts}).

% TODO: rewrite
%\RU{Почему \scanf переименовали в \TT{\_\_\_isoc99\_scanf}, я честно говоря, пока не знаю.}
%\EN{Why \scanf is renamed to \TT{\_\_\_isoc99\_scanf}, I do not know yet.}
% 
% Apparently it has to do with the ISO c99 standard compliance. By default GCC allows specifying a standard to adhere to.
% For example if you compile with -std=c89 the outputted assmebly file will contain scanf and not __isoc99__scanf. I guess current GCC version adhares to c99 by default.
% According to my understanding the two implementations differ in the set of suported modifyers (See printf man page)


Далее всё как и прежде~--- параметры заталкиваются через стек при помощи \MOV.

\myparagraph{Кстати}

Этот простой пример иллюстрирует то обстоятельство, что компилятор преобразует
список выражений в \CCpp-блоке просто в последовательный набор инструкций.
Между выражениями в \CCpp ничего нет, и в итоговом машинном коде между ними тоже ничего нет, 
управление переходит от одной инструкции к следующей за ней.

}
\PTBR{\subsubsection{x86: 3 argumentos}

\myparagraph{MSVC}

Quando compilamos esse código com o MSVC 2010 Express temos:

\begin{lstlisting}[style=customasmx86]
$SG3830	DB	'a=%d; b=%d; c=%d', 00H

...

	push	3
	push	2
	push	1
	push	OFFSET $SG3830
	call	_printf
	add	esp, 16					; 00000010H
\end{lstlisting}

Quase a mesma coisa, mas agora nós podemos ver que os argumentos da função \printf são empurrados na pilha na ordem reversa. O primeiro argumento é empurrado por último.

A propósito, variáveis do tipo \Tint em ambientes 32-bits tem 32-bits de largura, isso é 4 bytes.

Então, nós temos quatro argumentos aqui, $4*4=16$ --- eles ocupam exatamente 16 bytes na pilha: um ponteiro de 32-bits para uma string e três números do tipo \Tint.

\myindex{x86!\Instructions!ADD}
\myindex{x86!\Registers!ESP}
\myindex{cdecl}
Quando o ponteiro da pilha (registrador \ESP) volta para seu valor anterior pela instrução \INS{ADD ESP, X} depois de uma chamada de função,
geralmente o número de argumentos pode ser obtido simplismente por se dividir X, o argumento da função \ADD, por 4.

Lógico que isso é por causa da convenção de chamada do \emph{cdecl} e somente para ambientes 32-bits.

% TODO: Olly, GCC, GDB

}
\IT{\subsubsection{x86}

\myparagraph{MSVC}

Questo e' cio' che si ottiene dopo la compilazione con MSVC 2010:

\lstinputlisting[style=customasmx86]{patterns/04_scanf/1_simple/ex1_MSVC_EN.asm}

\TT{x} e' una variabile locale.

In base allo standard \CCpp deve essere visibile soltanto in questa funzione e non in altri ambiti (esterni alla funzione).
Tradizionalmente, le variabili locali sono memorizzate sullo stack. 
Ci sono probabilmente altri modi per allocarle, ma in x86 e' cosi'.

\myindex{x86!\Instructions!PUSH}
Lo scopo dell'istruzione che segue il prologo della funzione, \TT{PUSH ECX}, non e' quello di salvare lo stato di \ECX  
(si noti infatti l'assenza della corrispondente istruzione \TT{POP ECX} alla fine della funzione).

Infatti alloca 4 byte sullo stack per memorizzare la variabile \TT{x}.

\label{stack_frame}
\myindex{\Stack!Stack frame}
\myindex{x86!\Registers!EBP}
\TT{x} sara' acceduta con l'aiuto della macro \TT{\_x\$} (che e' uguale a -4) ed il registro \EBP che punta al frame corrente.

Durante l'esecuzione delle funziona, \EBP punta allo \gls{stack frame} corrente 
rendendo possibile accedere alle variabili locali ed agli argomenti della funzione attraverso \TT{EBP+offset}.

\myindex{x86!\Registers!ESP}
E' anche possibile usare \ESP per lo stesso scopo, tuttavia non e' molto conveniente poiche' cambia di frequente.
Il valore di \EBP puo' essere pensato come uno \emph{stato congelato} del valore in \ESP all'inizio dell'esecuzione della funzione.

% FIXME1 это уже было в 02_stack?
Questo e' un tipico layout di uno \gls{stack frame} in un ambiente a 32-bit:

\begin{center}
\begin{tabular}{ | l | l | }
\hline
\dots & \dots \\
\hline
EBP-8 & local variable \#2, \MarkedInIDAAs{} \TT{var\_8} \\
\hline
EBP-4 & local variable \#1, \MarkedInIDAAs{} \TT{var\_4} \\
\hline
EBP & saved value of \EBP \\
\hline
EBP+4 & return address \\
\hline
EBP+8 & \argument \#1, \MarkedInIDAAs{} \TT{arg\_0} \\
\hline
EBP+0xC & \argument \#2, \MarkedInIDAAs{} \TT{arg\_4} \\
\hline
EBP+0x10 & \argument \#3, \MarkedInIDAAs{} \TT{arg\_8} \\
\hline
\dots & \dots \\
\hline
\end{tabular}
\end{center}

La funzione \scanf nel nostro esempio ha due argomenti.
Il primo e' un puntatore alla stringa contenente \TT{\%d} e il secondo e' l'indirizzo della variabile \TT{x}.

\myindex{x86!\Instructions!LEA}
Per prima cosa l'indirizzo della variabile \TT{x} e' caricato nel registro \EAX dall'istruzione \TT{lea eax, DWORD PTR \_x\$[ebp]}.

\LEA sta per \emph{load effective address}, ed e' spesso usata per formare un indirizzo ~(\myref{sec:LEA}).

Potremmo dire che in questo caso \LEA memorizza semplicemente la somma del valore nel registro \EBP e della macro \TT{\_x\$} nel registro \EAX.

E' l'equivalente di \INS{lea eax, [ebp-4]}.

Quindi, 4 viene sottratto dal valore del registro \EBP ed il risultato e' memorizzato nel registro \EAX.
Successivamente il registro \EAX e' messo sullo stack (push) e \scanf viene chiamata.

\printf viene chiamata subito dopo con il suo primo argomento --- un puntatore alla stringa:
\TT{You entered \%d...\textbackslash{}n}.

Il secondo argomento e' preparato con: \TT{mov ecx, [ebp-4]}.
L'istruzione memorizza il valore della variabile \TT{x},  non il suo indirizzo, nel registro \ECX.

Successivamente il valore in \ECX e' memorizzato sullo stack e l'ultima \printf viene chiamata.

\EN{\input{patterns/04_scanf/1_simple/olly_EN}}
\RU{\input{patterns/04_scanf/1_simple/olly_RU}}
\IT{\input{patterns/04_scanf/1_simple/olly_IT}}
\DE{\input{patterns/04_scanf/1_simple/olly_DE}}
\FR{\input{patterns/04_scanf/1_simple/olly_FR}}
\JA{\input{patterns/04_scanf/1_simple/olly_JA}}


\myparagraph{GCC}

Proviamo a compilare questo codice con GCC 4.4.1 su Linux:

\lstinputlisting[style=customasmx86]{patterns/04_scanf/1_simple/ex1_GCC.asm}

\myindex{puts() instead of printf()}
GCC ha sostituito la chiamata a \printf con \puts. La ragione per cui cio' avviene e' stata spiegata in ~(\myref{puts}).

% TODO: rewrite
%\RU{Почему \scanf переименовали в \TT{\_\_\_isoc99\_scanf}, я честно говоря, пока не знаю.}
%\EN{Why \scanf is renamed to \TT{\_\_\_isoc99\_scanf}, I do not know yet.}
% 
% Apparently it has to do with the ISO c99 standard compliance. By default GCC allows specifying a standard to adhere to.
% For example if you compile with -std=c89 the outputted assmebly file will contain scanf and not __isoc99__scanf. I guess current GCC version adhares to c99 by default.
% According to my understanding the two implementations differ in the set of suported modifyers (See printf man page)

Come nell'esempio compilato con MSVC ---gli argomenti sono messi sullo stack utilizzando l'istruzione \MOV.


}
\DE{\subsubsection{x86}

\myparagraph{MSVC}
Den folgenden Code erhalten wie nach dem Kompilieren mit MSVC 2010:

\lstinputlisting[style=customasmx86]{patterns/04_scanf/1_simple/ex1_MSVC_DE.asm}

\TT{x} ist eine lokale Variable.

Gemäß dem \CCpp-Standard darf diese nur innerhalb dieser Funktion sichtbar sein und nicht aus einem anderen, äußeren Scope.
Traditionell werden lokale Variablen auf dem Stack gespeichert.
Es gibt möglicherweise andere Wege sie anzulegen, aber in x86 geschieht es auf diese Weise.


\myindex{x86!\Instructions!PUSH}
Das Ziel des Befehls direkt nach dem Funktionsprolog, \TT{PUSH ECX}), ist es nicht, den Status von \ECX zu sichern
(man beachte, dass Fehlen eines entsprechenden \TT{POP ECX} im Funktionsepilog).
Tatsächlich reserviert der Befehl 4 Byte auf dem Stack, um die Variable $x$ speichern zu können.

\label{stack_frame}
\myindex{\Stack!Stack frame}
\myindex{x86!\Registers!EBP}
Auf \TT{x} wird mithilfe des \TT{\_x\$} Makros (es entspricht -4) und des \EBP Registers, das auf den aktuellen Stack Frame zeigt, zugegriffen. 
Während der Dauer der Funktionsausführung zeigt \EBP auf den aktuellen \glslink{stack frame}{Stack Frame}, wodurch mittels \TT{EBP+offset} auf lokalen Variablen und Funktionsargumente zugegriffen werden kann.

\TT{x} is to be accessed with the assistance of the \TT{\_x\$} macro (it equals to -4) and the \EBP register pointing to the current frame.

\myindex{x86!\Registers!ESP}
Es ist auch möglich, das \ESP Register zu diesem Zweck zu verwenden, aber dies ist ungebräuchlich, da es sich häufig verändert.
Der Wert von \EBP kann als eingefrorener Wert des Wertes von \ESP zu Beginn der Funktionsausführung verstanden werden.

It is also possible to use \ESP for the same purpose, although that is not very convenient since it changes frequently.
The value of the \EBP could be perceived as a \emph{frozen state} of the value in \ESP at the start of the function's execution.

% FIXME1 это уже было в 02_stack?
Hier ist ein typisches Layour eines Stack Frames in einer 32-Bit-Umgebung:

\begin{center}
\begin{tabular}{ | l | l | }
\hline
\dots & \dots \\
\hline
EBP-8 & local variable \#2, \MarkedInIDAAs{} \TT{var\_8} \\
\hline
EBP-4 & local variable \#1, \MarkedInIDAAs{} \TT{var\_4} \\
\hline
EBP & saved value of \EBP \\
\hline
EBP+4 & return address \\
\hline
EBP+8 & \argument \#1, \MarkedInIDAAs{} \TT{arg\_0} \\
\hline
EBP+0xC & \argument \#2, \MarkedInIDAAs{} \TT{arg\_4} \\
\hline
EBP+0x10 & \argument \#3, \MarkedInIDAAs{} \TT{arg\_8} \\
\hline
\dots & \dots \\
\hline
\end{tabular}
\end{center}
Die Funktion \scanf in unserem Beispiel hat zwei Argumente.

Das erste ist ein Pointer auf den String \TT{\%d} und das zweite ist die Adresse der Variablen \TT{x}.

\myindex{x86!\Instructions!LEA}
Zunächst wird die Adresse der Variablen $x$ durch den Befehl \\
\TT{lea eax, DWORD PTR \_x\$[ebp]} in das \EAX Register geladen.

\LEA steht für \emph{load effective address} und wird häufig benutzt, um eine Adresse zu erstellen ~(\myref{sec:LEA}).
In diesem Fall speichert \LEA einfach die Summe des \EBP Registers und des \TT{\_\$} Makros im Register \EAX.
Dies entspricht dem Befehl \INS{lea eax, [ebp-4]}.

Es wird also 4 von Wert in \EBP abgezogen und das Ergebnis in das Register \EAX geladen.
Danach wird der Wert in \EAX auf dem Stack abgelegt und \scanf wird aufgerufen.

Anschließend wird \printf mit einem Argument aufgerufen--einen Pointer auf den String:
\TT{You entered \%d...\textbackslash{}n}.

Das zweite Argument wird mit \TT{mov ecx, [ebp-4]} vorbereitet.
Dieser Befehl speichert den Wert der Variablen $x$ (nicht seine Adresse) im Register \ECX.

Schließlich wird der Wert in \ECX auf dem Stack gespeichert und das letzte \printf wird aufgerufen.

\EN{\input{patterns/04_scanf/1_simple/olly_EN}}
\RU{\input{patterns/04_scanf/1_simple/olly_RU}}
\IT{\input{patterns/04_scanf/1_simple/olly_IT}}
\DE{\input{patterns/04_scanf/1_simple/olly_DE}}
\FR{\input{patterns/04_scanf/1_simple/olly_FR}}
\JA{\input{patterns/04_scanf/1_simple/olly_JA}}


\myparagraph{GCC}

Kompilieren wir diesen Code mit GCC 4.4.1 unter Linux:

\lstinputlisting[style=customasmx86]{patterns/04_scanf/1_simple/ex1_GCC.asm}

\myindex{puts() instead of printf()}
GCC ersetzt den Aufruf von \printf durch einen Aufruf von \puts. Der Grund hierfür wurde bereits in ~(\myref{puts}) erklärt.

% TODO: rewrite
%\RU{Почему \scanf переименовали в \TT{\_\_\_isoc99\_scanf}, я честно говоря, пока не знаю.}
%\EN{Why \scanf is renamed to \TT{\_\_\_isoc99\_scanf}, I do not know yet.}
% 
% Apparently it has to do with the ISO c99 standard compliance. By default GCC allows specifying a standard to adhere to.
% For example if you compile with -std=c89 the outputted assmebly file will contain scanf and not __isoc99__scanf. I guess current GCC version adhares to c99 by default.
% According to my understanding the two implementations differ in the set of suported modifyers (See printf man page)
Genau wie im MSVC Beispiel werden die Argumente mithilfe des Befehls \MOV auf dem Stack abgelegt.

\myparagraph{By the way}
Dieses einfache Beispiel ist übrigens eine Demonstration der Tatsache, dass der Compiler eine Liste von Ausdrücken in einem
\CCpp-Block in eine sequentielle Liste von Befehlen übersetzt.
Es gibt nichts zwischen zwei \CCpp-Anweisungen und genauso verhält es sich auch im Maschinencode.
Der Control Flow geht von einem Ausdruck direkt an den folgenden über.
}
\FR{\subsubsection{x86}

\myparagraph{MSVC}

Voici ce que l'on obtient après avoir compilé avec MSVC 2010:

\lstinputlisting[style=customasmx86]{patterns/04_scanf/1_simple/ex1_MSVC_FR.asm}

\TT{x} est une variable locale.

D'après le standard \CCpp elle ne doit être visible que dans cette fonction et dans
aucune autre portée.
Traditionnellement, les variables locales sont stockées sur la pile.
Il y a probablement d'autres moyens de les allouer, mais en x86, c'est la façon de faire.

\myindex{x86!\Instructions!PUSH}
Le but de l'instruction suivant le prologue de la fonction, \TT{PUSH ECX}, n'est
pas de sauver l'état de \ECX (noter l'absence d'un \TT{POP ECX} à la fin de la
fonction).

En fait, cela alloue 4 octets sur la pile pour stocker la variable \TT{x}.

\label{stack_frame}
\myindex{\Stack!Stack frame}
\myindex{x86!\Registers!EBP}
\TT{x} est accédée à l'aide de la macro \TT{\_x\$} (qui vaut -4) et du registre \EBP
qui pointe sur la structure de pile courante.

Pendant la durée de l'exécution de la fonction, \EBP pointe sur la \glslink{stack frame}{structure locale de pile}
courante, rendant possible l'accès aux variables locales et aux arguments de la
fonction via \TT{EBP+offset}.

\myindex{x86!\Registers!ESP}
Il est aussi possible d'utiliser \ESP dans le même but, bien que ça ne soit pas
très commode, car il change fréquemment.
La valeur de \EBP peut être perçue comme un \emph{état figé} de la valeur de \ESP
au début de l'exécution de la fonction.

% FIXME1 это уже было в 02_stack?
Voici une \glslink{stack frame}{structure de pile} typique dans un environnement 32-bit:

\begin{center}
\begin{tabular}{ | l | l | }
\hline
\dots & \dots \\
\hline
EBP-8 & variable locale \#2, \MarkedInIDAAs{} \TT{var\_8} \\
\hline
EBP-4 & variable locale \#1, \MarkedInIDAAs{} \TT{var\_4} \\
\hline
EBP & valeur sauvée de \EBP \\
\hline
EBP+4 & adresse de retour \\
\hline
EBP+8 & \argument \#1, \MarkedInIDAAs{} \TT{arg\_0} \\
\hline
EBP+0xC & \argument \#2, \MarkedInIDAAs{} \TT{arg\_4} \\
\hline
EBP+0x10 & \argument \#3, \MarkedInIDAAs{} \TT{arg\_8} \\
\hline
\dots & \dots \\
\hline
\end{tabular}
\end{center}

La fonction \scanf de notre exemple a deux arguments.

Le premier est un pointeur sur la chaîne contenant \TT{\%d} et le second est l'adresse
de la variable \TT{x}.

\myindex{x86!\Instructions!LEA}
Tout d'abord, l'adresse de la variable \TT{x} est chargée dans le registre \EAX
par l'instruction \\ \TT{lea eax, DWORD PTR \_x\$[ebp]}.

\LEA signifie \emph{load effective address} (charger l'adresse effective) et est souvent
utilisée pour composer une adresse ~(\myref{sec:LEA}).

Nous pouvons dire que dans ce cas, \LEA stocke simplement la somme de la valeur du
registre \EBP et de la macro \TT{\_x\$} dans le registre \EAX.

C'est la même chose que \INS{lea eax, [ebp-4]}.

Donc, 4 est soustrait de la valeur du registre \EBP et le résultat est chargé dans
le registre \EAX.
Ensuite, la valeur du registre \EAX est poussée sur la pile et \scanf est appelée.

\printf est appelée ensuite avec son premier argument --- un pointeur sur la chaîne:
\TT{You entered \%d...\textbackslash{}n}.

Le second argument est préparé avec: \TT{mov ecx, [ebp-4]}.
L'instruction stocke la valeur de la variable \TT{x} et non son adresse, dans le
registre \ECX.

Puis, la valeur de \ECX est stockée sur la pile et le dernier appel à \printf
est effectué.

\EN{\input{patterns/04_scanf/1_simple/olly_EN}}
\RU{\input{patterns/04_scanf/1_simple/olly_RU}}
\IT{\input{patterns/04_scanf/1_simple/olly_IT}}
\DE{\input{patterns/04_scanf/1_simple/olly_DE}}
\FR{\input{patterns/04_scanf/1_simple/olly_FR}}
\JA{\input{patterns/04_scanf/1_simple/olly_JA}}


\myparagraph{GCC}

Compilons ce code avec GCC 4.4.1 sous Linux:

\lstinputlisting[style=customasmx86]{patterns/04_scanf/1_simple/ex1_GCC.asm}

\myindex{puts() instead of printf()}
GCC a remplacé l'appel à \printf avec un appel à \puts. La raison de cela a été
expliquée dans ~(\myref{puts}).

% TODO: rewrite
%\RU{Почему \scanf переименовали в \TT{\_\_\_isoc99\_scanf}, я честно говоря, пока не знаю.}
%\EN{Why \scanf is renamed to \TT{\_\_\_isoc99\_scanf}, I do not know yet.}
% 
% Apparently it has to do with the ISO c99 standard compliance. By default GCC allows specifying a standard to adhere to.
% For example if you compile with -std=c89 the outputted assmebly file will contain scanf and not __isoc99__scanf. I guess current GCC version adhares to c99 by default.
% According to my understanding the two implementations differ in the set of suported modifyers (See printf man page)

Comme dans l'exemple avec MSVC---les arguments sont placés dans la pile avec l'instruction
\MOV.

\myparagraph{À propos}

Ce simple exemple est la démonstration du fait que le compilateur traduit
une liste d'expression en bloc-\CCpp en une liste séquentielle d'instructions.
% TODO FIXME: better translation / clarify ?
Il n'y a rien entre les expressions en \CCpp, et le résultat en code machine,
il n'y a rien entre le déroulement du flux de contrôle d'une expression à la suivante.
}
\PL{\subsubsection{x86}

\myparagraph{MSVC}

Tutaj znajduje się wynik kompilacji programu w MSVC 2010:

\lstinputlisting[style=customasmx86]{patterns/04_scanf/1_simple/ex1_MSVC_EN.asm}

\TT{x} jest zmienną lokalną.

Według standardu \CCpp zmienna lokalna może być widoczna tylko w konkretnej funkcji. Tradycyjnie zmienne lokalne są przechowywane na stosie. Prawdopodnie są inne moliwości przechowywania tych zmiennych, ale tak akurat jest w x86.

\myindex{x86!\Instructions!PUSH}
Zadaniem instrukcji rozpoczynającej funkcję, \TT{PUSH ECX}, nie jest zapisanie stanu \ECX (można zauważyć brak odpowiadającej instrukcji POP ECX na końcu funkcji).

Tak naprawdę instrukcja ta alokuje 4 bajty na stosie do przechowania zmiennej x.

\label{stack_frame}
\myindex{\Stack!Stack frame}
\myindex{x86!\Registers!EBP}
Dostęp do \TT{x} odbywa się z asystującym makrem \TT{\_x\$} (równym -4) i rejestrem \EBP rwskazującym bieżącą ramkę.

We fragmencie wykonującym funkcje, \EBP wskazuje bieżącą \gls{stack frame}
umożliwiając dostęp do zmiennych lokalnych i argumentów funkcji poprzez \TT{EBP+offset}.

\myindex{x86!\Registers!ESP}
Możliwe jest także użycie \ESP w takim samym celu, le nie jest to zbyt wygodne, ponieważ wartość tego rejestru często się zmienia.
Wartość \EBP może być postrzegana jako \emph{frozen state} wartości w \ESP z początku wykonania funkcji.

% FIXME1 это уже было в 02_stack?
Tutaj znajduje się typowa ramka stosu w układzie środowiska 32-bitowego:

\begin{center}
\begin{tabular}{ | l | l | }
\hline
\dots & \dots \\
\hline
EBP-8 & zmienna lokalna \#2, \MarkedInIDAAs{} \TT{var\_8} \\
\hline
EBP-4 & zmienna lokalna \#1, \MarkedInIDAAs{} \TT{var\_4} \\
\hline
EBP & zapisana wartość \EBP \\
\hline
EBP+4 & adres powrotu \\
\hline
EBP+8 & \argument \#1, \MarkedInIDAAs{} \TT{arg\_0} \\
\hline
EBP+0xC & \argument \#2, \MarkedInIDAAs{} \TT{arg\_4} \\
\hline
EBP+0x10 & \argument \#3, \MarkedInIDAAs{} \TT{arg\_8} \\
\hline
\dots & \dots \\
\hline
\end{tabular}
\end{center}

Funkcja \scanf w naszym przykładzie ma dwa argumenty.

Pierwszy jest wskaźnikiem na string \TT{\%d} a drugi jest adresem zmiennej \TT{x}.

\myindex{x86!\Instructions!LEA}
Na początku adres zmiennej \TT{x} jest ładowany do rejestru \EAX przy pomocy instrukcji \\
\TT{lea eax, DWORD PTR \_x\$[ebp]}.

\LEA oznacza \emph{load effective address} i jest często używana do formowania adresów ~(\myref{sec:LEA}).

Można powiedzieć, że w tym przypadku \LEA po prostu umieszcza sumę rejestru \EBP i makra \TT{\_x\$} w rejestrze \EAX.

To jest to samo co \INS{lea eax, [ebp-4]}.

Więc od rejestru \EBP jest odejmowane 4 i wynik zostaje umieszczony w rejestrze \EAX.
Następnie wartość rejestru \EAX jest odkładana na stosie i funkcja \scanf zostaje wywołana.

\printf wywołuje się z pierwszym argumentem- wskaźnikiem na string:
\TT{You entered \%d...\textbackslash{}n}.

Drugi argument jest przygotowywany za pomocą: \TT{mov ecx, [ebp-4]}.
Instrukcja kopiuje zmienną \TT{x} (nie jej adres) do rejestru \ECX.

Następnie wartość z \ECX jest odkładana na stos, a na koniec zostaje wywołana funkcja  \printf.

\EN{\input{patterns/04_scanf/1_simple/olly_EN}}
\RU{\input{patterns/04_scanf/1_simple/olly_RU}}
\IT{\input{patterns/04_scanf/1_simple/olly_IT}}
\DE{\input{patterns/04_scanf/1_simple/olly_DE}}
\FR{\input{patterns/04_scanf/1_simple/olly_FR}}
\JA{\input{patterns/04_scanf/1_simple/olly_JA}}


\myparagraph{GCC}

Tak wygląda skompilowany kod w GCC 4.4.1 w systemie Linux:

\lstinputlisting[style=customasmx86]{patterns/04_scanf/1_simple/ex1_GCC.asm}

\myindex{puts() instead of printf()}
GCC zamienia wywołanie funkcji \printf na wywołanie funkcji \puts. Powód tego został wyjaśniony w ~(\myref{puts}).

% TODO: rewrite
%\RU{Почему \scanf переименовали в \TT{\_\_\_isoc99\_scanf}, я честно говоря, пока не знаю.}
%\EN{Why \scanf is renamed to \TT{\_\_\_isoc99\_scanf}, I do not know yet.}
% 
% Apparently it has to do with the ISO c99 standard compliance. By default GCC allows specifying a standard to adhere to.
% For example if you compile with -std=c89 the outputted assmebly file will contain scanf and not __isoc99__scanf. I guess current GCC version adhares to c99 by default.
% According to my understanding the two implementations differ in the set of suported modifyers (See printf man page)

Jak w przykładzie MSVC---argumenty funkcji są umieszczane na stosie przy użyciu instrukcji \MOV.

\myparagraph{By the way}

Ten prosty przykład pokazuje jak faktycznie kompilatory tłumaczą
listy wyrażeń w \CCpp-block na sekwencyjne listy instrukcji.
Nie ma nic pomiędzy wyrażeniami w \CCpp a wynikowym kodem maszynowym.
}
\JA{\subsubsection{x86}

\myparagraph{MSVC}

MSVC 2010でコンパイルした後に得られるものは次のとおりです。

\lstinputlisting[style=customasmx86]{patterns/04_scanf/1_simple/ex1_MSVC_JA.asm}

\TT{x}はローカル変数です。

\CCpp 標準によれば、この関数でのみ表示でき、他の外部スコープでは表示できません。
従来、ローカル変数はスタックに格納されていました。
それらを割り当てる方法はおそらく他にもありますが、それはx86の方法です。

\myindex{x86!\Instructions!PUSH}
関数プロローグ、\TT{PUSH ECX}に続く命令の目的は、 \ECX 状態を保存することではありません
(関数の最後に対応する\TT{POP ECX}が存在しないことに注意してください)。

実際、\TT{x}変数を格納するためにスタックに4バイトを割り当てます。

\label{stack_frame}
\myindex{\Stack!Stack frame}
\myindex{x86!\Registers!EBP}
\TT{x}は、\TT{\_x\$} マクロ (-4に等しい)と現在のフレームを指す \EBP レジスタの助けを借りてアクセスされます。

関数の実行の範囲にわたって、 \EBP は現在の\gls{stack frame}を指しており、 \TT{EBP+オフセット}
を介してローカル変数と関数引数にアクセスすることができます。

\myindex{x86!\Registers!ESP}
同じ目的で \ESP を使用することもできますが、 \ESP は頻繁に変更されるためあまり便利ではありません。 
\EBP の値は、関数の実行開始時に \ESP の値が固定された状態として認識される可能性があります。

% FIXME1 это уже было в 02_stack?
32ビット環境での典型的な\gls{stack frame}レイアウトを次に示します。

\begin{center}
\begin{tabular}{ | l | l | }
\hline
\dots & \dots \\
\hline
EBP-8 & local variable \#2, \MarkedInIDAAs{} \TT{var\_8} \\
\hline
EBP-4 & local variable \#1, \MarkedInIDAAs{} \TT{var\_4} \\
\hline
EBP & saved value of \EBP \\
\hline
EBP+4 & return address \\
\hline
EBP+8 & \argument \#1, \MarkedInIDAAs{} \TT{arg\_0} \\
\hline
EBP+0xC & \argument \#2, \MarkedInIDAAs{} \TT{arg\_4} \\
\hline
EBP+0x10 & \argument \#3, \MarkedInIDAAs{} \TT{arg\_8} \\
\hline
\dots & \dots \\
\hline
\end{tabular}
\end{center}

この例の \scanf 関数には2つの引数があります。

最初のものは\TT{\%d}を含む文字列へのポインタで、2番目のものは\TT{x}変数のアドレスです。

\myindex{x86!\Instructions!LEA}
最初に、\TT{x}変数のアドレスが\TT{lea eax, DWORD PTR \_x\$[ebp]}命令によって \EAX レジスタにロードされます。

\LEA は\emph{ロード実効アドレス}の略で、アドレスを形成するためによく使用されます(~(\myref{sec:LEA}))。

この場合、\LEA は単に \EBP レジスタ値と\TT{\_x\$}マクロの合計を \EAX レジスタに格納すると言うことができます。

これは\INS{lea eax, [ebp-4]}と同じです。

したがって、 \EBP レジスタ値から4が減算され、その結果が \EAX レジスタにロードされます。
次に、 \EAX レジスタの値がスタックにプッシュされ、 \scanf が呼び出されます。

\printf は最初の引数で呼び出されています。文字列へのポインタ:
\TT{You entered \%d...\textbackslash{}n}

2番目の引数は\TT{mov ecx, [ebp-4]}で準備されています。
命令は、 \ECX レジスタにそのアドレスではなく\TT{x}変数値を格納します。

次に、 \ECX の値がスタックに格納され、最後の \printf が呼び出されます。

\EN{\input{patterns/04_scanf/1_simple/olly_EN}}
\RU{\input{patterns/04_scanf/1_simple/olly_RU}}
\IT{\input{patterns/04_scanf/1_simple/olly_IT}}
\DE{\input{patterns/04_scanf/1_simple/olly_DE}}
\FR{\input{patterns/04_scanf/1_simple/olly_FR}}
\JA{\input{patterns/04_scanf/1_simple/olly_JA}}


\myparagraph{GCC}

Linux上のGCC 4.4.1でこのコードをコンパイルしようとしましょう。

\lstinputlisting[style=customasmx86]{patterns/04_scanf/1_simple/ex1_GCC.asm}

\myindex{puts() instead of printf()}
GCCは \printf 呼び出しを \puts の呼び出しで置き換えました。 この理由は、~(\myref{puts})で説明されました。

% TODO: rewrite
%\RU{Почему \scanf переименовали в \TT{\_\_\_isoc99\_scanf}, я честно говоря, пока не знаю.}
%\EN{Why \scanf is renamed to \TT{\_\_\_isoc99\_scanf}, I do not know yet.}
% 
% Apparently it has to do with the ISO c99 standard compliance. By default GCC allows specifying a standard to adhere to.
% For example if you compile with -std=c89 the outputted assmebly file will contain scanf and not __isoc99__scanf. I guess current GCC version adhares to c99 by default.
% According to my understanding the two implementations differ in the set of suported modifyers (See printf man page)

MSVCの例のように、引数は \MOV 命令を使用してスタックに配置されます。

\myparagraph{ところで}

ところで、この単純な例は、コンパイラが \CCpp ブロックの式のリストを命令の連続したリストに
変換するという事実のデモンストレーションです。
\CCpp の式の間には何もないので、結果のマシンコードには、
ある式から次の式への制御フローの間には何もありません。
}

\EN{\subsection{x64: \Optimizing MSVC 2013}

\lstinputlisting[caption=\Optimizing MSVC 2013 x64,style=customasmx86]{\CURPATH/MSVC2013_x64_Ox_EN.asm}

First, MSVC inlined the \strlen{} function code, because it concluded this 
is to be faster than the usual \strlen{} work + the cost of calling it and returning from it.
This is called inlining: \myref{inline_code}.

\myindex{x86!\Instructions!OR}
\myindex{\CStandardLibrary!strlen()}
\label{using_OR_instead_of_MOV}
The first instruction of the inlined \strlen{} is\\
\TT{OR RAX, 0xFFFFFFFFFFFFFFFF}. 

MSVC often uses \TT{OR} instead of \TT{MOV RAX, 0xFFFFFFFFFFFFFFFF}, because resulting opcode is shorter.

And of course, it is equivalent: all bits are set, and a number with all bits set is $-1$ 
in two's complement arithmetic: \myref{sec:signednumbers}.

Why would the $-1$ number be used in \strlen{}, one might ask.
Due to optimizations, of course.
Here is the code that MSVC generated:

\lstinputlisting[caption=Inlined \strlen{} by MSVC 2013 x64,style=customasmx86]{\CURPATH/strlen_MSVC_EN.asm}

Try to write shorter if you want to initialize the counter at 0!
OK, let' try:

\lstinputlisting[caption=Our version of \strlen{},style=customasmx86]{\CURPATH/my_strlen_EN.asm}

We failed. We have to use additional \INS{JMP} instruction!

So what the MSVC 2013 compiler did is to move the \TT{INC} instruction to the place before 
the actual character loading.

If the first character is 0, that's OK, \RAX is 0 at this moment, 
so the resulting string length is 0.

The rest in this function seems easy to understand.

\subsection{x64: \NonOptimizing GCC 4.9.1}

\lstinputlisting[style=customasmx86]{\CURPATH/GCC491_x64_O0_EN.asm}

Comments are added by the author of the book.

After the execution of \strlen{}, the control is passed to the L2 label, 
and there two clauses are checked, one after another.

\myindex{\CLanguageElements!Short-circuit}
The second will never be checked, if the first one (\emph{str\_len==0}) is false 
(this is \q{short-circuit}).

Now let's see this function in short form:

\begin{itemize}
\item First for() part (call to \strlen{})
\item goto L2
\item L5: for() body. goto exit, if needed
\item for() third part (decrement of str\_len)
\item L2: 
for() second part: check first clause, then second. goto loop body begin or exit.
\item L4: // exit
\item return s
\end{itemize}

\subsection{x64: \Optimizing GCC 4.9.1}
\label{string_trim_GCC_x64_O3}

\lstinputlisting[style=customasmx86]{\CURPATH/GCC491_x64_O3_EN.asm}

Now this is more complex.

The code before the loop's body start is executed only once, but it has the \CRLF{} 
characters check too!
What is this code duplication for?

The common way to implement the main loop is probably this:

\begin{itemize}
\item (loop start) check for 
\CRLF{} characters, make decisions
\item store zero character
\end{itemize}

But GCC has decided to reverse these two steps. 

Of course, \emph{store zero character} cannot be first step, so another check is needed:

\begin{itemize}
\item workout first character. match it to \CRLF{}, exit if character is not \CRLF{}

\item (loop begin) store zero character

\item check for \CRLF{} characters, make decisions
\end{itemize}

Now the main loop is very short, which is good for latest \ac{CPU}s.

The code doesn't use the str\_len variable, but str\_len-1.
So this is more like an index in a buffer.

Apparently, GCC notices that the str\_len-1 statement is used twice.

So it's better to allocate a variable which always holds a value that's smaller than 
the current string length by one, 
and decrement it (this is the same effect as decrementing the str\_len variable).
}
\RU{\subsection{x64: \Optimizing MSVC 2013}

\lstinputlisting[caption=\Optimizing MSVC 2013 x64,style=customasmx86]{\CURPATH/MSVC2013_x64_Ox_RU.asm}

В начале, MSVC вставил тело функции \strlen{} прямо в код, потому что решил, что так будет
быстрее чем обычная работа \strlen{} + время на вызов её и возврат из нее.

Это также называется \emph{inlining}: \myref{inline_code}.

\myindex{x86!\Instructions!OR}
\myindex{\CStandardLibrary!strlen()}
\label{using_OR_instead_of_MOV}
Первая инструкция функции \strlen{} вставленная здесь,\\
это \TT{OR RAX, 0xFFFFFFFFFFFFFFFF}. 
MSVC часто использует \TT{OR} вместо \TT{MOV RAX, 0xFFFFFFFFFFFFFFFF}, потому что опкод получается короче.

И конечно, это эквивалентно друг другу: все биты просто выставляются, а все выставленные
биты это -1 в дополнительном коде (two's complement): \myref{sec:signednumbers}.

Кто-то мог бы спросить, зачем вообще нужно использовать число -1 в функции \strlen{}?

Вследствие оптимизации, конечно.
Вот что сделал MSVC:

\lstinputlisting[caption=Вставленная \strlen{} сгенерированная MSVC 2013 x64,style=customasmx86]{\CURPATH/strlen_MSVC_RU.asm}

Попробуйте написать короче, если хотите инициализировать счетчик нулем!

Ну, например:

\lstinputlisting[caption=Наша версия \strlen{},style=customasmx86]{\CURPATH/my_strlen_RU.asm}

Не получилось. Нам придется вводить дополнительную инструкцию JMP!

Что сделал MSVC 2013, так это передвинул инструкцию \TT{INC} в место перед загрузкой символа.

Если самый первый символ --- нулевой, всё нормально, \RAX содержит 0 в этот момент, так что
итоговая длина строки будет 0.

Остальную часть функции проще понять.

\subsection{x64: \NonOptimizing GCC 4.9.1}

\lstinputlisting[style=customasmx86]{\CURPATH/GCC491_x64_O0_RU.asm}

Комментарии автора.
После исполнения \strlen{}, управление передается на метку L2,
и там проверяются два выражения, одно после другого.

\myindex{\CLanguageElements!Short-circuit}
Второе никогда не будет проверяться, если первое выражение не истинно (\emph{str\_len==0})
(это \q{short-circuit}).

Теперь посмотрим на эту функцию в коротком виде:

\begin{itemize}
\item Первая часть for() (вызов \strlen{})
\item goto L2
\item L5: Тело for(). переход на выход, если нужно
\item Третья часть for() (декремент str\_len)
\item L2: Вторая часть for(): проверить первое выражение, затем второе. 
переход на начало тела цикла, или выход.

\item L4: // выход
\item return s
\end{itemize}

\subsection{x64: \Optimizing GCC 4.9.1}
\label{string_trim_GCC_x64_O3}

\lstinputlisting[style=customasmx86]{\CURPATH/GCC491_x64_O3_RU.asm}

Тут более сложный результат.
Код перед циклом исполняется только один раз, но также содержит проверку символов \CRLF{}!

Зачем нужна это дублирование кода?

Обычная реализация главного цикла это, наверное, такая:

\begin{itemize}
\item (начало цикла) проверить символы \CRLF{}, принять решения

\item записать нулевой символ
\end{itemize}

Но GCC решил поменять местами эти два шага. 
Конечно, шаг \emph{записать нулевой символ} не может быть первым, так что нужна еще одна
проверка:


\begin{itemize}
\item обработать первый символ. сравнить его с \CRLF{}, выйти если символ не равен \CRLF{}

\item (начало цикла) записать нулевой символ

\item проверить символы \CRLF{}, принять решения

\end{itemize}

Теперь основной цикл очень короткий, а это очень хорошо для современных процессоров.

Код не использует переменную str\_len, но str\_len-1.

Так что это больше похоже на индекс в буфере.
Должно быть, GCC заметил, что выражение str\_len-1 используется дважды.

Так что будет лучше выделить переменную, которая всегда содержит значение равное 
текущей длине строки минус 1, и уменьшать его на 1 (это тот же эффект, что и уменьшать
переменную str\_len).

}
\PTBR{\input{patterns/04_scanf/1_simple/x64_PTBR}}
\IT{\subsubsection{x64}

\myindex{x86-64}
La situazione e' simile, con l'unica differenza che, per il passaggio degli argomenti, i registri sono usati al posto dello stack.

\myparagraph{MSVC}

\lstinputlisting[caption=MSVC 2012 x64,style=customasmx86]{patterns/04_scanf/1_simple/ex1_MSVC_x64_EN.asm}

\myparagraph{GCC}

\lstinputlisting[caption=\Optimizing GCC 4.4.6 x64,style=customasmx86]{patterns/04_scanf/1_simple/ex1_GCC_x64_EN.s}

}
\DE{\subsection{x64}

\myindex{x86-64}

Die Geschichte bei x86-64 Funktions Argumenten ist ein wenig anders (zumindest für die ersten vier bis sechs)
sie werden über die Register übergeben z.b. der \gls{callee} liest direkt aus den Registern anstatt vom Stack 
zu lesen.

\subsubsection{MSVC}

\Optimizing MSVC:

\lstinputlisting[caption=\Optimizing MSVC 2012 x64,style=customasmx86]{patterns/05_passing_arguments/x64_MSVC_Ox_EN.asm}

Wie wir sehen können, die compact Funktion \ttf nimmt alle Argumente aus den Registern.

Die \LEA Instruktion wird hier für Addition benutzt,
scheinbar hat der Compiler die Instruktion für schneller befunden als
die \TT{ADD} Instruktion.

\myindex{x86!\Instructions!LEA}

\LEA wird auch benutzt in der \main Funktion um das erste und das dritte \ttf Argument vor zu bereiten.
Der Compiler muss entschieden haben das dies schneller abgearbeitet wird als die Werte in die Register 
zu laden mit der \MOV Instruktion.

Lasst uns einen Blick auf nicht optimierte MSVC Ausgabe werfen:

\lstinputlisting[caption=MSVC 2012 x64,style=customasmx86]{patterns/05_passing_arguments/x64_MSVC_IDA_EN.asm}

Es sieht ein bisschen wie ein Puzzle aus, weil alle drei Argumente aus den Registern auf dem Stack
gespeichert werden aus irgend einem Grund.

\myindex{Shadow space}
\label{shadow_space}
Dies bezeichnet man als \q{shadow space}

\footnote{\href{http://go.yurichev.com/17256}{MSDN}}: 
So wird sich wahrscheinlich jede Win64 EXE verhalten und alle vier Register Werte auf dem Stack speichern.

Das wird aus zwei Gründen so gemacht:

1) Es ist ziemlich übertrieben ein ganzes Register (oder gar vier Register) zu Reservieren für eine
Argument Übergabe, also werden die Argumente über den Stack zugänglich gemacht.
2) Der Debugger weiß immer wo die Funktions Argumente zu finden sind bei einem breakpoint\footnote{\href{http://go.yurichev.com/17257}{MSDN}}.


Also, so können größere Funktionen ihre Eingabe Argumente im \q{shadows space} speichern wenn die Funktion
auf die Argumente während der Laufzeit zugreifen will, kleinere Funktionen (wie unsere) zeigen dieses Verhalten 
nicht. 

Es liegt in der Verantwortung vom \gls{caller} den \q{shadow space} auf dem Stack zu allozieren.

\subsubsection{GCC}

Optimierter GCC generiert mehr oder minder verständlichen Code:

\lstinputlisting[caption=\Optimizing GCC 4.4.6 x64,style=customasmx86]{patterns/05_passing_arguments/x64_GCC_O3_EN.s}

\NonOptimizing GCC:

\lstinputlisting[caption=GCC 4.4.6 x64,style=customasmx86]{patterns/05_passing_arguments/x64_GCC_EN.s}

\myindex{Shadow space}

Bei System V *NIX Systemen (\SysVABI) ist kein \q{shadow space} nötig, aber der \gls{callee} will vielleicht
seine Argumente irgendwo speichern im Fall das keine oder zu wenig Register frei sind.

\subsubsection{GCC: uint64\_t statt int}

Unser Beispiel funktioniert mit 32-Bit \Tint, weshalb auch 32-Bit Register Bereiche benutzt werden (mit dem Präfix \TT{E-}).

Es lassen sich auch ohne Probleme 64-Bit Werte benutzen:

\lstinputlisting{patterns/05_passing_arguments/ex64.c}

\lstinputlisting[caption=\Optimizing GCC 4.4.6 x64,style=customasmx86]{patterns/05_passing_arguments/ex64_GCC_O3_IDA_EN.asm}

Der Code ist der gleiche, aber diesmal werden die \emph{full size} 64-Bit Register benutzt (mit dem \TT{R-} Präfix).

}
\FR{\subsubsection{x64: 8 arguments}

\myindex{x86-64}
\label{example_printf8_x64}
Pour voir comment les autres arguments sont passés par la pile, changeons encore
notre exemple en augmentant le nombre d'arguments à 9 (chaîne de format de
\printf + 8 variables \Tint):

\lstinputlisting[style=customc]{patterns/03_printf/2.c}

\myparagraph{MSVC}

Comme il a déjà été mentionné, les 4 premiers arguments sont passés par les registres
\RCX, \RDX, \Reg{8}, \Reg{9} sous Win64, tandis les autres le sont---par la pile.
C'est exactement de que l'on voit ici.
Toutefois, l'instruction \MOV est utilisée ici à la place de \PUSH, donc les valeurs
sont stockées sur la pile d'une manière simple.

\lstinputlisting[caption=MSVC 2012 x64,style=customasmx86]{patterns/03_printf/x86/2_MSVC_x64_FR.asm}

Le lecteur observateur pourrait demander pourquoi 8 octets sont alloués sur la
pile pour les valeurs \Tint, alors que 4 suffisent?
Oui, il faut se rappeler: 8 octets sont alloués pour tout type de données plus
petit que 64 bits.
Ceci est instauré pour des raisons de commodités: cela rend facile le calcul
de l'adresse de n'importe quel argument.
En outre, ils sont tous situés à des adresses mémoires alignées.
Il en est de même dans les environnements 32-bit: 4 octets sont réservés pour tout
types de données.

% also for local variables?

\myparagraph{GCC}

Le tableau est similaire pour les OS x86-64 *NIX, excepté que les 6 premiers arguments
sont passés par les registres \RDI, \RSI, \RDX, \RCX, \Reg{8}, \Reg{9}.
Tout les autres---par la pile.
GCC génère du code stockant le pointeur de chaîne dans \EDI au lieu de \RDI{}---nous
l'avons noté précédemment:
\myref{hw_EDI_instead_of_RDI}.

Nous avions également noté que le registre \EAX a été vidé avant l'appel à
\printf: \myref{SysVABI_input_EAX}.

\lstinputlisting[caption=GCC 4.4.6 x64 \Optimizing,style=customasmx86]{patterns/03_printf/x86/2_GCC_x64_FR.s}

\myparagraph{GCC + GDB}
\myindex{GDB}

Essayons cet exemple dans \ac{GDB}.

\begin{lstlisting}
$ gcc -g 2.c -o 2
\end{lstlisting}

\begin{lstlisting}
$ gdb 2
GNU gdb (GDB) 7.6.1-ubuntu
...
Reading symbols from /home/dennis/polygon/2...done.
\end{lstlisting}

\begin{lstlisting}[caption=mettons le point d'arrêt à \printf{,} et lançons]
(gdb) b printf
Breakpoint 1 at 0x400410
(gdb) run
Starting program: /home/dennis/polygon/2 

Breakpoint 1, __printf (format=0x400628 "a=%d; b=%d; c=%d; d=%d; e=%d; f=%d; g=%d; h=%d\n") at printf.c:29
29	printf.c: No such file or directory.
\end{lstlisting}

Les registres \RSI/\RDX/\RCX/\Reg{8}/\Reg{9} ont les valeurs attendues.
\RIP contient l'adresse de la toute première instruction de la fonction \printf.

\begin{lstlisting}
(gdb) info registers
rax            0x0	0
rbx            0x0	0
rcx            0x3	3
rdx            0x2	2
rsi            0x1	1
rdi            0x400628	4195880
rbp            0x7fffffffdf60	0x7fffffffdf60
rsp            0x7fffffffdf38	0x7fffffffdf38
r8             0x4	4
r9             0x5	5
r10            0x7fffffffdce0	140737488346336
r11            0x7ffff7a65f60	140737348263776
r12            0x400440	4195392
r13            0x7fffffffe040	140737488347200
r14            0x0	0
r15            0x0	0
rip            0x7ffff7a65f60	0x7ffff7a65f60 <__printf>
...
\end{lstlisting}

\begin{lstlisting}[caption=inspectons la chaîne de format]
(gdb) x/s $rdi
0x400628:	"a=%d; b=%d; c=%d; d=%d; e=%d; f=%d; g=%d; h=%d\n"
\end{lstlisting}

Affichons la pile avec la commande x/g cette fois---\emph{g} est l'unité pour \emph{giant words}, i.e., mots de 64-bit.

\begin{lstlisting}
(gdb) x/10g $rsp
0x7fffffffdf38:	0x0000000000400576	0x0000000000000006
0x7fffffffdf48:	0x0000000000000007	0x00007fff00000008
0x7fffffffdf58:	0x0000000000000000	0x0000000000000000
0x7fffffffdf68:	0x00007ffff7a33de5	0x0000000000000000
0x7fffffffdf78:	0x00007fffffffe048	0x0000000100000000
\end{lstlisting}

Le tout premier élément de la pile, comme dans le cas précédent, est la \ac{RA}.
3 valeurs sont aussi passées par la pile: 6, 7, 8.
Nous voyons également que 8 est passé avec les 32-bits de poids fort non
effacés: \GTT{0x00007fff00000008}.
C'est en ordre, car les valeurs sont d'un type \Tint, qui est 32-bit.
Donc, la partie haute du registre ou l'élément de la pile peuvent contenir des
\q{restes de données aléatoires}.

Si vous regardez où le contrôle reviendra après l'exécution de \printf,
\ac{GDB} affiche la fonction \main en entier:

\begin{lstlisting}[style=customasmx86]
(gdb) set disassembly-flavor intel
(gdb) disas 0x0000000000400576
Dump of assembler code for function main:
   0x000000000040052d <+0>:	push   rbp
   0x000000000040052e <+1>:	mov    rbp,rsp
   0x0000000000400531 <+4>:	sub    rsp,0x20
   0x0000000000400535 <+8>:	mov    DWORD PTR [rsp+0x10],0x8
   0x000000000040053d <+16>:	mov    DWORD PTR [rsp+0x8],0x7
   0x0000000000400545 <+24>:	mov    DWORD PTR [rsp],0x6
   0x000000000040054c <+31>:	mov    r9d,0x5
   0x0000000000400552 <+37>:	mov    r8d,0x4
   0x0000000000400558 <+43>:	mov    ecx,0x3
   0x000000000040055d <+48>:	mov    edx,0x2
   0x0000000000400562 <+53>:	mov    esi,0x1
   0x0000000000400567 <+58>:	mov    edi,0x400628
   0x000000000040056c <+63>:	mov    eax,0x0
   0x0000000000400571 <+68>:	call   0x400410 <printf@plt>
   0x0000000000400576 <+73>:	mov    eax,0x0
   0x000000000040057b <+78>:	leave  
   0x000000000040057c <+79>:	ret    
End of assembler dump.
\end{lstlisting}

Laissons se terminer l'exécution de \printf, exécutez l'instruction mettant \EAX
à zéro, et notez que le registre \EAX à une valeur d'exactement zéro.
\RIP pointe maintenant sur l'instruction \INS{LEAVE}, i.e, la pénultième de la
fonction \main.

\begin{lstlisting}
(gdb) finish
Run till exit from #0  __printf (format=0x400628 "a=%d; b=%d; c=%d; d=%d; e=%d; f=%d; g=%d; h=%d\n") at printf.c:29
a=1; b=2; c=3; d=4; e=5; f=6; g=7; h=8
main () at 2.c:6
6		return 0;
Value returned is $1 = 39
(gdb) next
7	};
(gdb) info registers
rax            0x0	0
rbx            0x0	0
rcx            0x26	38
rdx            0x7ffff7dd59f0	140737351866864
rsi            0x7fffffd9	2147483609
rdi            0x0	0
rbp            0x7fffffffdf60	0x7fffffffdf60
rsp            0x7fffffffdf40	0x7fffffffdf40
r8             0x7ffff7dd26a0	140737351853728
r9             0x7ffff7a60134	140737348239668
r10            0x7fffffffd5b0	140737488344496
r11            0x7ffff7a95900	140737348458752
r12            0x400440	4195392
r13            0x7fffffffe040	140737488347200
r14            0x0	0
r15            0x0	0
rip            0x40057b	0x40057b <main+78>
...
\end{lstlisting}
}
\JA{\subsubsection{x64}

\myindex{x86-64}
ここの画像は、スタックではなくレジスタが引数の受け渡しに使用されるという違いと似ています。

\myparagraph{MSVC}

\lstinputlisting[caption=MSVC 2012 x64,style=customasmx86]{patterns/04_scanf/1_simple/ex1_MSVC_x64_EN.asm}

\myparagraph{GCC}

\lstinputlisting[caption=\Optimizing GCC 4.4.6 x64,style=customasmx86]{patterns/04_scanf/1_simple/ex1_GCC_x64_EN.s}

}

\EN{\subsubsection{ARM}

\myparagraph{\OptimizingKeilVI (\ThumbMode)}

\lstinputlisting[style=customasmARM]{patterns/04_scanf/1_simple/ARM_IDA.lst}

\myindex{\CLanguageElements!\Pointers}

In order for \scanf to be able to read item it needs a parameter---pointer to an \Tint.
\Tint is 32-bit, so we need 4 bytes to store it somewhere in memory, and it fits exactly in a 32-bit register.
\myindex{IDA!var\_?}
A place for the local variable \GTT{x} is allocated in the stack and \IDA
has named it \emph{var\_8}. It is not necessary, however, to allocate a such since \ac{SP} (\gls{stack pointer}) is already pointing to that space and it can be used directly.

So, \ac{SP}'s value is copied to the \Reg{1} register and, together with the format-string, passed to \scanf.

\INS{PUSH/POP} instructions behaves differently in ARM than in x86 (it's the other way around).
They are synonyms to \INS{STM/STMDB/LDM/LDMIA} instructions.
And \INS{PUSH} instruction first writes a value into the stack, \emph{and then} subtracts \ac{SP} by 4.
\INS{POP} first adds 4 to \ac{SP}, \emph{and then} reads a value from the stack.
Hence, after \INS{PUSH}, \ac{SP} points to an unused space in stack.
It is used by \scanf, and by \printf after.

\INS{LDMIA} means \emph{Load Multiple Registers Increment address After each transfer}.
\INS{STMDB} means \emph{Store Multiple Registers Decrement address Before each transfer}.

\myindex{ARM!\Instructions!LDR}
Later, with the help of the \INS{LDR} instruction, this value is moved from the stack to the \Reg{1} register in order to be passed to \printf.

\myparagraph{ARM64}

\lstinputlisting[caption=\NonOptimizing GCC 4.9.1 ARM64,numbers=left,style=customasmARM]{patterns/04_scanf/1_simple/ARM64_GCC491_O0_EN.s}

There is 32 bytes are allocated for stack frame, which is bigger than it needed. Perhaps some memory aligning issue?
The most interesting part is finding space for the $x$ variable in the stack frame (line 22).
Why 28? Somehow, compiler decided to place this variable at the end of stack frame instead of beginning.
The address is passed to \scanf, which just stores the user input value in the memory at that address.
This is 32-bit value of type \Tint.
The value is fetched at line 27 and then passed to \printf.

}
\RU{\subsubsection{ARM}

\myparagraph{\OptimizingKeilVI (\ThumbMode)}

\lstinputlisting[style=customasmARM]{patterns/04_scanf/1_simple/ARM_IDA.lst}

\myindex{\CLanguageElements!\Pointers}
Чтобы \scanf мог вернуть значение, ему нужно передать указатель на переменную типа \Tint.
\Tint~--- 32-битное значение, для его хранения нужно только 4 байта, и оно помещается в 32-битный регистр.

\myindex{IDA!var\_?}
Место для локальной переменной \GTT{x} выделяется в стеке, \IDA наименовала её \emph{var\_8}. 
Впрочем, место для неё выделять не обязательно, т.к. \glslink{stack pointer}{указатель стека} \ac{SP} уже указывает на место, 
свободное для использования.
Так что значение указателя \ac{SP} копируется в регистр \Reg{1}, и вместе с format-строкой, 
передается в \scanf.

Инструкции \INS{PUSH/POP} в ARM работают иначе, чем в x86 (тут всё наоборот).
Это синонимы инструкций \INS{STM/STMDB/LDM/LDMIA}.
И инструкция \INS{PUSH} в начале записывает в стек значение, \emph{затем} вычитает 4 из \ac{SP}.
\INS{POP} в начале прибавляет 4 к \ac{SP}, \emph{затем} читает значение из стека.
Так что после \INS{PUSH}, \ac{SP} указывает на неиспользуемое место в стеке.
Его и использует \scanf, а затем и \printf.

\INS{LDMIA} означает \emph{Load Multiple Registers Increment address After each transfer}.
\INS{STMDB} означает \emph{Store Multiple Registers Decrement address Before each transfer}.

\myindex{ARM!\Instructions!LDR}
Позже, при помощи инструкции \INS{LDR}, это значение перемещается из стека в регистр \Reg{1}, чтобы быть переданным в \printf.

\myparagraph{ARM64}

\lstinputlisting[caption=\NonOptimizing GCC 4.9.1 ARM64,numbers=left,style=customasmARM]{patterns/04_scanf/1_simple/ARM64_GCC491_O0_RU.s}

Под стековый фрейм выделяется 32 байта, что больше чем нужно. Может быть, это связано с выравниваем по границе памяти?
Самая интересная часть~--- это поиск места под переменную $x$ в стековом фрейме (строка 22).
Почему 28? Почему-то, компилятор решил расположить эту переменную в конце стекового фрейма, а не в начале.
Адрес потом передается в \scanf, которая просто сохраняет значение, введенное пользователем, в памяти по этому адресу.
Это 32-битное значение типа \Tint.
Значение загружается в строке 27 и затем передается в \printf.

}
\IT{\subsubsection{ARM}

\myparagraph{\OptimizingKeilVI (\ThumbMode)}

\lstinputlisting[style=customasmARM]{patterns/04_scanf/1_simple/ARM_IDA.lst}

\myindex{\CLanguageElements!\Pointers}

Affinche' \scanf possa leggere l'input, necessita di un parametro ---puntatore ad un \Tint.
\Tint e' 32-bit, quindi servono 4 byte per memorizzarlo da qualche parte in memoria, e entra perfettamente in un registro a 32-bit.
\myindex{IDA!var\_?}
Uno spazio per la variabile locale \GTT{x} e' allocato nello stack e \IDA
lo ha chiamato \emph{var\_8}. Non e' comunque necessario allocarlo in questo modo poiche' \ac{SP} (\gls{stack pointer}) punta gia' a quella posizione e puo' essere usato direttamente.

Successivamente il valore di \ac{SP} e' copiato nel registro \Reg{1} e sono passati, insieme alla format-string, a \scanf.

% TBT here
%\INS{PUSH/POP} instructions behaves differently in ARM than in x86 (it's the other way around).
%They are synonyms to \INS{STM/STMDB/LDM/LDMIA} instructions.
%And \INS{PUSH} instruction first writes a value into the stack, \emph{and then} subtracts \ac{SP} by 4.
%\INS{POP} first adds 4 to \ac{SP}, \emph{and then} reads a value from the stack.
%Hence, after \INS{PUSH}, \ac{SP} points to an unused space in stack.
%It is used by \scanf, and by \printf after.

%\INS{LDMIA} means \emph{Load Multiple Registers Increment address After each transfer}.
%\INS{STMDB} means \emph{Store Multiple Registers Decrement address Before each transfer}.

\myindex{ARM!\Instructions!LDR}
Questo valore, con l'aiuto dell'istruzione \INS{LDR} , viene poi spostato dallo stakc al registro \Reg{1} per essere passato a \printf.

\myparagraph{ARM64}

\lstinputlisting[caption=\NonOptimizing GCC 4.9.1 ARM64,numbers=left,style=customasmARM]{patterns/04_scanf/1_simple/ARM64_GCC491_O0_EN.s}

Ci sono 32 byte allocati per lo stack frame, che e' piu' grande del necessario. Forse a causa di meccanismi di allineamento della memoria?
La parte piu' interessante e' quella in cui trova spazio per la variabile $x$ nello stack frame (riga 22).
Perche' 28? Il compilatore ha in qualche modo deciso di piazzare questa variabile alla fine dello stack frame anziche' all'inizio.
L'indirizzo e' passato a \scanf, che memorizzera' il valore immesso dall'utente nella memoria a quell'indirizzo.
Si tratta di un valore a 32-bit di tipo \Tint.
Il valore e' recuperato successivamente a riga 27 e passato a \printf.

}
\DE{\subsubsection{ARM}

\myparagraph{\OptimizingKeilVI (\ThumbMode)}

\lstinputlisting[style=customasmARM]{patterns/04_scanf/1_simple/ARM_IDA.lst}

\myindex{\CLanguageElements!\Pointers}
Damit \scanf Elemente einlesen kann, benötigt die Funktion einen Paramter--einen Pointer vom Typ \Tint.
\Tint hat die Größe 32 Bit, wir benötigen also 4 Byte, um den Wert im Speicher abzulegen, und passt daher genau in ein 32-Bit-Register.
\myindex{IDA!var\_?}
Auf dem Stack wird Platz für die lokalen Variable \GTT{x} reserviert und IDA bezeichnet diese Variable mit \emph{var\_8}. 
Eigentlich ist aber an dieser Stelle gar nicht notwendig, Platz auf dem Stack zu reservieren, da \ac{SP} (\gls{stack pointer} 
bereits auf die Adresse zeigt und auch direkt verwendet werden kann.

Der Wert von \ac{SP} wird also in das \Reg{1} Register kopiert und zusammen mit dem Formatierungsstring an \scanf übergeben.

% TBT here
%\INS{PUSH/POP} instructions behaves differently in ARM than in x86 (it's the other way around).
%They are synonyms to \INS{STM/STMDB/LDM/LDMIA} instructions.
%And \INS{PUSH} instruction first writes a value into the stack, \emph{and then} subtracts \ac{SP} by 4.
%\INS{POP} first adds 4 to \ac{SP}, \emph{and then} reads a value from the stack.
%Hence, after \INS{PUSH}, \ac{SP} points to an unused space in stack.
%It is used by \scanf, and by \printf after.

%\INS{LDMIA} means \emph{Load Multiple Registers Increment address After each transfer}.
%\INS{STMDB} means \emph{Store Multiple Registers Decrement address Before each transfer}.

\myindex{ARM!\Instructions!LDR}
Später wird mithilfe des \INS{LDR} Befehls dieser Wert vom Stack in das \Reg{1} Register verschoben um an \printf übergeben werden zu können.

\myparagraph{ARM64}

\lstinputlisting[caption=\NonOptimizing GCC 4.9.1 ARM64,numbers=left,style=customasmARM]{patterns/04_scanf/1_simple/ARM64_GCC491_O0_DE.s}

Im Stack Frame werden 32 Byte reserviert, was deutlich mehr als benötigt ist. Vielleicht handelt es sich um eine Frage des Aligning (dt. Angleichens) von Speicheradressen.
Der interessanteste Teil ist, im Stack Frame einen Platz für die Variable $x$ zu finden (Zeile 22).
Warum 28? Irgendwie hat der Compiler entschieden die Variable am Ende des Stack Frames anstatt an dessen Beginn abzulegen.
Die Adresse wird an \scanf übergeben; diese Funktion speichert den Userinput an der genannten Adresse im Speicher.
Es handelt sich hier um einen 32-Bit-Wert vom Typ \Tint. 
Der Wert wird in Zeile 27 abgeholt und dann an \printf übergeben.


}
\FR{\subsubsection{ARM}

\myparagraph{\OptimizingKeilVI (\ThumbMode)}

\lstinputlisting[style=customasmARM]{patterns/04_scanf/1_simple/ARM_IDA.lst}

\myindex{\CLanguageElements!\Pointers}

Afin que \scanf puisse lire l'item, elle a besoin d'un paramètre---un pointeur sur un \Tint.
Le type \Tint est 32-bit, donc nous avons besoin de 4 octets pour le stocker quelque
part en mémoire, et il tient exactement dans un registre 32-bit.
\myindex{IDA!var\_?}
De l'espace pour la variable locale \GTT{x} est allouée sur la pile et \IDA l'a
nommée \emph{var\_8}. Il n'est toutefois pas nécessaire de définir cette macro, puisque
le \ac{SP} (\glslink{stack pointer}{pointeur de pile}) pointe déjà sur cet espace et
peut être utilisé directement.

Donc, la valeur de \ac{SP} est copiée dans la registre \Reg{1} et, avec la chaîne
de format, passée à \scanf.

Les instructions \INS{PUSH/POP} se comportent différemment en ARM et en x86 (c'est l'inverse)
Il y a des sysnonymes aux instructions \INS{STM/STMDB/LDM/LDMIA}.
Et l'instruction \INS{PUSH} écrit d'abord une valeur sur la pile, \emph{et ensuite}
soustrait 4 de \ac{SP}.
De ce fait, après \INS{PUSH}, \ac{SP} pointe sur de l'espace inutilisé sur la pile.
Il est utilisé par \scanf, et après par \printf.

\INS{LDMIA} signifie \emph{Load Multiple Registers Increment address After each transfer}
(charge plusieurs registres incrémente l'adresse après chaque transfert).
\INS{STMDB} signifie \emph{Store Multiple Registers Decrement address Before each transfer}
(socke plusieurs registres décrémente l'adresse avant chaque transfert).

\myindex{ARM!\Instructions!LDR}
Plus tard, avec l'aide de l'instruction \INS{LDR}, cette valeur est copiée depuis
la pile vers le registre \Reg{1} afin de la passer à \printf.

\myparagraph{ARM64}

\lstinputlisting[caption=GCC 4.9.1 ARM64 \NonOptimizing,numbers=left,style=customasmARM]{patterns/04_scanf/1_simple/ARM64_GCC491_O0_FR.s}

Il y a 32 octets alloués pour la structure de pile, ce qui est plus que nécessaire.
Peut-être dans un soucis d'alignement de mémoire?
La partie la plus intéressante est de trouver de l'espace pour la variable $x$ dans
la structure de pile (ligne 22).
Pourquoi 28? Pour une certaine raison, le compilateur a décidé de stocker cette
variable à la fin de la structure de pile locale au lieu du début.
L'adresse est passée à \scanf, qui stocke l'entrée de l'utilisateur en mémoire à
cette adresse.
Il s'agit d'une valeur sur 32-bit de type \Tint.
La valeur est prise à la ligne 27 puis passée à \printf.

}
\JA{\subsubsection{ARM}

\myparagraph{\OptimizingKeilVI (\ThumbMode)}

\lstinputlisting[style=customasmARM]{patterns/04_scanf/1_simple/ARM_IDA.lst}

\myindex{\CLanguageElements!\Pointers}

\scanf がitemを読み込むためには、 \Tint へのparameter.pointerが必要です。 
\Tint は32ビットなので、メモリのどこかに格納するには4バイトが必要で、32ビットのレジスタに正確に収まります。 
\myindex{IDA!var\_?}
ローカル変数\GTT{x}の場所がスタックに割り当てられ、 
\IDA の名前は\emph{var\_8}です。 ただし、\ac{SP}(\gls{stack pointer})がすでにその領域を指しているため、その領域を直接割り当てることはできません。 

\INS{PUSH/POP}命令は、ARMとx86とでは動作が異なります(これは逆です)。 
これらは\INS{STM/STMDB/LDM/LDMIA}命令の同義語です。 
そして、\INS{PUSH}命令は最初に値をスタックに書き込み、\emph{次に} \ac{SP} を4で減算します。
\INS{POP}は最初に\ac{SP}に4を加算してから、スタックから値を読み取ります。 
したがって、\INS{PUSH}後、\ac{SP}はスタック内の未使用スペースを指します。 
それは \scanf によって、そして後に \printf によって使用されます。

\INS{LDMIA} は \emph{Load Multiple Registers Increment address After each transfer}の略です。
\INS{STMDB} は \emph{Store Multiple Registers Decrement address Before each transfer}の略です。

したがって、\ac{SP}の値は\Reg{1}レジスタにコピーされ、フォーマット文字列とともに \scanf に渡されます。 
その後、\INS{LDR}命令の助けを借りて、この値はスタックから\Reg{1}レジスタに移動され、 \printf に渡されます。

\myparagraph{ARM64}

\lstinputlisting[caption=\NonOptimizing GCC 4.9.1 ARM64,numbers=left,style=customasmARM]{patterns/04_scanf/1_simple/ARM64_GCC491_O0_JA.s}

スタックフレームには32バイトが割り当てられており、必要なサイズよりも大きくなっています。 たぶんメモリのアラインメントの問題でしょうか? 
最も興味深いのはスタックフレーム内の$x$変数のためのスペースを見つけることです(22行目)。 
なぜ28なのでしょう? 何らかの理由で、コンパイラは、この変数をスタックフレームの最後に置きます。 
アドレスは \scanf に渡され、\scanf はユーザ入力値をそのアドレスのメモリに格納するだけです。 
これは \Tint 型の32ビット値です。 
値は27行目から取得され、 \printf に渡されます。
}

\EN{\subsection{Win32 PE}
\label{win32_pe}
\myindex{Windows!Win32}

\acs{PE} is an executable file format used in Windows.
The difference between .exe, .dll and .sys is that .exe and .sys usually do not have exports, only imports.

\myindex{OEP}

A \ac{DLL}, just like any other PE-file, has an entry point (\ac{OEP}) (the function DllMain() is located there)
but this function usually does nothing.
.sys is usually a device driver.
As of drivers, Windows requires the checksum to be present in the PE file and for it to be correct
\footnote{For example, Hiew(\myref{Hiew}) can calculate it}.

\myindex{Windows!Windows Vista}
Starting at Windows Vista, a driver's files must also be signed with a digital signature. It will fail to load otherwise.

\myindex{MS-DOS}
Every PE file begins with tiny DOS program that prints a
message like \q{This program cannot be run in DOS mode.}---if you run this program in DOS or Windows 3.1 (\ac{OS}-es which are not aware of the PE format),
this message will be printed.

\subsubsection{Terminology}

\myindex{VA}
\myindex{Base address}
\myindex{RVA}
\myindex{Windows!IAT}
\myindex{Windows!INT}

\begin{itemize}
\item Module---a separate file, .exe or .dll.

\item Process---a program loaded into memory and currently running.  Commonly consists of one .exe file and bunch of .dll files.

\item Process memory---the memory a process works with.  Each process has its own.
There usually are loaded modules, memory of the stack, \gls{heap}(s), etc.

\item \ac{VA}---an address which is to be used in program while runtime.

\item Base address (of module)---the address within the process memory at which the module is to be loaded.
\ac{OS} loader may change it, if the base address is already occupied by another module just loaded before.

\item \ac{RVA}---the \ac{VA}-address minus the base address.

Many addresses in PE-file tables use \ac{RVA}-addresses.

%\item
%Data directory --- ...

\item \ac{IAT}---an array of addresses of imported symbols \footnote{\PietrekPE}.
Sometimes, the \TT{IMAGE\_DIRECTORY\_ENTRY\_IAT} data directory points at the \ac{IAT}.
\label{IDA_idata}
It is worth noting that \ac{IDA} (as of 6.1) may allocate a pseudo-section named \TT{.idata} for
\ac{IAT}, even if the \ac{IAT} is a part of another section!

\item \ac{INT}---an array of names of symbols to be imported\footnote{\PietrekPE}.
\end{itemize}

\subsubsection{Base address}

The problem is that several module authors can prepare DLL files for others to use and it is not possible
to reach an agreement which addresses is to be assigned to whose modules.

So that is why if two necessary DLLs for a process have the same base address,
one of them will be loaded at this base address, and the other---at some other free space in process memory,
and each virtual addresses in the second DLL will be corrected.

\par With \ac{MSVC} the linker often generates the .exe files with a base address of \TT{0x400000}
\footnote{The origin of this address choice is described here: \href{http://go.yurichev.com/17041}{MSDN}},
and with the code section starting at \TT{0x401000}.
This means that the \ac{RVA} of the start of the code section is \TT{0x1000}.

DLLs are often generated by MSVC's linker with a base address of \TT{0x10000000}
\footnote{This can be changed by the /BASE linker option}.

\myindex{ASLR}

There is also another reason to load modules at various base addresses, in this case random ones.
It is \ac{ASLR}\footnote{\href{http://go.yurichev.com/17140}{wikipedia}}.

\myindex{Shellcode}

A shellcode trying to get executed on a compromised system must call system functions, hence, know their addresses.

In older \ac{OS} (in \gls{Windows NT} line: before Windows Vista),
system DLL (like kernel32.dll, user32.dll) were always loaded at known addresses,
and if we also recall
that their versions rarely changed, the addresses of functions were
fixed and shellcode could call them directly.

In order to avoid this, the \ac{ASLR}
method loads your program and all modules it needs at random base addresses, different every time.

\ac{ASLR} support is denoted in a PE file by setting the flag
\par \TT{IMAGE\_DLL\_CHARACTERISTICS\_DYNAMIC\_BASE} \InSqBrackets{see \Russinovich}.

\subsubsection{Subsystem}

There is also a \emph{subsystem} field, usually it is:

\myindex{Native API}

\begin{itemize}
\item native\footnote{Meaning, the module use Native API instead of Win32} (.sys-driver),

\item console (console application) or

\item \ac{GUI} (non-console).
\end{itemize}

\subsubsection{OS version}

A PE file also specifies the minimal Windows version it needs in order to be loadable.

The table of version numbers stored in the PE file and corresponding Windows codenames is
here\footnote{\href{http://go.yurichev.com/17044}{wikipedia}}.

\myindex{Windows!Windows NT4}
\myindex{Windows!Windows 2000}
For example, \ac{MSVC} 2005 compiles .exe files for running on Windows NT4 (version 4.00), but \ac{MSVC} 2008 does not
(the generated files have a version of 5.00, at least Windows 2000 is needed to run them).

\myindex{Windows!Windows XP}

\ac{MSVC} 2012 generates .exe files of version 6.00 by default,
targeting at least Windows Vista.
However, by changing the compiler's options\footnote{\href{http://go.yurichev.com/17045}{MSDN}},
it is possible to force it to compile for Windows XP.

\subsubsection{Sections}

Division in sections, as it seems, is present in all executable file formats.

It is devised in order to separate code from data, and data---from constant data.

\begin{itemize}
\item Either the \emph{IMAGE\_SCN\_CNT\_CODE} or \emph{IMAGE\_SCN\_MEM\_EXECUTE} flags will be set on the code section---this is executable code.

\item On data section---\emph{IMAGE\_SCN\_CNT\_INITIALIZED\_DATA},\\
\emph{IMAGE\_SCN\_MEM\_READ} and \emph{IMAGE\_SCN\_MEM\_WRITE} flags.

\item On an empty section with uninitialized data---\\
\emph{IMAGE\_SCN\_CNT\_UNINITIALIZED\_DATA}, \emph{IMAGE\_SCN\_MEM\_READ} \\
        and \emph{IMAGE\_SCN\_MEM\_WRITE}.

\item On a constant data section (one that's protected from writing), the flags \\
\emph{IMAGE\_SCN\_CNT\_INITIALIZED\_DATA} and \emph{IMAGE\_SCN\_MEM\_READ} can be set, \\
but not \emph{IMAGE\_SCN\_MEM\_WRITE}.
A process going to crash if it tries to write to this section.
\end{itemize}

\myindex{TLS}
\myindex{BSS}
Each section in PE-file may have a name, however, it is not very important.
Often (but not always) the code section is named \TT{.text},
the data section---\TT{.data}, the constant data section --- \TT{.rdata} \emph{(readable data)}.
Other popular section names are:

\myindex{MIPS}
\begin{itemize}
\item \TT{.idata}---imports section.
\ac{IDA} may create a pseudo-section named like this: \myref{IDA_idata}.
\item \TT{.edata}---exports section (rare)
\item \TT{.pdata}---section holding all information about exceptions in Windows NT for MIPS, \ac{IA64} and x64: \myref{SEH_win64}
\item \TT{.reloc}---relocs section
\item \TT{.bss}---uninitialized data (\ac{BSS})
\item \TT{.tls}---thread local storage (\ac{TLS})
\item \TT{.rsrc}---resources
\item \TT{.CRT}---may present in binary files compiled by ancient MSVC versions
\end{itemize}

PE file packers/encryptors often garble section names or replace the names with their own.

\ac{MSVC} allows you to declare data in arbitrarily named section
\footnote{\href{http://go.yurichev.com/17047}{MSDN}}.

Some compilers and linkers can add a section with debugging symbols and
other debugging information (MinGW for instance).
\myindex{Windows!PDB}
However it is not so in latest versions of \ac{MSVC} (separate \gls{PDB} files are used there for this purpose).\\
\\
That is how a PE section is described in the file:

\begin{lstlisting}[style=customc]
typedef struct _IMAGE_SECTION_HEADER {
  BYTE  Name[IMAGE_SIZEOF_SHORT_NAME];
  union {
    DWORD PhysicalAddress;
    DWORD VirtualSize;
  } Misc;
  DWORD VirtualAddress;
  DWORD SizeOfRawData;
  DWORD PointerToRawData;
  DWORD PointerToRelocations;
  DWORD PointerToLinenumbers;
  WORD  NumberOfRelocations;
  WORD  NumberOfLinenumbers;
  DWORD Characteristics;
} IMAGE_SECTION_HEADER, *PIMAGE_SECTION_HEADER;
\end{lstlisting}
\footnote{\href{http://go.yurichev.com/17048}{MSDN}}

\myindex{Hiew}
A word about terminology: \emph{PointerToRawData} is called \q{Offset} in Hiew
and \emph{VirtualAddress} is called \q{RVA} there.

\subsubsection{Data section}

Data section in file can be smaller than in memory.
For example, some variables can be initialized, some are not.
Compiler and linker will collect them all into one section, but the first part of it is initialized and allocated in file,
while another is absent in file (of course, to make it smaller).
\emph{VirtualSize} will be equal to the size of section in memory, and \emph{SizeOfRawData} --- to
size of section in file.

IDA can show the border between initialized and not initialized parts like that:

\begin{lstlisting}[style=customasmx86]
...

.data:10017FFA                 db    0
.data:10017FFB                 db    0
.data:10017FFC                 db    0
.data:10017FFD                 db    0
.data:10017FFE                 db    0
.data:10017FFF                 db    0
.data:10018000                 db    ? ;
.data:10018001                 db    ? ;
.data:10018002                 db    ? ;
.data:10018003                 db    ? ;
.data:10018004                 db    ? ;
.data:10018005                 db    ? ;

...
\end{lstlisting}

\subsubsection{Relocations (relocs)}
\label{subsec:relocs}

\ac{AKA} FIXUP-s (at least in Hiew).

They are also present in almost all executable file formats
\footnote{Even in .exe files for MS-DOS}.
Exceptions are shared dynamic libraries compiled with \ac{PIC}, or any other \ac{PIC}-code.

What are they for?

Obviously, modules can be loaded on various base addresses, but how to deal with global variables, for example?
They must be accessed by address.  One solution is \PICcode{} (\myref{sec:PIC}).
But it is not always convenient.

That is why a relocations table is present.
There the addresses of points that must be corrected are enumerated,
in case of loading at a different base address.

% TODO тут бы пример с HIEW или objdump..
For example, there is a global variable at address \TT{0x410000} and this is how it is accessed:

\begin{lstlisting}[style=customasmx86]
A1 00 00 41 00         mov         eax,[000410000]
\end{lstlisting}

The base address of the module is \TT{0x400000}, the \ac{RVA} of the global variable is \TT{0x10000}.

If the module is loaded at base address \TT{0x500000}, the real address of the global variable must be \TT{0x510000}.

\myindex{x86!\Instructions!MOV}

As we can see, the address of variable is encoded in the instruction \TT{MOV}, after the byte \TT{0xA1}.

That is why the address of the 4 bytes after \TT{0xA1}, is written in the relocs table.

If the module is loaded at a different base address, the \ac{OS} loader enumerates all addresses in the table,

finds each 32-bit word the address points to, subtracts the original base address from it
(we get the \ac{RVA} here), and adds the new base address to it.

If a module is loaded at its original base address, nothing happens.

All global variables can be treated like that.

Relocs may have various types, however, in Windows for x86 processors, the type is usually \\
\emph{IMAGE\_REL\_BASED\_HIGHLOW}.

\myindex{Hiew}

By the way, relocs are darkened in Hiew, for example: \figref{fig:scanf_ex3_hiew_1}.

\myindex{\olly}
\olly underlines the places in memory to which relocs are to be applied, for example: \figref{fig:switch_lot_olly3}.

\subsubsection{Exports and imports}

\label{PE_exports_imports}
As we all know, any executable program must use the \ac{OS}'s services and other DLL-libraries somehow.

It can be said that functions from one module (usually DLL) must be connected somehow to the points of their
calls in other modules (.exe-file or another DLL).

For this, each DLL has an \q{exports} table, which consists of functions plus their addresses in a module.

And every .exe file or DLL has \q{imports}, a table of functions it needs for execution including
list of DLL filenames.

After loading the main .exe-file, the \ac{OS} loader processes imports table:
it loads the additional DLL-files, finds function names
among the DLL exports and writes their addresses down in the \ac{IAT} of the main .exe-module.

\myindex{Windows!Win32!Ordinal}

As we can see, during loading the loader must compare a lot of function names, but string comparison is not a very
fast procedure, so there is a support for \q{ordinals} or \q{hints},
which are function numbers stored in the table, instead of their names.

That is how they can be located faster when loading a DLL.
Ordinals are always present in the \q{export} table.

\myindex{MFC}
For example, a program using the \ac{MFC} library usually loads mfc*.dll by ordinals,
and in such programs there are no \ac{MFC} function names in \ac{INT}.

% TODO example!
When loading such programs in \IDA, it will ask for a path to the mfc*.dll files
in order to determine the function names.

If you don't tell \IDA the path to these DLLs, there will be \emph{mfc80\_123} instead of function names.

\myparagraph{Imports section}

Often a separate section is allocated for the imports table and everything related to it (with name like \TT{.idata}),
however, this is not a strict rule.

Imports are also a confusing subject because of the terminological mess. Let's try to collect all information in one place.

\begin{figure}[H]
\centering
\myincludegraphics{OS/PE/unnamed0.png}
\caption{
A scheme that unites all PE-file structures related to imports}
\end{figure}

The main structure is the array \emph{IMAGE\_IMPORT\_DESCRIPTOR}.
Each element for each DLL being imported.

Each element holds the \ac{RVA} address of the text string (DLL name) (\emph{Name}).

\emph{OriginalFirstThunk} is the \ac{RVA} address of the \ac{INT} table.
This is an array of \ac{RVA} addresses, each of which points to a text string with a function name.
Each string is prefixed by a 16-bit integer
(\q{hint})---\q{ordinal} of function.

While loading, if it is possible to find a function by ordinal,
then the strings comparison will not occur. The array is terminated by zero.

There is also a pointer to the \ac{IAT} table named \emph{FirstThunk}, it is just the \ac{RVA} address
of the place where the loader writes the addresses of the resolved functions.

The points where the loader writes addresses are marked by \IDA like this: \emph{\_\_imp\_CreateFileA}, etc.

There are at least two ways to use the addresses written by the loader.

\myindex{x86!\Instructions!CALL}
\begin{itemize}
\item The code will have instructions like \emph{call \_\_imp\_CreateFileA},
and since the field with the address of the imported function is a global variable in some sense,
the address of the \emph{call} instruction (plus 1 or 2) is to be added to the relocs table,
for the case when the module is loaded at a different base address.

But, obviously, this may enlarge relocs table significantly.

Because there are might be a lot of calls to imported functions in the module.

Furthermore, large relocs table slows down the process of loading modules.

\myindex{x86!\Instructions!JMP}
\myindex{thunk-functions}
\item For each imported function, there is only one jump allocated, using the \JMP instruction
plus a reloc to it.
Such points are also called \q{thunks}.

All calls to the imported functions are just \CALL instructions to the corresponding \q{thunk}.
In this case, additional relocs are not necessary because these \CALL{}-s
have relative addresses and do not need to be corrected.
\end{itemize}

These two methods can be combined.

Possible, the linker creates individual \q{thunk}s if there are too many calls to the function,
but not done by default. \\
\\
By the way, the array of function addresses to which FirstThunk is pointing is not necessary to be located in the \ac{IAT} section.
For example, the author of these lines once wrote the PE\_add\_import\footnote{\href{http://go.yurichev.com/17049}{yurichev.com}}
utility for adding imports to an existing .exe-file.

Some time earlier, in the previous versions of the utility,
at the place of the function you want to substitute with a call to another DLL,
my utility wrote the following code:

\begin{lstlisting}[style=customasmx86]
MOV EAX, [yourdll.dll!function]
JMP EAX
\end{lstlisting}

FirstThunk points to the first instruction. In other words, when loading yourdll.dll,
the loader writes the address of the \emph{function} function right in the code.

It also worth noting that a code section is usually write-protected, so my utility adds the \\
\emph{IMAGE\_SCN\_MEM\_WRITE}
flag for code section. Otherwise, the program to crash while loading with error code
5 (access denied). \\
\\
One might ask: what if I supply a program with a set of DLL files which is not supposed to change (including addresses of all DLL functions),
is it possible to speed up the loading process?

Yes, it is possible to write the addresses of the functions to be imported into the FirstThunk arrays in advance.
The \emph{Timestamp} field is present in the \\
\emph{IMAGE\_IMPORT\_DESCRIPTOR} structure.

If a value is present there, then the loader compares this value with the date-time of the DLL file.

If the values are equal, then the loader does not do anything, and the loading of the process can be faster.
This is called \q{old-style binding}
\footnote{\href{http://go.yurichev.com/17050}{MSDN}. There is also the \q{new-style binding}.}.
\myindex{BIND.EXE}

The BIND.EXE utility in Windows SDK is for this.
For speeding up the loading of your program, Matt Pietrek in \PietrekPEURL, suggests to do the binding shortly after your program
installation on the computer of the end user. \\
\\
PE-files packers/encryptors may also compress/encrypt imports table.

In this case, the Windows loader, of course, will not load all necessary DLLs.
\myindex{Windows!Win32!LoadLibrary}
\myindex{Windows!Win32!GetProcAddress}

Therefore, the packer/encryptor does this on its own, with the help of
\emph{LoadLibrary()} and the \emph{GetProcAddress()} functions.

That is why these two functions are often present in \ac{IAT} in packed files.\\
\\
In the standard DLLs from the Windows installation, \ac{IAT} often is located right at the beginning of the PE file.
Supposedly, it is made so for optimization.

While loading, the .exe file is not loaded into memory as a whole (recall huge install programs which are
started suspiciously fast), it is \q{mapped}, and loaded into memory in parts as they are accessed.

Probably, Microsoft developers decided it will be faster.

\subsubsection{Resources}

\label{PEresources}

Resources in a PE file are just a set of icons, pictures, text strings, dialog descriptions.

Perhaps they were separated from the main code, so all these things could be multilingual,
and it would be simpler to pick text or picture for the language that is currently set in the \ac{OS}. \\
\\
As a side effect, they can be edited easily and saved back to the executable file, even if one does not have special knowledge,
by using the ResHack editor, for example (\myref{ResHack}).

\subsubsection{.NET}

\myindex{.NET}

.NET programs are not compiled into machine code but into a special bytecode.
\myindex{OEP}
Strictly speaking, there is bytecode instead of the usual x86 code
in the .exe file, however, the entry point (\ac{OEP}) points to this tiny fragment of x86 code:

\begin{lstlisting}[style=customasmx86]
jmp         mscoree.dll!_CorExeMain
\end{lstlisting}

The .NET loader is located in mscoree.dll, which processes the PE file.
\myindex{Windows!Windows XP}

It was so in all pre-Windows XP \ac{OS}es. Starting from XP, the \ac{OS} loader is able to detect the .NET file
and run it without executing that \JMP instruction
\footnote{\href{http://go.yurichev.com/17051}{MSDN}}.

\myindex{TLS}
\subsubsection{TLS}

This section holds initialized data for the \ac{TLS}(\myref{TLS}) (if needed).
When a new thread start, its \ac{TLS} data is initialized using the data from this section. \\
\\
\myindex{TLS!Callbacks}

Aside from that, the PE file specification also provides initialization of the
\ac{TLS} section, the so-called TLS callbacks.

If they are present, they are to be called before the control is passed to the main entry point (\ac{OEP}).

This is used widely in the PE file packers/encryptors.

\subsubsection{Tools}

\myindex{objdump}
\myindex{Cygwin}
\myindex{Hiew}
\label{ResHack}

\begin{itemize}
\item objdump (present in cygwin) for dumping all PE-file structures.

\item Hiew(\myref{Hiew}) as editor.

\item pefile---Python-library for PE-file processing \footnote{\url{http://go.yurichev.com/17052}}.

\item ResHack \acs{AKA} Resource Hacker---resources editor\footnote{\url{http://go.yurichev.com/17052}}.

\item PE\_add\_import\footnote{\url{http://go.yurichev.com/17049}}---
simple tool for adding symbol(s) to PE executable import table.

\item PE\_patcher\footnote{\href{http://go.yurichev.com/17054}{yurichev.com}}---simple tool for patching PE executables.

\item PE\_search\_str\_refs\footnote{\href{http://go.yurichev.com/17055}{yurichev.com}}---simple tool for searching for a function in PE executables which use some text string.
\end{itemize}

\subsubsection{Further reading}

% FIXME: bibliography per chapter or section
\begin{itemize}
\item Daniel Pistelli---The .NET File Format \footnote{\url{http://go.yurichev.com/17056}}
\end{itemize}

}
\RU{\mysection{Пример вычисления адреса сети}

Как мы знаем, TCP/IP-адрес (IPv4) состоит из четырех чисел в пределах $0 \ldots 255$, т.е. 4 байта.

4 байта легко помещаются в 32-битную переменную, так что адрес хоста в IPv4, сетевая маска или адрес сети
могут быть 32-битными числами.

С точки зрения пользователя, маска сети определяется четырьмя числами в формате вроде \\
255.255.255.0,
но сетевые инженеры (сисадмины) используют более компактную нотацию (\ac{CIDR}),
вроде  \q{/8}, \q{/16}, итд.

Эта нотация просто определяет количество бит в сетевой маске, начиная с \ac{MSB}.

\small
\begin{center}
\begin{tabular}{ | l | l | l | l | l | l | }
\hline
\HeaderColor Маска & 
\HeaderColor Хосты & 
\HeaderColor Свободно &
\HeaderColor Сетевая маска &
\HeaderColor В шест.виде &
\HeaderColor \\
\hline
/30  & 4        & 2        & 255.255.255.252  & 0xfffffffc  & \\
\hline
/29  & 8        & 6        & 255.255.255.248  & 0xfffffff8  & \\
\hline
/28  & 16       & 14       & 255.255.255.240  & 0xfffffff0  & \\
\hline
/27  & 32       & 30       & 255.255.255.224  & 0xffffffe0  & \\
\hline
/26  & 64       & 62       & 255.255.255.192  & 0xffffffc0  & \\
\hline
/24  & 256      & 254      & 255.255.255.0    & 0xffffff00  & сеть класса C \\
\hline
/23  & 512      & 510      & 255.255.254.0    & 0xfffffe00  & \\
\hline
/22  & 1024     & 1022     & 255.255.252.0    & 0xfffffc00  & \\
\hline
/21  & 2048     & 2046     & 255.255.248.0    & 0xfffff800  & \\
\hline
/20  & 4096     & 4094     & 255.255.240.0    & 0xfffff000  & \\
\hline
/19  & 8192     & 8190     & 255.255.224.0    & 0xffffe000  & \\
\hline
/18  & 16384    & 16382    & 255.255.192.0    & 0xffffc000  & \\
\hline
/17  & 32768    & 32766    & 255.255.128.0    & 0xffff8000  & \\
\hline
/16  & 65536    & 65534    & 255.255.0.0      & 0xffff0000  & сеть класса B \\
\hline
/8   & 16777216 & 16777214 & 255.0.0.0        & 0xff000000  & сеть класса A \\
\hline
\end{tabular}
\end{center}
\normalsize

Вот простой пример, вычисляющий адрес сети используя сетевую маску и адрес хоста.

\lstinputlisting[style=customc]{\CURPATH/netmask.c}

\subsection{calc\_network\_address()}

Функция \TT{calc\_network\_address()} самая простая: 

она просто умножает (логически, используя \AND) адрес хоста на сетевую маску, в итоге давая адрес
сети.

\lstinputlisting[caption=\Optimizing MSVC 2012 /Ob0,numbers=left,style=customasmx86]{\CURPATH/calc_network_address_MSVC_2012_Ox.asm}

На строке 22 мы видим самую важную инструкцию \AND --- так вычисляется адрес сети.

\subsection{form\_IP()}

Функция \TT{form\_IP()} просто собирает все 4 байта в одно 32-битное значение.

Вот как это обычно происходит:

\begin{itemize}
\item Выделите переменную для возвращаемого значения. Обнулите её.

\item 
Возьмите четвертый (самый младший) байт, сложите его (логически, инструкцией \OR) с возвращаемым
значением. Оно содержит теперь 4-й байт.

\item Возьмите третий байт, сдвиньте его на 8 бит влево.
Получится значение в виде \TT{0x0000bb00}, где \TT{bb} это третий байт.
Сложите итоговое значение (логически, инструкцией \OR) с возвращаемым значением.
Возвращаемое значение пока что содержит \TT{0x000000aa}, так что логическое сложение
в итоге выдаст значение вида \TT{0x0000bbaa}.

\item 
Возьмите второй байт, сдвиньте его на 16 бит влево.
Вы получите значение вида \TT{0x00cc0000}, где \TT{cc} это второй байт.
Сложите (логически) результат и возвращаемое значение.
Выходное значение содержит пока что \TT{0x0000bbaa}, так что логическое сложение
в итоге выдаст значение вида \TT{0x00ccbbaa}.

\item 
Возьмите первый байт, сдвиньте его на 24 бита влево.
Вы получите значение вида \TT{0xdd000000}, где \TT{dd} это первый байт.
Сложите (логически) результат и выходное значение.
Выходное значение содержит пока что \TT{0x00ccbbaa}, так что сложение выдаст в итоге значение
вида \TT{0xddccbbaa}.

\end{itemize}

И вот как работает неоптимизирующий MSVC 2012:

\lstinputlisting[caption=\NonOptimizing MSVC 2012,style=customasmx86]{\CURPATH/form_IP_MSVC_2012_RU.asm}

Хотя, порядок операций другой, но, конечно, порядок роли не играет.

\Optimizing MSVC 2012 делает то же самое, но немного иначе:

\lstinputlisting[caption=\Optimizing MSVC 2012 /Ob0,style=customasmx86]{\CURPATH/form_IP_MSVC_2012_Ox_RU.asm}

Можно сказать, что каждый байт записывается в младшие 8 бит возвращаемого значения,
и затем возвращаемое значение сдвигается на один байт влево на каждом шаге.

Повторять 4 раза, для каждого байта.

\par
Вот и всё! 
К сожалению, наверное, нет способа делать это иначе.
Не существует более-менее популярных \ac{CPU} или \ac{ISA}, где имеется инструкция для сборки значения из бит или байт.
Обычно всё это делает сдвигами бит и логическим сложением (OR).

\subsection{print\_as\_IP()}

\TT{print\_as\_IP()} делает наоборот: расщепляет 32-битное значение на 4 байта.

Расщепление работает немного проще: просто сдвигайте входное значение на 24, 16, 8 или 0 бит,
берите биты с нулевого по седьмой (младший байт), вот и всё:

\lstinputlisting[caption=\NonOptimizing MSVC 2012,style=customasmx86]{\CURPATH/print_as_IP_MSVC_2012_RU.asm}

\Optimizing MSVC 2012 делает почти всё то же самое, только без ненужных перезагрузок входного значения:

\lstinputlisting[caption=\Optimizing MSVC 2012 /Ob0,style=customasmx86]{\CURPATH/print_as_IP_MSVC_2012_Ox.asm}

\subsection{form\_netmask() и set\_bit()}

\TT{form\_netmask()} делает сетевую маску из \ac{CIDR}-нотации.

Конечно, было бы куда эффективнее использовать здесь какую-то уже готовую таблицу, но мы рассматриваем
это именно так, сознательно, для демонстрации битовых сдвигов.
Мы также сделаем отдельную функцию \TT{set\_bit()}. 

Не очень хорошая идея выделять отдельную функцию для такой примитивной операции, но так будет проще понять,
как это всё работает.

\lstinputlisting[caption=\Optimizing MSVC 2012 /Ob0,style=customasmx86]{\CURPATH/form_netmask_MSVC_2012_Ox.asm}

\TT{set\_bit()} примитивна: просто сдвигает единицу на нужное количество бит, затем складывает (логически) с
входным значением \q{input}.
\TT{form\_netmask()} имеет цикл: он выставит столько бит (начиная с \ac{MSB}), 
сколько передано в аргументе \TT{netmask\_bits}.

\subsection{Итог}

Вот и всё!
Мы запускаем и видим:

\begin{lstlisting}
netmask=255.255.255.0
network address=10.1.2.0
netmask=255.0.0.0
network address=10.0.0.0
netmask=255.255.255.128
network address=10.1.2.0
netmask=255.255.255.192
network address=10.1.2.64
\end{lstlisting}
}
\IT{\subsubsection{MIPS}

Nello stack locale viene allocato spazio per la variabile $x$ , a cui viene fatto riferimento come $\$sp+24$.
\myindex{MIPS!\Instructions!LW}

Il suo indirizzo è passato a \scanf, il valore immesso dall'utente è caricato usand l'istruzione \INS{LW} (\q{Load Word}) ed è infine passato a \printf.

\lstinputlisting[caption=\Optimizing GCC 4.4.5 (\assemblyOutput),style=customasmMIPS]{patterns/04_scanf/1_simple/MIPS/ex1.O3_EN.s}

IDA mostra il layout dello stack nel modo seguente:

\lstinputlisting[caption=\Optimizing GCC 4.4.5 (IDA),style=customasmMIPS]{patterns/04_scanf/1_simple/MIPS/ex1.O3.IDA_EN.lst}

% TODO non-optimized version?
}
\DE{\mysection{\Stack}
\label{sec:stack}
\myindex{\Stack}

Der Stack ist eine der fundamentalen Datenstrukturen in der Informatik.
\footnote{\href{http://go.yurichev.com/17119}{wikipedia.org/wiki/Call\_Stack}}.
\ac{AKA} \ac{LIFO}.

Technisch betrachtet ist es ein Stapelspeicher innerhalb des Prozessspeichers der zusammen mit den \ESP (x86), \RSP (x64) oder dem \ac{SP} (ARM) Register als ein Zeiger in diesem Speicherblock fungiert.

\myindex{ARM!\Instructions!PUSH}
\myindex{ARM!\Instructions!POP}
\myindex{x86!\Instructions!PUSH}
\myindex{x86!\Instructions!POP}

Die häufigsten Stack-Zugriffsinstruktionen sind die \PUSH- und \POP-Instruktionen (in beidem x86 und ARM Thumb-Modus). \PUSH subtrahiert vom \ESP/\RSP/\ac{SP} 4 Byte im 32-Bit Modus (oder 8 im 64-Bit Modus) und schreibt dann den Inhalt des Zeigers an die Adresse auf die von \ESP/\RSP/\ac{SP} gezeigt wird.

\POP ist die umgekehrte Operation: Die Daten des Zeigers für die Speicherregion auf die von \ac{SP}
gezeigt wird werden ausgelesen und die Inhalte in den Instruktionsoperanden geschreiben (oft ist das ein Register). Dann werden 4 (beziehungsweise 8) Byte zum \gls{stack pointer} addiert.

Nach der Stackallokation, zeigt der \gls{stack pointer} auf den Boden des Stacks.
\PUSH verringert den \gls{stack pointer} und \POP erhöht ihn.
Der Boden des Stacks ist eigentlich der Anfang der Speicherregion die für den Stack reserviert wurde.
Das wirkt zunächst seltsam, aber so funktioniert es.

ARM unterstützt beides, aufsteigende und absteigende Stacks.

\myindex{ARM!\Instructions!STMFD}
\myindex{ARM!\Instructions!LDMFD}
\myindex{ARM!\Instructions!STMED}
\myindex{ARM!\Instructions!LDMED}
\myindex{ARM!\Instructions!STMFA}
\myindex{ARM!\Instructions!LDMFA}
\myindex{ARM!\Instructions!STMEA}
\myindex{ARM!\Instructions!LDMEA}

Zum Beispiel die \ac{STMFD}/\ac{LDMFD} und \ac{STMED}/\ac{LDMED} Instruktionen sind alle dafür gedacht mit einem absteigendem Stack zu arbeiten ( wächst nach unten, fängt mit hohen Adressen an und entwickelt sich zu niedrigeren Adressen). Die \ac{STMFA}/\ac{LDMFA} und \ac{STMEA}/\ac{LDMEA} Instruktionen sind dazu gedacht mit einem aufsteigendem Stack zu arbeiten (wächst nach oben und fängt mit niedrigeren Adressen an und wächst nach oben).

% It might be worth mentioning that STMED and STMEA write first,
% and then move the pointer, and that LDMED and LDMEA move the pointer first, and then read.
% In other words, ARM not only lets the stack grow in a non-standard direction,
% but also in a non-standard order.
% Maybe this can be in the glossary, which would explain why E stands for "empty".

\subsection{Warum wächst der Stack nach unten?}
\label{stack_grow_backwards}

Intuitiv, würden man annehmen das der Stack nach oben wächst z.B Richtung höherer Adressen, so wie bei jeder anderen Datenstruktur.

Der Grund das der Stack rückwärts wächst ist wohl historisch bedingt. Als Computer so groß waren das sie einen ganzen Raum beansprucht haben war es einfach Speicher in zwei Sektionen zu unterteilen, einen Teil für den \gls{heap} und einen Teil für den Stack. Sicher war zu dieser Zeit nicht bekannt wie groß der \gls{heap} und der Stack wachsen würden, während der Programm Laufzeit, also war die Lösung die einfachste mögliche.

\input{patterns/02_stack/stack_and_heap}

In \RitchieThompsonUNIX können wir folgendes lesen:

\begin{framed}
\begin{quotation}
Der user-core eines Programm Images wird in drei logische Segmente unterteilt. Das Programm-Text Segment beginnt bei 0 im virtuellen Adress Speicher. Während der Ausführung wird das Segment als schreibgeschützt markiert und eine einzelne Kopie des Segments wird unter allen Prozessen geteilt die das Programm ausführen. An der ersten 8K grenze über dem Programm Text Segment im Virtuellen Speicher, fängt der ``nonshared'' Bereich an, der nach Bedarf von Syscalls erweitert werden kann. Beginnend bei der höchsten Adresse im Virtuellen Speicher ist das Stack Segment, das Automatisch nach unten wächst während der Hardware Stackpointer sich ändert.
\end{quotation}
\end{framed}

Das erinnert daran wie manche Schüler Notizen zu  zwei Vorträgen in einem Notebook dokumentieren:
Notizen für den ersten Vortrag werden normal notiert, und Notizen zur zum zweiten Vortrag werden 
ans Ende des Notizbuches geschrieben, indem man das Notizbuch umdreht. Die Notizen treffen sich irgendwann
im Notizbuch aufgrund des fehlenden Freien Platzes.

% I think if we want to expand on this analogy,
% one might remember that the line number increases as as you go down a page.
% So when you decrease the address when pushing to the stack, visually,
% the stack does grow upwards.
% Of course, the problem is that in most human languages,
% just as with computers,
% we write downwards, so this direction is what makes buffer overflows so messy.

\subsection{Für was wird der Stack benutzt?}

% subsections
\EN{\input{patterns/02_stack/01_saving_ret_addr_EN}}
\RU{\input{patterns/02_stack/01_saving_ret_addr_RU}}
\DE{\input{patterns/02_stack/01_saving_ret_addr_DE}}
\FR{\input{patterns/02_stack/01_saving_ret_addr_FR}}
\PTBR{\input{patterns/02_stack/01_saving_ret_addr_PTBR}}
\IT{\input{patterns/02_stack/01_saving_ret_addr_IT}}
\PL{\input{patterns/02_stack/01_saving_ret_addr_PL}}
\JA{\input{patterns/02_stack/01_saving_ret_addr_JA}}

\EN{\input{patterns/02_stack/02_args_passing_EN}}
\RU{\input{patterns/02_stack/02_args_passing_RU}}
\PTBR{\input{patterns/02_stack/02_args_passing_PTBR}}
\DE{\input{patterns/02_stack/02_args_passing_DE}}
\IT{\input{patterns/02_stack/02_args_passing_IT}}
\FR{\input{patterns/02_stack/02_args_passing_FR}}
\JA{\input{patterns/02_stack/02_args_passing_JA}}
\PL{\input{patterns/02_stack/02_args_passing_PL}}


\EN{\input{patterns/02_stack/03_local_vars_EN}}
\RU{\input{patterns/02_stack/03_local_vars_RU}}
\DE{\input{patterns/02_stack/03_local_vars_DE}}
\PTBR{\input{patterns/02_stack/03_local_vars_PTBR}}
\EN{\input{patterns/02_stack/04_alloca/main_EN}}
\FR{\input{patterns/02_stack/04_alloca/main_FR}}
\RU{\input{patterns/02_stack/04_alloca/main_RU}}
\PTBR{\input{patterns/02_stack/04_alloca/main_PTBR}}
\IT{\input{patterns/02_stack/04_alloca/main_IT}}
\DE{\input{patterns/02_stack/04_alloca/main_DE}}
\PL{\input{patterns/02_stack/04_alloca/main_PL}}
\JA{\input{patterns/02_stack/04_alloca/main_JA}}

\subsubsection{(Windows) SEH}
\myindex{Windows!Structured Exception Handling}

\ifdefined\RUSSIAN
В стеке хранятся записи \ac{SEH} для функции (если они присутствуют).
Читайте больше о нем здесь: (\myref{sec:SEH}).
\fi % RUSSIAN

\ifdefined\ENGLISH
\ac{SEH} records are also stored on the stack (if they are present).
Read more about it: (\myref{sec:SEH}).
\fi % ENGLISH

\ifdefined\BRAZILIAN
\ac{SEH} também são guardados na pilha (se estiverem presentes).
\PTBRph{}: (\myref{sec:SEH}).
\fi % BRAZILIAN

\ifdefined\ITALIAN
I record \ac{SEH}, se presenti, sono anch'essi memorizzati nello stack.
Maggiori informazioni qui: (\myref{sec:SEH}).
\fi % ITALIAN

\ifdefined\FRENCH
Les enregistrements \ac{SEH} sont aussi stockés dans la pile (s'ils sont présents).
Lire à ce propos: (\myref{sec:SEH}).
\fi % FRENCH


\ifdefined\POLISH
Na stosie są przechowywane wpisy \ac{SEH} dla funkcji (jeśli są one obecne).
Więcej o tym tutaj: (\myref{sec:SEH}).
\fi % POLISH

\ifdefined\JAPANESE
\ac{SEH}レコードはスタックにも格納されます(存在する場合)。
それについてもっと読む:(\myref{sec:SEH})
\fi % JAPANESE

\ifdefined\ENGLISH
\subsubsection{Buffer overflow protection}

More about it here~(\myref{subsec:bufferoverflow}).
\fi

\ifdefined\RUSSIAN
\subsubsection{Защита от переполнений буфера}

Здесь больше об этом~(\myref{subsec:bufferoverflow}).
\fi

\ifdefined\BRAZILIAN
\subsubsection{Proteção contra estouro de buffer}

Mais sobre aqui~(\myref{subsec:bufferoverflow}).
\fi

\ifdefined\ITALIAN
\subsubsection{Protezione da buffer overflow}

Maggiori informazioni qui~(\myref{subsec:bufferoverflow}).
\fi

\ifdefined\FRENCH
\subsubsection{Protection contre les débordements de tampon}

Lire à ce propos~(\myref{subsec:bufferoverflow}).
\fi


\ifdefined\POLISH
\subsubsection{Metody zabiezpieczenia przed przepełnieniem stosu}

Więcej o tym tutaj~(\myref{subsec:bufferoverflow}).
\fi

\ifdefined\JAPANESE
\subsubsection{バッファオーバーフロー保護}

詳細はこちら~(\myref{subsec:bufferoverflow})
\fi

\subsubsection{Automatisches deallokieren der Daten auf dem Stack}

Vielleicht ist der Grund warum man lokale Variablen und SEH Einträge auf dem Stack speichert, weil sie beim 
verlassen der Funktion automatisch aufgeräumt werden. Man braucht dabei nur eine Instruktion um die Position
des Stackpointers zu korrigieren (oftmals ist es die \ADD Instruktion). Funktions Argumente, könnte man sagen 
werden auch am Ende der Funktion deallokiert. Im Kontrast dazu, alles was auf dem \emph{heap} gespeichert wird muss
explizit deallokiert werden. 

% sections
\EN{\input{patterns/02_stack/07_layout_EN}}
\RU{\input{patterns/02_stack/07_layout_RU}}
\DE{\input{patterns/02_stack/07_layout_DE}}
\PTBR{\input{patterns/02_stack/07_layout_PTBR}}
\EN{\input{patterns/02_stack/08_noise/main_EN}}
\FR{\input{patterns/02_stack/08_noise/main_FR}}
\RU{\input{patterns/02_stack/08_noise/main_RU}}
\IT{\input{patterns/02_stack/08_noise/main_IT}}
\DE{\input{patterns/02_stack/08_noise/main_DE}}
\PL{\input{patterns/02_stack/08_noise/main_PL}}
\JA{\input{patterns/02_stack/08_noise/main_JA}}

\input{patterns/02_stack/exercises}
}
\FR{\subsubsection{MIPS}

Une place est allouée sur la pile locale pour la variable $x$, et elle doit être appelée par $\$sp+24$.
\myindex{MIPS!\Instructions!LW}

Son adresse est passée à \scanf, et la valeur entrée par l'utilisateur est chargée en utilisant
l'instruction \INS{LW} (\q{Load Word}), puis passée à \printf.

\lstinputlisting[caption=GCC 4.4.5 \Optimizing (\assemblyOutput),style=customasmMIPS]{patterns/04_scanf/1_simple/MIPS/ex1.O3_FR.s}

IDA affiche la disposition de la pile comme suit:

\lstinputlisting[caption=GCC 4.4.5 \Optimizing (IDA),style=customasmMIPS]{patterns/04_scanf/1_simple/MIPS/ex1.O3.IDA_FR.lst}

% TODO non-optimized version?
}
\JA{\subsubsection{MIPS}

ローカルスタック内の場所は$x$変数に割り当てられ、$\$sp+24$ と呼ばれます。
\myindex{MIPS!\Instructions!LW}

そのアドレスは \scanf に渡され、ユーザー入力値は\INC{LW}(\q{Load Word})を使用してロードされます。
そしてそれから \printf に渡されます。

\lstinputlisting[caption=\Optimizing GCC 4.4.5 (\assemblyOutput),style=customasmMIPS]{patterns/04_scanf/1_simple/MIPS/ex1.O3_JA.s}

IDAはスタックレイアウトを次のように表示します。

\lstinputlisting[caption=\Optimizing GCC 4.4.5 (IDA),style=customasmMIPS]{patterns/04_scanf/1_simple/MIPS/ex1.O3.IDA_JA.lst}

% TODO non-optimized version?
}

