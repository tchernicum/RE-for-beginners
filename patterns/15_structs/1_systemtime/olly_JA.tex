\clearpage
\subsubsection{\olly}
\myindex{\olly}

この例を\TT{/GS- /MD}オプション付きでMSVC 2010でコンパイルし \olly で実行してみましょう。

データのウィンドウを開き、\TT{GetSystemTime()}関数の最初の引数として渡されたアドレスにスタックし、
実行されるまで待機しましょう。 このようになります。

\begin{figure}[H]
\centering
\myincludegraphics{patterns/15_structs/1_systemtime/olly_systemtime1.png}
\caption{\olly: \TT{GetSystemTime()} が実行された}
\label{fig:struct_olly_1}
\end{figure}

私のコンピュータ上での関数のシステム時間は2014年12月9日、22時29分52秒です。

\lstinputlisting[caption=\printf output]{patterns/15_structs/1_systemtime/console.txt}

このような16バイトをデータウィンドウに
みることができます。
\begin{lstlisting}
DE 07 0C 00 02 00 09 00 16 00 1D 00 34 00 D4 03
\end{lstlisting}

各2バイトが構造体のフィールドを表します。
\gls{endianness}は\emph{リトルエンディアン}なので、
低位バイトが最初に見え、高位バイトがその後です。

したがって、これらは現在メモリに格納されている値です。

\begin{center}
\begin{tabular}{ | l | l | l | }
\hline
\headercolor{} 16進数 & 
\headercolor{} 10進数 & 
\headercolor{} フィールド名 \\
\hline
0x07DE & 2014	& wYear \\
\hline
0x000C & 12	& wMonth \\
\hline
0x0002 & 2	& wDayOfWeek \\
\hline
0x0009 & 9	& wDay \\
\hline
0x0016 & 22	& wHour \\
\hline
0x001D & 29	& wMinute \\
\hline
0x0034 & 52	& wSecond \\
\hline	
0x03D4 & 980	& wMilliseconds \\
\hline
\end{tabular}
\end{center}

同じ値がスタックウィンドウに表示されますが、32ビットの値としてグループ分けされています。

そして、 \printf は必要な値だけを取り出してコンソールに出力します。

いくつかの値は \printf (\TT{wDayOfWeek} と \TT{wMilliseconds})
によって出力されませんが、使用可能なメモリ上にあります。
