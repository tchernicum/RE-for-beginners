\subsubsection{x86}

Компилируется это все в:

\lstinputlisting[caption=MSVC 2012 /GS- /Ob0,label=src:struct_packing_4,numbers=left,style=customasmx86]{patterns/15_structs/4_packing/packing_RU.asm}

Кстати, мы передаем всю структуру, но в реальности, как видно, структура в начале копируется
во временную структуру (выделение места под нее в стеке происходит в строке 10,
а все 4 поля, по одному, копируются в строках 12 \ldots\ 19), 
затем передается только указатель на нее (или адрес).

Структура копируется, потому что неизвестно, будет ли функция \ttf модифицировать структуру или нет.
И если да, то структура внутри \main должна остаться той же.

Мы могли бы использовать указатели на \CCpp, и итоговый код был бы почти такой же,
только копирования не было бы.

Мы видим здесь что адрес каждого поля в структуре выравнивается по 4-байтной границе. 
Так что каждый \Tchar здесь занимает те же 4 байта что и \Tint. Зачем? 
Затем что процессору удобнее обращаться по таким адресам и кэшировать данные из памяти.

Но это не экономично по размеру данных.

Попробуем скомпилировать тот же исходник с опцией (\TT{/Zp1}) 
(\emph{/Zp[n] pack structures on n-byte boundary}).

\lstinputlisting[caption=MSVC 2012 /GS- /Zp1,label=src:struct_packing_1,numbers=left,style=customasmx86]{patterns/15_structs/4_packing/packing_msvc_Zp1_RU.asm}

Теперь структура занимает 10 байт и все \Tchar занимают по байту. Что это дает? 
Экономию места. Недостаток ~--- процессор будет обращаться к этим полям не так эффективно 
по скорости, как мог бы.

\label{short_struct_copying_using_MOV}
Структура так же копируется в \main. Но не по одному полю, а 10 байт, при помощи трех
пар \MOV.

Почему не 4?
Компилятор рассудил, что будет лучше скопировать 10 байт
при помощи 3 пар \MOV, чем копировать два 32-битных слова и два байта при помощи 4 пар \MOV.

Кстати, подобная реализация копирования при помощи \MOV взамен вызова функции \TT{memcpy()}, например, это
очень распространенная практика, потому что это в любом случае работает быстрее чем вызов \TT{memcpy()} ---
если речь идет о коротких блоках, конечно: \myref{copying_short_blocks}.

Как нетрудно догадаться, если структура используется много в каких исходниках и объектных файлах, 
все они должны быть откомпилированы с одним и тем же соглашением об упаковке структур.

\newcommand{\FNURLMSDNZP}{\footnote{\href{http://go.yurichev.com/17067}
{MSDN: Working with Packing Structures}}}
\newcommand{\FNURLGCCPC}{\footnote{\href{http://go.yurichev.com/17068}
{Structure-Packing Pragmas}}}

Помимо ключа MSVC \TT{/Zp}, указывающего, по какой границе упаковывать поля структур, есть также 
опция компилятора \TT{\#pragma pack}, её можно указывать прямо в исходнике. 
Это справедливо и для MSVC\FNURLMSDNZP и GCC\FNURLGCCPC{}.

Давайте теперь вернемся к \TT{SYSTEMTIME}, которая состоит из 16-битных полей. 
Откуда наш компилятор знает что их надо паковать по однобайтной границе?

В файле \TT{WinNT.h} попадается такое:

\begin{lstlisting}[caption=WinNT.h,style=customc]
#include "pshpack1.h"
\end{lstlisting}

И такое:

\begin{lstlisting}[caption=WinNT.h,style=customc]
#include "pshpack4.h"                   // 4 byte packing is the default
\end{lstlisting}

Сам файл PshPack1.h выглядит так:

\lstinputlisting[caption=PshPack1.h,style=customc]{patterns/15_structs/4_packing/tmp1.c}

Собственно, так и задается компилятору, как паковать объявленные после \TT{\#pragma pack} структуры.

\clearpage
\subsubsection{\olly}
\myindex{\olly}

Компилируем этот пример в MSVC 2010 с ключами \TT{/GS- /MD} и запускаем в \olly.
Открываем окна данных и стека по адресу, который передается в качестве первого аргумента в функцию \TT{GetSystemTime()}, 
ждем пока эта функция исполнится, и видим следующее:

\begin{figure}[H]
\centering
\myincludegraphics{patterns/15_structs/1_systemtime/olly_systemtime1.png}
\caption{\olly: \TT{GetSystemTime()} только что исполнилась}
\label{fig:struct_olly_1}
\end{figure}

Точное системное время на моем компьютере, в которое исполнилась функция, это 9 декабря 2014, 22:29:52:

\lstinputlisting[caption=Вывод \printf]{patterns/15_structs/1_systemtime/console.txt}

Таким образом, в окне данных мы видим следующие 16 байт: 
\begin{lstlisting}
DE 07 0C 00 02 00 09 00 16 00 1D 00 34 00 D4 03
\end{lstlisting}

Каждые два байта отражают одно поле структуры. 
А так как порядок байт (\gls{endianness}) \emph{little endian},
то в начале следует младший байт, затем старший.
Следовательно, вот какие 16-битные числа сейчас записаны в памяти:

\begin{center}
\begin{tabular}{ | l | l | l | }
\hline
\headercolor{} Шестнадцатеричное число & 
\headercolor{} десятичное число & 
\headercolor{} имя поля \\
\hline
0x07DE & 2014	& wYear \\
\hline
0x000C & 12	& wMonth \\
\hline
0x0002 & 2	& wDayOfWeek \\
\hline
0x0009 & 9	& wDay \\
\hline
0x0016 & 22	& wHour \\
\hline
0x001D & 29	& wMinute \\
\hline
0x0034 & 52	& wSecond \\
\hline	
0x03D4 & 980	& wMilliseconds \\
\hline
\end{tabular}
\end{center}

В окне стека, видны те же значения, только они сгруппированы как 32-битные значения.

Затем \printf просто берет нужные значения и выводит их на консоль.

Некоторые поля \printf не выводит (\TT{wDayOfWeek} и
\TT{wMilliseconds}), но они находятся в памяти и доступны для использования.


