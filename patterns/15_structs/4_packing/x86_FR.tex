\subsubsection{x86}

Le résultat de la compilation est:

\lstinputlisting[caption=MSVC 2012 /GS- /Ob0,label=src:struct_packing_4,numbers=left,style=customasmx86]{patterns/15_structs/4_packing/packing_FR.asm}

Nous passons la structure comme un tout, mais en réalité nous pouvons constater que la structure est copiée 
dans un espace temporaire. De l'espace est réservé pour cela ligne 10 et les 4 champs sont copiées par les 
lignes de 12 \ldots\ 19), puis le pointeur sur l'espace temporaire est passé à la fonction.

La structure est recopiée au cas où la fonction \ttf{} viendrait à en modifier le contenu. Si cela arrive, 
la copie de la structure qui existe dans \main restera inchangée.

Nous pourrions également utiliser des pointeurs \CCpp. Le résulta demeurerait le même, sans qu'il soit 
nécessaire de procéder à la copie.

Nous observons que l'adresse de chaque champ est alignée sur un multiple de 4  octets. C'est pourquoi chaque 
\Tchar occupe 4 octets (de même qu'un \Tint). Pourquoi en est-il ainsi? La réponse se situe au niveau de la 
CPU. Il est plus facile et performant pour elle d'accéder la mémoire et de gérer le cache de données en 
utilisant des adresses alignées.

En revanche ce n'est pas très économique en terme d'espace.

Tentons maintenant une compilation avec l'option (\TT{/Zp1}) (\emph{/Zp[n] indique qu'il faut compresser les 
structures en utilisant des frontières tous les n octets}).

\lstinputlisting[caption=MSVC 2012 /GS- /Zp1,label=src:struct_packing_1,numbers=left,style=customasmx86]{patterns/15_structs/4_packing/packing_msvc_Zp1_FR.asm}

La structure n'occupe plus que 10 octets et chaque valeur de type \Tchar n'occupe plus qu'un octet. Quelles 
sont les conséquences ? Nous économisons de la place au prix d'un accès à ces champs moins rapide que ne 
pourrait le faire la CPU.

\label{short_struct_copying_using_MOV}

La structure est également copiée dans \main. Cette opération ne s'effectue pas champ par champ mais par 
blocs en utilisant trois instructions \MOV. Et pourquoi pas 4 ?

Tout simplement parce que le compilateur a décidé qu'il était préférable d'effectuer la copie en utilisant 
3 paires d'instructions \MOV plutôt que de copier deux mots de 32 bits puis 2 fois un octet ce qui aurait 
nécessité 4 paires d'instructions \MOV.

Ce type d'implémentation de la copie qui repose sur les instructions \MOV plutôt que sur l'appel à la 
fonction \TT{memcpy()} est très répandu. La raison en est que pour de petits blocs, cette approche est 
plus rapide qu'un appel à \TT{memcpy()}: \myref{copying_short_blocks}.

Comme vous pouvez le deviner, si la structure est utilisée dans de nombreux fichiers sources et objets, ils 
doivent tous être compilés avec la même convention de compactage de la structure.

\newcommand{\FNURLMSDNZP}{\footnote{\href{http://go.yurichev.com/17067}
{MSDN: Working with Packing Structures}}}
\newcommand{\FNURLGCCPC}{\footnote{\href{http://go.yurichev.com/17068}
{Structure-Packing Pragmas}}}

Au delà de l'option MSVC \TT{/Zp} qui permet de définir l'alignement des champs des structures, il existe 
également l'option du compilateur \TT{\#pragma pack} qui peut être utilisée directement dans le code source.
Elle est supportée aussi bien par MSVC\FNURLMSDNZP que pars GCC\FNURLGCCPC{}.

Revenons à la structure \TT{SYSTEMTIME} qui contient des champs de 16 bits. Comment notre compilateur sait-il 
les aligner sur des frontières de 1 octet ?

Le fichier \TT{WinNT.h} contient ces instructions:

\begin{lstlisting}[caption=WinNT.h,style=customc]
#include "pshpack1.h"
\end{lstlisting}

et celles-ci:

\begin{lstlisting}[caption=WinNT.h,style=customc]
#include "pshpack4.h"                   // L'alignement sur 4 octets est la valeur par défaut
\end{lstlisting}

Le fichier PshPack1.h ressemble à ceci:

\lstinputlisting[caption=PshPack1.h,style=customc]{patterns/15_structs/4_packing/tmp1.c}

Ces instructions indiquent au compilateur comment compresser les structures définies après \TT{\#pragma pack}.

\clearpage
\myparagraph{\olly et les champs alignés par défaut}
\myindex{\olly}

Examinons dans \olly notre exemple lorsque les champs sont alignés par défaut sur des frontières de 4 octets:

\begin{figure}[H]
\centering
\myincludegraphics{patterns/15_structs/4_packing/olly_packing_4.png}
\caption{\olly: Before \printf execution}
\label{fig:packing_olly_4}
\end{figure}

Nous voyons nos quatre champs dans la fenêtre de données.

Mais d'où viennent ces octets aléatoires (0x30, 0x37, 0x01) situé à côté des premier (a) et troisième (c)
champs ?

Si nous revenons à notre listing \myref{src:struct_packing_4}, nous constatons que ces deux champs sont de
type \Tchar. Seul un octet est écrit pour chacun d'eux: 1 et 3 respectivement (lignes 6 et 8).

Les trois autres octets des deux mots de 32 bits ne sont pas modifiés en mémoire! Des débris aléatoires des 
précédentes opérations demeurent donc là.

\myindex{x86!\Instructions!MOVSX}

Ces débris n'influencent nullement le résultat de la fonction \printf parce que les valeurs qui lui sont 
passées sont préparés avec l'instruction \MOVSX qui opère sur des octets et non pas sur des mots: 
\lstref{src:struct_packing_4} (lignes 34 et 38).

L'instruction \MOVSX (extension de signe) est utilisée ici car le type \Tchar est par défaut une valeur 
signée pour MSVC et GCC. Si l'un des types \TT{unsigned char} ou \TT{uint8\_t} était utilisé ici, ce serait 
l'instruction \MOVZX que le compilateur aurait choisi.

\clearpage
\myparagraph{\olly et les champs alignés sur des frontières de 1 octet}
\myindex{\olly}

Les choses sont beaucoup plus simples ici. Les 4 champs occupent 10 octets et les valeurs sont stockées 
côte-à-côte.

\begin{figure}[H]
\centering
\myincludegraphics{patterns/15_structs/4_packing/olly_packing_1.png}
\caption{\olly: Avant appel de la fonction \printf}
\label{fig:packing_olly_1}
\end{figure}

