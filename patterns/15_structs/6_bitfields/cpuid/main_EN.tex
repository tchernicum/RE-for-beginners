\subsubsection{CPUID example}

The \CCpp language allows to define the exact number of bits for each structure field.
It is very useful if one needs to save memory space. 
For example, one bit is enough for a \Tbool variable.
But of course, it is not rational if speed is important.
% FIXME!
% another use of this is to parse binary protocols/packets, for example
% the definition of struct iphdr in include/linux/ip.h

\newcommand{\FNCPUID}{\footnote{\href{http://go.yurichev.com/17069}{wikipedia}}}

\myindex{x86!\Instructions!CPUID}
\label{cpuid}

Let's consider the \CPUID\FNCPUID instruction example.
This instruction returns information about the current CPU and its features.

If the \EAX is set to 1 before the instruction's execution, 
\CPUID returning this information packed into the \EAX register:

\begin{center}
\begin{tabular}{ | l | l | }
\hline
3:0 (4 bits)& Stepping \\
7:4 (4 bits) & Model \\
11:8 (4 bits) & Family \\
13:12 (2 bits) & Processor Type \\
19:16 (4 bits) & Extended Model \\
27:20 (8 bits) & Extended Family \\
\hline
\end{tabular}
\end{center}

\newcommand{\FNGCCAS}{\footnote{\href{http://go.yurichev.com/17070}
{More about internal GCC assembler}}}

MSVC 2010 has \CPUID macro, but GCC 4.4.1 does not.
So let's make this function by ourselves for GCC with the help of its built-in assembler\FNGCCAS.

\lstinputlisting[style=customc]{patterns/15_structs/6_bitfields/cpuid/CPUID.c}

After \CPUID fills \EAX/\EBX/\ECX/\EDX, these registers are to be written in the \TT{b[]} array.
Then, we have a pointer to the \TT{CPUID\_1\_EAX} structure and we point it to the value in \EAX from the \TT{b[]} array.

In other words, we treat a 32-bit \Tint value as a structure.
Then we read specific bits from the structure.

\myparagraph{MSVC}

Let's compile it in MSVC 2008 with \Ox option:

\lstinputlisting[caption=\Optimizing MSVC 2008,style=customasmx86]{patterns/15_structs/6_bitfields/cpuid/CPUID_msvc_Ox.asm}

\myindex{x86!\Instructions!SHR}

The \TT{SHR} instruction shifting the value in \EAX by the number of bits that must be
\emph{skipped}, e.g., we ignore some bits \emph{at the right side}.

\myindex{x86!\Instructions!AND}

The \AND instruction clears the unneeded bits \emph{on the left}, or, in other words, 
leaves only those bits in the \EAX register we need.

\clearpage
\mysubparagraph{\olly}
\myindex{\olly}

Let's try this example in \olly.
The input value of the function (2) is loaded into \EAX: 

\begin{figure}[H]
\centering
\myincludegraphics{patterns/08_switch/2_lot/olly1.png}
\caption{\olly: function's input value is loaded in \EAX}
\label{fig:switch_lot_olly1}
\end{figure}

\clearpage
The input value is checked, is it bigger than 4? 
If not, the \q{default} jump is not taken:
\begin{figure}[H]
\centering
\myincludegraphics{patterns/08_switch/2_lot/olly2.png}
\caption{\olly: 2 is no bigger than 4: no jump is taken}
\label{fig:switch_lot_olly2}
\end{figure}

\clearpage
Here we see a jumptable:

\begin{figure}[H]
\centering
\myincludegraphics{patterns/08_switch/2_lot/olly3.png}
\caption{\olly: calculating destination address using jumptable}
\label{fig:switch_lot_olly3}
\end{figure}

Here we've clicked \q{Follow in Dump} $\rightarrow$ \q{Address constant}, so now we see the \emph{jumptable} in the data window.
These are 5 32-bit values\footnote{They are underlined by \olly because
these are also FIXUPs: \myref{subsec:relocs}, we are going to come back to them later}.
\ECX is now 2, so the third element (can be indexed as 2\footnote{About indexing, see also: \ref{arrays_at_one}}) of the table is to be used.
It's also possible to click \q{Follow in Dump} $\rightarrow$ 
\q{Memory address} and \olly will show the element addressed by the \JMP instruction. 
That's \TT{0x010B103A}.

\clearpage
After the jump we are at \TT{0x010B103A}: the code printing \q{two} will now be executed:

\begin{figure}[H]
\centering
\myincludegraphics{patterns/08_switch/2_lot/olly4.png}
\caption{\olly: now we at the \emph{case:} label}
\label{fig:switch_lot_olly4}
\end{figure}


\myparagraph{GCC}

Let's try GCC 4.4.1 with \Othree option.

\lstinputlisting[caption=\Optimizing GCC 4.4.1,style=customasmx86]{patterns/15_structs/6_bitfields/cpuid/CPUID_gcc_O3.asm}

Almost the same.
The only thing worth noting is that GCC somehow combines the calculation of\\
\TT{extended\_model\_id} and \TT{extended\_family\_id} into one block,
instead of calculating them separately before each \printf call.
