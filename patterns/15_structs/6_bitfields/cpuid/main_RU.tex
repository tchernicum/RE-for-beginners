\subsubsection{Пример CPUID}

Язык \CCpp позволяет указывать, сколько именно бит отвести для каждого поля структуры. 
Это удобно если нужно экономить место в памяти. К примеру, для переменной типа \Tbool достаточно одного бита.
Но, это не очень удобно, если нужна скорость.

% FIXME!
% another use of this is to parse binary protocols/packets, for example
% the definition of struct iphdr in include/linux/ip.h

\newcommand{\FNCPUID}{\footnote{\href{http://go.yurichev.com/17069}{wikipedia}}}

\myindex{x86!\Instructions!CPUID}
\label{cpuid}
Рассмотрим пример с инструкцией \CPUID\FNCPUID. 
Эта инструкция возвращает информацию о том, какой процессор имеется в наличии и какие возможности он имеет.

Если перед исполнением инструкции в \EAX будет 1, 
то \CPUID вернет упакованную в \EAX такую информацию о процессоре:

\begin{center}
\begin{tabular}{ | l | l | }
\hline
3:0 (4 бита)& Stepping \\
7:4 (4 бита) & Model \\
11:8 (4 бита) & Family \\
13:12 (2 бита) & Processor Type \\
19:16 (4 бита) & Extended Model \\
27:20 (8 бит) & Extended Family \\
\hline
\end{tabular}
\end{center}

\newcommand{\FNGCCAS}{\footnote{\href{http://go.yurichev.com/17070}
{Подробнее о встроенном ассемблере GCC}}}

MSVC 2010 имеет макрос для \CPUID, а GCC 4.4.1 ~--- нет. 
Поэтому для GCC сделаем эту функцию сами, используя его встроенный ассемблер\FNGCCAS.

\lstinputlisting[style=customc]{patterns/15_structs/6_bitfields/cpuid/CPUID.c}

После того как \CPUID заполнит \EAX/\EBX/\ECX/\EDX, у нас они отразятся в массиве \TT{b[]}. 
Затем, мы имеем указатель на структуру \TT{CPUID\_1\_EAX}, и мы указываем его на значение 
\EAX из массива \TT{b[]}.

Иными словами, мы трактуем 32-битный \Tint как структуру.

Затем мы читаем отдельные биты из структуры.

\myparagraph{MSVC}

Компилируем в MSVC 2008 с опцией \Ox:

\lstinputlisting[caption=\Optimizing MSVC 2008,style=customasmx86]{patterns/15_structs/6_bitfields/cpuid/CPUID_msvc_Ox.asm}

\myindex{x86!\Instructions!SHR}
Инструкция \TT{SHR} сдвигает значение из \EAX на то количество бит, 
которое нужно \emph{пропустить}, то есть, мы игнорируем некоторые биты \emph{справа}.

\myindex{x86!\Instructions!AND}
А инструкция \AND очищает биты \emph{слева} которые нам не нужны, или же, говоря иначе, 
она оставляет по маске только те биты в \EAX, которые нам сейчас нужны.

\clearpage
\subsubsection{\olly}
\myindex{\olly}

Компилируем этот пример в MSVC 2010 с ключами \TT{/GS- /MD} и запускаем в \olly.
Открываем окна данных и стека по адресу, который передается в качестве первого аргумента в функцию \TT{GetSystemTime()}, 
ждем пока эта функция исполнится, и видим следующее:

\begin{figure}[H]
\centering
\myincludegraphics{patterns/15_structs/1_systemtime/olly_systemtime1.png}
\caption{\olly: \TT{GetSystemTime()} только что исполнилась}
\label{fig:struct_olly_1}
\end{figure}

Точное системное время на моем компьютере, в которое исполнилась функция, это 9 декабря 2014, 22:29:52:

\lstinputlisting[caption=Вывод \printf]{patterns/15_structs/1_systemtime/console.txt}

Таким образом, в окне данных мы видим следующие 16 байт: 
\begin{lstlisting}
DE 07 0C 00 02 00 09 00 16 00 1D 00 34 00 D4 03
\end{lstlisting}

Каждые два байта отражают одно поле структуры. 
А так как порядок байт (\gls{endianness}) \emph{little endian},
то в начале следует младший байт, затем старший.
Следовательно, вот какие 16-битные числа сейчас записаны в памяти:

\begin{center}
\begin{tabular}{ | l | l | l | }
\hline
\headercolor{} Шестнадцатеричное число & 
\headercolor{} десятичное число & 
\headercolor{} имя поля \\
\hline
0x07DE & 2014	& wYear \\
\hline
0x000C & 12	& wMonth \\
\hline
0x0002 & 2	& wDayOfWeek \\
\hline
0x0009 & 9	& wDay \\
\hline
0x0016 & 22	& wHour \\
\hline
0x001D & 29	& wMinute \\
\hline
0x0034 & 52	& wSecond \\
\hline	
0x03D4 & 980	& wMilliseconds \\
\hline
\end{tabular}
\end{center}

В окне стека, видны те же значения, только они сгруппированы как 32-битные значения.

Затем \printf просто берет нужные значения и выводит их на консоль.

Некоторые поля \printf не выводит (\TT{wDayOfWeek} и
\TT{wMilliseconds}), но они находятся в памяти и доступны для использования.



\myparagraph{GCC}

Попробуем GCC 4.4.1 с опцией \Othree.

\lstinputlisting[caption=\Optimizing GCC 4.4.1,style=customasmx86]{patterns/15_structs/6_bitfields/cpuid/CPUID_gcc_O3.asm}

Практически, то же самое. Единственное что стоит отметить это то, что GCC решил зачем-то объединить 
вычисление \\
\TT{extended\_model\_id} и \TT{extended\_family\_id} в один блок, 
вместо того чтобы вычислять их перед соответствующим вызовом \printf.

