\subsubsection{MIPS}

\ifdefined\RUSSIAN
\lstinputlisting[caption=\NonOptimizing GCC 4.4.5 (IDA),style=customasmMIPS]{patterns/09_loops/simple/MIPS_O0_IDA_RU.lst}

\myindex{MIPS!\Pseudoinstructions!B}
Новая для нас инструкция это \INS{B}. Вернее, это псевдоинструкция (\INS{BEQ}).
\fi

\ifdefined\ENGLISH
\lstinputlisting[caption=\NonOptimizing GCC 4.4.5 (IDA),style=customasmMIPS]{patterns/09_loops/simple/MIPS_O0_IDA_EN.lst}

\myindex{MIPS!\Pseudoinstructions!B}
The instruction that's new to us is \TT{B}. It is actually the pseudo instruction (\INS{BEQ}).
\fi

\ifdefined\FRENCH
\lstinputlisting[caption=GCC 4.4.5 \NonOptimizing (IDA),style=customasmMIPS]{patterns/09_loops/simple/MIPS_O0_IDA_FR.lst}

\myindex{MIPS!\Pseudoinstructions!B}
L'instruction qui est nouvelle pour nous est \TT{B}. C'est la pseudo instruction (\INS{BEQ}).
\fi

\ifdefined\JAPANESE
\lstinputlisting[caption=GCC 4.4.5 \NonOptimizing (IDA),style=customasmMIPS]{patterns/09_loops/simple/MIPS_O0_IDA_JA.lst}

\myindex{MIPS!\Pseudoinstructions!B}
新しい命令は\TT{B}です。 実際には擬似命令(\INS{BEQ})です。
\fi

