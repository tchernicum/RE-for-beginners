\subsubsection{ARM}

\myparagraph{\NonOptimizingKeilVI (\ARMMode)}

\lstinputlisting[label=Keil_number_sign,style=customasmARM]{patterns/09_loops/simple/ARM/Keil_ARM_O0.asm}

Der Zähler $i$ wird im Register \Reg{4} gespeichert. 
Der Befehl \INS{MOV R4, \#2} initialisiert $i$.
Die Befehle \INS{MOV R0, R4} und \INS{BL printing\_function} bilden den Körper
der Schleife; der erste Befehl bereitet das Argument für die \ttf Funktion vor
und der zweite ruft die Funktion auf.

\myindex{ARM!\Instructions!ADD}
Der Befehl \INS{ADD R4, R4, \#1} erhöht die $i$ Variable in jedem Durchlauf um
1.
\myindex{ARM!\Instructions!CMP}
\myindex{ARM!\Instructions!BLT}
\INS{CMP R4, \#0xA} vergleicht $i$ mit \TT{0xA} (10). 
Der nächste Befehl \INS{BLT} (\emph{Branch Less Than}) springt, falls $i$ kleiner
als 10 ist. Sonst wird 0 in das Register \Reg{0} geschrieben (unsere Funktion
liefert den Wert 0 zurück) und die Funktionsausführung wird beendet.

\myparagraph{\OptimizingKeilVI (\ThumbMode)}

\lstinputlisting[style=customasmARM]{patterns/09_loops/simple/ARM/Keil_thumb_O3.asm}

Praktisch das gleiche.

\myparagraph{\OptimizingXcodeIV (\ThumbTwoMode)}
\label{ARM_unrolled_loops}

\lstinputlisting[style=customasmARM]{patterns/09_loops/simple/ARM/xcode_thumb_O3.asm}

In meiner \ttf Funktion befand sich tatsächlich Folgendes:

\begin{lstlisting}[style=customc]
void printing_function(int i)
{
    printf ("%d\n", i);
};
\end{lstlisting}

\myindex{Unrolled loop}
\myindex{Inline code}
Also hat LLVM die Schleife nicht nur \emph{unrolled} sondern auch die einfache
Funktion \ttf \emph{inlined} und den Körper der Schleife acht Mal generiert,
anstatt die Schleife aufzurufen. 

Dies ist möglich, wenn eine Funktion sehr einfach ist (wie meine) und wenn sie
nicht allzu oft aufgerufen wird (wie hier).


\myparagraph{ARM64: \Optimizing GCC 4.9.1}

\lstinputlisting[caption=\Optimizing GCC 4.9.1,style=customasmARM]{patterns/09_loops/simple/ARM/ARM64_GCC491_O3_DE.s}

\myparagraph{ARM64: \NonOptimizing GCC 4.9.1}

\lstinputlisting[caption=\NonOptimizing GCC 4.9.1 -fno-inline,style=customasmARM]{patterns/09_loops/simple/ARM/ARM64_GCC491_O0_DE.s}

