\subsection{Renvoyer des valeurs}

Les pointeurs sont souvent utilisés pour renvoyer des valeurs depuis les fonctions
(rappelez-vous le cas~(\myref{label_scanf}) de \scanf).

Par exemple, lorsqu'une fonction doit renvoyer deux valeurs.

\subsubsection{Exemple avec des variables globales}

\lstinputlisting[style=customc]{patterns/061_pointers/2/global.c}

Ceci se compile en:

\lstinputlisting[caption=MSVC 2010 \Optimizing (/Ob0),style=customasmx86]{patterns/061_pointers/2/global.asm}

\myindex{\olly}
\clearpage
Regardons ceci dans \olly:

\begin{figure}[H]
\centering
\myincludegraphics{patterns/061_pointers/2/olly_global1.png}
\caption{\olly: 
les adresses des variables globales sont passées à \ttfone}
\label{fig:pointers_olly_global_1}
\end{figure}

Tout d'abord, les adresses des variables globales sont passées à \ttfone.
Nous pouvons cliquer sur \q{Follow in dump}
sur l'élément de la pile, et nous voyons l'espace alloué dans le segment de données
pour les deux variables.

\clearpage
Ces variables sont mises à zéro, car les données non-initialisées (de
\ac{BSS}) sont effacées avant le début de l'exécution, [voir \CNineNineStd{} 6.7.8p10].

Elles se trouvent dans le segment de données, nous pouvons le vérifier en appuyant
sur Alt-M et en regardant la carte de la mémoire:

\begin{figure}[H]
\centering
\myincludegraphics{patterns/061_pointers/2/olly_global5.png}
\caption{\olly: carte de la mémoire}
\label{fig:pointers_olly_global_5}
\end{figure}

\clearpage
Traçons l'exécution (F7) jusqu'au début de \ttfone:

\begin{figure}[H]
\centering
\myincludegraphics{patterns/061_pointers/2/olly_global2.png}
\caption{\olly: \ttfone commence}
\label{fig:pointers_olly_global_2}
\end{figure}

Deux valeurs sont visibles sur la pile: 456 (\TT{0x1C8}) et 123 (\TT{0x7B}), et aussi
les adresses des deux variables globales.

\clearpage
Suivons l'exécution jusqu'à la fin de \ttfone.
Dans la fenêtre en bas à gauche, nous voyons comment le résultat du calcul apparaît
dans les variables globales:

\begin{figure}[H]
\centering
\myincludegraphics{patterns/061_pointers/2/olly_global3.png}
\caption{\olly: l'exécution de \ttfone est terminée}
\label{fig:pointers_olly_global_3}
\end{figure}

\clearpage

Maintenant les valeurs des variables globales sont chargées dans des registres, prêtes
à être passées à \printf (via la pile):

\begin{figure}[H]
\centering
\myincludegraphics{patterns/061_pointers/2/olly_global4.png}
\caption{\olly: 
les adresses des variables globales sont passées à \printf}
\label{fig:pointers_olly_global_4}
\end{figure}

\subsubsection{Exemple avec des variables locales}

Modifions légèrement notre exemple:

\lstinputlisting[caption=maintenant les variables \TT{sum} et \TT{product} sont locales,style=customc]{patterns/061_pointers/2/local_FR.c}

Le code de \ttfone ne va pas changer.
Seul celui de \main va changer:

\lstinputlisting[caption=MSVC 2010 \Optimizing (/Ob0),style=customasmx86]{patterns/061_pointers/2/local.asm}

\newcommand{\PtrsAddresses}{\TT{0x2EF854} et \TT{0x2EF858}\xspace}

\clearpage
Regardons à nouveau avec \olly.
Les adresses des variables locales dans la pile sont \PtrsAddresses.
Voyons comment elles sont poussées sur la pile:

\begin{figure}[H]
\centering
\myincludegraphics{patterns/061_pointers/2/olly_stk1.png}
\caption{\olly: les adresses des variables locales sont poussées sur la pile}
\label{fig:pointers_olly_stk_1}
\end{figure}

\clearpage
\ttfone commence.
Jusqu'ici, il n'y a que des restes de données sur la pile en \PtrsAddresses :

\begin{figure}[H]
\centering
\myincludegraphics{patterns/061_pointers/2/olly_stk2.png}
\caption{\olly: \ttfone commence}
\label{fig:pointers_olly_stk_2}
\end{figure}

\clearpage
\ttfone se termine:

\begin{figure}[H]
\centering
\myincludegraphics{patterns/061_pointers/2/olly_stk3.png}
\caption{\olly: \ttfone termine son exécution}
\label{fig:pointers_olly_stk_3}
\end{figure}

Nous trouvons maintenant \TT{0xDB18} et \TT{0x243} aux adresses \PtrsAddresses.
Ces valeurs sont les résultats de \ttfone.

\subsubsection{\Conclusion{}}
 
\ttfone peut renvoyer des pointeurs sur n'importe quel emplacement en mémoire, situés
n'importe où.

C'est par essence l'utilité des pointeurs.

À propos, les \emph{references} \Cpp fonctionnent exactement pareil. Voir à ce propos:
(\myref{cpp_references}).
