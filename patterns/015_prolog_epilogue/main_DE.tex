\mysection{Funktionsprolog und Funktionsepilog}
\label{sec:prologepilog}
\myindex{Function epilogue}
\myindex{Function prologue}

Ein Funktionsprolog besteht aus einer Abfolge von Instruktionen zu Beginn einer
Funktion. Dieser Prolog sieht häufig ähnlich aus wie das folgende Codefragment:

\begin{lstlisting}[style=customasmx86]
    push    ebp
    mov     ebp, esp
    sub     esp, X
\end{lstlisting}

Was diese Instruktionen tun: speichern des Wertes im \EBP Register auf dem
Stack, setzen des Wertes von \EBP auf den Wert in \ESP, zuweisen von Speicherplatz für lokale
Variablen auf dem Stack.

Der Wert in \EBP verändert sich während der Funktionsausführung nicht und wird
für den Zugriff auf lokale Variablen und Funktionsargumente verwendet. Zum
gleichen Zweck kann auch \ESP verwendet werden, aber da sich der Wert von \ESP
während der Funktionsausführung verändert, ist diese Vorgehensweise nicht allzu
gebräuchlich.

Der Funktionsepilog gibt den zugewiesenen Speicher auf dem Stack wieder frei,
setzt das \EBP Register auf den ursprünglichen Wert zurück und gibt den control
flow an den \gls{caller} zurück:

\begin{lstlisting}[style=customasmx86]

    mov    esp, ebp
    pop    ebp
    ret    0
\end{lstlisting}

% what about calling convention?
Für gewöhnlich werden Funktionsprolog und -epilog von Disassemblern erkannt und
zur Markierung und Abgrenzung von Funktionen verwendet.

 
\subsection{\Rekursion}

\myindex{\Recursion}
Funktionsprolog und -epilog können sich negativ auf die Performanz bei Rekursion
auswirken. 

Mehr zum Thema Rekursion im folgenden Buch: \myref{Recursion_and_tail_call}.
