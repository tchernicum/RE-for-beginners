\newcommand{\ARMELF}{\InSqBrackets{\emph{ELF for the ARM 64-bit Architecture (AArch64)}, (2013)}\footnote{\AlsoAvailableAs \url{http://go.yurichev.com/17288}}}

\subsection{Relocations en ARM64}
\label{ARM64_relocs}

Comme nous le savons, il y a des instructions 4-octet en ARM64, donc il est impossible
d'écrire un nombre large dans un registre en utilisant une seule instruction.

Cependant, une image exécutable peut être chargée à n'importe quelle adresse aléatoire
en mémoire, c'est pourquoi les relocations existent.

\myindex{ARM!\Instructions!ADRP/ADD pair}

L'adresse est formée en utilisant la paire d'instructions \TT{ADRP} et \ADD en ARM64.

La première charge l'adresse d'une page de 4KiB et la seconde ajoute le reste.
Compilons l'exemple de \q{\HelloWorldSectionName} (\lstref{hw_c}) avec GCC (Linaro)
4.9 sous win32:

\begin{lstlisting}[caption=GCC (Linaro) 4.9 et objdump du fichier objet,style=customasmARM]
...>aarch64-linux-gnu-gcc.exe hw.c -c

...>aarch64-linux-gnu-objdump.exe -d hw.o

...

0000000000000000 <main>:
   0:   a9bf7bfd        stp     x29, x30, [sp,#-16]!
   4:   910003fd        mov     x29, sp
   8:   90000000        adrp    x0, 0 <main>
   c:   91000000        add     x0, x0, #0x0
  10:   94000000        bl      0 <printf>
  14:   52800000        mov     w0, #0x0                        // #0
  18:   a8c17bfd        ldp     x29, x30, [sp],#16
  1c:   d65f03c0        ret

...>aarch64-linux-gnu-objdump.exe -r hw.o

...

RELOCATION RECORDS FOR [.text]:
OFFSET           TYPE              VALUE
0000000000000008 R_AARCH64_ADR_PREL_PG_HI21  .rodata
000000000000000c R_AARCH64_ADD_ABS_LO12_NC  .rodata
0000000000000010 R_AARCH64_CALL26  printf
\end{lstlisting}

Donc, il y a 3 relocations dans ce fichier objet.

\begin{itemize}
\item 
La première prend l'adresse de la page, coupe les 12 bits du bas et écrit les 21 bits
du haut restants dans le champs de bit de l'instruction \TT{ADRP}. Ceci car nous n'avons
pas besoin d'encoder les 12 bits bas, et l'instruction \INS{ADRP} possède seulement
de l'espace pour 21 bits.

\item 
La seconde met les 12 bits de l'adresse relative au début de la page dans le champ
de bits de l'instruction \ADD.

\item 
La dernière, celle de 26-bit, est appliquée à l'instruction à l'adresse \TT{0x10}
où le saut à la fonction \printf se trouve.

Toutes les adresses des instructions ARM64 (et ARM en mode ARM) ont zéro dans les
deux bits les plus bas (car toutes les instructions ont une taille de 4 octets),
donc on doit seulement encoder les 26 bits du haut de l'espace d'adresse de 28-bit
($\pm 128$MB).

\end{itemize}

Il n'y a pas de telles relocations dans le fichier exécutable: car l'adresse où se
trouve la chaîne \q{Hello!} est connue, la page, et l'adresse de \puts sont aussi
connues.

Donc, il y a déjà des valeurs mises dans les instructions \TT{ADRP}, \ADD et \TT{BL}
(l'éditeur de liens à écrit des valeurs lors de l'édition de liens):

\begin{lstlisting}[caption=objdump du fichier exécutable,style=customasmARM]
0000000000400590 <main>:
  400590:       a9bf7bfd        stp     x29, x30, [sp,#-16]!
  400594:       910003fd        mov     x29, sp
  400598:       90000000        adrp    x0, 400000 <_init-0x3b8>
  40059c:       91192000        add     x0, x0, #0x648
  4005a0:       97ffffa0        bl      400420 <puts@plt>
  4005a4:       52800000        mov     w0, #0x0                        // #0
  4005a8:       a8c17bfd        ldp     x29, x30, [sp],#16
  4005ac:       d65f03c0        ret

...

Contents of section .rodata:
 400640 01000200 00000000 48656c6c 6f210000  ........Hello!..
\end{lstlisting}

\myindex{ARM!\Instructions!BL}

À titre d'exemple, essayons de désassembler manuellement l'instruction BL.\\
\TT{0x97ffffa0} est $0b10010111111111111111111110100000$.
D'après \InSqBrackets{\ARMSixFourRef C5.6.26}, \emph{imm26} correspond aux derniers
26 bits:\\
$imm26 = 0b11111111111111111110100000$.
Il s'agit de \TT{0x3FFFFA0}, mais le \ac{MSB} est 1,
donc le nombre est négatif, et nous pouvons le convertir manuellement en une forme
pratique pour nous.
D'après les règles de la négation (\myref{sec:signednumbers:negation}), il faut simplement
inverser tous les bits (ça donne \TT{0b1011111=0x5F}), et ajouter 1 (\TT{0x5F+1=0x60}).
Donc le nombre sous sa forme signée est \TT{-0x60}.
Multiplions \TT{-0x60} par 4 (car l'adresse est stockée dans l'opcode est divisée
par 4): ça fait \TT{-0x180}.
Maintenant calculons l'adresse de destination: \TT{0x4005a0} + (\TT{-0x180}) = \TT{0x400420}
(noter s'il vous plaît: nous considérons l'adresse de l'instruction BL, pas la valeur
courante du \ac{PC}, qui peut être différente!).
Donc l'adresse de destination est \TT{0x400420}.\\
\\
Plus d'informations relatives aux relocations en ARM64: \ARMELF.
