\subsubsection{Zapisywanie adresu powrotu}

\myparagraph{x86}

\myindex{x86!\Instructions!CALL}
Przy wywołaniu funkcji przez \CALL najpierw na stos jest odkładany adres, wskazujący na miejsce po 
instrukcji \CALL, następnie robi się przejście bezwzględne (prawie jak \TT{JMP}) pod adres, zapisany w operandzie.

\myindex{x86!\Instructions!PUSH}
\myindex{x86!\Instructions!JMP}
\CALL~--- jest analogiczny do pary instrukcji \INS{PUSH address\_after\_call / JMP}.

\myindex{x86!\Instructions!RET}
\myindex{x86!\Instructions!POP}
\RET zdejmuje ze stosu wartość i przekazuje zarządzanie pod ten adres~--- 
jest to analogiczne do działania pary instrukcji \TT{POP tmp / JMP tmp}.

\myindex{\Stack!\MLStackOverflow}
\myindex{\Recursion}
Bardzo łatwo przepełnić stos, poprzez rekurencję:

\begin{lstlisting}[style=customc]
void f()
{
	f();
};
\end{lstlisting}

MSVC 2008 uprzedza:

\begin{lstlisting}
c:\tmp6>cl ss.cpp /Fass.asm
Microsoft (R) 32-bit C/C++ Optimizing Compiler Version 15.00.21022.08 for 80x86
Copyright (C) Microsoft Corporation.  All rights reserved.

ss.cpp
c:\tmp6\ss.cpp(4) : warning C4717: 'f' : recursive on all control paths, function will cause runtime stack overflow
\end{lstlisting}

\dots ale, niemniej jednak, tworzy potrzebny kod:

\lstinputlisting[style=customasmx86]{patterns/02_stack/1.asm}

\dots do tego, jeśli optymalizacja jest wyłączona (\TT{\Ox}), to będzie ciekawiej, bez przepełnienia stosu, 
ale będzie działało \emph{poprawnie}\footnote{ironia}:

\lstinputlisting[style=customasmx86]{patterns/02_stack/2.asm}

GCC 4.4.1 generuje taki sam kod w obu przypadkach, chociaż i nie wydaje stosownego komunikatu.

\myparagraph{ARM}

\myindex{ARM!\Registers!Link Register}
Programy dla ARM również korzystają ze stosu do zapisywania \ac{RA}, gdzie trzeba wrócić, ale trochę w inny sposób.
Jak już było wspomniane w sekcji \q{\HelloWorldSectionName}~(\myref{sec:hw_ARM}),
\ac{RA} jest zapisywany do rejestru \ac{LR} (\gls{link register}).
Ale jeśli wynika potrzeba wywołania jeszcze jakiejś funkcji i trzeba skorzystać z rejestru \ac{LR} jeszcze
raz, to jego zawartość najlepiej gdzieś zapisać.
\myindex{Function prologue}
\myindex{ARM!\Instructions!PUSH}
\myindex{ARM!\Instructions!POP}

Zwykle to się odbywa w prologu funkcji, często widzimy tam instrukcje w stylu \INS{PUSH \{R4-R7,LR\}}, a w epilogu
\INS{POP \{R4-R7,PC\}}~--- w ten sposób są zapisywane rejestry, z których będzie korzystała bieżąca funkcja, w tym rejestr \ac{LR}.

\myindex{ARM!Leaf function}
Niemniej jednak, jeśli jakaś funkcja nie wywołuje żadnych innych funkcji w trakcie swojej roboty, według terminologii \ac{RISC} jest ona nazywana
\emph{\gls{leaf function}}\footnote{\href{http://go.yurichev.com/17064}{infocenter.arm.com/help/index.jsp?topic=/com.arm.doc.faqs/ka13785.html}}. 
Wskutek tego, \q{leaf}-funkcja nie zapisuje rejestru \ac{LR} (dlatego że ona go nie zmienia).
A jeśli funkcja jest niewielkich rozmiarów, korzysta z małej ilości rejestrów, to może nie korzystać ze stosu w ogóle.
W ten sposób, w ARM możliwe jest wywoływanie małych leaf-funkcji nie korzystając ze stosu.
jest to szybsze niż w starych x86, dlatego że nie korzysta się z pamięci zewnętrznej do stosu
\footnote{Kiedyś, bardzo dawno temu, na PDP-11 i VAX na wykonanie instrukcjii CALL (wywołanie innych funkcji) mogło być zatracone
nawet 50\% czasu (przwdopodobnie przez pracę z pamięcią zewnętrzną),
dlatego było uważane, że dużo małych funkcji to \glslink{anti-pattern}
\InSqBrackets{\TAOUP Chapter 4, Part II}.}.
Również to może być też korzystne, kiedy pamięć pod stos jeszcze nie jest zaalokowana, lub jest niedostępna,

kilka przykładów takich funkcji:
\myref{ARM_leaf_example1}, \myref{ARM_leaf_example2}, 
\myref{ARM_leaf_example3}, \myref{ARM_leaf_example4}, \myref{ARM_leaf_example5},
\myref{ARM_leaf_example6}, \myref{ARM_leaf_example7}, \myref{ARM_leaf_example10}.


