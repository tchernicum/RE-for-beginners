\subsection{Division}

\subsubsection{Division durch Verschieben}
\label{division_by_shifting}

Beispiel der Division durch 4:

\begin{lstlisting}[style=customc]
unsigned int f(unsigned int a)
{
	return a/4;
};
\end{lstlisting}

Wir betrachten (MSVC 2010):

\begin{lstlisting}[caption=MSVC 2010,style=customasmx86]
_a$ = 8		; size = 4
_f	PROC
	mov	eax, DWORD PTR _a$[esp-4]
	shr	eax, 2
	ret	0
_f	ENDP
\end{lstlisting}

\label{SHR}
\myindex{x86!\Instructions!SHR}

Der Befehl \SHR (\emph{Shift Right}) verschiebt die Zahl in diesem Beispiel um 2
Bits nach rechts. Die beiden freien Bits am linken Rand (dies sind die beiden
höchstwertigsten Bits) werden auf null gesetzt. 
Die beiden niederwertigsten Bits werden entfernt. 
Diese beiden entfernten Bits entsprechen genau dem Rest der Division.

\myindex{x86!\Instructions!SHR}

Der \SHR Befehl funktioniert genau wie \SHL, aber in die entgegengesetzte
Richtung.

\input{shift_right}
Das Vorgehen kann leicht verdeutlicht werden, wenn wir es an der Zahl 23 im
Dezimalsystem veranschaulichen. Die 23 kann einfach durch 10 geteilt werden,
indem die letzte Ziffer (3--Rest der Division) entfernt wird. Die 2 bleibt bei
der Division als ganzzahliger \glslink{quotient}{Quotient} übrig.

Der Rest wird also entfernt, was aber kein Problem darstellt, da wir hier
ausschließlich mit ganzzahligen Werten arbeiten und nicht mit \glslink{real number}{reellen Zahlen}.

Division durch 4 in ARM:

\begin{lstlisting}[caption=\NonOptimizingKeilVI (\ARMMode),style=customasmARM]
f PROC
        LSR      r0,r0,#2
        BX       lr
        ENDP
\end{lstlisting}

Division by 4 in MIPS:

\begin{lstlisting}[caption=\Optimizing GCC 4.4.5 (IDA),style=customasmMIPS]
        jr      $ra
        srl     $v0, $a0, 2 ; branch delay slot
\end{lstlisting}

\myindex{MIPS!\Instructions!SRL}
Der Befehl SRL steht für \q{Shift Right Logical}.
