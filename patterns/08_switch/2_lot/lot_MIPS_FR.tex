\subsubsection{MIPS}

\lstinputlisting[caption=GCC 4.4.5 \Optimizing (IDA),style=customasmMIPS]{patterns/08_switch/2_lot/MIPS_O3_IDA_FR.lst}

\myindex{MIPS!\Instructions!SLTIU}

La nouvelle instruction pour nous est \INS{SLTIU} (\q{Set on Less Than Immediate Unsigned}
Mettre si inférieur à la valeur immédiate non signée).
\myindex{MIPS!\Instructions!SLTU}

Ceci est la même que \INS{SLTU} (\q{Set on Less Than Unsigned}), mais \q{I} signifie
\q{immediate}, i.e., un nombre doit être spécifié dans l'instruction elle-même.

\myindex{MIPS!\Instructions!BNEZ}
\INS{BNEZ} est \q{Branch if Not Equal to Zero}.

Le code est très proche de l'autre \ac{ISA}s.
\myindex{MIPS!\Instructions!SLL}
\INS{SLL} (\q{Shift Word Left Logical}) effectue une multiplication par 4.

MIPS est un CPU 32-bit après tout, donc toutes les adresses de la \emph{jumtable}
sont 32-bits.

