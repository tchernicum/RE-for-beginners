\subsection{A lot of cases}

\TT{switch()}ステートメントに大量のケースが含まれている場合、コンパイラが多くの \JE/\JNE 命令で大きすぎるコードを
出力することはあまり便利ではありません。

\lstinputlisting[label=switch_lot_c,style=customc]{patterns/08_switch/2_lot/lot.c}

\subsubsection{x86}

\myparagraph{\NonOptimizing MSVC}

We get (MSVC 2010):

\lstinputlisting[caption=MSVC 2010,style=customasmx86]{patterns/08_switch/2_lot/lot_msvc_JA.asm}

\myindex{jumptable}

ここでは、さまざまな引数を持つ \printf 呼び出しのセットを見ていきます。
すべては、プロセスのメモリだけでなく、コンパイラによって割り当てられた内部シンボリックラベルも持っています。
これらのラベルはすべて \TT{\$LN11@f} 内部テーブルにも記載されています。

関数の開始時に、 $a$ が4より大きい場合、制御フローはラベル \TT{\$LN1@f}に渡されます。
引数 \TT{'something unknown'}をとって \printf が呼び出されます。

しかし、 $a$ の値が4以下の場合は、4を乗算して \TT{\$LN11@f}テーブルアドレスで加算します。 
これはテーブル内のアドレスがどのように構築され、必要な要素を正確に指し示すものです。 
たとえば、 $a$ が2に等しいとしましょう。$2*4 = 8$(すべてのテーブル要素は
32ビットプロセスのアドレスなので、すべての要素が4バイト幅です)。
\TT{\$LN11@f}テーブルのアドレス+ 8は\TT{\$LN4@f}ラベルが格納されているテーブル要素です。
\JMP はテーブルから\TT{\$LN4@f}アドレスを取り出し、それにジャンプします。

このテーブルはしばしば\emph{jumptable} または \emph{branch table}\footnote{The whole method was once called 
\emph{computed GOTO} in early versions of Fortran:
\href{http://go.yurichev.com/17122}{wikipedia}.
Not quite relevant these days, but what a term!}と呼ばれます。

それから、対応する \printf は引数 \TT{'two'}で呼び出されます。
実際、\TT{jmp DWORD PTR \$LN11@f[ecx*4]}命令は\emph{jump to the DWORD that is stored at address} \TT{\$LN11@f + ecx * 4}

\TT{npad}(\myref{sec:npad})は、4バイト(または16バイト)の境界に整列したアドレスに格納されるように次のラベルを整列するアセンブリ言語マクロです。
これは、メモリバス、キャッシュメモリなどを介してメモリから32ビット値をフェッチすることができるため、
プロセッサが整列している場合にはより効果的な方法でプロセッサに非常に適しています。

\clearpage
\mysubparagraph{\olly}
\myindex{\olly}

\olly でこの例を試してみましょう。
関数への入力値2が \EAX にロードされます。

\begin{figure}[H]
\centering
\myincludegraphics{patterns/08_switch/2_lot/olly1.png}
\caption{\olly: 関数への入力値が \EAX にロードされる}
\label{fig:switch_lot_olly1}
\end{figure}

\clearpage
入力値が4より大きいかチェックされます。
そうでなければ、\q{default} ジャンプは実行されません。

\begin{figure}[H]
\centering
\myincludegraphics{patterns/08_switch/2_lot/olly2.png}
\caption{\olly: 2は4より大きいか:ジャンプは実行されない}
\label{fig:switch_lot_olly2}
\end{figure}

\clearpage
ここで、ジャンプテーブルを見ることができます。

\begin{figure}[H]
\centering
\myincludegraphics{patterns/08_switch/2_lot/olly3.png}
\caption{\olly: ジャンプテーブルを用いて行先のアドレスを計算する}
\label{fig:switch_lot_olly3}
\end{figure}

ここで、\q{Follow in Dump} $\rightarrow$ \q{Address constant} をチェックします。そして、データウィンドウに\emph{jumptable}が見えます。
5つの32ビット値があります。\footnote{これらはまた要修正\myref{subsec:relocs}であるため、 \olly で下線が引かれています。後でそれらに戻ってくるつもりです}
\ECX は2になりました。したがって、テーブルの3番目の要素(2として\footnote{インデックスについては以下を参照: \ref{arrays_at_one}}索引付けできます)が使用されます。
\q{Follow in Dump} $\rightarrow$ \q{Memory address}をクリックすることができ、
\olly は \JMP 命令で指示された要素を表示します。
それは\TT{0x010B103A}です。

\clearpage
ジャンプの後、\TT{0x010B103A}にいます。\q{two}を表示するコードが実行されます。

\begin{figure}[H]
\centering
\myincludegraphics{patterns/08_switch/2_lot/olly4.png}
\caption{\olly: 今や \emph{case:} ラベルにいます}
\label{fig:switch_lot_olly4}
\end{figure}


\myparagraph{\NonOptimizing GCC}
\label{switch_lot_GCC}

GCC 4.4.1が生成するものを見てみましょう:

\lstinputlisting[caption=GCC 4.4.1,style=customasmx86]{patterns/08_switch/2_lot/lot_gcc.asm}

\myindex{x86!\Registers!JMP}

引数\TT{arg\_0}は2ビット左にシフトすることで4倍されます
(これは4倍の乗算とほぼ同じです)。~(\myref{SHR})
\TT{off\_804855C}配列からラベルのアドレスを取り出し、 \EAX に格納してから、\TT{JMP EAX}が実際のジャンプを行います。


\subsubsection{ARM: \OptimizingKeilVI (\ARMMode)}
\label{sec:SwitchARMLot}

\lstinputlisting[caption=\OptimizingKeilVI (\ARMMode),style=customasmARM]{patterns/08_switch/2_lot/lot_ARM_ARM_O3.asm}

このコードでは、すべての命令の固定サイズが4バイトのARMモード機能を使用しています。

$a$ の最大値は4で、それ以上の値を指定すると、\emph{<<something unknown\textbackslash{}n>>}文字列が
出力されることに注意しましょう。

\myindex{ARM!\Instructions!CMP}
\myindex{ARM!\Instructions!ADDCC}
最初の\TT{CMP R0, \#5}命令は、 $a$ の入力値を5と比較します。

\footnote{ADD---addition}
次の\TT{ADDCC PC, PC,R0, LSL \#2}命令は、$R0 < 5$(\emph{CC=Carry clear / Less than})の場合にのみ実行されます。
したがって、\TT{ADDCC}がトリガしない場合($R0 \geq 5$の場合)、\emph{default\_case}ラベルにジャンプします。

しかし$R0 < 5$と\TT{ADDCC}がトリガされた場合、次のことが起こります:

\Reg{0}の値には4が掛けられます。
実際、命令のサフィックスの\TT{LSL \#2}は\q{2ビット左シフト}の略です。
しかし、セクション\q{\ShiftsSectionName}の~(\myref{division_by_shifting})で後で見るように、2ビット左シフトは4を乗算するのと同じです。

次に、$R0*4$を\ac{PC}の現在の値に追加し、下にある\TT{B}(\emph{Branch})命令の1つにジャンプします。

\TT{ADDCC}命令の実行時に、\ac{PC}の値は\TT{ADDCC}命令が置かれているアドレス(\TT{0x178})よりも8バイト先(\TT{0x180})であり、
言い換えれば2命令先にあります。

\myindex{ARM!Pipeline}

これはARMプロセッサのパイプラインがどのように動作するかを示しています。
\TT{ADDCC}が実行されると、現時点でプロセッサは次の命令の後に命令を処理し始めているので、
\ac{PC}がそこを指しているのはそのためです。 
これは覚えておく必要があります。

$a=0$ の場合、\ac{PC}の値に加算され、\ac{PC}の実際の値は\ac{PC}(8バイト先)に書き込まれ、
\emph{loc\_180}というラベルへのジャンプが起こります。これは、\TT{ADDCC}命令の先の8バイト先です。

$a=1$ の場合、\ac{PC}には $PC+8+a*4 = PC+8+1*4 = PC+12 = 0x184$ が書き込まれます。
\emph{loc\_184}というラベルが付いたアドレスです。

1を $a$ に加えるごとに、結果の\ac{PC}は4ずつ増加します。

4はARMモードの命令長であり、各\TT{B}命令の長さ4でそれらは5つあります。

これらの5つの\TT{B}命令のそれぞれは、制御を\emph{switch()}にプログラムされたものにさらに渡します。

対応する文字列のポインタローディングが発生します。

\subsubsection{ARM: \OptimizingKeilVI (\ThumbMode)}

\lstinputlisting[caption=\OptimizingKeilVI (\ThumbMode),style=customasmARM]{patterns/08_switch/2_lot/lot_ARM_thumb_O3.asm}

\myindex{ARM!\ThumbMode}
\myindex{ARM!\ThumbTwoMode}

ThumbモードとThumb-2モードのすべての命令が同じサイズであることを確認することはできません。
これらのモードでは、x86の場合と同様に、命令の長さが可変であるといえます。

\myindex{jumptable}

したがって、そこにあるケースの数(デフォルトケースを含まない)に関する情報と、
対応するケースでコントロールを渡す必要があるラベルを持つそれぞれのオフセットが含まれている特別なテーブルが追加されています。

\myindex{ARM!Mode switching}
\myindex{ARM!\Instructions!BX}

\emph{\_\_ARM\_common\_switch8\_thumb}という名前のテーブルと
パスコントロールを扱うために特別な関数がここにあります。
\TT{BX PC}で始まり、その機能はプロセッサをARMモードに切り替えることです。
次に、テーブル処理の機能が表示されます。

今ここで説明するにはあまりにも進んでいるので、省略しましょう。
% TODO explain it...

\myindex{ARM!\Registers!Link Register}

関数が\ac{LR}レジスタをテーブルへのポインタとして使用することは興味深いことです。

実際、この関数を呼び出した後、\ac{LR}にはテーブルが始まる\TT{BL \_\_ARM\_common\_switch8\_thumb}命令の後のアドレスが入ります。

また、コードを再利用するために別の関数として生成されるので、
コンパイラはすべてのswitch()文に対して同じコードを生成しないことにも注意してください。

\IDA はそれをサービス関数とテーブルとして認識し、\TT{jumptable 000000FA case 0}のような
ラベルのコメントを追加します。

\subsubsection{MIPS}

\lstinputlisting[caption=\Optimizing GCC 4.4.5 (IDA),style=customasmMIPS]{patterns/08_switch/2_lot/MIPS_O3_IDA_JA.lst}

\myindex{MIPS!\Instructions!SLTIU}

私たちの新しい命令は \INS{SLTIU} です(\q{Set on Less Than Immediate Unsigned})。
\myindex{MIPS!\Instructions!SLTU}

\INS{SLTU}と同じですが、\q{I}は\q{immediate}を表します。
つまり、命令自体に数値を指定する必要があります。

\myindex{MIPS!\Instructions!BNEZ}
\INS{BNEZ} は \q{Branch if Not Equal to Zero}です。

コードは他の\ac{ISA}に似ています。
\myindex{MIPS!\Instructions!SLL}
\INS{SLL} (\q{Shift Word Left Logical})は4を掛けます。

結局のところ、MIPSは32ビットCPUなので、\emph{jumptable}のすべてのアドレスは32ビットのものです。


\subsubsection{\Conclusion{}}

\emph{switch()} の大まかなスケルトン:

% TODO: ARM, MIPS skeleton
\lstinputlisting[caption=x86,style=customasmx86]{patterns/08_switch/2_lot/skel1_JA.lst}

ジャンプテーブルのアドレスへのジャンプはこの命令で用いて実装されるでしょう:\TT{JMP jump\_table[REG*4]}
もしくはx64では\TT{JMP jump\_table[REG*8]} 。

\emph{jumptable}は単にポインタの配列で、後で説明します:\myref{array_of_pointers_to_strings}
