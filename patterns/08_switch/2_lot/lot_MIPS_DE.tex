\subsubsection{MIPS}

\lstinputlisting[caption=\Optimizing GCC 4.4.5 (IDA),style=customasmMIPS]{patterns/08_switch/2_lot/MIPS_O3_IDA_DE.lst}

\myindex{MIPS!\Instructions!SLTIU}
Der für uns neue Befehl ist \INS{SLTIU} (\q{Set on Less Than Immediate Unsigned}).

\myindex{MIPS!\Instructions!SLTU}
Dies ist das gleiche wie \INS{SLTU} (\q{Set on Less Than Unsigned}); das \q{I} steht dabei für \q{immediate}, d.h.
für den Befehl muss eine Zahl angegeben werden. 

\myindex{MIPS!\Instructions!BNEZ}
\INS{BNEZ} ist \q{Branch if Not Equal to Zero}.

Der Code ist den anderen \ac{ISA}s sehr ähnlich.
\myindex{MIPS!\Instructions!SLL}
\INS{SLL} (\q{Shift Word Left Logical}) führt eine Multiplikation mit 4 durch.
Da MIPS eine 32-Bit CPU ist, sind auch die Adressen in der \emph{Jumptable} 32 Bit groß.