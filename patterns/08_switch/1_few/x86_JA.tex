\subsubsection{x86}

\myparagraph{\NonOptimizing MSVC}

結果 (MSVC 2010):

\lstinputlisting[caption=MSVC 2010,style=customasmx86]{patterns/08_switch/1_few/few_msvc.asm}

実際、switch()でいくつかのcaseを持つ私たちの関数は、この構造に似ています。

\lstinputlisting[label=switch_few_ifelse,style=customc]{patterns/08_switch/1_few/few_analogue.c}

\myindex{\CLanguageElements!switch}
\myindex{\CLanguageElements!if}

いくつかのcaseでswitch()を使用する場合、ソースコード内の実際のswitch()か、
単にif文の組であるかどうかを確認することは不可能です。
\myindex{\SyntacticSugar}

これはswitch()が多段にネストされたif文との糖衣構文のようなものであることを意味します。

コンパイラが入力変数 $a$ を一時的なローカル変数\TT{tv64}に移動することを除いて、
生成されたコードには特に新しいことはありません。
\footnote{スタック内のローカル変数には接頭辞\TT{tv}が付きます。MSVCが内部変数として使用するために命名しています。}

これをGCC 4.4.1でコンパイルすると、最大限の最適化(\Othree option)を有効にしても
ほぼ同じ結果になります。

\myparagraph{\Optimizing MSVC}

% TODO separate various kinds of \TT
% idea: enclose command lines in a specific environment, like \cmdline{} 
% assembly instructions in \asm{} (now both \TT and \q{} are used),
% variables in,  like \var{}
% messages (string constants) in something else, like \strconst
% to separate them all. Now they all use \TT, which is not best
% \INS{} for all instructions including operands? --DY
では、MSVC(\Ox)の最適化を有効にしましょう:\TT{cl 1.c /Fa1.asm /Ox}

\label{JMP_instead_of_RET}
\lstinputlisting[caption=MSVC,style=customasmx86]{patterns/08_switch/1_few/few_msvc_Ox.asm}

ここで、汚いハックを見ることができます。

\myindex{x86!\Instructions!JZ}
\myindex{x86!\Instructions!JE}
\myindex{x86!\Instructions!SUB}

最初に、 $a$ の値を \EAX に置き、0を引きます。 EAXの値が0かどうかを確認するために行われますが、
そうであれば、 \ZF フラグがセットされます(例えば、0からの減算は0)
最初の条件ジャンプ \JE (\emph{Jump if Equal} またはあ同義語 \JZ~---\emph{Jump if Zero})は実行され、
制御フローは\TT{\$LN4@f}ラベルに渡されます。ここでは、 \TT{'zero'}メッセージが出力されます。
最初のジャンプが実行されない場合は、入力値から1が減算され、結果が0の場合、対応するジャンプが実行されます。

また、ジャンプが全く実行されない場合、制御フローは文字列引数\TT{'something unknown'}を \printf に渡します。

\label{jump_to_last_printf}
\myindex{\Stack}

次に、文字列ポインタが $a$ 変数に置かれ、 \printf が \CALL ではなく \JMP を介して呼び出されます。 
簡単に説明するとこうなります:
\gls{caller} は値をスタックにプッシュし、 \CALL 経由で関数を呼び出します。
\CALL 自体は戻りアドレス(\ac{RA})をスタックにプッシュし、関数アドレスへの無条件ジャンプを行います。
スタックポインタを移動させる命令が含まれていないため、任意の実行時点での関数は、次のスタックレイアウトを持ちます。

\begin{itemize}
\item\ESP---points to \ac{RA}
\item\TT{ESP+4}---points to the $a$ variable 
\end{itemize}

反対に、\printf をここで呼び出さなければならないときは、文字列を指し示す必要がある最初の\printf 引数を除いて、
全く同じスタックレイアウトが必要です。それが私たちのコードがすることです。

ファンクションの最初の引数を文字列のアドレスに置き換え、
関数 \ttf を直接呼び出しずに直接 \printf を呼び出すかのように、 \printf にジャンプします。
\printf は文字列を \gls{stdout} に出力し、 \RET 命令を実行します。スタックから\ac{RA}を取り出し、
制御フローは \ttf ではなく \ttf 関数の終りをバイパスして、 \ttf の \gls{caller} です。

\myindex{\CStandardLibrary!longjmp()}
\newcommand{\URLSJ}{\href{http://go.yurichev.com/17121}{wikipedia}}

% TODO \myref{}
\printf はすべての場合に  \ttf 関数の終わりで右に呼ばれるので、これはすべて可能です。
ある意味では、\TT{longjmp()}\footnote{\URLSJ}関数に似ています。
そしてもちろん、それはスピードのためにすべて行われます。

ARMコンパイラと同様のケースは、\q{\PrintfSeveralArgumentsSectionName}セクションに記載されています。
こちら:~(\myref{ARM_B_to_printf})

\clearpage
\mysubparagraph{\olly}

この例は扱いにくいので、 \olly でトレースしてみましょう。

\olly はそのようなswitch()構文を検出することができ、有用なコメントを追加することができます。
\EAX の値は最初は2で、それは関数への入力値です:

\begin{figure}[H]
\centering
\myincludegraphics{patterns/08_switch/1_few/olly1.png}
\caption{\olly: \EAX 
は最初の(そして唯一の)関数への引数を含んでいます}
\label{fig:switch_few_olly1}
\end{figure}

\clearpage
0は \EAX から2を引いた値です。
もちろん、 \EAX にはまだ2が入っています。
しかし、 \ZF フラグは0になり、結果の値がゼロでないことを示します。

\begin{figure}[H]
\centering
\myincludegraphics{patterns/08_switch/1_few/olly2.png}
\caption{\olly: \SUB の実行}
\label{fig:switch_few_olly2}
\end{figure}

\clearpage
\DEC が実行され、 \EAX には1が入ります。
しかし1はゼロではないので、 \ZF フラグはまだ0です:

\begin{figure}[H]
\centering
\myincludegraphics{patterns/08_switch/1_few/olly3.png}
\caption{\olly: 最初の \DEC 実行}
\label{fig:switch_few_olly3}
\end{figure}

\clearpage
次の \DEC が実行されます。
\EAX は最終的に0になり、結果がゼロであるため \ZF フラグが設定されます。

\begin{figure}[H]
\centering
\myincludegraphics{patterns/08_switch/1_few/olly4.png}
\caption{\olly: 2回目の \DEC 実行}
\label{fig:switch_few_olly4}
\end{figure}

\olly は、このジャンプが今行われることを示しています。

\clearpage
\q{two}という文字列へのポインタが今スタックに書き込まれます:

\begin{figure}[H]
\centering
\myincludegraphics{patterns/08_switch/1_few/olly5.png}
\caption{\olly: 
文字列へのポインタは、最初の引数の場所に書き込まれる}
\label{fig:switch_few_olly5}
\end{figure}

% TODO: homogenize numbers
% now they are inconsistent: sometimes plain text, sometimes in math mode
% some kind of \expr{} both for numbers and expressions? --DY
注意:関数の現在の引数は2であり、2はスタックに\TT{0x001EF850}のアドレスにあります。

\clearpage
\MOV はアドレス\TT{0x001EF850}の文字列にポインタを書き込みます(スタックウィンドウを参照)。
その後、ジャンプが発生します。
これはMSVCR100.DLLの \printf 関数の最初の命令です(この例は/MDスイッチでコンパイルされています)。

\begin{figure}[H]
\centering
\myincludegraphics{patterns/08_switch/1_few/olly6.png}
\caption{\olly: MSVCR100.DLLでの \printf の最初の命令}
\label{fig:switch_few_olly6}
\end{figure}

今や \printf は\TT{0x00FF3010}の文字列を唯一の引数として扱い、文字列を出力します。

\clearpage
これが \printf の最後の命令です。

\begin{figure}[H]
\centering
\myincludegraphics{patterns/08_switch/1_few/olly7.png}
\caption{\olly: MSVCR100.DLLの \printf の最後の命令}
\label{fig:switch_few_olly7}
\end{figure}

文字列 \q{two} はコンソールウィンドウに表示されます。

\clearpage
F7またはF8を押して(\stepover) リターンすると\dots \ttf ではなく、 \main にいきます。

\begin{figure}[H]
\centering
\myincludegraphics{patterns/08_switch/1_few/olly8.png}
\caption{\olly: \main へのリターン}
\label{fig:switch_few_olly8}
\end{figure}

はい、 \printf の中心から \main に直接ジャンプしました。
なぜならスタックの\ac{RA}は \ttf ではなく、 \main の場所を指しているからです。
\CALL \TT{0x00FF1000}は \ttf を呼び出した実際の命令です。


