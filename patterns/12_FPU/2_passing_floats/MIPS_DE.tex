\subsubsection{MIPS}

\lstinputlisting[caption=\Optimizing GCC 4.4.5 (IDA),style=customasmMIPS]{patterns/12_FPU/2_passing_floats/MIPS_O3_IDA_DE.lst}
Und wieder sehen wir hier, dass der Befehl \INS{LUI} einen 32-Bit-Teil einer
\Tdouble Zahl nach \$V0 lädt.
Und wiederum ist es schwer nachzuvollziehen warum dies geschieht.

\myindex{MIPS!\Instructions!MFC1}
Der für uns neue Befehl an dieser Stelle ist \INS{MFC1}(\q{Move From Coprocessor
1}). Die Nummer des FPU-Koprozessors ist 1, daher die \q{1} im Namen des
Befehls. 
Dieser Befehl überträgt Werte aus den Registern des Koprozessors in die Register
der CPU (\ac{GPR}).
Auf diese Weise wird das Ergebnis von \TT{pow()} schließlich in die Register
\$A3 und \$A2 verschoben und \printf übernimmt einen 64-Bit-Wert von doppelter
Genauigkeit aus diesem Registerpaar.
