\mysection{\FPUChapterName}
\label{sec:FPU}

\newcommand{\FNURLSTACK}{\footnote{\href{http://go.yurichev.com/17123}{wikipedia.org/wiki/Stack\_machine}}}
\newcommand{\FNURLFORTH}{\footnote{\href{http://go.yurichev.com/17124}{wikipedia.org/wiki/Forth\_(programming\_language)}}}
\newcommand{\FNURLIEEE}{\footnote{\href{http://go.yurichev.com/17125}{wikipedia.org/wiki/IEEE\_floating\_point}}}
\newcommand{\FNURLSP}{\footnote{\href{http://go.yurichev.com/17126}{wikipedia.org/wiki/Single-precision\_floating-point\_format}}}
\newcommand{\FNURLDP}{\footnote{\href{http://go.yurichev.com/17127}{wikipedia.org/wiki/Double-precision\_floating-point\_format}}}
\newcommand{\FNURLEP}{\footnote{\href{http://go.yurichev.com/17128}{wikipedia.org/wiki/Extended\_precision}}}

The \ac{FPU} is a device within the main \ac{CPU}, specially designed to deal with floating point numbers.

It was called \q{coprocessor} in the past and it stays somewhat aside of the main \ac{CPU}.

\subsection{IEEE 754}

A number in the IEEE 754 format consists of a \emph{sign}, a \emph{significand} (also called \emph{fraction}) and an \emph{exponent}.

\subsection{x86}

It is worth looking into stack machines\FNURLSTACK or learning the basics of the Forth language\FNURLFORTH,
before studying the \ac{FPU} in x86.

\myindex{Intel!80486}
\myindex{Intel!FPU}
It is interesting to know that in the past (before the 80486 CPU) the coprocessor was a separate chip 
and it was not always pre-installed on the motherboard. It was possible to buy it separately and install it
\footnote{For example, John Carmack used fixed-point arithmetic 
(\href{http://go.yurichev.com/17356}{wikipedia.org/wiki/Fixed-point\_arithmetic}) values in his Doom video game, stored in 
32-bit \ac{GPR} registers (16 bit for integral part and another 16 bit for fractional part), so Doom
could work on 32-bit computers without FPU, i.e., 80386 and 80486 SX.}.

Starting with the 80486 DX CPU, the \ac{FPU} is integrated in the \ac{CPU}.

\myindex{x86!\Instructions!FWAIT}
The \INS{FWAIT} instruction reminds us of that fact---it switches the \ac{CPU} to a waiting state, so it can wait until the \ac{FPU} has finished with its work.

Another rudiment is the fact that the \ac{FPU} instruction 
opcodes start with the so called \q{escape}-opcodes (\GTT{D8..DF}), i.e., 
opcodes passed to a separate coprocessor.

\myindex{IEEE 754}
\label{FPU_is_stack}

The FPU has a stack capable to holding 8 80-bit registers, and each register can hold a number 
in the IEEE 754\FNURLIEEE format.

They are \ST{0}..\ST{7}. For brevity, \IDA and \olly show \ST{0} as \GTT{ST}, 
which is represented in some textbooks and manuals as \q{Stack Top}.

\subsection{ARM, MIPS, x86/x64 SIMD}

In ARM and MIPS the FPU is not a stack, but a set of registers, which can be accessed randomly, like \ac{GPR}.

The same ideology is used in the SIMD extensions of x86/x64 CPUs.

\subsection{\CCpp}

\myindex{float}
\myindex{double}

The standard \CCpp languages offer at least two floating number types, \Tfloat (\emph{single-precision}\FNURLSP, 32 bits)
\footnote{the single precision floating point number format is also addressed in 
the \emph{\WorkingWithFloatAsWithStructSubSubSectionName}~(\myref{sec:floatasstruct}) section}
and \Tdouble (\emph{double-precision}\FNURLDP, 64 bits).

In \InSqBrackets{\TAOCPvolII 246} we can find the \emph{single-precision} means that the floating point value can be placed into a single
[32-bit] machine word, \emph{double-precision} means it can be stored in two words (64 bits).

\myindex{long double}

GCC also supports the \emph{long double} type (\emph{extended precision}\FNURLEP, 80 bit), which MSVC doesn't.

The \Tfloat type requires the same number of bits as the \Tint type in 32-bit environments, 
but the number representation is completely different.

\ifdefined\RUSSIAN
\subsection{Простой пример}

Рассмотрим простой пример:
\fi

\ifdefined\ENGLISH
\subsection{Simple example}

Let's consider this simple example:
\fi

\ifdefined\GERMAN
\subsection{\DEph{}}

\DEph{}

\fi

\ifdefined\FRENCH
\subsection{Exemple simple}

Considérons cet exemple simple:
\fi

\ifdefined\JAPANESE
\subsection{簡単な例}

この簡単な例を考えてみましょう。
\fi

\lstinputlisting[style=customc]{patterns/12_FPU/1_simple/simple.c}

\subsubsection{x86}

% subsubsections
\EN{\myparagraph{MSVC}

Compile it in MSVC 2010:

\lstinputlisting[caption=MSVC 2010: \ttf{},style=customasmx86]{patterns/12_FPU/1_simple/MSVC_EN.asm}

\FLD takes 8 bytes from stack and loads the number into the \ST{0} register, automatically converting 
it into the internal 80-bit format (\emph{extended precision}).

\myindex{x86!\Instructions!FDIV}

\FDIV divides the value in \ST{0} by the number stored at address \\
\GTT{\_\_real@40091eb851eb851f}~---the value 3.14 is encoded there. 
The assembly syntax doesn't support floating point numbers, so 
what we see here is the hexadecimal representation of 3.14 in 64-bit IEEE 754 format.

After the execution of \FDIV \ST{0} holds the \gls{quotient}.

\myindex{x86!\Instructions!FDIVP}

By the way, there is also the \FDIVP instruction, which divides \ST{1} by \ST{0}, 
popping both these values from stack and then pushing the result. 
If you know the Forth language\FNURLFORTH,
you can quickly understand that this is a stack machine\FNURLSTACK.

The subsequent \FLD instruction pushes the value of $b$ into the stack.

After that, the quotient is placed in \ST{1}, and \ST{0} has the value of $b$.

\myindex{x86!\Instructions!FMUL}

The next \FMUL instruction does multiplication: $b$ from \ST{0} is multiplied by value at\\
\GTT{\_\_real@4010666666666666} (the number 4.1 is there) and leaves the result in the \ST{0} register.

\myindex{x86!\Instructions!FADDP}

The last \FADDP instruction adds the two values at top of stack, storing the result in \ST{1} 
and then popping the value of \ST{0}, thereby leaving the result at the top of the stack, in \ST{0}.

The function must return its result in the \ST{0} register, 
so there are no any other instructions except the function epilogue after \FADDP.

\input{patterns/12_FPU/1_simple/olly_EN.tex}
}
\RU{\myparagraph{MSVC}

Компилируем в MSVC 2010:

\lstinputlisting[caption=MSVC 2010: \ttf{},style=customasmx86]{patterns/12_FPU/1_simple/MSVC_RU.asm}

\FLD берет 8 байт из стека и загружает их в регистр \ST{0}, автоматически конвертируя во внутренний 
80-битный формат (\emph{extended precision}).

\myindex{x86!\Instructions!FDIV}
\FDIV делит содержимое регистра \ST{0} на число, лежащее по адресу \GTT{\_\_real@40091eb851eb851f}~--- 
там закодировано значение 3,14. Синтаксис ассемблера не поддерживает подобные числа, 
поэтому мы там видим шестнадцатеричное представление числа 3,14 в формате IEEE 754.

После выполнения \FDIV в \ST{0} остается \glslink{quotient}{частное}.

\myindex{x86!\Instructions!FDIVP}
Кстати, есть ещё инструкция \FDIVP, которая делит \ST{1} на \ST{0}, 
выталкивает эти числа из стека и заталкивает результат. 
Если вы знаете язык Forth\FNURLFORTH, то это как раз оно и есть~--- стековая машина\FNURLSTACK.

Следующая \FLD заталкивает в стек значение $b$.

После этого в \ST{1} перемещается результат деления, а в \ST{0} теперь $b$.

\myindex{x86!\Instructions!FMUL}
Следующий \FMUL умножает $b$ из \ST{0} на значение \\
\GTT{\_\_real@4010666666666666} --- там лежит число 4,1~--- и оставляет результат в \ST{0}.

\myindex{x86!\Instructions!FADDP}
Самая последняя инструкция \FADDP складывает два значения из вершины стека 
в \ST{1} и затем выталкивает значение, лежащее в \ST{0}. 
Таким образом результат сложения остается на вершине стека в \ST{0}.

Функция должна вернуть результат в \ST{0}, так что больше ничего здесь не производится, 
кроме эпилога функции.

\input{patterns/12_FPU/1_simple/olly_RU.tex}
}
\DE{\myparagraph{MSVC}

Kompilieren mit MSVC 2010 liefert:

\lstinputlisting[caption=MSVC 2010: \ttf{},style=customasmx86]{patterns/12_FPU/1_simple/MSVC_DE.asm}

\FLD nimmt 8 Byte vom Stack und lädt die Zahl in das \ST{0} Register, wobei
diese automatisch in das interne 80-bit-Format (\emph{erweiterte Genauigkeit})
konvertiert wird.

\myindex{x86!\Instructions!FDIV}
\FDIV teilt den Wert in \ST{0} durch die Zahl, die an der Adresse
\GTT{\_\_real@40091eb851eb851f} gespeichert ist~---der Wert 3.14 ist hier
kodiert.
Die Syntax des Assemblers erlaubt keine Fließkommazahlen, sodass wir hier die
hexadezimale Darstellung von 3.14 im 64-bit IEEE 754 Format finden.

Nach der Ausführung von \FDIV enthält \ST{0} den \glslink{quotient}{Quotienten}.

\myindex{x86!\Instructions!FDIVP}
Es gibt übrigens auch noch den \FDIVP Befehl, welcher \ST{1} durch \ST{0}
teilt, beide Werte vom Stack holt und das Ergebnis ebenfalls auf dem Stack
ablegt.
Wer mit der Sprache Forth \FNURLFORTH vertraut ist, erkennt schnell, dass es sich
hier um eine Stackmaschine\FNURLSTACK handelt.

Der nachfolgende \FLD Befehl speichert den Wert von $b$ auf dem Stack.

Anschließend wir der Quotient in \ST{1} abgelegt und \ST{0} enthält den Wert von
$b$.

\myindex{x86!\Instructions!FMUL}
Der nächste \FMUL Befehl führt folgende Multiplikation aus: $b$ aus Register
\ST{0} wird mit dem Wert an der Speicherstelle \GTT{\_\_real@4010666666666666}
(hier befindet sich die Zahl 4.1) multipliziert und hinterlässt das Ergebnis im
\ST{ß} Register.

\myindex{x86!\Instructions!FADDP}
Der letzte \FADDP Befehl addiert die beiden Werte, die auf dem Stack zuoberst
liegen, speichet das Ergebnis in \ST{1} und holt dann den Wert von \ST{0} vom
Stack, wobei das oberste Element auf dem Stack in \ST{0} gespeichert wird.

Die Funktion muss ihr Ergebnis im \ST{0} Register zurückgeben, sodass außer dem
Funktionsepilog nach \FADDP keine weiteren Befehle mehr folgen.

\input{patterns/12_FPU/1_simple/olly_DE.tex}
}
\FR{\myparagraph{MSVC}

Compilons-le avec MSVC 2010:

\lstinputlisting[caption=MSVC 2010: \ttf{},style=customasmx86]{patterns/12_FPU/1_simple/MSVC_FR.asm}

\FLD prend 8 octets depuis la pile et charge le nombre dans le registre \ST{0}, en
le convertissant automatiquement dans le format interne sur 80-bit (\emph{précision
étendue}):

\myindex{x86!\Instructions!FDIV}

\FDIV divise la valeur dans \ST{0} par le nombre stocké à l'adresse \\
\GTT{\_\_real@40091eb851eb851f}~---la valeur 3.14 est encodée ici.
La syntaxe assembleur ne supporte pas les nombres à virgule flottante, donc ce que
l'on voit ici est la représentation hexadécimale de 3.14 au format 64-bit IEEE 754.

Après l'exécution de \FDIV, \ST{0} contient le \gls{quotient}.

\myindex{x86!\Instructions!FDIVP}

À propos, il y a aussi l'instruction \FDIVP, qui divise \ST{1} par \ST{0}, prenant
ces deux valeurs dans la pile et poussant le résultant.
Si vous connaissez le langage Forth\FNURLFORTH, vous pouvez comprendre rapidement
que ceci est une machine à pile\FNURLSTACK.

L'instruction \FLD subséquente pousse la valeur de $b$ sur la pile.

Après cela, le quotient est placé dans \ST{1}, et \ST{0} a la valeur de $b$.

\myindex{x86!\Instructions!FMUL}

L'instruction suivante effectue la multiplication: $b$ de \ST{0} est multiplié par
la valeur en\\
\GTT{\_\_real@4010666666666666} (le nombre 4.1 est là) et met le résultat
dans le registre \ST{0}.

\myindex{x86!\Instructions!FADDP}

La dernière instruction \FADDP ajoute les deux valeurs au sommet de la pile, stockant
le résultat dans \ST{1} et supprimant la valeur de \ST{0}, laissant ainsi le résultat
au sommet de la pile, dans \ST{0}.

La fonction doit renvoyer son résultat dans le registre \ST{0}, donc il n'y a aucune
autre instruction après \FADDP, excepté l'épilogue de la fonction.

\input{patterns/12_FPU/1_simple/olly_FR.tex}
}
\JA{\myparagraph{MSVC}

MSVC 2010でコンパイルしましょう。

\lstinputlisting[caption=MSVC 2010: \ttf{},style=customasmx86]{patterns/12_FPU/1_simple/MSVC_JA.asm}

\FLD はスタックから8バイトを取り出し、その数値を\ST{0}レジスタにロードし、内部80ビットフォーマット
(\emph{拡張精度})に自動的に変換します。

\myindex{x86!\Instructions!FDIV}

\FDIV は、\ST{0}の値をアドレス\GTT{\_\_real@40091eb851eb851f}~に格納された数値で除算します。
値3.14はそこにエンコードされます。
アセンブリ構文は浮動小数点数をサポートしていないので、64ビットIEEE 754形式での3.14の16進表現です。

\FDIV \ST{0}の実行後に\gls{quotient}が保持されます。

\myindex{x86!\Instructions!FDIVP}

ちなみに、 \FDIVP 命令もあります。これは、\ST{1}を\ST{0}で除算し、
これらの値をスタックからポップし、その結果をプッシュします。
あなたがForth言語\FNURLFORTH を知っていれば、
すぐにこれがスタックマシン\FNURLSTACK であることがわかります。

後続の \FLD 命令は、 $b$ の値をスタックにプッシュします。

その後、商は\ST{1}に置かれ、\ST{0}は $b$ の値を持ちます。

\myindex{x86!\Instructions!FMUL}

次の \FMUL 命令は乗算を行います。\ST{0}の $b$ は\GTT{\_\_real@4010666666666666}
(そこには4.1が入る)の値で乗算され、結果は\ST{0}レジスタに残ります。

\myindex{x86!\Instructions!FADDP}

最後の \FADDP 命令は、スタックの先頭に2つの値を加算し、結果を\ST{1}に格納した後、
\ST{0}の値をポップし、\ST{0}のスタックの先頭に結果を残します。

関数はその結果を\ST{0}レジスタに戻す必要があるため、
\FADDP 後の関数エピローグ以外の命令はありません。

\input{patterns/12_FPU/1_simple/olly_JA.tex}
}

\EN{\myparagraph{GCC}

GCC 4.4.1 (with \Othree option) emits the same code, just slightly different:

\lstinputlisting[caption=\Optimizing GCC 4.4.1,style=customasmx86]{patterns/12_FPU/1_simple/GCC_EN.asm}

The difference is that, first of all, 3.14 is pushed to the stack (into \ST{0}), and then the value 
in \GTT{arg\_0} is divided by the value in \ST{0}.

\myindex{x86!\Instructions!FDIVR}

\FDIVR stands for \emph{Reverse Divide}~---to divide with divisor and dividend swapped with each other. 
There is no likewise instruction for multiplication since it is 
a commutative operation, so we just have \FMUL without its \GTT{-R} counterpart.

\myindex{x86!\Instructions!FADDP}

\FADDP adds the two values but also pops one value from the stack. 
After that operation, \ST{0} holds the sum.

}
\RU{\myparagraph{GCC}

GCC 4.4.1 (с опцией \Othree) генерирует похожий код, хотя и с некоторой разницей:

\lstinputlisting[caption=\Optimizing GCC 4.4.1,style=customasmx86]{patterns/12_FPU/1_simple/GCC_RU.asm}

Разница в том, что в стек сначала заталкивается 3,14 (в \ST{0}), а затем значение 
из \GTT{arg\_0} делится на то, что лежит в регистре \ST{0}.

\myindex{x86!\Instructions!FDIVR}
\FDIVR означает \emph{Reverse Divide}~--- делить, поменяв делитель и делимое местами. 
Точно такой же инструкции для умножения нет, потому что она была бы бессмысленна (ведь умножение 
операция коммутативная), так что остается только \FMUL без соответствующей ей \GTT{-R} инструкции.

\myindex{x86!\Instructions!FADDP}
\FADDP не только складывает два значения, но также и выталкивает из стека одно значение. 
После этого в \ST{0} остается только результат сложения.
}
\DE{\myparagraph{GCC}

GCC 4.4.1 (mit der Option \Othree) erzeugt fast den gleichen Code, nur leicht
verändert.

\lstinputlisting[caption=\Optimizing GCC 4.4.1,style=customasmx86]{patterns/12_FPU/1_simple/GCC_DE.asm} Der Unterschied
besteht darin, dass zuerst 3.14 auf dem Stack (in \ST{0}) abgelegt wird und danach der Wert in \GTT{arg\_0} durch den Wert in \ST{0}
geteilt wird.

\myindex{x86!\Instructions!FDIVR}
\FDIVR steht für \emph{Reverse Divide}~--teilen, wobei Dividend und Divisor
miteinander vertauscht werden. Da es sich bei der Multiplikation um eine
kommutative Operation handelt, gibt es keinen vergleichbaren Befehl für die
Multiplikation. Wir haben es lediglich \FMUL ohne \GTT{-R} Gegenstück zur
Verfügung.

\myindex{x86!\Instructions!FADDP}
\FADDP addiert die beiden Werte und holt auch einen Wert vom Stack. 
Nach der Ausführung steht die Summe in \ST{0}.

}
\FR{\myparagraph{GCC}

GCC 4.4.1 (avec l'option \Othree) génère le même code, juste un peu différent:

\lstinputlisting[caption=GCC 4.4.1 \Optimizing,style=customasmx86]{patterns/12_FPU/1_simple/GCC_FR.asm}

La différence est que, tout d'abord, 3.14 est poussé sur la pile (dans \ST{0}), et
ensuite la valeur dans \GTT{arg\_0} est divisée par la valeur dans \ST{0}.

\myindex{x86!\Instructions!FDIVR}

\FDIVR signifie \emph{Reverse Divide}~---pour diviser avec le diviseur et le dividende
échangés l'un avec l'autre.
Il n'y a pas d'instruction de ce genre pour la multiplication puisque c'est une opération
commutative, donc nous avons seulement \FMUL sans son homologue \GTT{-R}.

\myindex{x86!\Instructions!FADDP}

\FADDP ajoute les deux valeurs mais supprime aussi une valeur de la pile.
Après cette opération, \ST{0} contient la somme.

}
\JA{\myparagraph{GCC}

GCC 4.4.1( \Othree オプション付き)は、わずかに異なる同じコードを出力します:

\lstinputlisting[caption=\Optimizing GCC 4.4.1,style=customasmx86]{patterns/12_FPU/1_simple/GCC_JA.asm}

違いは、まず3.14がスタック(\ST{0})にプッシュされ、\GTT{arg\_0}の値が\ST{0}の値で除算される点です。

\myindex{x86!\Instructions!FDIVR}

\FDIVR は、\emph{Reverse Divide}の略で、除数と配当を入れ替えて割ります。 
同様に乗算命令はありません。これは可換演算であるため、 \FMUL には\GTT{-R}の部分がなくてもかまいません。

\myindex{x86!\Instructions!FADDP}

\FADDP は2つの値を加算するだけでなく、スタックから値を1つポップします。 
その操作の後、\ST{0}は合計を保持します。
}


\EN{\subsubsection{ARM: \OptimizingXcodeIV (\ARMMode)}

Until ARM got standardized floating point support, several processor manufacturers added their own 
instructions extensions.
Then, VFP (\emph{Vector Floating Point}) was standardized.

One important difference from x86 is that in ARM, there
is no stack, you work just with registers.

\lstinputlisting[label=ARM_leaf_example10,caption=\OptimizingXcodeIV (\ARMMode),style=customasmARM]{patterns/12_FPU/1_simple/ARM/Xcode_ARM_O3_EN.asm}

\myindex{ARM!D-\registers{}}
\myindex{ARM!S-\registers{}}

So, we see here new some registers used, with D prefix.

These are 64-bit registers, there are 32 of them, and they can be used both for floating-point numbers 
(double) but also for SIMD (it is called NEON here in ARM).

There are also 32 32-bit S-registers, intended to be used for single precision 
floating pointer numbers (float).

It is easy to memorize: D-registers are for double precision numbers, while
S-registers---for single precision numbers.
More about it: \myref{ARM_VFP_registers}.

Both constants (3.14 and 4.1) are stored in memory in IEEE 754 format.

\myindex{ARM!\Instructions!VLDR}
\myindex{ARM!\Instructions!VMOV}
\INS{VLDR} and \INS{VMOV}, as it can be easily deduced, are analogous to the \INS{LDR} and \MOV instructions,
but they work with D-registers.

It has to be noted that these instructions, just like the D-registers, are intended not only for
floating point numbers, 
but can be also used for SIMD (NEON) operations and this will also be shown soon.

The arguments are passed to the function in a common way, via the R-registers, however
each number that has double precision has a size of 64 bits, so two R-registers are needed to pass each one.

\INS{VMOV D17, R0, R1} at the start, composes two 32-bit values from \Reg{0} and \Reg{1} into one 64-bit value
and saves it to \GTT{D17}.

\INS{VMOV R0, R1, D16} is the inverse operation: what has been in \GTT{D16} 
is split in two registers, \Reg{0} and \Reg{1}, because a double-precision number 
that needs 64 bits for storage, is returned in \Reg{0} and \Reg{1}.

\myindex{ARM!\Instructions!VDIV}
\myindex{ARM!\Instructions!VMUL}
\myindex{ARM!\Instructions!VADD}
\INS{VDIV}, \INS{VMUL} and \INS{VADD}, 
are instruction for processing floating point numbers that compute \gls{quotient}, 
\gls{product} and sum, respectively.

The code for Thumb-2 is same.

\subsubsection{ARM: \OptimizingKeilVI (\ThumbMode)}

\lstinputlisting[style=customasmARM]{patterns/12_FPU/1_simple/ARM/Keil_O3_thumb_EN.asm}

Keil generated code for a processor without FPU or NEON support.

The double-precision floating-point numbers are passed via generic R-registers,
and instead of FPU-instructions, service library functions are called\\
(like \GTT{\_\_aeabi\_dmul}, \GTT{\_\_aeabi\_ddiv}, \GTT{\_\_aeabi\_dadd})
which emulate multiplication, division and addition for floating-point numbers.

Of course, that is slower than FPU-coprocessor, but it's still better than nothing.

By the way, similar FPU-emulating libraries were very popular in the x86 world when coprocessors were rare
and expensive, and were installed only on expensive computers.

\myindex{ARM!soft float}
\myindex{ARM!armel}
\myindex{ARM!armhf}
\myindex{ARM!hard float}

The FPU-coprocessor emulation is called \emph{soft float} or \emph{armel} (\emph{emulation}) in the ARM world, 
while using the coprocessor's FPU-instructions is called \emph{hard float} or \emph{armhf}.

\iffalse
% TODO разобраться...
\myindex{Raspberry Pi}

For example, the Linux kernel for Raspberry Pi is compiled in two variants.

In the \emph{soft float} case, arguments are passed via R-registers, and in the \emph{hard float} case---via D-registers.

And that is what stops you from using armhf-libraries from armel-code or vice versa,
and that is why all the code in Linux distributions must be compiled according to a single convention.
\fi

\subsubsection{ARM64: \Optimizing GCC (Linaro) 4.9}

Very compact code:

\lstinputlisting[caption=\Optimizing GCC (Linaro) 4.9,style=customasmARM]{patterns/12_FPU/1_simple/ARM/ARM64_GCC_O3_EN.s}

\subsubsection{ARM64: \NonOptimizing GCC (Linaro) 4.9}

\lstinputlisting[caption=\NonOptimizing GCC (Linaro) 4.9,style=customasmARM]{patterns/12_FPU/1_simple/ARM/ARM64_GCC_O0_EN.s}

\NonOptimizing GCC is more verbose.

There is a lot of unnecessary value shuffling, including some clearly redundant code 
(the last two \INS{FMOV} instructions). Probably, GCC 4.9 is not yet good in generating ARM64 code.

What is worth noting is that ARM64 has 64-bit registers, and the D-registers are 64-bit ones as well.

So the compiler is free to save values of type \Tdouble in \ac{GPR}s instead of the local stack.
This isn't possible on 32-bit CPUs.

And again, as an exercise, you can try to optimize this function manually, without introducing
new instructions like \INS{FMADD}.
}
\RU{\subsubsection{ARM: \OptimizingXcodeIV (\ARMMode)}

Пока в ARM не было стандартного набора инструкций для работы с числами с плавающей точкой, разные производители процессоров
могли добавлять свои расширения для работы с ними.
Позже был принят стандарт VFP (\emph{Vector Floating Point}).

Важное отличие от x86 в том, что там вы работаете с FPU-стеком, а здесь стека нет, вы работаете просто с регистрами.

\lstinputlisting[label=ARM_leaf_example10,caption=\OptimizingXcodeIV (\ARMMode),style=customasmARM]{patterns/12_FPU/1_simple/ARM/Xcode_ARM_O3_RU.asm}

\myindex{ARM!D-\registers{}}
\myindex{ARM!S-\registers{}}
Итак, здесь мы видим использование новых регистров с префиксом D.

Это 64-битные регистры. Их 32 и их можно
использовать для чисел с плавающей точкой двойной точности (double) и для 
SIMD (в ARM это называется NEON).

Имеются также 32 32-битных S-регистра. Они применяются для работы с числами 
с плавающей точкой одинарной точности (float).

Запомнить легко: D-регистры предназначены для чисел double-точности, 
а S-регистры~--- для чисел single-точности.

Больше об этом: \myref{ARM_VFP_registers}.

Обе константы (3,14 и 4,1) хранятся в памяти в формате IEEE 754.

\myindex{ARM!\Instructions!VLDR}
\myindex{ARM!\Instructions!VMOV}
Инструкции \INS{VLDR} и \INS{VMOV}, как можно догадаться, это аналоги обычных \INS{LDR} и \MOV, но они работают с D-регистрами.

Важно отметить, что эти инструкции, как и D-регистры, предназначены не только для работы 
с числами с плавающей точкой, но пригодны также и для работы с SIMD (NEON), и позже это также будет видно.

Аргументы передаются в функцию обычным путем через R-регистры, однако 
каждое число, имеющее двойную точность, занимает 64 бита, так что для передачи каждого нужны два R-регистра.

\INS{VMOV D17, R0, R1} в самом начале составляет два 32-битных значения из \Reg{0} и \Reg{1} в одно 64-битное и сохраняет в \GTT{D17}.

\INS{VMOV R0, R1, D16} в конце это обратная процедура: то что было в \GTT{D16} 
остается в двух регистрах \Reg{0} и \Reg{1}, потому что число с двойной точностью, 
занимающее 64 бита, возвращается в паре регистров \Reg{0} и \Reg{1}.

\myindex{ARM!\Instructions!VDIV}
\myindex{ARM!\Instructions!VMUL}
\myindex{ARM!\Instructions!VADD}
\INS{VDIV}, \INS{VMUL} и \INS{VADD}, это инструкции для работы с числами 
с плавающей точкой, вычисляющие, соответственно, \glslink{quotient}{частное}, \glslink{product}{произведение} и сумму.

Код для Thumb-2 такой же.

\subsubsection{ARM: \OptimizingKeilVI (\ThumbMode)}

\lstinputlisting[style=customasmARM]{patterns/12_FPU/1_simple/ARM/Keil_O3_thumb_RU.asm}

Keil компилировал для процессора, в котором может и не быть поддержки FPU или NEON.
Так что числа с двойной точностью передаются в парах обычных R-регистров,
а вместо FPU-инструкций вызываются сервисные библиотечные функции\\
\GTT{\_\_aeabi\_dmul}, \GTT{\_\_aeabi\_ddiv}, \GTT{\_\_aeabi\_dadd}, эмулирующие умножение, деление и сложение чисел с плавающей точкой.

Конечно, это медленнее чем FPU-сопроцессор, но это лучше, чем ничего.

Кстати, похожие библиотеки для эмуляции сопроцессорных инструкций были очень распространены в x86 
когда сопроцессор был редким и дорогим и присутствовал далеко не во всех компьютерах.

\myindex{ARM!soft float}
\myindex{ARM!armel}
\myindex{ARM!armhf}
\myindex{ARM!hard float}
Эмуляция FPU-сопроцессора в ARM называется \emph{soft float} или \emph{armel} (\emph{emulation}),
а использование FPU-инструкций сопроцессора~--- \emph{hard float} или \emph{armhf}.

\iffalse
% TODO разобраться...
\myindex{Raspberry Pi}
Ядро Linux, например, для Raspberry Pi может поставляться в двух вариантах.

В случае \emph{soft float}, аргументы будут передаваться через R-регистры, 
а в случае \emph{hard float}, через D-регистры.


И это то, что помешает использовать, например, armhf-библиотеки
из armel-кода или наоборот, поэтому, весь код в дистрибутиве Linux должен быть скомпилирован
в соответствии с выбранным соглашением о вызовах.

\fi

\subsubsection{ARM64: \Optimizing GCC (Linaro) 4.9}

Очень компактный код:

\lstinputlisting[caption=\Optimizing GCC (Linaro) 4.9,style=customasmARM]{patterns/12_FPU/1_simple/ARM/ARM64_GCC_O3_RU.s}

\subsubsection{ARM64: \NonOptimizing GCC (Linaro) 4.9}

\lstinputlisting[caption=\NonOptimizing GCC (Linaro) 4.9,style=customasmARM]{patterns/12_FPU/1_simple/ARM/ARM64_GCC_O0_RU.s}

\NonOptimizing GCC более многословный.
Здесь много ненужных перетасовок значений, включая явно избыточный код 
(последние две инструкции \INS{GMOV}).
Должно быть, GCC 4.9 пока ещё не очень хорош для генерации кода под ARM64.
Интересно заметить что у ARM64 64-битные регистры и D-регистры так же 64-битные.
Так что компилятор может сохранять значения типа \Tdouble в \ac{GPR} вместо локального стека.
Это было невозможно на 32-битных CPU.
И снова, как упражнение, вы можете попробовать соптимизировать эту функцию вручную, без добавления
новых инструкций вроде \INS{FMADD}.

}
\DE{%TODO
\subsubsection{ARM: \OptimizingXcodeIV (\ARMMode)}
Bis die Unterstützung für Fließkommaarithmetik in ARM standardisiert wurde,
fügten einige Hersteller von Prozessoren ihre eigenen Befehlserweiterungen
hinzu. 
Schließlich wurde VFP (\emph{Vector Floating Point}) standardisiert.

Ein wichtiger Unterschied zum x86 ist, dass in es in ARM keinen Stack gibt,
sondern man nur mit den Registern arbeitet.

\lstinputlisting[label=ARM_leaf_example10,caption=\OptimizingXcodeIV (\ARMMode),style=customasmARM]{patterns/12_FPU/1_simple/ARM/Xcode_ARM_O3_DE.asm}

\myindex{ARM!D-\registers{}}
\myindex{ARM!S-\registers{}}

Hier sehen wir, dass einige neue Register mit einem D als Präfix verwendet
werden.

Bei diesen handelt es sich um 64-bit-Register; es gibt 32 von ihnen und sie
können sowohl für Fließkommazahlen (doppelte Genauigkeit (double)) als auch für
SIMD (heißt hier in ARM NEON) benutzt werden.

Es gibt also 32 32-bit-S-Register vorgesehen für Fließkommazahlen in einfacher
Genauigkeit (float).

Es ist leicht zu merken: D-Register sind für Zahlen in doppelter Genauigkeit,
während S-Register für einfache Genauigkeit (engl. single) vorgesehen sind.
Mehr dazu hier:\myref{ARM_VFP_registers}

Beide Konstanten (3.14 und 4.1) werden im IEEE 754 Format im Speicher abgelegt.

\myindex{ARM!\Instructions!VLDR}
\myindex{ARM!\Instructions!VMOV}
Wie man leicht sieht sind \INS{VLDR} und \INS{VMOV} analog zu den \INS{LDR} und
\MOV Befehlen, aber arbeiten auf D-Registern.

Es muss angemerkt werden, dass diese Befehle genau wie die D-Register nicht nur
für Fließkommazahlen vorgesehen sind, sondern ebenfalls für SIMD (NEON)
Operationen verwendet werden können, was wir im folgenden zeigen werden.

Die Paraemter werden der Funktion auf übliche Weise über die R-Register
übergeben, aber da jede Zahl in doppelter Genauigkeit eine Größe von 64 Bit hat
werden jeweils zwei R-Register benötigt, um eine Zahl zu übergeben.

Der Befehl \INS{VMOV D17, R0, R1} zu Beginn, fasst zwei 32-Bit-Werte aus
\Reg{0} und \Reg{1} zu einem 64-Bit-Wert zusammen und speichert diesen in
\GTT{D17}.

\INS{VMOV R0, R1, D16} ist die umgekehrte Operation: was vorher in \GTT{D16}
war, wird in zwei Register, \Reg{0} und \Reg{1} aufgeteilt, denn eine Zahl in
doppelter Genauigkeit, die 64 Bit Speicherplatz benötigt, wird über \Reg{0} und
\Reg{1} zurückgegeben.

\myindex{ARM!\Instructions!VDIV}
\myindex{ARM!\Instructions!VMUL}
\myindex{ARM!\Instructions!VADD}
\INS{VDIV}, \INS{VMUL} und \INS{VADD} sind Befehle zur Verarbeitung von
Fließkommazahlen, die \glslink{quotient}{Quotient}, \glslink{product}{Produkt} bzw. Summe berechnen.

Der Code für Thumb-2 ist identisch.

\subsubsection{ARM: \OptimizingKeilVI (\ThumbMode)}

\lstinputlisting[style=customasmARM]{patterns/12_FPU/1_simple/ARM/Keil_O3_thumb_DE.asm}

Keil erzeugte Code für einen Prozessor ohne FPU oder NEON Unterstützung.

Die Fließkommazahlen in doppelter Genauigkeit werden über die üblichen
R-Register übergeben und anstelle von FPU-Befehlen werden Programmbibliotheken
(wie z.B. \GTT{\_\_aeabi\_dmul}, \GTT{\_\_aeabi\_ddiv}, \GTT{\_\_aeabi\_dadd})
aufgerufen, welche Multiplikation, Division und Addition auf Fließkommazahlen
emulieren. 

Diese Vorgehensweise ist natürlich langsamer als der FPU-Koprozessor, aber es
ist besser als nichts.

Übrigens waren ähnliche FPU-emulierende Programmbibliotheken auch in der
x86-Welt sehr beliebt als Koprozessoren selten und teuer waren und nur auf
wertvollen Computern installiert waren.

\myindex{ARM!soft float}
\myindex{ARM!armel}
\myindex{ARM!armhf}
\myindex{ARM!hard float}
Die Emulation des FPU-Koprozessors wird \emph{soft float} oder \emph{armel} (in der
ARM-Welt) genannt, wohingegen die FPU-Befehle des Koprozessors \emph{hard float}
oder \emph{armhf} genannt werden.

\iffalse
% TODO разобраться...
\myindex{Raspberry Pi}
Der Linux Kernel des Raspberry Pi beispielsweise wird in zwei Varianten
kompiliert.

Im Falle von \emph{soft float} werden Parameter über R-Register übergeben und im
Falle von \emph{hard float} über D-Register.

Diese Tatsache hält einen davon ab armhf-Programmbibliotheken für armel-Code
oder umgekehrt zu verwenden und dies ist der Grund warum der gesamte Code in
Linux-Distributionen speziell für eine der beiden Konventionen kompiliert wird.
\fi

\subsubsection{ARM64: \Optimizing GCC (Linaro) 4.9}

Sehr kompakter Code:

\lstinputlisting[caption=\Optimizing GCC (Linaro) 4.9,style=customasmARM]{patterns/12_FPU/1_simple/ARM/ARM64_GCC_O3_DE.s}

\subsubsection{ARM64: \NonOptimizing GCC (Linaro) 4.9}

\lstinputlisting[caption=\NonOptimizing GCC (Linaro) 4.9,style=customasmARM]{patterns/12_FPU/1_simple/ARM/ARM64_GCC_O0_DE.s}

\NonOptimizing GCC ist geschwätziger.
Hier findet eine Menge unnützes Verschieben von Werten statt, inklusive einigem
eindeutig redundantem Code (die letzten beiden \INS{FMOV} Befehle). Vermutlich
ist GCC 4.9 noch nicht besonders gut im Erzeugen von ARM64 Code.

Bemerkenswert ist, dass ARM64 64-Bit-Register besitzt und die D-Register
ebenfalls 64 Bit breit sind.

Dadurch steht es dem Compiler frei Werte von Typ \Tdouble in \ac{GPR}s anstelle
auf dem lokalen Stack zu speichern. Dies ist in 32-bit-CPUs nicht möglich.

Wiederum kann man als Übung versuchen diese Funktion manuell zu optimieren ohne
neue Befehl wie \INS{FMADD} einzuführen. 
}
\FR{\subsubsection{ARM: \OptimizingXcodeIV (\ARMMode)}

Jusqu'à la standardisation du support de la virgule flottante, certains fabricants
de processeur ont ajouté leur propre instructions étendues.
Ensuite, VFP (\emph{Vector Floating Point}) a été standardisé.

Une différence importante par rapport au x86 est qu'en ARM, il n'y a pas de pile,
vous travaillez seulement avec des registres.

\lstinputlisting[label=ARM_leaf_example10,caption=\OptimizingXcodeIV (\ARMMode),style=customasmARM]{patterns/12_FPU/1_simple/ARM/Xcode_ARM_O3_FR.asm}

\myindex{ARM!D-\registers{}}
\myindex{ARM!S-\registers{}}

Donc, nous voyons ici que des nouveaux registres sont utilisés, avec le préfixe D.

Ce sont des registres 64-bits, il y en a 32, et ils peuvent être utilisés tant pour
des nombres à virgules flottantes (double) que pour des opérations SIMD (c'est appelé
NEON ici en ARM).

Il y a aussi 32 S-registres 32 bits, destinés à être utilisés pour les nombres à
virgules flottantes simple précision (float).

C'est facile à retenir: les registres D sont pour les nombres en double précision,
tandis que les registres S----pour les nombres en simple précision
Pour aller plus loin: \myref{ARM_VFP_registers}.

Les deux constantes (3.14 et 4.1) sont stockées en mémoire au format IEEE 754.

\myindex{ARM!\Instructions!VLDR}
\myindex{ARM!\Instructions!VMOV}
\INS{VLDR} et \INS{VMOV}, comme il peut en être facilement déduit, sont analogues
aux instructions \INS{LDR} et \MOV, mais travaillent avec des registres D.

Il est à noter que ces instructions, tout comme les registres D, sont destinées non
seulement pour les nombres à virgules flottantes, mais peuvent aussi être utilisées
pour des opérations SIMD (NEON) et cela va être montré bientôt.

Les arguments sont passés à la fonction de manière classique, via les R-registres,
toutefois, chaque nombre en double précision a une taille de 64 bits, donc deux
R-registres sont nécessaires pour passer chacun d'entre eux.

\INS{VMOV D17, R0, R1} au début, combine les deux valeurs 32-bit de \Reg{0} et \Reg{1}
en une valeur 64-bit et la sauve dans \GTT{D17}.

\INS{VMOV R0, R1, D16} est l'opération inverse: ce qui est dans \GTT{D16} est séparé
dans deux registres, \Reg{0} et \Reg{1}, car un nombre en double précision qui
nécessite 64 bit pour le stockage, est renvoyé dans \Reg{0} et \Reg{1}.

\myindex{ARM!\Instructions!VDIV}
\myindex{ARM!\Instructions!VMUL}
\myindex{ARM!\Instructions!VADD}
\INS{VDIV}, \INS{VMUL} and \INS{VADD}, 
sont des instructions pour traiter des nombres à virgule flottante, qui calculent
respectivement le \gls{quotient}, \glslink{product}{produit} et la somme.

Le code pour Thumb-2 est similaire.

\subsubsection{ARM: \OptimizingKeilVI (\ThumbMode)}

\lstinputlisting[style=customasmARM]{patterns/12_FPU/1_simple/ARM/Keil_O3_thumb_FR.asm}

Code généré par Keil pour un processeur sans FPU ou support pour NEON.

Les nombres en virgule flottante double précision sont passés par des R-registres
génériques et au lieu d'instructions FPU, des fonctions d'une bibliothèque de service
sont appelées (comme \GTT{\_\_aeabi\_dmul}, \GTT{\_\_aeabi\_ddiv}, \GTT{\_\_aeabi\_dadd})
qui émulent la multiplication, la division et l'addition pour les nombres à virgule
flottante.

Bien sûr, c'est plus lent qu'un coprocesseur FPU, mais toujours mieux que rien.

À propos, de telles bibliothèques d'émulation de FPU étaient très populaires dans
le monde x86 lorsque les coprocesseurs étaient rares et chers, et étaient installés
seulement dans des ordinateurs coûteux.

\myindex{ARM!soft float}
\myindex{ARM!armel}
\myindex{ARM!armhf}
\myindex{ARM!hard float}

L'émulation d'un coprocesseur FPU est appelée \emph{soft float} ou \emph{armel} (\emph{emulation})
dans le monde ARM, alors que l'utilisation des instructions d'un coprocesseur FPU
est appelée \emph{hard float} ou \emph{armhf}.

\iffalse
% TODO разобраться...
\myindex{Raspberry Pi}

Par exemple, le noyau Linux pour Raspberry Pi est compilé en deux variantes.

Dans le case \emph{soft float}, les arguments sont passés par les R-registres, et dans
le cas \emph{hard float}---par les registes-D.

Et c'est ce qui empêche d'utiliser des bibliothèques armfh pour de code armel ou
vice-versa, et c'est pourquoi le code dans les distributions Linux doit être compilé
suivant une seule convention.
\fi

\subsubsection{ARM64: GCC \Optimizing (Linaro) 4.9}

Code très compact:

\lstinputlisting[caption=GCC \Optimizing (Linaro) 4.9,style=customasmARM]{patterns/12_FPU/1_simple/ARM/ARM64_GCC_O3_FR.s}

\subsubsection{ARM64: GCC \NonOptimizing (Linaro) 4.9}

\lstinputlisting[caption=GCC \NonOptimizing (Linaro) 4.9,style=customasmARM]{patterns/12_FPU/1_simple/ARM/ARM64_GCC_O0_FR.s}

GCC \NonOptimizing est plus verbeux.

Il y a des nombreuses modifications de valeur inutiles, incluant du code clairement
redondant (les deux dernières instructions \INS{FMOV}). Sans doute que GCC 4.9 n'est
pas encore très bon pour la génération de code ARM64.

Il est utile de noter qu'ARM64 possède des registres 64-bit, et que les D-registres
sont aussi 64-bit.

Donc le compilateur est libre de sauver des valeurs de type \Tdouble dans \ac{GPR}s
au lieu de la pile locale.
Ce n'est pas possible sur des CPUs 32-bit.

Et encore, à titre d'exercice, vous pouvez essayer d'optimiser manuellement cette
fonction, sans introduire de nouvelles instructions comme \INS{FMADD}.
}
\JA{\subsubsection{ARM: \OptimizingXcodeIV (\ARMMode)}

ARMが標準化された浮動小数点サポートを得るまで、いくつかのプロセッサーメーカーは独自の命令拡張を追加しました。 
次に、VFP(\emph{Vector Floating Point})を標準化しました。

x86との重要な違いの1つは、ARMではスタックがなく、
レジスタだけで動作するということです。

\lstinputlisting[label=ARM_leaf_example10,caption=\OptimizingXcodeIV (\ARMMode),style=customasmARM]{patterns/12_FPU/1_simple/ARM/Xcode_ARM_O3_JA.asm}

\myindex{ARM!D-\registers{}}
\myindex{ARM!S-\registers{}}

そこで、ここではDの接頭辞を使用して新しいレジスタをいくつか見ていきます。

これらは64ビットレジスタで、32個あり、浮動小数点数(double)とSIMD(ARMではNEONと呼ばれます)
の両方に使用できます。

32ビットの32ビットSレジスタもあり、単精度浮動小数点数
(浮動小数点数)として使用されます。

暗記するのは簡単です.Dレジスタは倍精度の数値用であり、Sレジスタは単精度の数値です。
詳細は:\myref{ARM_VFP_registers}

両方の定数(3.14と4.1)はIEEE 754形式でメモリに格納されます。

\myindex{ARM!\Instructions!VLDR}
\myindex{ARM!\Instructions!VMOV}
\INS{VLDR}と\INS{VMOV}は、簡単に推測できるように、\INS{LDR}命令と \MOV 命令に似ていますが、
Dレジスタで動作します。

これらの命令は、Dレジスタと同様に、浮動小数点数だけでなく、
SIMD(NEON)演算にも使用でき、これもすぐに表示されることに注意してください。

引数はRレジスタを介して共通の方法で関数に渡されますが、
倍精度の各数値のサイズは64ビットなので、各レジスタを渡すには2つのRレジスタが必要です。

\INS{VMOV D17, R0, R1}は\Reg{0}と\Reg{1}から2つの32ビット値を1つの64ビット値に合成し、
\GTT{D17}に保存します。

\INS{VMOV R0, R1, D16}は逆の演算です。\GTT{D16}にあったものは、
\Reg{0}と\Reg{1}の2つのレジスタに分割されます。
これは、格納に64ビット必要な倍精度数が\Reg{0}と\Reg{1}に返されるためです。

\myindex{ARM!\Instructions!VDIV}
\myindex{ARM!\Instructions!VMUL}
\myindex{ARM!\Instructions!VADD}
\INS{VDIV}、\INS{VMUL}、\INS{VADD}はそれぞれ\gls{quotient}、\gls{product}、
和を計算する浮動小数点数を処理する命令です。

Thumb-2のコードは同じです。

\subsubsection{ARM: \OptimizingKeilVI (\ThumbMode)}

\lstinputlisting[style=customasmARM]{patterns/12_FPU/1_simple/ARM/Keil_O3_thumb_JA.asm}

KeilはFPUまたはNEONをサポートしていないプロセッサ用のコードを生成しました。

倍精度浮動小数点数は、汎用Rレジスタを介して渡され、
FPU命令の代わりに浮動小数点数の乗算、除算、加算をエミュレートするサービスライブラリ関数
(\GTT{\_\_aeabi\_dmul}、 \GTT{\_\_aeabi\_ddiv}、 \GTT{\_\_aeabi\_dadd}など)が呼び出されます。

もちろん、それはFPUコプロセッサよりも遅いですが、何もないよりはましです。

ところで、同様のFPUエミュレートライブラリは、コプロセッサが貴重で高価で、
高価なコンピュータにしかインストールされていなかったx86の世界で非常に人気がありました。

\myindex{ARM!soft float}
\myindex{ARM!armel}
\myindex{ARM!armhf}
\myindex{ARM!hard float}

FPUコプロセッサエミュレーションは、ARMワールドでは\emph{ソフトフロート}または\emph{armel}(\emph{エミュレーション})と呼ばれ、
コプロセッサのFPU命令はハードフロートまたは\emph{armhf}と呼ばれます。

\iffalse
% TODO разобраться...
\myindex{Raspberry Pi}

For example, the Linux kernel for Raspberry Pi is compiled in two variants.

In the \emph{soft float} case, arguments are passed via R-registers, and in the \emph{hard float} case---via D-registers.

And that is what stops you from using armhf-libraries from armel-code or vice versa,
and that is why all the code in Linux distributions must be compiled according to a single convention.
\fi

\subsubsection{ARM64: \Optimizing GCC (Linaro) 4.9}

とってもコンパクトなコードです。

\lstinputlisting[caption=\Optimizing GCC (Linaro) 4.9,style=customasmARM]{patterns/12_FPU/1_simple/ARM/ARM64_GCC_O3_JA.s}

\subsubsection{ARM64: \NonOptimizing GCC (Linaro) 4.9}

\lstinputlisting[caption=\NonOptimizing GCC (Linaro) 4.9,style=customasmARM]{patterns/12_FPU/1_simple/ARM/ARM64_GCC_O0_JA.s}

\NonOptimizing GCCはもっと冗長です。

いくつかの明確に冗長なコード(最後の2つの\INS{FMOV}命令)を含む、不要な値のシャッフルが多くあります。 
おそらく、GCC 4.9はまだARM64コードを生成するのに適していません。

注目すべきことは、ARM64には64ビットのレジスタがあり、Dレジスタには64ビットのレジスタも含まれているということです。

したがって、コンパイラはローカルスタックではなく\ac{GPR}に \Tdouble 型の値を自由に保存できます。 
これは32ビットCPUでは不可能です。

また、エクササイズとして、\INS{FMADD}のような新しい命令を導入することなく、
この機能を手動で最適化してみることができます。
}



\iffalse
A BUG HERE! to be fixed...
\EN{\subsubsection{MIPS}
% FIXME better start at non-optimizing version?

The function uses a lot of S- registers which must be preserved, so that's why its 
values are saved in the function prologue and restored in the epilogue.

\lstinputlisting[caption=\Optimizing GCC 4.4.5 (IDA),style=customasmMIPS]{patterns/13_arrays/1_simple/MIPS_O3_IDA_EN.lst}

Something interesting: there are two loops and the first one doesn't need $i$, it needs only 
$i*2$ (increased by 2 at each iteration) and also the address in memory (increased by 4 at each iteration).

So here we see two variables, one (in \$V0) increasing by 2 each time, and another (in \$V1) --- by 4.

The second loop is where \printf is called and it reports the value of $i$ to the user, 
so there is a variable
which is increased by 1 each time (in \$S0) and also a memory address (in \$S1) increased by 4 each time.

That reminds us of loop optimizations: \myref{loop_iterators}.

Their goal is to get rid of multiplications.

}
\RU{\subsubsection{MIPS}
% FIXME better start at non-optimizing version?
Функция использует много S-регистров, которые должны быть сохранены. Вот почему их значения сохраняются
в прологе функции и восстанавливаются в эпилоге.

\lstinputlisting[caption=\Optimizing GCC 4.4.5 (IDA),style=customasmMIPS]{patterns/13_arrays/1_simple/MIPS_O3_IDA_RU.lst}

Интересная вещь: здесь два цикла и в первом не нужна переменная $i$, а нужна только переменная
$i*2$ (скачущая через 2 на каждой итерации) и ещё адрес в памяти (скачущий через 4 на каждой итерации).

Так что мы видим здесь две переменных: одна (в \$V0) увеличивается на 2 каждый раз, и вторая (в \$V1) --- на 4.

Второй цикл содержит вызов \printf. Он должен показывать значение $i$ пользователю,
поэтому здесь есть переменная, увеличивающаяся на 1 каждый раз (в \$S0), а также адрес в памяти (в \$S1) 
увеличивающийся на 4 каждый раз.

Это напоминает нам оптимизацию циклов: \myref{loop_iterators}.
Цель оптимизации в том, чтобы избавиться от операций умножения.

}
\DE{\subsubsection{MIPS}

\lstinputlisting[caption=\Optimizing GCC 4.4.5 (IDA),style=customasmMIPS]{patterns/08_switch/1_few/MIPS_O3_IDA_DE.lst}

\myindex{MIPS!\Instructions!JR}
Die Funktion endet stets mit einem Aufruf von \puts, weshalb wir hier einen Sprung zu \puts (\INS{JR}: \q{Jump
Register}) anstelle von \q{jump and link} finden.
Dieses Feature haben wir bereit in \myref{JMP_instead_of_RET} besprochen.

\myindex{MIPS!Load delay slot}
Wir finden auch oft \INS{NOP} Befehle nach \INS{LW} Befehlen.
Dies ist \q{load delay slot}: ein anderer \emph{delay slot} in MIPS.
\myindex{MIPS!\Instructions!LW}
Ein Befehl neben \INS{LW} kann in dem Moment ausgeführt werden, in dem \INS{LW} Werte aus dem Speicher lädt.
Der nächste Befehl muss aber nicht das Ergebnis von \INS{LW} verwenden.
Moderne MIPS CPUs haben die Eigenschaft abwarten zu können, ob der folgende Befehl das Ergebnis von \INS{LW} verwendet,
sodass dieses Vorgehen überholt wirkt, aber GCC fügt für ältere MIPS CPUs immer noch NOPs hinzu.
Im Allgemeinen können diese aber ignoriert werden.}
\FR{\subsubsection{MIPS}

MIPS peut supporter plusieurs coprocesseurs (jusqu'à 4), le zérotième\footnote{Barbarisme
pour rappeler que les indices commencent à zéro.} est un coprocesseur contrôleur
spécial, et celui d'indice 1 est le FPU.

Comme en ARM, le coprocesseur MIPS n'est pas une machine à pile, il comprend 32 registres
32-bit (\$F0-\$F31):
\myref{MIPS_FPU_registers}.

Lorsque l'on doit travailler avec des valeurs \Tdouble 64-bit, une paire de F-registres
32-bit est utilisée.

\lstinputlisting[caption=GCC 4.4.5 \Optimizing (IDA),style=customasmMIPS]{patterns/12_FPU/1_simple/MIPS_O3_IDA_FR.lst}

Les nouvelles instructions ici sont:

\myindex{MIPS!\Instructions!LWC1}
\myindex{MIPS!\Instructions!DIV.D}
\myindex{MIPS!\Instructions!MUL.D}
\myindex{MIPS!\Instructions!ADD.D}
\begin{itemize}

\item \INS{LWC1} charge un mot de 32-bit dans un registre du premier coprocesseur
(d'où le \q{1} dans le nom de l'instruction).
\myindex{MIPS!\Pseudoinstructions!L.D}

Une paire d'instructions \INS{LWC1} peut être combinée en une pseudo instruction \INS{L.D}.

\item \INS{DIV.D}, \INS{MUL.D}, \INS{ADD.D} effectuent respectivement la division,
la multiplication, et l'addition (\q{.D} est le suffixe standard pour la double précision,
\q{.S} pour la simple précision)

\end{itemize}

\myindex{MIPS!\Instructions!LUI}
\myindex{\CompilerAnomaly}
\label{MIPS_FPU_LUI}

Il y a une anomalie bizarre du compilateur: l'instruction \INS{LUI} que nous avons
marqué avec un point d'interrogation.
Il m'est difficile de comprendre pourquoi charger une partie de la constante de type
64-bit \Tdouble dans le registre \$V0. Cette instruction n'a pas d'effet.
% TODO did you try checking out compiler source code?
Si quelqu'un en sait plus sur ceci, s'il vous plaît, envoyez moi un email\footnote{\EMAIL}.

}
\JA{\subsubsection{MIPS}

MIPSはいくつかのコプロセッサ(最大4個)をサポートすることができます。
そのうちの0番目\footnote{0から始まる}は特別な制御コプロセッサであり、最初のコプロセッサはFPUです。

ARMと同様に、MIPSコプロセッサはスタックマシンではなく、32個の32ビットレジスタ(\$F0-\$F31)を持ちます。
\myref{MIPS_FPU_registers}.

64ビットの \Tdouble 値を扱う必要がある場合、32ビットのFレジスタのペアが使用されます。

\lstinputlisting[caption=\Optimizing GCC 4.4.5 (IDA),style=customasmMIPS]{patterns/12_FPU/1_simple/MIPS_O3_IDA_JA.lst}

新しい命令は以下です。

\myindex{MIPS!\Instructions!LWC1}
\myindex{MIPS!\Instructions!DIV.D}
\myindex{MIPS!\Instructions!MUL.D}
\myindex{MIPS!\Instructions!ADD.D}
\begin{itemize}

\item \INS{LWC1}は32ビットワードを第1コプロセッサのレジスタにロードします(命令名は\q{1})。
\myindex{MIPS!\Pseudoinstructions!L.D}

一対の\INS{LWC1}命令を組み合わせて\INS{L.D}疑似命令にすることができます。

\item \INS{DIV.D}、 \INS{MUL.D}、 \INS{ADD.D}はそれぞれ除算、乗算、加算を行います
(接尾辞の\q{.D}は倍精度、\q{.S}は単精度を表します)

\end{itemize}

\myindex{MIPS!\Instructions!LUI}
\myindex{\CompilerAnomaly}
\label{MIPS_FPU_LUI}

また、奇妙なコンパイラの例外があります\INS{LUI}命令に疑問符がついています。 
\$V0 レジスタに64ビット定数の \Tdouble 型の一部をロードする理由を理解することは難しいです。 
これらの命令は何の効果もありません。 
% TODO did you try checking out compiler source code?
これについて何か知っているなら、著者に電子メール\footnote{\EMAIL}を送ってください。
}
\fi


\subsection{\RU{Передача чисел с плавающей запятой в аргументах}\EN{Passing floating point numbers via arguments}\DEph{}\FR{Passage de nombres en virgule flottante par les arguments}}
\myindex{\CStandardLibrary!pow()}

\lstinputlisting[style=customc]{patterns/12_FPU/2_passing_floats/pow.c}

\EN{\subsubsection{x86}

Let's see what we get in (MSVC 2010):

\lstinputlisting[caption=MSVC 2010,style=customasmx86]{patterns/12_FPU/2_passing_floats/MSVC_EN.asm}

\myindex{x86!\Instructions!FLD}
\myindex{x86!\Instructions!FSTP}

\FLD and \FSTP move variables between the data segment and the FPU stack. 
\GTT{pow()}\footnote{a standard C function, raises a number to the given power (exponentiation)}
takes both values from the stack and returns its result in the \ST{0} register.
\printf takes 8 bytes from the local stack and interprets them as \Tdouble type variable.

By the way, a pair of \MOV instructions could be used here for moving values from the memory
into the stack, because the values in memory are stored in IEEE 754 format, and pow() also takes them in this
format, so no conversion is necessary.
That's how it's done in the next example, for ARM: \myref{FPU_passing_floats_ARM}.

}
\RU{\subsubsection{x86}

Посмотрим, что у нас вышло (MSVC 2010):

\lstinputlisting[caption=MSVC 2010,style=customasmx86]{patterns/12_FPU/2_passing_floats/MSVC_RU.asm}

\myindex{x86!\Instructions!FLD}
\myindex{x86!\Instructions!FSTP}
\FLD и \FSTP перемещают переменные из сегмента данных в FPU-стек или обратно. 
\GTT{pow()}\footnote{стандартная функция Си, возводящая число в степень} достает оба значения из стека и 
возвращает результат в \ST{0}. 
\printf берет 8 байт из стека и трактует их как переменную типа \Tdouble.

Кстати, с тем же успехом можно было бы перекладывать эти два числа из памяти в стек при помощи пары \MOV:
 
ведь в памяти числа в формате IEEE 754, pow() также принимает их в том же
формате, и никакая конверсия не требуется.

Собственно, так и происходит в следующем примере с ARM: \myref{FPU_passing_floats_ARM}.

}
\DE{\subsubsection{x86}
Schauen wir uns an, was wir in MSVC 2010 erhalten:

\lstinputlisting[caption=MSVC 2010,style=customasmx86]{patterns/12_FPU/2_passing_floats/MSVC_DE.asm}

\myindex{x86!\Instructions!FLD}
\myindex{x86!\Instructions!FSTP}
% TODO bug to be fixed here:
\FLD und \FSTP verschieben Variablen zwischen Datensegment und dem FPU
Stack.\GTT{pow()}\footnote{eine Standard-C-Funktion, die eine Zahl potenziert}
nimmt beide Werte vom Stack der FPU und gibt ihr Ergebnis über das \ST{0} Register zurück. 
Die Funktion \printf nimmt 8 Byte vom lokalen Stack und interpretiert diese als
Variable von Typ \Tdouble.

Übrigens könnte hier auch ein Paar \MOV Befehle verwendet werden, um die Werte
aus dem Speicher zu holen und auf den Stack zu legen, denn die Werte sind im
Speicher im IEEE 754 Format abgelegt und pow() arbeitet mit diesem Format,
sodass keine Umwandlung notwendig ist.
Genau so wird es im folgenden Beispiel für ARM auch
gemacht:\myref{FPU_passing_floats_ARM}
}
\FR{\subsubsection{x86}

Regardons ce que nous obtenons avec MSVC 2010:

\lstinputlisting[caption=MSVC 2010,style=customasmx86]{patterns/12_FPU/2_passing_floats/MSVC_FR.asm}

\myindex{x86!\Instructions!FLD}
\myindex{x86!\Instructions!FSTP}

\FLD et \FSTP déplacent des variables entre le segment de données et la pile du FPU.
\GTT{pow()}\footnote{une fonction C standard, qui élève un nombre à la puissance
donnée (puissance)} prend deux valeurs depuis la pile et renvoie son résultat dans
le registre \ST{0}.
\printf prend 8 octets de la pile locale et les interprète comme des variables de
type \Tdouble.

À propos, une paire d'instructions \MOV pourrait être utilisée ici pour déplacer
les valeurs depuis la mémoire vers la pile, car les valeurs en mémoire sont stockées
au format IEEE 754, et pow() les prend aussi dans ce format, donc aucune conversion
n'est nécessaire.
C'est fait ainsi dans l'exemple suivant, pour ARM: \myref{FPU_passing_floats_ARM}.

}
\JA{\subsubsection{x86}

(MSVC 2010)で見てみましょう

\lstinputlisting[caption=MSVC 2010,style=customasmx86]{patterns/12_FPU/2_passing_floats/MSVC_JA.asm}

\myindex{x86!\Instructions!FLD}
\myindex{x86!\Instructions!FSTP}

\FLD および \FSTP は、データセグメントとFPUスタックとの間の変数を移動します。
\GTT{pow()} \footnote{標準的なC関数であり、与えられたべき乗(指数関数)}はスタックから両方の値をとり、
その結果を\ST{0}レジスタに返します。  \printf はローカルスタックから8バイトを取り出し、double型の変数として解釈します。

ちなみに、メモリ内の値はIEEE 754形式で格納され、pow()もこの形式で格納されているため、
値をメモリからスタックに移動するための一対の \MOV 命令を使用でき、変換は不要です。
これはARMのための次の例で行われます:\myref{FPU_passing_floats_ARM}
}

\EN{\subsubsection{ARM: \OptimizingKeilVI (\ARMMode)}
\myindex{\CLanguageElements!switch}

\lstinputlisting[style=customasmARM]{patterns/08_switch/1_few/few_ARM_ARM_O3.asm}

Again, by investigating this code we cannot say if it was a switch() in the original source code, 
or just a pack of if() statements.

\myindex{ARM!\Instructions!ADRcc}

Anyway, we see here predicated instructions again (like \ADREQ (\emph{Equal}))
which is triggered only in case $R0=0$, and then loads the address of the string \emph{<<zero\textbackslash{}n>>}
into \Reg{0}.
\myindex{ARM!\Instructions!BEQ}
The next instruction \ac{BEQ} redirects control flow to \TT{loc\_170}, if $R0=0$.

An astute reader may ask, will \ac{BEQ} trigger correctly since \ADREQ it
has already filled the \Reg{0} register before with another value?

Yes, it will since \ac{BEQ} checks the flags set by the \CMP instruction, 
and \ADREQ does not modify any flags at all.

The rest of the instructions are already familiar to us. 
There is only one call to \printf , at the end, and we have already examined this trick here~(\myref{ARM_B_to_printf}).
At the end, there are three paths to \printf{}.

\myindex{ARM!\Instructions!ADRcc}
\myindex{ARM!\Instructions!CMP}
The last instruction, \TT{CMP R0, \#2}, is needed to check if $a=2$.

If it is not true, then \ADRNE loads a pointer to the string \emph{<<something unknown \textbackslash{}n>>}
into \Reg{0}, since $a$ has already been checked to be equal to 0 or 1,
and we can sure that the $a$ variable is not equal to these numbers at this point.
And if $R0=2$, 
a pointer to the string \emph{<<two\textbackslash{}n>>}
will be loaded by \ADREQ into \Reg{0}.

\subsubsection{ARM: \OptimizingKeilVI (\ThumbMode)}

\lstinputlisting[style=customasmARM]{patterns/08_switch/1_few/few_ARM_thumb_O3.asm}

% FIXME а каким можно? к каким нельзя? \myref{} ->

As was already mentioned, it is not possible to add conditional predicates to most instructions in Thumb
mode, so the Thumb-code here is somewhat similar to the easily understandable x86 \ac{CISC}-style code.

\subsubsection{ARM64: \NonOptimizing GCC (Linaro) 4.9}

\lstinputlisting[style=customasmARM]{patterns/08_switch/1_few/ARM64_GCC_O0_EN.lst}

The type of the input value is \Tint, hence register \RegW{0} is used to hold it instead of the whole
\RegX{0} register.

The string pointers are passed to \puts using an \INS{ADRP}/\INS{ADD} instructions pair just like it was demonstrated in the 
\q{\HelloWorldSectionName} example:~\myref{pointers_ADRP_and_ADD}.

\subsubsection{ARM64: \Optimizing GCC (Linaro) 4.9}

\lstinputlisting[style=customasmARM]{patterns/08_switch/1_few/ARM64_GCC_O3_EN.lst}

Better optimized piece of code.
\TT{CBZ} (\emph{Compare and Branch on Zero}) instruction does jump if \RegW{0} is zero.
There is also a direct jump to \puts instead of calling it, like it was explained before:~\myref{JMP_instead_of_RET}.

}
\RU{\subsubsection{ARM + \OptimizingKeilVI (\ARMMode)}

\begin{lstlisting}[caption=\OptimizingKeilVI (\ARMMode),style=customasmARM]
02 0C C0 E3          BIC     R0, R0, #0x200
01 09 80 E3          ORR     R0, R0, #0x4000
1E FF 2F E1          BX      LR
\end{lstlisting}

\myindex{ARM!\Instructions!BIC}
\INS{BIC} (\emph{BItwise bit Clear}) это инструкция сбрасывающая заданные биты. 
Это как аналог \AND, но только с инвертированным операндом.

Т.е. это аналог инструкций \NOT+\AND.

\myindex{ARM!\Instructions!ORR}
\INS{ORR} это \q{логическое или}, аналог \OR в x86.

Пока всё понятно.

\subsubsection{ARM + \OptimizingKeilVI (\ThumbMode)}

\begin{lstlisting}[caption=\OptimizingKeilVI (\ThumbMode),style=customasmARM]
01 21 89 03          MOVS    R1, 0x4000
08 43                ORRS    R0, R1
49 11                ASRS    R1, R1, #5   ; сгенерировать 0x200 и записать в R1
88 43                BICS    R0, R1
70 47                BX      LR
\end{lstlisting}

Вероятно, Keil решил, что код в режиме Thumb,
получающий \TT{0x200} из \TT{0x4000}, 
более компактный, нежели код, 
записывающий \TT{0x200} в какой-нибудь регистр.
% TODO1 пример, как компилятор при помощи сдвигов оптизирует такое: a=0x1000; b=0x2000; c=0x4000, etc

\myindex{ARM!\Instructions!ASRS}
Поэтому при помощи инструкции \INS{ASRS} (\ASRdesc), это значение вычисляется как $\TT{0x4000} \gg 5$.

\subsubsection{ARM + \OptimizingXcodeIV (\ARMMode)}
\label{anomaly:LLVM}
\myindex{\CompilerAnomaly}

\begin{lstlisting}[caption=\OptimizingXcodeIV (\ARMMode),label=ARM_leaf_example3,style=customasmARM]
42 0C C0 E3          BIC             R0, R0, #0x4200
01 09 80 E3          ORR             R0, R0, #0x4000
1E FF 2F E1          BX              LR
\end{lstlisting}

Код, который был сгенерирован LLVM, в исходном коде, на самом деле, выглядел бы так:

\begin{lstlisting}[style=customc]
    REMOVE_BIT (rt, 0x4200);
    SET_BIT (rt, 0x4000);
\end{lstlisting}

И он делает в точности что нам нужно. 
Но почему \TT{0x4200}? 
Возможно, это артефакт оптимизатора LLVM
\footnote{Это был LLVM build 2410.2.00 входящий в состав Apple Xcode 4.6.3}.
Возможно, ошибка оптимизатора компилятора, но создаваемый код всё же работает верно.

Об аномалиях компиляторов, подробнее читайте здесь ~(\myref{anomaly:Intel}).

\OptimizingXcodeIV для режима Thumb генерирует точно такой же код.

\subsubsection{ARM: ещё об инструкции \INS{BIC}}
\myindex{ARM!\Instructions!BIC}

Если немного переделать пример:

\begin{lstlisting}[style=customc]
int f(int a)
{
    int rt=a;

    REMOVE_BIT (rt, 0x1234);

    return rt;
};
\end{lstlisting}

То оптимизирующий Keil 5.03 в режиме ARM сделает такое:

\begin{lstlisting}[style=customasmARM]
f PROC
        BIC      r0,r0,#0x1000
        BIC      r0,r0,#0x234
        BX       lr
        ENDP
\end{lstlisting}

Здесь две инструкции \INS{BIC}, т.е. биты \TT{0x1234} сбрасываются в два прохода.

Это потому что в инструкции \INS{BIC} нельзя закодировать значение \TT{0x1234}, 
но можно закодировать \TT{0x1000} либо \TT{0x234}.

\subsubsection{ARM64: \Optimizing GCC (Linaro) 4.9}

\Optimizing GCC, компилирующий для ARM64, может использовать \AND вместо \INS{BIC}:

\begin{lstlisting}[caption=\Optimizing GCC (Linaro) 4.9,style=customasmARM]
f:
	and	w0, w0, -513	; 0xFFFFFFFFFFFFFDFF
	orr	w0, w0, 16384	; 0x4000
	ret
\end{lstlisting}

\subsubsection{ARM64: \NonOptimizing GCC (Linaro) 4.9}

\NonOptimizing GCC генерирует больше избыточного кода, но он работает также:

\begin{lstlisting}[caption=\NonOptimizing GCC (Linaro) 4.9,style=customasmARM]
f:
	sub	sp, sp, #32
	str	w0, [sp,12]
	ldr	w0, [sp,12]
	str	w0, [sp,28]
	ldr	w0, [sp,28]
	orr	w0, w0, 16384	; 0x4000
	str	w0, [sp,28]
	ldr	w0, [sp,28]
	and	w0, w0, -513	; 0xFFFFFFFFFFFFFDFF
	str	w0, [sp,28]
	ldr	w0, [sp,28]
	add	sp, sp, 32
	ret
\end{lstlisting}

}
\DE{\subsubsection{ARM: \OptimizingKeilVI (\ARMMode)}
\myindex{\CLanguageElements!switch}

\lstinputlisting[style=customasmARM]{patterns/08_switch/1_few/few_ARM_ARM_O3.asm}
Auch hier können wir bei Untersuchung des Code nicht sagen, ob im Quellcode ein switch() oder eine Folge von
if()-Ausdrücken vorliegt.


\myindex{ARM!\Instructions!ADRcc}
Wir finden hier Befehle mit Prädikaten wieder (wie \ADREQ (\emph{Equal})), welcher nur dann ausgeführt wird, wenn $R0=0$
und dann die Adresse des Strings IT{<<zero\textbackslash{}n>>} nach \Reg{0} lädt.

\myindex{ARM!\Instructions!BEQ}
Der folgende \ac{BEQ} Befehl übergibt den Control Flow an \TT{loc\_170}, falls $R0=0$.
Ein aufmerksamer Leser könnte sich fragen, ob \ac{BEQ} korrekt ausgelöst wird, da \ADREQ das \Reg{0} Register bereits
mit einem anderen Wert befüllt hat.
Es wird korrekt ausgelöst, da \ac{BEQ} die Flags, die vom \CMP Befehl gesetzt wurden, prüft und \ADREQ die Flags nicht
verändert.

Die übrigen Befehle kennen wir bereits.
Es gibt nur einen Aufruf von \printf am Ende und wir haben diesen Trick bereits hier
kennengelernt~(\myref{ARM_B_to_printf}). Am Ende gibt es drei Wege zur Ausführung von \printf.

\myindex{ARM!\Instructions!ADRcc}
\myindex{ARM!\Instructions!CMP}
Der letzte Befehl, \TT{CMP R0, \#2}, wird benötigt, um zu prüfen, ob $a=2$.
Wenn dies nicht der Fall ist, lädt \ADRNE einen Pointer auf den String \emph{<<something unknown \textbackslash{}n>>} nach
\Reg{0}, da $a$ bereits auf Gleichheit mit 0 oder 1 geprüft wurde und wir können sicher sein, dass die Variable $a$ an
dieser Stelle keinen dieser beiden Werte enthält.
Falls $R0=2$ ist, lädt \ADREQ einen Pointer auf den String \emph{<<two\textbackslash{}n>>} nach \Reg{0}. 

\subsubsection{ARM: \OptimizingKeilVI (\ThumbMode)}

\lstinputlisting[style=customasmARM]{patterns/08_switch/1_few/few_ARM_thumb_O3.asm}

% FIXME а каким можно? к каким нельзя? \myref{} ->
Wie bereits erwähnt ist es bei den meisten Befehlen im Thumb mode nicht möglich Prädikate für Bedingungen hinzuzufügen,
sodass der Thumb-Code hier dem leicht verständlichen x86 \ac{CISC}-style Code sehr ähnlich ist.

\subsubsection{ARM64: \NonOptimizing GCC (Linaro) 4.9}

\lstinputlisting[style=customasmARM]{patterns/08_switch/1_few/ARM64_GCC_O0_DE.lst}
Der Datentyp des Eingabewertes ist \Tint, deshalb wird das Register \RegW{0} anstatt des \RegX{0} Registers verwendet,
um ihn aufzunehmen.

Die Pointer auf die Strings werden an \puts mit einem \INS{ADRP}/\INS{ADD} Befehlspaar übergeben, genauso wie wir es im
\q{\HelloWorldSectionName} Beispiel gezeigt haben:~\myref{pointers_ADRP_and_ADD}.

\subsubsection{ARM64: \Optimizing GCC (Linaro) 4.9}

\lstinputlisting[style=customasmARM]{patterns/08_switch/1_few/ARM64_GCC_O3_DE.lst}
Ein besser optimiertes Stück Code. 
Der Befehl \TT{CBZ} (\emph{Compare and Branch on Zero}) springt, falls \RegW{0} gleich null ist.
Es gibt auch einen direkten Sprung zu \puts anstelle eines Aufrufs, so wie bereits hier
erklärt:~\myref{JMP_instead_of_RET}.

}
\FR{\subsubsection{ARM + \NonOptimizingXcodeIV (\ThumbTwoMode)}
\label{FPU_passing_floats_ARM}

\lstinputlisting[style=customasmARM]{patterns/12_FPU/2_passing_floats/Xcode_thumb_O0.asm}

Comme nous l'avons déjà mentionné, les pointeurs sur des nombres flottants 64-bit
sont passés dans une paire de R-registres.

Ce code est un peu redondant (probablement car l'optimisation est désactivée),
puisqu'il est possible de charger les valeurs directement dans les R-registres sans
toucher les D-registres.

Donc, comme nous le voyons, la fonction \GTT{\_pow} reçoit son premier argument dans
\Reg{0} et \Reg{1}, et le second dans \Reg{2} et \Reg{3}.
La fonction laisse son résultat dans \Reg{0} et \Reg{1}.
Le résultat de \GTT{\_pow} est déplacé dans \GTT{D16}, puis dans la paire \Reg{1}
et \Reg{2}, d'où \printf prend le nombre résultant.

\subsubsection{ARM + \NonOptimizingKeilVI (\ARMMode)}

\lstinputlisting[style=customasmARM]{patterns/12_FPU/2_passing_floats/Keil_ARM_O0.asm}

Les D-registres ne sont pas utilisés ici, juste des paires de R-registres.

\subsubsection{ARM64 + GCC (Linaro) 4.9 \Optimizing}

\lstinputlisting[caption=GCC (Linaro) 4.9 \Optimizing,style=customasmARM]{patterns/12_FPU/2_passing_floats/ARM64_FR.s}

Les constantes sont chargées dans \RegD{0} et \RegD{1}: \TT{pow()} les prend d'ici.
Le résultat sera dans \RegD{0} après l'exécution de \TT{pow()}.
Il est passé à  \printf sans aucune modification ni déplacement, car \printf
prend ces arguments de \glslink{integral type}{type intégral} et pointeurs depuis
des X-registres, et les arguments en virgule flottante depuis des D-registres.

}
\JA{\subsubsection{ARM + \NonOptimizingXcodeIV (\ThumbTwoMode)}
\label{FPU_passing_floats_ARM}

\lstinputlisting[style=customasmARM]{patterns/12_FPU/2_passing_floats/Xcode_thumb_O0.asm}

前に述べたように、64ビットの浮動ポインタ番号はRレジスタのペアで渡されます。

D-レジスタに触れることなく直接Rレジスタに値をロードすることができるので、
このコードは少し冗長です(最適化がオフになっているためです)。

したがって、見てきたように、\GTT{\_pow}関数は\Reg{0}と\Reg{1}で最初の引数を受け取り、\Reg{2}と\Reg{3}で2番目の引数を受け取ります。
関数は、その結果を\Reg{0} と\Reg{1} のままにします。 
\GTT{\_pow}の結果は\GTT{D16}に移動し、次に\Reg{1}と\Reg{2}のペアで \printf が結果の数値を取得します。

\subsubsection{ARM + \NonOptimizingKeilVI (\ARMMode)}

\lstinputlisting[style=customasmARM]{patterns/12_FPU/2_passing_floats/Keil_ARM_O0.asm}

Dレジスタはここでは使用されず、Rレジスタのペアのみが使用されます。

\subsubsection{ARM64 + \Optimizing GCC (Linaro) 4.9}

\lstinputlisting[caption=\Optimizing GCC (Linaro) 4.9,style=customasmARM]{patterns/12_FPU/2_passing_floats/ARM64_JA.s}

定数は\RegD{0}と\RegD{1}にロードされます。\TT{pow()}は定数をそこから取り出します。
結果は、\TT{pow()}の実行後に\RegD{0}に格納されます。 
変更や移動をせずに \printf に渡す必要があります。これは、
\printf は、Xレジスタからの\glslink{integral type}{integral types}とポインタとDレジスタからの浮動小数点引数の引数をとります。
}

\iffalse
BUG HERE
\EN{\subsubsection{MIPS}
% FIXME better start at non-optimizing version?

The function uses a lot of S- registers which must be preserved, so that's why its 
values are saved in the function prologue and restored in the epilogue.

\lstinputlisting[caption=\Optimizing GCC 4.4.5 (IDA),style=customasmMIPS]{patterns/13_arrays/1_simple/MIPS_O3_IDA_EN.lst}

Something interesting: there are two loops and the first one doesn't need $i$, it needs only 
$i*2$ (increased by 2 at each iteration) and also the address in memory (increased by 4 at each iteration).

So here we see two variables, one (in \$V0) increasing by 2 each time, and another (in \$V1) --- by 4.

The second loop is where \printf is called and it reports the value of $i$ to the user, 
so there is a variable
which is increased by 1 each time (in \$S0) and also a memory address (in \$S1) increased by 4 each time.

That reminds us of loop optimizations: \myref{loop_iterators}.

Their goal is to get rid of multiplications.

}
\RU{\subsubsection{MIPS}
% FIXME better start at non-optimizing version?
Функция использует много S-регистров, которые должны быть сохранены. Вот почему их значения сохраняются
в прологе функции и восстанавливаются в эпилоге.

\lstinputlisting[caption=\Optimizing GCC 4.4.5 (IDA),style=customasmMIPS]{patterns/13_arrays/1_simple/MIPS_O3_IDA_RU.lst}

Интересная вещь: здесь два цикла и в первом не нужна переменная $i$, а нужна только переменная
$i*2$ (скачущая через 2 на каждой итерации) и ещё адрес в памяти (скачущий через 4 на каждой итерации).

Так что мы видим здесь две переменных: одна (в \$V0) увеличивается на 2 каждый раз, и вторая (в \$V1) --- на 4.

Второй цикл содержит вызов \printf. Он должен показывать значение $i$ пользователю,
поэтому здесь есть переменная, увеличивающаяся на 1 каждый раз (в \$S0), а также адрес в памяти (в \$S1) 
увеличивающийся на 4 каждый раз.

Это напоминает нам оптимизацию циклов: \myref{loop_iterators}.
Цель оптимизации в том, чтобы избавиться от операций умножения.

}
\DE{\subsubsection{MIPS}

\lstinputlisting[caption=\Optimizing GCC 4.4.5 (IDA),style=customasmMIPS]{patterns/12_FPU/2_passing_floats/MIPS_O3_IDA_DE.lst}
Und wieder sehen wir hier, dass der Befehl \INS{LUI} einen 32-Bit-Teil einer
\Tdouble Zahl nach \$V0 lädt.
Und wiederum ist es schwer nachzuvollziehen warum dies geschieht.

\myindex{MIPS!\Instructions!MFC1}
Der für uns neue Befehl an dieser Stelle ist \INS{MFC1}(\q{Move From Coprocessor
1}). Die Nummer des FPU-Koprozessors ist 1, daher die \q{1} im Namen des
Befehls. 
Dieser Befehl überträgt Werte aus den Registern des Koprozessors in die Register
der CPU (\ac{GPR}).
Auf diese Weise wird das Ergebnis von \TT{pow()} schließlich in die Register
\$A3 und \$A2 verschoben und \printf übernimmt einen 64-Bit-Wert von doppelter
Genauigkeit aus diesem Registerpaar.
}
\FR{\subsubsection{MIPS}

\lstinputlisting[caption=\Optimizing GCC 4.4.5 (IDA),style=customasmMIPS]{patterns/12_FPU/2_passing_floats/MIPS_O3_IDA_FR.lst}

À nouveau, nous voyons ici \INS{LUI} qui charge une partie 32-bit d'un nombre \Tdouble
dans \$V0.
À nouveau, c'est difficile de comprendre pourquoi.

\myindex{MIPS!\Instructions!MFC1}

La nouvelle instruction pour nous ici est \INS{MFC1} (\q{Move From Coprocessor 1}
charger depuis le coprocesseur 1).
Le FPU est le coprocesseur numéro 1, d'où le \q{1} dans le nom de l'instruction.
Cette instruction transfère des valeurs depuis des registres du coprocesseur dans
les registres du CPU (\ac{GPR}).
Donc à la fin, le résultat de \TT{pow()} est transféré dans les registres \$A3 et
\$A2, et \printf prend une valeur double 64-bit depuis cette paire de registres.

}
\JA{\subsubsection{MIPS}

\lstinputlisting[caption=\Optimizing GCC 4.4.5 (IDA),style=customasmMIPS]{patterns/12_FPU/2_passing_floats/MIPS_O3_IDA_JA.lst}

そしてここでも、\INS{LUI}は \Tdouble の32ビット部分を \$V0 にロードしています。 
そして、これを理解するのは難しいです。

\myindex{MIPS!\Instructions!MFC1}

ここで私たちの新しい指示は\INS{MFC1}です(\q{Coprocessor 1から移動})。 
FPUはコプロセッサ番号1なので、命令名に\q{1}が入ります。 
この命令は、コプロセッサのレジスタからCPU(\ac{GPR})のレジスタに値を転送します。 
したがって、最後に\TT{pow()}の結果はレジスタ \$A3 と \$A2 に移動され、
\printf はこのレジスタのペアから64ビットのdouble値をとります。
}
\fi


\subsection{\RU{Пример со сравнением}\EN{Comparison example}\DEph{}\FR{Exemple de comparaison}\JA{比較の例}}

\RU{Попробуем теперь вот это:}\EN{Let's try this:}\DEph{}\JA{これを試してみましょう}

\lstinputlisting[style=customc]{patterns/12_FPU/3_comparison/d_max.c}

\RU{Несмотря на кажущуюся простоту этой функции, понять, как она работает, будет чуть сложнее.}%
\EN{Despite the simplicity of the function, it will be harder to understand how it works.}%
\DEph{}
\FR{Malgré la simplicité de la fonction, il va être difficile de comprendre comment elle fonctionne.}
\JA{機能の単純さにもかかわらず、それがどのように機能するかを理解することは難しいでしょう。}

% subsections
\subsubsection{x86}

% subsubsections
\EN{\myparagraph{\NonOptimizing MSVC}

MSVC 2010 generates the following:

\lstinputlisting[caption=\NonOptimizing MSVC 2010,style=customasmx86]{patterns/12_FPU/3_comparison/x86/MSVC/MSVC_EN.asm}

\myindex{x86!\Instructions!FLD}

So, \FLD loads \GTT{\_b} into \ST{0}.

\label{Czero_etc}
\newcommand{\Czero}{\GTT{C0}\xspace}
\newcommand{\Ctwo}{\GTT{C2}\xspace}
\newcommand{\Cthree}{\GTT{C3}\xspace}
\newcommand{\CThreeBits}{\Cthree/\Ctwo/\Czero}

\myindex{x86!\Instructions!FCOMP}

\FCOMP compares the value in \ST{0} with what is in \GTT{\_a} 
and sets \CThreeBits bits in FPU status word register, accordingly. 
This is a 16-bit register that reflects the current state of the FPU.

After the bits are set, the \FCOMP instruction also pops one variable from the stack. 
This is what distinguishes it from \FCOM, which is just compares values, leaving the stack in the same state.

Unfortunately, CPUs before Intel P6
\footnote{Intel P6 is Pentium Pro, Pentium II, etc.} don't have any conditional 
jumps instructions which check the \CThreeBits bits. 
Perhaps, it is a matter of history (recall: FPU was a separate chip in past).\\
Modern CPU starting at Intel P6 have \FCOMI/\FCOMIP/\FUCOMI/\FUCOMIP 
instructions~---which do the same, but modify the \ZF/\PF/\CF CPU flags.

\myindex{x86!\Instructions!FNSTSW}

The \FNSTSW instruction copies FPU the status word register to \AX. 
\CThreeBits bits are placed at positions 14/10/8, 
they are at the same positions in the \AX register and all they are placed in the high part of \AX{}~---\AH{}.

\begin{itemize}
\item If $b>a$ in our example, then \CThreeBits bits are to be set as following: 0, 0, 0.
\item If $a>b$, then the bits are: 0, 0, 1.
\item If $a=b$, then the bits are: 1, 0, 0.
\item

If the result is unordered (in case of error), then the set bits are: 1, 1, 1.
\end{itemize}
% TODO: table here?

This is how \CThreeBits bits are located in the \AX register:

\input{C3_in_AX}

This is how \CThreeBits bits are located in the \AH register:

\input{C3_in_AH}

After the execution of \INS{test ah, 5}\footnote{5=101b}, 
only \Czero and \Ctwo bits (on 0 and 2 position) are considered, all other bits are just
ignored.

\label{parity_flag}
\myindex{x86!\Registers!\Flags!Parity flag}

Now let's talk about the \emph{parity flag}, another notable historical rudiment.

This flag is set to 1 if the number of ones in the result of the last calculation is even, and to 0 if it is odd.

Let's look into Wikipedia\footnote{\href{http://go.yurichev.com/17131}{wikipedia.org/wiki/Parity\_flag}}:

\begin{framed}
\begin{quotation}
One common reason to test the parity flag actually has nothing to do with parity. The FPU has four condition flags 
(C0 to C3), but they cannot be tested directly, and must instead be first copied to the flags register. 
When this happens, C0 is placed in the carry flag, C2 in the parity flag and C3 in the zero flag. 
The C2 flag is set when e.g. incomparable floating point values (NaN or unsupported format) are compared 
with the FUCOM instructions.
\end{quotation}
\end{framed}

As noted in Wikipedia, the parity flag used sometimes in FPU code, let's see how.

\myindex{x86!\Instructions!JP}

The \PF flag is to be set to 1 if both \Czero and \Ctwo are set to 0 or both are 1, in which case
the subsequent \JP (\emph{jump if PF==1}) is triggering. 
If we recall the values of \CThreeBits for various cases,
we can see that the conditional jump 
\JP is triggering in two cases: if $b>a$ or $a=b$ 
(\Cthree bit is not considered here, since it has been cleared by the \INS{test ah, 5} instruction).

It is all simple after that. 
If the conditional jump has been triggered, 
\FLD loads the value of \GTT{\_b} 
in \ST{0}, and if it hasn't been triggered, the value of \GTT{\_a} is loaded there.

\mysubparagraph{And what about checking \Ctwo?}

The \Ctwo flag is set in case of error (\gls{NaN}, etc.), but our code doesn't check it.

If the programmer cares about FPU errors, he/she must add additional checks.

\input{patterns/12_FPU/3_comparison/x86/MSVC/olly_EN.tex}
}
\RU{\myparagraph{\NonOptimizing MSVC}

Вот что выдал MSVC 2010:

\lstinputlisting[caption=\NonOptimizing MSVC 2010,style=customasmx86]{patterns/12_FPU/3_comparison/x86/MSVC/MSVC_RU.asm}

\myindex{x86!\Instructions!FLD}
Итак, \FLD загружает \GTT{\_b} в регистр \ST{0}.

\label{Czero_etc}
\newcommand{\Czero}{\GTT{C0}\xspace}
\newcommand{\Ctwo}{\GTT{C2}\xspace}
\newcommand{\Cthree}{\GTT{C3}\xspace}
\newcommand{\CThreeBits}{\Cthree/\Ctwo/\Czero}

\myindex{x86!\Instructions!FCOMP}
\FCOMP сравнивает содержимое \ST{0} с тем, что лежит в \GTT{\_a} и выставляет биты \CThreeBits в 
регистре статуса FPU. Это 16-битный регистр отражающий текущее состояние FPU.

После этого инструкция \FCOMP также выдергивает одно значение из стека. 
Это отличает её от \FCOM, которая просто сравнивает значения, оставляя стек в таком же состоянии.

К сожалению, у процессоров до Intel P6
\footnote{Intel P6 это Pentium Pro, Pentium II, и последующие модели} нет инструкций условного перехода,
проверяющих биты \CThreeBits.
Возможно, так сложилось исторически (вспомните о том, что FPU когда-то был вообще отдельным чипом).\\
А у Intel P6 появились инструкции \FCOMI/\FCOMIP/\FUCOMI/\FUCOMIP, делающие то же самое, 
только напрямую модифицирующие флаги \ZF/\PF/\CF.

\myindex{x86!\Instructions!FNSTSW}
Так что \FNSTSW копирует содержимое регистра статуса в \AX. 
Биты \CThreeBits занимают позиции, 
соответственно, 14, 10, 8. В этих позициях они и остаются в регистре \AX, 
и все они расположены в старшей части регистра~--- \AH.

\begin{itemize}
\item Если $b>a$ в нашем случае, то биты \CThreeBits должны быть выставлены так: 0, 0, 0.
\item Если $a>b$, то биты будут выставлены: 0, 0, 1.
\item Если $a=b$, то биты будут выставлены так: 1, 0, 0.
\item Если результат не определен (в случае ошибки), то биты будут выставлены так: 1, 1, 1.
\end{itemize}
% TODO: table here?

Вот как биты \CThreeBits расположены в регистре \AX:

\input{C3_in_AX}

Вот как биты \CThreeBits расположены в регистре \AH:

\input{C3_in_AH}

После исполнения \INS{test ah, 5}\footnote{5=101b} % FIXME: subscript here!
будут учтены только биты \Czero и \Ctwo (на позициях 0 и 2), остальные просто проигнорированы.

\label{parity_flag}
\myindex{x86!\Registers!\Flags!Флаг четности}
Теперь немного о \emph{parity flag}\footnote{флаг четности}. 
Ещё один замечательный рудимент эпохи.

Этот флаг выставляется в 1 если количество единиц в последнем результате четно. 
И в 0 если нечетно.

Заглянем в Wikipedia\footnote{\href{http://go.yurichev.com/17131}{wikipedia.org/wiki/Parity\_flag}}:

\begin{framed}
\begin{quotation}
One common reason to test the parity flag actually has nothing to do with parity. The FPU has four condition flags 
(C0 to C3), but they cannot be tested directly, and must instead be first copied to the flags register. 
When this happens, C0 is placed in the carry flag, C2 in the parity flag and C3 in the zero flag. 
The C2 flag is set when e.g. incomparable floating point values (NaN or unsupported format) are compared 
with the FUCOM instructions.
\end{quotation}
\end{framed}

Как упоминается в Wikipedia, флаг четности иногда используется в FPU-коде и сейчас мы увидим как.

\myindex{x86!\Instructions!JP}
Флаг \PF будет выставлен в 1, если \Czero и \Ctwo оба 1 или оба 0. 
И тогда сработает последующий \JP (\emph{jump if PF==1}). 
Если мы вернемся чуть назад и посмотрим значения \CThreeBits 
для разных вариантов, то увидим, что условный переход \JP сработает в двух случаях: если $b>a$ или если $a=b$ 
(ведь бит \Cthree перестал учитываться после исполнения \INS{test ah, 5}).

Дальше всё просто. Если условный переход сработал, то \FLD загрузит значение \INS{\_b} в \ST{0}, 
а если не сработал, то загрузится \GTT{\_a} и произойдет выход из функции.

\mysubparagraph{А как же проверка флага \Ctwo?}

Флаг \Ctwo включается в случае ошибки (\gls{NaN}, итд.), но наш код его не проверяет.

Если программисту нужно знать, не произошла ли FPU-ошибка, он должен позаботиться об этом
дополнительно, добавив соответствующие проверки.

\input{patterns/12_FPU/3_comparison/x86/MSVC/olly_RU.tex}
}
\DE{\myparagraph{\NonOptimizing MSVC}

MSVC 2010 erzeugt den folgenden Code:

\lstinputlisting[caption=\NonOptimizing MSVC
2010,style=customasmx86]{patterns/12_FPU/3_comparison/x86/MSVC/MSVC_DE.asm}

\myindex{x86!\Instructions!FLD}

Der Befehl \FLD lädt \GTT{\_b} nach \ST{0}.

\label{Czero_etc}
\newcommand{\Czero}{\GTT{C0}\xspace}
\newcommand{\Ctwo}{\GTT{C2}\xspace}
\newcommand{\Cthree}{\GTT{C3}\xspace}
\newcommand{\CThreeBits}{\Cthree/\Ctwo/\Czero}

\myindex{x86!\Instructions!FCOMP}
\FCOMP verlgeicht den Wert in \ST{0} mit dem Wert, der sich in \GTT{\_a}
befindet und setzt die \CThreeBits im FPU Status Register entsprechend.
Das Statusregister ist ein 16-Bit-Register, das den aktueller Zustand der FPU
abbildet.

Nachdem die Bits gesetzt worden sind, nimmer der \FCOMP Befehl auch eine
Variable vom Stack. Dieses Verhalten unterscheidet ihn von \FCOM, der einfach
zwei Werte vergleicht und den Stack unangetastet lässt.

Leider verfügen CPUs vor Intel P6\footnote{Intel P6 ist Pentium Pro, Pentium II,
etc.}über keinerlei bedingte Sprungbefehle, die die \CThreeBits prüfen.

After the bits are set, the \FCOMP instruction also pops one variable from the stack. 
This is what distinguishes it from \FCOM, which is just compares values, leaving the stack in the same state.
Vielleicht ist diese Tatsache historisch begründet (man erinnere sich: die FPU
war früher ein eigener Chip).\\
Moderne CPUs, beginnend mit Intel P6 haben \FCOMI/\FCOMIP/\FUCOMI/\FUCOMIP
Befehle~--welche im Prinzip das gleiche tun, aber die \ZF/\PF/\CF Flags der CPU
verändern können.

\myindex{x86!\Instructions!FNSTSW}
Der \FNSTSW Befehl kopiert das FPU Statusregister nach \AX.
\CThreeBits werden an den Stellen 14/10/8 abgelegt, sie befinden sich im \AX
Register an den gleichen Stellen und sie werden alle in höherwertigen Teil von
\AX{}~---\AH{} abgelegt.

\begin{itemize}
\item Falls in unserem Beispiel $b>a$, dann werden die \CThreeBits Bits wie
folgt gesetzt: 0, 0, 0.
\item Falls $a>b$, dann ist das Bitmuster: 0, 0, 1.
\item Falls $a=b$, dann ist das Bitmuster: 1, 0, 0.
\item

Wenn das Ergebnis (z.B. im Fehlerfall) ungeordnet ist, dann werden die Bits wie
folgt gesetzt: 1,1,1.
\end{itemize}
% TODO: table here?
So werden die \CThreeBits Bits im \AX Register angeordnet:

\input{C3_in_AX}

So werden die \CThreeBits Bits im \AH Register angeordnet:

\input{C3_in_AH}
Nach der Ausführung von \INS{test ah, 5}\footnote{5=101b} werden nur die \Czero
und \Ctwo Bits (an den Stellen 0 und 2) betrachtet, alle übrigen Bits werden
einfach überlesen.

\label{parity_flag}
\myindex{x86!\Registers!\Flags!Parity flag}
Werfen wir nun einen Blick auf ein anderes bemerkenswertes historisches
Überbleibsel: das \emph{parity flag}.

Dieses Flag wird auf 1 gesetzt, falls die Anzahl der Einsen im Ergebnis der
letzten Berechnung gerade ist und auf 1, falls dies nicht der Fall ist.

Schlagen wir in der Wikipedia
nach\footnote{\href{http://go.yurichev.com/17131}{wikipedia.org/wiki/Parity\_flag}}:

%TODO Quotation has been translated from English wiki article, since the
% correspondig German article doesn't offer such information.
\begin{framed}
\begin{quotation}
Ein guter Grund das Parity Flag abzufragen, hat tatsächlich gar nichts mit
Parität zu tun. Die FPU hat vier Bedingungsflags (C0 bis C3), aber diese können
nicht direkt abgefragt werden, sondern müssen zunächst in das Flags Register
kopiert werden. Wenn dies geschieht, wird C0 im Carry Flag abgelegt, C2 im
Parity Flag und C3 im Zero Flag.
Das C2 Flag ist gesetzt, wenn z.B. unvergleichbare Fließkommawerte (NaN oder
nicht unterstütztes Format) über der \FUCOM Befehl miteinander verglichen
werden.\textit{(Übersetzung aus der englischen Wikipedia.)}
\end{quotation}
\end{framed}

Wie in der Wikipedia dargestellt wird das Parity Flag manchmal im FPU Code
verwendet; schauen wir uns genauer an wie das funktioniert.

\myindex{x86!\Instructions!JP}
Das \PF Flag wird auf 1 gesetzt, wenn sowohl \Czero als auch \Ctwo beide 0 oder
beide 1 sind. In diesem Fall wird der nachfolgende Sprung \JP(\emph{jump if
PF==1}) ausgeführt.
Wenn wir die Werte der \CThreeBits in den unterschiedlichen Fällen betrachten,
dann sehen wir, dass der bedingte Sprung \JP in zwei Fällen ausgeführt wird:
wenn $b>a$ oder wenn $a=b$ (das \Cthree Bit wird hier nicht betrachtet, da es
durch den Befehl \INS{test ah,5}) gelöscht wurde).

Der Rest ist leicht nachvollziehbar.
Denn der bedingte Sprung ausgeführt wurde, lädt \FLD den Wert von \GTT{\_b} nach
\ST{0} und wenn nicht, wird der Wert von \GTT{\_a} dorthin geladen.

\mysubparagraph{Was ist mit der Abfrage von \Ctwo?}
Das \Ctwo Flag wird im Fehlerfall (\gls{NaN}, etc.) gesetzt, aber unser Code
prüft dies nicht. 
Wenn sich der Programmierer für FPU Fehler interessiert, muss er zusätzliche
Abfragen hinzufügen.

\input{patterns/12_FPU/3_comparison/x86/MSVC/olly_DE.tex}
}
\FR{\myparagraph{MSVC \NonOptimizing}

MSVC 2010 génère ce qui suit:

\lstinputlisting[caption=MSVC 2010 \NonOptimizing,style=customasmx86]{patterns/12_FPU/3_comparison/x86/MSVC/MSVC_FR.asm}

\myindex{x86!\Instructions!FLD}

Ainsi, \FLD charge \GTT{\_b} dans \ST{0}.

\label{Czero_etc}
\newcommand{\Czero}{\GTT{C0}\xspace}
\newcommand{\Ctwo}{\GTT{C2}\xspace}
\newcommand{\Cthree}{\GTT{C3}\xspace}
\newcommand{\CThreeBits}{\Cthree/\Ctwo/\Czero}

\myindex{x86!\Instructions!FCOMP}

\FCOMP compare la valeur dans \ST{0} avec ce qui est dans \GTT{\_a} et met les bits
\CThreeBits du mot registre d'état du FPU, suivant le résultat.
Ceci est un registre 16-bit qui reflète l'état courant du FPU.

Après que les bits ont été mis, l'instruction \FCOMP dépile une variable depuis la
pile.
C'est ce qui la différencie de \FCOM, qui compare juste les valeurs, laissant la
pile dans le même état.

Malheureusement, les CPUs avant les Intel P6\footnote{Intel P6 comprend les Pentium
Pro, Pentium II, etc.} ne possèdent aucune instruction de saut conditionnel qui teste
les bits \CThreeBits.
Peut-être est-ce une raison historique (rappel: le FPU était une puce séparée dans
le passé).\\Les CPU modernes, à partir des Intel P6 possèdent les instructions \FCOMI/\FCOMIP/\FUCOMI/\FUCOMIP~---qui
font la même chose, mais modifient les flags \ZF/\PF/\CF du CPU.

\myindex{x86!\Instructions!FNSTSW}

L'instruction \FNSTSW copie le le mot du registre d'état du FPU dans \AX.
Les bits \CThreeBits sont placés aux positions 14/10/8, ils sont à la même position
dans le registre \AX et tous sont placés dans la partie haute de \AX{}~---\AH{}.

\begin{itemize}
\item Si $b>a$ dans notre exemple, alors les bits \CThreeBits sont mis comme ceci: 0, 0, 0.
\item Si $a>b$, alors les bits sont: 0, 0, 1.
\item Si $a=b$, alors les bits sont: 1, 0, 0.

Si le résultat n'est pas ordonné (en cas d'erreur), alors les bits sont: 1, 1, 1.
\end{itemize}
% TODO: table here?

Voici comment les bits \CThreeBits sont situés dans le registre \AX:

\input{C3_in_AX}

Voici comment les bits \CThreeBits sont situés dans le registre \AH:

\input{C3_in_AH}

Après l'exécution de \INS{test ah, 5}\footnote{5=101b}, seul les bits \Czero et \Ctwo
(en position 0 et 2) sont considérés, tous les autres bits sont simplement ignorés.

\label{parity_flag}
\myindex{x86!\Registers!\Flags!Parity flag}

Parlons maintenant du \emph{parity flag} (flag de parité), un autre rudiment historique
remarquable.

Ce flag est mis à 1 si le nombre de un dans le résultat du dernier calcul est pair,
et à 0 s'il est impair.

Regardons sur Wikipédia\footnote{\href{http://go.yurichev.com/17131}{wikipedia.org/wiki/Parity\_flag}}:

\begin{framed}
\begin{quotation}
Une raison commune de tester le bit de parité n'a rien à voir avec la parité. Le FPU
possède quatre flags de condition (C0 à C3), mais ils ne peuvent pas être testés
directement, et doivent d'abord être copiés dans le registre d'états.
Lorsque ça se produit, C0 est mis dans le flag de retenue, C2 dans le flag
de parité et C3 dans le flag de zéro.
Le flag C2 est mis lorsque e.g. des valeurs en virgule flottantes incomparable
(NaN ou format non supporté) sont comparées avec l'instruction \FUCOM.
\end{quotation}
\end{framed}

Comme indiqué dans Wikipédia, le flag de parité est parfois utilisé dans du code
FPU, voyons comment.

\myindex{x86!\Instructions!JP}

Le flag \PF est mis à 1 si à la fois \Czero et \Ctwo sont mis à 0 ou si les deux
sont à 1, auquel cas le \JP (\emph{jump if PF==1}) subséquent est déclenché.
Si l'on se rappelle les valeurs de \CThreeBits pour différents cas, nous pouvons
voir que le saut conditionnel \JP est déclenché dans deux cas: si $b>a$ ou $a=b$
(le bit \Cthree n'est pris en considération ici, puisqu'il a été mis à 0 par l'instruction
\INS{test ah, 5}).

C'est très simple ensuite.
Si le saut conditionnel a été déclenché, \FLD charge la valeur de \GTT{\_b} dans
\ST{0}, et sinon, la valeur de \GTT{\_a} est chargée ici.

\mysubparagraph{Et à propos du test de \Ctwo?}

Le flag \Ctwo est mis en cas d'erreur (\gls{NaN}, etc.), mais notre code ne le teste
pas.

Si le programmeur veut prendre en compte les erreurs FPU, il doit ajouter des tests
supplémentaires.

\input{patterns/12_FPU/3_comparison/x86/MSVC/olly_FR.tex}
}
\JA{\myparagraph{\NonOptimizing MSVC}

MSVC 2010は以下のコードを生成します。

\lstinputlisting[caption=\NonOptimizing MSVC 2010,style=customasmx86]{patterns/12_FPU/3_comparison/x86/MSVC/MSVC_JA.asm}

\myindex{x86!\Instructions!FLD}

\FLD は\GTT{\_b}を\ST{0}にロードします。

\label{Czero_etc}
\newcommand{\Czero}{\GTT{C0}\xspace}
\newcommand{\Ctwo}{\GTT{C2}\xspace}
\newcommand{\Cthree}{\GTT{C3}\xspace}
\newcommand{\CThreeBits}{\Cthree/\Ctwo/\Czero}

\myindex{x86!\Instructions!FCOMP}

\FCOMP は\ST{0}の値と\GTT{\_a}の値を比較し、
それに応じてFPUステータスワードレジスタの \CThreeBits ビットを設定します。
これは、FPUの現在の状態を反映する16ビットのレジスタです。

ビットがセットされると、 \FCOMP 命令はスタックから1つの変数もポップします。
これは、値を比較してスタックを同じ状態にしておく \FCOM とは区別されます。

残念ながら、インテルP6 
\footnote{インテルP6はPentium Pro、Pentium IIなどです。}
より前のCPUには、 \CThreeBits ビットをチェックする条件付きジャンプ命令はありません。
おそらく、それは歴史の問題です。(思い起こしてみてください:FPUは過去に別のチップでした)

インテルP6で始まる最新のCPUは、\FCOMI/\FCOMIP/\FUCOMI/\FUCOMIP 命令を持っていて、
同じことをしますが、 \ZF/\PF/\CF CPUフラグを変更します。

\myindex{x86!\Instructions!FNSTSW}

\FNSTSW 命令は状態レジスタであるFPUを \AX にコピーします。 
\CThreeBits ビットは14/10/8の位置に配置され、
\AX レジスタの同じ位置にあり、 \AX{}~---\AH{} の上位部分に配置されます。

\begin{itemize}
\item この例では $b>a$ の場合、 \CThreeBits ビットは0,0,0と設定します。
\item $a>b$ の場合、ビットは0,0,1です。
\item $a=b$ の場合、ビットは1,0,0です。
\item 結果が順序付けられていない場合(エラーの場合)、セットされたビットは1,1,1,1です。
\end{itemize}
% TODO: table here?

これは、 \CThreeBits ビットが \AX レジスタにどのように配置されるかを示しています。

\input{C3_in_AX}

これは、 \CThreeBits ビットが \AH レジスタにどのように配置されるかを示しています。

\input{C3_in_AH}

\INS{test ah, 5}\footnote{5=101b}の実行後、
\Czero と \Ctwo ビット(0と2の位置)のみが考慮され、他のビットはすべて無視されます。

\label{parity_flag}
\myindex{x86!\Registers!\Flags!Parity flag}

さて、\emph{パリティーフラグ}と注目すべきもう1つの歴史的基礎についてお話しましょう。

このフラグは、最後の計算結果の1の数が偶数の場合は1に設定され、奇数の場合は0に設定されます。

Wikipedia\footnote{\href{http://go.yurichev.com/17131}{wikipedia.org/wiki/Parity\_flag}}を見てみましょう:

\begin{framed}
\begin{quotation}
パリティフラグをテストする一般的な理由の1つに、無関係なFPUフラグをチェックすることがあります。 FPUには4つの条件フラグ
(C0~C3)がありますが、直接テストすることはできず、最初にフラグレジスタにコピーする必要があります。 
これが起こると、C0はキャリーフラグに、C2はパリティフラグに、C3はゼロフラグに置かれます。 
C2フラグは、例えば比較できない浮動小数点値(NaNまたはサポートされていない形式)がFUCOM命令と比較されます。
\end{quotation}
\end{framed}

Wikipediaで述べられているように、パリティフラグはFPUコードで使用されることがあります。

\myindex{x86!\Instructions!JP}

\Czero と \Ctwo の両方が0に設定されている場合、 \PF フラグは1に設定されます。その場合、
後続の \JP (\emph{jump if PF==1})が実行されます。 
いろいろな場合の \CThreeBits の値を思い出すと、
条件ジャンプ \JP は、 $b>a$ または $a=b$ の場合に実行されます。
(\INS{test ah, 5}命令によってクリアされているので、 \Cthree ビットはここでは考慮されていません)

それ以降はすべて簡単です。 
条件付きジャンプが実行された場合、
\FLD は\ST{0}の\GTT{\_b}の値をロードし、
実行されていなければ\GTT{\_a}の値をロードします。

\mysubparagraph{\Ctwo? のチェックは?}

\Ctwo フラグはエラー(\gls{NaN}など)の場合に設定されますが、コードではチェックされません。

プログラマがFPUエラーを気にする場合は、チェックを追加する必要があります。

\input{patterns/12_FPU/3_comparison/x86/MSVC/olly_JA.tex}
}

\EN{\subsection{Win32 PE}
\label{win32_pe}
\myindex{Windows!Win32}

\acs{PE} is an executable file format used in Windows.
The difference between .exe, .dll and .sys is that .exe and .sys usually do not have exports, only imports.

\myindex{OEP}

A \ac{DLL}, just like any other PE-file, has an entry point (\ac{OEP}) (the function DllMain() is located there)
but this function usually does nothing.
.sys is usually a device driver.
As of drivers, Windows requires the checksum to be present in the PE file and for it to be correct
\footnote{For example, Hiew(\myref{Hiew}) can calculate it}.

\myindex{Windows!Windows Vista}
Starting at Windows Vista, a driver's files must also be signed with a digital signature. It will fail to load otherwise.

\myindex{MS-DOS}
Every PE file begins with tiny DOS program that prints a
message like \q{This program cannot be run in DOS mode.}---if you run this program in DOS or Windows 3.1 (\ac{OS}-es which are not aware of the PE format),
this message will be printed.

\subsubsection{Terminology}

\myindex{VA}
\myindex{Base address}
\myindex{RVA}
\myindex{Windows!IAT}
\myindex{Windows!INT}

\begin{itemize}
\item Module---a separate file, .exe or .dll.

\item Process---a program loaded into memory and currently running.  Commonly consists of one .exe file and bunch of .dll files.

\item Process memory---the memory a process works with.  Each process has its own.
There usually are loaded modules, memory of the stack, \gls{heap}(s), etc.

\item \ac{VA}---an address which is to be used in program while runtime.

\item Base address (of module)---the address within the process memory at which the module is to be loaded.
\ac{OS} loader may change it, if the base address is already occupied by another module just loaded before.

\item \ac{RVA}---the \ac{VA}-address minus the base address.

Many addresses in PE-file tables use \ac{RVA}-addresses.

%\item
%Data directory --- ...

\item \ac{IAT}---an array of addresses of imported symbols \footnote{\PietrekPE}.
Sometimes, the \TT{IMAGE\_DIRECTORY\_ENTRY\_IAT} data directory points at the \ac{IAT}.
\label{IDA_idata}
It is worth noting that \ac{IDA} (as of 6.1) may allocate a pseudo-section named \TT{.idata} for
\ac{IAT}, even if the \ac{IAT} is a part of another section!

\item \ac{INT}---an array of names of symbols to be imported\footnote{\PietrekPE}.
\end{itemize}

\subsubsection{Base address}

The problem is that several module authors can prepare DLL files for others to use and it is not possible
to reach an agreement which addresses is to be assigned to whose modules.

So that is why if two necessary DLLs for a process have the same base address,
one of them will be loaded at this base address, and the other---at some other free space in process memory,
and each virtual addresses in the second DLL will be corrected.

\par With \ac{MSVC} the linker often generates the .exe files with a base address of \TT{0x400000}
\footnote{The origin of this address choice is described here: \href{http://go.yurichev.com/17041}{MSDN}},
and with the code section starting at \TT{0x401000}.
This means that the \ac{RVA} of the start of the code section is \TT{0x1000}.

DLLs are often generated by MSVC's linker with a base address of \TT{0x10000000}
\footnote{This can be changed by the /BASE linker option}.

\myindex{ASLR}

There is also another reason to load modules at various base addresses, in this case random ones.
It is \ac{ASLR}\footnote{\href{http://go.yurichev.com/17140}{wikipedia}}.

\myindex{Shellcode}

A shellcode trying to get executed on a compromised system must call system functions, hence, know their addresses.

In older \ac{OS} (in \gls{Windows NT} line: before Windows Vista),
system DLL (like kernel32.dll, user32.dll) were always loaded at known addresses,
and if we also recall
that their versions rarely changed, the addresses of functions were
fixed and shellcode could call them directly.

In order to avoid this, the \ac{ASLR}
method loads your program and all modules it needs at random base addresses, different every time.

\ac{ASLR} support is denoted in a PE file by setting the flag
\par \TT{IMAGE\_DLL\_CHARACTERISTICS\_DYNAMIC\_BASE} \InSqBrackets{see \Russinovich}.

\subsubsection{Subsystem}

There is also a \emph{subsystem} field, usually it is:

\myindex{Native API}

\begin{itemize}
\item native\footnote{Meaning, the module use Native API instead of Win32} (.sys-driver),

\item console (console application) or

\item \ac{GUI} (non-console).
\end{itemize}

\subsubsection{OS version}

A PE file also specifies the minimal Windows version it needs in order to be loadable.

The table of version numbers stored in the PE file and corresponding Windows codenames is
here\footnote{\href{http://go.yurichev.com/17044}{wikipedia}}.

\myindex{Windows!Windows NT4}
\myindex{Windows!Windows 2000}
For example, \ac{MSVC} 2005 compiles .exe files for running on Windows NT4 (version 4.00), but \ac{MSVC} 2008 does not
(the generated files have a version of 5.00, at least Windows 2000 is needed to run them).

\myindex{Windows!Windows XP}

\ac{MSVC} 2012 generates .exe files of version 6.00 by default,
targeting at least Windows Vista.
However, by changing the compiler's options\footnote{\href{http://go.yurichev.com/17045}{MSDN}},
it is possible to force it to compile for Windows XP.

\subsubsection{Sections}

Division in sections, as it seems, is present in all executable file formats.

It is devised in order to separate code from data, and data---from constant data.

\begin{itemize}
\item Either the \emph{IMAGE\_SCN\_CNT\_CODE} or \emph{IMAGE\_SCN\_MEM\_EXECUTE} flags will be set on the code section---this is executable code.

\item On data section---\emph{IMAGE\_SCN\_CNT\_INITIALIZED\_DATA},\\
\emph{IMAGE\_SCN\_MEM\_READ} and \emph{IMAGE\_SCN\_MEM\_WRITE} flags.

\item On an empty section with uninitialized data---\\
\emph{IMAGE\_SCN\_CNT\_UNINITIALIZED\_DATA}, \emph{IMAGE\_SCN\_MEM\_READ} \\
        and \emph{IMAGE\_SCN\_MEM\_WRITE}.

\item On a constant data section (one that's protected from writing), the flags \\
\emph{IMAGE\_SCN\_CNT\_INITIALIZED\_DATA} and \emph{IMAGE\_SCN\_MEM\_READ} can be set, \\
but not \emph{IMAGE\_SCN\_MEM\_WRITE}.
A process going to crash if it tries to write to this section.
\end{itemize}

\myindex{TLS}
\myindex{BSS}
Each section in PE-file may have a name, however, it is not very important.
Often (but not always) the code section is named \TT{.text},
the data section---\TT{.data}, the constant data section --- \TT{.rdata} \emph{(readable data)}.
Other popular section names are:

\myindex{MIPS}
\begin{itemize}
\item \TT{.idata}---imports section.
\ac{IDA} may create a pseudo-section named like this: \myref{IDA_idata}.
\item \TT{.edata}---exports section (rare)
\item \TT{.pdata}---section holding all information about exceptions in Windows NT for MIPS, \ac{IA64} and x64: \myref{SEH_win64}
\item \TT{.reloc}---relocs section
\item \TT{.bss}---uninitialized data (\ac{BSS})
\item \TT{.tls}---thread local storage (\ac{TLS})
\item \TT{.rsrc}---resources
\item \TT{.CRT}---may present in binary files compiled by ancient MSVC versions
\end{itemize}

PE file packers/encryptors often garble section names or replace the names with their own.

\ac{MSVC} allows you to declare data in arbitrarily named section
\footnote{\href{http://go.yurichev.com/17047}{MSDN}}.

Some compilers and linkers can add a section with debugging symbols and
other debugging information (MinGW for instance).
\myindex{Windows!PDB}
However it is not so in latest versions of \ac{MSVC} (separate \gls{PDB} files are used there for this purpose).\\
\\
That is how a PE section is described in the file:

\begin{lstlisting}[style=customc]
typedef struct _IMAGE_SECTION_HEADER {
  BYTE  Name[IMAGE_SIZEOF_SHORT_NAME];
  union {
    DWORD PhysicalAddress;
    DWORD VirtualSize;
  } Misc;
  DWORD VirtualAddress;
  DWORD SizeOfRawData;
  DWORD PointerToRawData;
  DWORD PointerToRelocations;
  DWORD PointerToLinenumbers;
  WORD  NumberOfRelocations;
  WORD  NumberOfLinenumbers;
  DWORD Characteristics;
} IMAGE_SECTION_HEADER, *PIMAGE_SECTION_HEADER;
\end{lstlisting}
\footnote{\href{http://go.yurichev.com/17048}{MSDN}}

\myindex{Hiew}
A word about terminology: \emph{PointerToRawData} is called \q{Offset} in Hiew
and \emph{VirtualAddress} is called \q{RVA} there.

\subsubsection{Data section}

Data section in file can be smaller than in memory.
For example, some variables can be initialized, some are not.
Compiler and linker will collect them all into one section, but the first part of it is initialized and allocated in file,
while another is absent in file (of course, to make it smaller).
\emph{VirtualSize} will be equal to the size of section in memory, and \emph{SizeOfRawData} --- to
size of section in file.

IDA can show the border between initialized and not initialized parts like that:

\begin{lstlisting}[style=customasmx86]
...

.data:10017FFA                 db    0
.data:10017FFB                 db    0
.data:10017FFC                 db    0
.data:10017FFD                 db    0
.data:10017FFE                 db    0
.data:10017FFF                 db    0
.data:10018000                 db    ? ;
.data:10018001                 db    ? ;
.data:10018002                 db    ? ;
.data:10018003                 db    ? ;
.data:10018004                 db    ? ;
.data:10018005                 db    ? ;

...
\end{lstlisting}

\subsubsection{Relocations (relocs)}
\label{subsec:relocs}

\ac{AKA} FIXUP-s (at least in Hiew).

They are also present in almost all executable file formats
\footnote{Even in .exe files for MS-DOS}.
Exceptions are shared dynamic libraries compiled with \ac{PIC}, or any other \ac{PIC}-code.

What are they for?

Obviously, modules can be loaded on various base addresses, but how to deal with global variables, for example?
They must be accessed by address.  One solution is \PICcode{} (\myref{sec:PIC}).
But it is not always convenient.

That is why a relocations table is present.
There the addresses of points that must be corrected are enumerated,
in case of loading at a different base address.

% TODO тут бы пример с HIEW или objdump..
For example, there is a global variable at address \TT{0x410000} and this is how it is accessed:

\begin{lstlisting}[style=customasmx86]
A1 00 00 41 00         mov         eax,[000410000]
\end{lstlisting}

The base address of the module is \TT{0x400000}, the \ac{RVA} of the global variable is \TT{0x10000}.

If the module is loaded at base address \TT{0x500000}, the real address of the global variable must be \TT{0x510000}.

\myindex{x86!\Instructions!MOV}

As we can see, the address of variable is encoded in the instruction \TT{MOV}, after the byte \TT{0xA1}.

That is why the address of the 4 bytes after \TT{0xA1}, is written in the relocs table.

If the module is loaded at a different base address, the \ac{OS} loader enumerates all addresses in the table,

finds each 32-bit word the address points to, subtracts the original base address from it
(we get the \ac{RVA} here), and adds the new base address to it.

If a module is loaded at its original base address, nothing happens.

All global variables can be treated like that.

Relocs may have various types, however, in Windows for x86 processors, the type is usually \\
\emph{IMAGE\_REL\_BASED\_HIGHLOW}.

\myindex{Hiew}

By the way, relocs are darkened in Hiew, for example: \figref{fig:scanf_ex3_hiew_1}.

\myindex{\olly}
\olly underlines the places in memory to which relocs are to be applied, for example: \figref{fig:switch_lot_olly3}.

\subsubsection{Exports and imports}

\label{PE_exports_imports}
As we all know, any executable program must use the \ac{OS}'s services and other DLL-libraries somehow.

It can be said that functions from one module (usually DLL) must be connected somehow to the points of their
calls in other modules (.exe-file or another DLL).

For this, each DLL has an \q{exports} table, which consists of functions plus their addresses in a module.

And every .exe file or DLL has \q{imports}, a table of functions it needs for execution including
list of DLL filenames.

After loading the main .exe-file, the \ac{OS} loader processes imports table:
it loads the additional DLL-files, finds function names
among the DLL exports and writes their addresses down in the \ac{IAT} of the main .exe-module.

\myindex{Windows!Win32!Ordinal}

As we can see, during loading the loader must compare a lot of function names, but string comparison is not a very
fast procedure, so there is a support for \q{ordinals} or \q{hints},
which are function numbers stored in the table, instead of their names.

That is how they can be located faster when loading a DLL.
Ordinals are always present in the \q{export} table.

\myindex{MFC}
For example, a program using the \ac{MFC} library usually loads mfc*.dll by ordinals,
and in such programs there are no \ac{MFC} function names in \ac{INT}.

% TODO example!
When loading such programs in \IDA, it will ask for a path to the mfc*.dll files
in order to determine the function names.

If you don't tell \IDA the path to these DLLs, there will be \emph{mfc80\_123} instead of function names.

\myparagraph{Imports section}

Often a separate section is allocated for the imports table and everything related to it (with name like \TT{.idata}),
however, this is not a strict rule.

Imports are also a confusing subject because of the terminological mess. Let's try to collect all information in one place.

\begin{figure}[H]
\centering
\myincludegraphics{OS/PE/unnamed0.png}
\caption{
A scheme that unites all PE-file structures related to imports}
\end{figure}

The main structure is the array \emph{IMAGE\_IMPORT\_DESCRIPTOR}.
Each element for each DLL being imported.

Each element holds the \ac{RVA} address of the text string (DLL name) (\emph{Name}).

\emph{OriginalFirstThunk} is the \ac{RVA} address of the \ac{INT} table.
This is an array of \ac{RVA} addresses, each of which points to a text string with a function name.
Each string is prefixed by a 16-bit integer
(\q{hint})---\q{ordinal} of function.

While loading, if it is possible to find a function by ordinal,
then the strings comparison will not occur. The array is terminated by zero.

There is also a pointer to the \ac{IAT} table named \emph{FirstThunk}, it is just the \ac{RVA} address
of the place where the loader writes the addresses of the resolved functions.

The points where the loader writes addresses are marked by \IDA like this: \emph{\_\_imp\_CreateFileA}, etc.

There are at least two ways to use the addresses written by the loader.

\myindex{x86!\Instructions!CALL}
\begin{itemize}
\item The code will have instructions like \emph{call \_\_imp\_CreateFileA},
and since the field with the address of the imported function is a global variable in some sense,
the address of the \emph{call} instruction (plus 1 or 2) is to be added to the relocs table,
for the case when the module is loaded at a different base address.

But, obviously, this may enlarge relocs table significantly.

Because there are might be a lot of calls to imported functions in the module.

Furthermore, large relocs table slows down the process of loading modules.

\myindex{x86!\Instructions!JMP}
\myindex{thunk-functions}
\item For each imported function, there is only one jump allocated, using the \JMP instruction
plus a reloc to it.
Such points are also called \q{thunks}.

All calls to the imported functions are just \CALL instructions to the corresponding \q{thunk}.
In this case, additional relocs are not necessary because these \CALL{}-s
have relative addresses and do not need to be corrected.
\end{itemize}

These two methods can be combined.

Possible, the linker creates individual \q{thunk}s if there are too many calls to the function,
but not done by default. \\
\\
By the way, the array of function addresses to which FirstThunk is pointing is not necessary to be located in the \ac{IAT} section.
For example, the author of these lines once wrote the PE\_add\_import\footnote{\href{http://go.yurichev.com/17049}{yurichev.com}}
utility for adding imports to an existing .exe-file.

Some time earlier, in the previous versions of the utility,
at the place of the function you want to substitute with a call to another DLL,
my utility wrote the following code:

\begin{lstlisting}[style=customasmx86]
MOV EAX, [yourdll.dll!function]
JMP EAX
\end{lstlisting}

FirstThunk points to the first instruction. In other words, when loading yourdll.dll,
the loader writes the address of the \emph{function} function right in the code.

It also worth noting that a code section is usually write-protected, so my utility adds the \\
\emph{IMAGE\_SCN\_MEM\_WRITE}
flag for code section. Otherwise, the program to crash while loading with error code
5 (access denied). \\
\\
One might ask: what if I supply a program with a set of DLL files which is not supposed to change (including addresses of all DLL functions),
is it possible to speed up the loading process?

Yes, it is possible to write the addresses of the functions to be imported into the FirstThunk arrays in advance.
The \emph{Timestamp} field is present in the \\
\emph{IMAGE\_IMPORT\_DESCRIPTOR} structure.

If a value is present there, then the loader compares this value with the date-time of the DLL file.

If the values are equal, then the loader does not do anything, and the loading of the process can be faster.
This is called \q{old-style binding}
\footnote{\href{http://go.yurichev.com/17050}{MSDN}. There is also the \q{new-style binding}.}.
\myindex{BIND.EXE}

The BIND.EXE utility in Windows SDK is for this.
For speeding up the loading of your program, Matt Pietrek in \PietrekPEURL, suggests to do the binding shortly after your program
installation on the computer of the end user. \\
\\
PE-files packers/encryptors may also compress/encrypt imports table.

In this case, the Windows loader, of course, will not load all necessary DLLs.
\myindex{Windows!Win32!LoadLibrary}
\myindex{Windows!Win32!GetProcAddress}

Therefore, the packer/encryptor does this on its own, with the help of
\emph{LoadLibrary()} and the \emph{GetProcAddress()} functions.

That is why these two functions are often present in \ac{IAT} in packed files.\\
\\
In the standard DLLs from the Windows installation, \ac{IAT} often is located right at the beginning of the PE file.
Supposedly, it is made so for optimization.

While loading, the .exe file is not loaded into memory as a whole (recall huge install programs which are
started suspiciously fast), it is \q{mapped}, and loaded into memory in parts as they are accessed.

Probably, Microsoft developers decided it will be faster.

\subsubsection{Resources}

\label{PEresources}

Resources in a PE file are just a set of icons, pictures, text strings, dialog descriptions.

Perhaps they were separated from the main code, so all these things could be multilingual,
and it would be simpler to pick text or picture for the language that is currently set in the \ac{OS}. \\
\\
As a side effect, they can be edited easily and saved back to the executable file, even if one does not have special knowledge,
by using the ResHack editor, for example (\myref{ResHack}).

\subsubsection{.NET}

\myindex{.NET}

.NET programs are not compiled into machine code but into a special bytecode.
\myindex{OEP}
Strictly speaking, there is bytecode instead of the usual x86 code
in the .exe file, however, the entry point (\ac{OEP}) points to this tiny fragment of x86 code:

\begin{lstlisting}[style=customasmx86]
jmp         mscoree.dll!_CorExeMain
\end{lstlisting}

The .NET loader is located in mscoree.dll, which processes the PE file.
\myindex{Windows!Windows XP}

It was so in all pre-Windows XP \ac{OS}es. Starting from XP, the \ac{OS} loader is able to detect the .NET file
and run it without executing that \JMP instruction
\footnote{\href{http://go.yurichev.com/17051}{MSDN}}.

\myindex{TLS}
\subsubsection{TLS}

This section holds initialized data for the \ac{TLS}(\myref{TLS}) (if needed).
When a new thread start, its \ac{TLS} data is initialized using the data from this section. \\
\\
\myindex{TLS!Callbacks}

Aside from that, the PE file specification also provides initialization of the
\ac{TLS} section, the so-called TLS callbacks.

If they are present, they are to be called before the control is passed to the main entry point (\ac{OEP}).

This is used widely in the PE file packers/encryptors.

\subsubsection{Tools}

\myindex{objdump}
\myindex{Cygwin}
\myindex{Hiew}
\label{ResHack}

\begin{itemize}
\item objdump (present in cygwin) for dumping all PE-file structures.

\item Hiew(\myref{Hiew}) as editor.

\item pefile---Python-library for PE-file processing \footnote{\url{http://go.yurichev.com/17052}}.

\item ResHack \acs{AKA} Resource Hacker---resources editor\footnote{\url{http://go.yurichev.com/17052}}.

\item PE\_add\_import\footnote{\url{http://go.yurichev.com/17049}}---
simple tool for adding symbol(s) to PE executable import table.

\item PE\_patcher\footnote{\href{http://go.yurichev.com/17054}{yurichev.com}}---simple tool for patching PE executables.

\item PE\_search\_str\_refs\footnote{\href{http://go.yurichev.com/17055}{yurichev.com}}---simple tool for searching for a function in PE executables which use some text string.
\end{itemize}

\subsubsection{Further reading}

% FIXME: bibliography per chapter or section
\begin{itemize}
\item Daniel Pistelli---The .NET File Format \footnote{\url{http://go.yurichev.com/17056}}
\end{itemize}

}
\RU{\mysection{Пример вычисления адреса сети}

Как мы знаем, TCP/IP-адрес (IPv4) состоит из четырех чисел в пределах $0 \ldots 255$, т.е. 4 байта.

4 байта легко помещаются в 32-битную переменную, так что адрес хоста в IPv4, сетевая маска или адрес сети
могут быть 32-битными числами.

С точки зрения пользователя, маска сети определяется четырьмя числами в формате вроде \\
255.255.255.0,
но сетевые инженеры (сисадмины) используют более компактную нотацию (\ac{CIDR}),
вроде  \q{/8}, \q{/16}, итд.

Эта нотация просто определяет количество бит в сетевой маске, начиная с \ac{MSB}.

\small
\begin{center}
\begin{tabular}{ | l | l | l | l | l | l | }
\hline
\HeaderColor Маска & 
\HeaderColor Хосты & 
\HeaderColor Свободно &
\HeaderColor Сетевая маска &
\HeaderColor В шест.виде &
\HeaderColor \\
\hline
/30  & 4        & 2        & 255.255.255.252  & 0xfffffffc  & \\
\hline
/29  & 8        & 6        & 255.255.255.248  & 0xfffffff8  & \\
\hline
/28  & 16       & 14       & 255.255.255.240  & 0xfffffff0  & \\
\hline
/27  & 32       & 30       & 255.255.255.224  & 0xffffffe0  & \\
\hline
/26  & 64       & 62       & 255.255.255.192  & 0xffffffc0  & \\
\hline
/24  & 256      & 254      & 255.255.255.0    & 0xffffff00  & сеть класса C \\
\hline
/23  & 512      & 510      & 255.255.254.0    & 0xfffffe00  & \\
\hline
/22  & 1024     & 1022     & 255.255.252.0    & 0xfffffc00  & \\
\hline
/21  & 2048     & 2046     & 255.255.248.0    & 0xfffff800  & \\
\hline
/20  & 4096     & 4094     & 255.255.240.0    & 0xfffff000  & \\
\hline
/19  & 8192     & 8190     & 255.255.224.0    & 0xffffe000  & \\
\hline
/18  & 16384    & 16382    & 255.255.192.0    & 0xffffc000  & \\
\hline
/17  & 32768    & 32766    & 255.255.128.0    & 0xffff8000  & \\
\hline
/16  & 65536    & 65534    & 255.255.0.0      & 0xffff0000  & сеть класса B \\
\hline
/8   & 16777216 & 16777214 & 255.0.0.0        & 0xff000000  & сеть класса A \\
\hline
\end{tabular}
\end{center}
\normalsize

Вот простой пример, вычисляющий адрес сети используя сетевую маску и адрес хоста.

\lstinputlisting[style=customc]{\CURPATH/netmask.c}

\subsection{calc\_network\_address()}

Функция \TT{calc\_network\_address()} самая простая: 

она просто умножает (логически, используя \AND) адрес хоста на сетевую маску, в итоге давая адрес
сети.

\lstinputlisting[caption=\Optimizing MSVC 2012 /Ob0,numbers=left,style=customasmx86]{\CURPATH/calc_network_address_MSVC_2012_Ox.asm}

На строке 22 мы видим самую важную инструкцию \AND --- так вычисляется адрес сети.

\subsection{form\_IP()}

Функция \TT{form\_IP()} просто собирает все 4 байта в одно 32-битное значение.

Вот как это обычно происходит:

\begin{itemize}
\item Выделите переменную для возвращаемого значения. Обнулите её.

\item 
Возьмите четвертый (самый младший) байт, сложите его (логически, инструкцией \OR) с возвращаемым
значением. Оно содержит теперь 4-й байт.

\item Возьмите третий байт, сдвиньте его на 8 бит влево.
Получится значение в виде \TT{0x0000bb00}, где \TT{bb} это третий байт.
Сложите итоговое значение (логически, инструкцией \OR) с возвращаемым значением.
Возвращаемое значение пока что содержит \TT{0x000000aa}, так что логическое сложение
в итоге выдаст значение вида \TT{0x0000bbaa}.

\item 
Возьмите второй байт, сдвиньте его на 16 бит влево.
Вы получите значение вида \TT{0x00cc0000}, где \TT{cc} это второй байт.
Сложите (логически) результат и возвращаемое значение.
Выходное значение содержит пока что \TT{0x0000bbaa}, так что логическое сложение
в итоге выдаст значение вида \TT{0x00ccbbaa}.

\item 
Возьмите первый байт, сдвиньте его на 24 бита влево.
Вы получите значение вида \TT{0xdd000000}, где \TT{dd} это первый байт.
Сложите (логически) результат и выходное значение.
Выходное значение содержит пока что \TT{0x00ccbbaa}, так что сложение выдаст в итоге значение
вида \TT{0xddccbbaa}.

\end{itemize}

И вот как работает неоптимизирующий MSVC 2012:

\lstinputlisting[caption=\NonOptimizing MSVC 2012,style=customasmx86]{\CURPATH/form_IP_MSVC_2012_RU.asm}

Хотя, порядок операций другой, но, конечно, порядок роли не играет.

\Optimizing MSVC 2012 делает то же самое, но немного иначе:

\lstinputlisting[caption=\Optimizing MSVC 2012 /Ob0,style=customasmx86]{\CURPATH/form_IP_MSVC_2012_Ox_RU.asm}

Можно сказать, что каждый байт записывается в младшие 8 бит возвращаемого значения,
и затем возвращаемое значение сдвигается на один байт влево на каждом шаге.

Повторять 4 раза, для каждого байта.

\par
Вот и всё! 
К сожалению, наверное, нет способа делать это иначе.
Не существует более-менее популярных \ac{CPU} или \ac{ISA}, где имеется инструкция для сборки значения из бит или байт.
Обычно всё это делает сдвигами бит и логическим сложением (OR).

\subsection{print\_as\_IP()}

\TT{print\_as\_IP()} делает наоборот: расщепляет 32-битное значение на 4 байта.

Расщепление работает немного проще: просто сдвигайте входное значение на 24, 16, 8 или 0 бит,
берите биты с нулевого по седьмой (младший байт), вот и всё:

\lstinputlisting[caption=\NonOptimizing MSVC 2012,style=customasmx86]{\CURPATH/print_as_IP_MSVC_2012_RU.asm}

\Optimizing MSVC 2012 делает почти всё то же самое, только без ненужных перезагрузок входного значения:

\lstinputlisting[caption=\Optimizing MSVC 2012 /Ob0,style=customasmx86]{\CURPATH/print_as_IP_MSVC_2012_Ox.asm}

\subsection{form\_netmask() и set\_bit()}

\TT{form\_netmask()} делает сетевую маску из \ac{CIDR}-нотации.

Конечно, было бы куда эффективнее использовать здесь какую-то уже готовую таблицу, но мы рассматриваем
это именно так, сознательно, для демонстрации битовых сдвигов.
Мы также сделаем отдельную функцию \TT{set\_bit()}. 

Не очень хорошая идея выделять отдельную функцию для такой примитивной операции, но так будет проще понять,
как это всё работает.

\lstinputlisting[caption=\Optimizing MSVC 2012 /Ob0,style=customasmx86]{\CURPATH/form_netmask_MSVC_2012_Ox.asm}

\TT{set\_bit()} примитивна: просто сдвигает единицу на нужное количество бит, затем складывает (логически) с
входным значением \q{input}.
\TT{form\_netmask()} имеет цикл: он выставит столько бит (начиная с \ac{MSB}), 
сколько передано в аргументе \TT{netmask\_bits}.

\subsection{Итог}

Вот и всё!
Мы запускаем и видим:

\begin{lstlisting}
netmask=255.255.255.0
network address=10.1.2.0
netmask=255.0.0.0
network address=10.0.0.0
netmask=255.255.255.128
network address=10.1.2.0
netmask=255.255.255.192
network address=10.1.2.64
\end{lstlisting}
}
\DE{\mysection{\Stack}
\label{sec:stack}
\myindex{\Stack}

Der Stack ist eine der fundamentalen Datenstrukturen in der Informatik.
\footnote{\href{http://go.yurichev.com/17119}{wikipedia.org/wiki/Call\_Stack}}.
\ac{AKA} \ac{LIFO}.

Technisch betrachtet ist es ein Stapelspeicher innerhalb des Prozessspeichers der zusammen mit den \ESP (x86), \RSP (x64) oder dem \ac{SP} (ARM) Register als ein Zeiger in diesem Speicherblock fungiert.

\myindex{ARM!\Instructions!PUSH}
\myindex{ARM!\Instructions!POP}
\myindex{x86!\Instructions!PUSH}
\myindex{x86!\Instructions!POP}

Die häufigsten Stack-Zugriffsinstruktionen sind die \PUSH- und \POP-Instruktionen (in beidem x86 und ARM Thumb-Modus). \PUSH subtrahiert vom \ESP/\RSP/\ac{SP} 4 Byte im 32-Bit Modus (oder 8 im 64-Bit Modus) und schreibt dann den Inhalt des Zeigers an die Adresse auf die von \ESP/\RSP/\ac{SP} gezeigt wird.

\POP ist die umgekehrte Operation: Die Daten des Zeigers für die Speicherregion auf die von \ac{SP}
gezeigt wird werden ausgelesen und die Inhalte in den Instruktionsoperanden geschreiben (oft ist das ein Register). Dann werden 4 (beziehungsweise 8) Byte zum \gls{stack pointer} addiert.

Nach der Stackallokation, zeigt der \gls{stack pointer} auf den Boden des Stacks.
\PUSH verringert den \gls{stack pointer} und \POP erhöht ihn.
Der Boden des Stacks ist eigentlich der Anfang der Speicherregion die für den Stack reserviert wurde.
Das wirkt zunächst seltsam, aber so funktioniert es.

ARM unterstützt beides, aufsteigende und absteigende Stacks.

\myindex{ARM!\Instructions!STMFD}
\myindex{ARM!\Instructions!LDMFD}
\myindex{ARM!\Instructions!STMED}
\myindex{ARM!\Instructions!LDMED}
\myindex{ARM!\Instructions!STMFA}
\myindex{ARM!\Instructions!LDMFA}
\myindex{ARM!\Instructions!STMEA}
\myindex{ARM!\Instructions!LDMEA}

Zum Beispiel die \ac{STMFD}/\ac{LDMFD} und \ac{STMED}/\ac{LDMED} Instruktionen sind alle dafür gedacht mit einem absteigendem Stack zu arbeiten ( wächst nach unten, fängt mit hohen Adressen an und entwickelt sich zu niedrigeren Adressen). Die \ac{STMFA}/\ac{LDMFA} und \ac{STMEA}/\ac{LDMEA} Instruktionen sind dazu gedacht mit einem aufsteigendem Stack zu arbeiten (wächst nach oben und fängt mit niedrigeren Adressen an und wächst nach oben).

% It might be worth mentioning that STMED and STMEA write first,
% and then move the pointer, and that LDMED and LDMEA move the pointer first, and then read.
% In other words, ARM not only lets the stack grow in a non-standard direction,
% but also in a non-standard order.
% Maybe this can be in the glossary, which would explain why E stands for "empty".

\subsection{Warum wächst der Stack nach unten?}
\label{stack_grow_backwards}

Intuitiv, würden man annehmen das der Stack nach oben wächst z.B Richtung höherer Adressen, so wie bei jeder anderen Datenstruktur.

Der Grund das der Stack rückwärts wächst ist wohl historisch bedingt. Als Computer so groß waren das sie einen ganzen Raum beansprucht haben war es einfach Speicher in zwei Sektionen zu unterteilen, einen Teil für den \gls{heap} und einen Teil für den Stack. Sicher war zu dieser Zeit nicht bekannt wie groß der \gls{heap} und der Stack wachsen würden, während der Programm Laufzeit, also war die Lösung die einfachste mögliche.

\input{patterns/02_stack/stack_and_heap}

In \RitchieThompsonUNIX können wir folgendes lesen:

\begin{framed}
\begin{quotation}
Der user-core eines Programm Images wird in drei logische Segmente unterteilt. Das Programm-Text Segment beginnt bei 0 im virtuellen Adress Speicher. Während der Ausführung wird das Segment als schreibgeschützt markiert und eine einzelne Kopie des Segments wird unter allen Prozessen geteilt die das Programm ausführen. An der ersten 8K grenze über dem Programm Text Segment im Virtuellen Speicher, fängt der ``nonshared'' Bereich an, der nach Bedarf von Syscalls erweitert werden kann. Beginnend bei der höchsten Adresse im Virtuellen Speicher ist das Stack Segment, das Automatisch nach unten wächst während der Hardware Stackpointer sich ändert.
\end{quotation}
\end{framed}

Das erinnert daran wie manche Schüler Notizen zu  zwei Vorträgen in einem Notebook dokumentieren:
Notizen für den ersten Vortrag werden normal notiert, und Notizen zur zum zweiten Vortrag werden 
ans Ende des Notizbuches geschrieben, indem man das Notizbuch umdreht. Die Notizen treffen sich irgendwann
im Notizbuch aufgrund des fehlenden Freien Platzes.

% I think if we want to expand on this analogy,
% one might remember that the line number increases as as you go down a page.
% So when you decrease the address when pushing to the stack, visually,
% the stack does grow upwards.
% Of course, the problem is that in most human languages,
% just as with computers,
% we write downwards, so this direction is what makes buffer overflows so messy.

\subsection{Für was wird der Stack benutzt?}

% subsections
\input{patterns/02_stack/01_saving_ret_addr}
\input{patterns/02_stack/02_args_passing}
\EN{\input{patterns/02_stack/03_local_vars_EN}}
\RU{\input{patterns/02_stack/03_local_vars_RU}}
\DE{\input{patterns/02_stack/03_local_vars_DE}}
\PTBR{\input{patterns/02_stack/03_local_vars_PTBR}}
\input{patterns/02_stack/04_alloca/main}
\input{patterns/02_stack/05_SEH}
\input{patterns/02_stack/06_BO_protection}

\subsubsection{Automatisches deallokieren der Daten auf dem Stack}

Vielleicht ist der Grund warum man lokale Variablen und SEH Einträge auf dem Stack speichert, weil sie beim 
verlassen der Funktion automatisch aufgeräumt werden. Man braucht dabei nur eine Instruktion um die Position
des Stackpointers zu korrigieren (oftmals ist es die \ADD Instruktion). Funktions Argumente, könnte man sagen 
werden auch am Ende der Funktion deallokiert. Im Kontrast dazu, alles was auf dem \emph{heap} gespeichert wird muss
explizit deallokiert werden. 

% sections
\EN{\input{patterns/02_stack/07_layout_EN}}
\RU{\input{patterns/02_stack/07_layout_RU}}
\DE{\input{patterns/02_stack/07_layout_DE}}
\PTBR{\input{patterns/02_stack/07_layout_PTBR}}
\input{patterns/02_stack/08_noise/main}
\input{patterns/02_stack/exercises}
}
\FR{\myparagraph{MSVC 2010 \Optimizing}

\lstinputlisting[caption=MSVC 2010 \Optimizing,style=customasmx86]{patterns/12_FPU/3_comparison/x86/MSVC_Ox/MSVC_FR.asm}

\myindex{x86!\Instructions!FCOM}

\FCOM diffère de \FCOMP dans le sens où il compare seulement les deux valeurs, et
ne change pas la pile du FPU.
Contrairement à l'exemple précédent, ici les opérandes sont dans l'ordre inverse,
c'est pourquoi le résultat de la comparaison dans \CThreeBits est différent.

\begin{itemize}
\item si $a>b$ dans notre exemple, alors les bits \CThreeBits sont mis comme suit: 0, 0, 0.
\item si $b>a$, alors les bits sont: 0, 0, 1.
\item si $a=b$, alors les bits sont: 1, 0, 0.
\end{itemize}
% TODO: table?

L'instruction \INS{test ah, 65} laisse seulement deux bits~---\Cthree et \Czero.
Les deux seront à zéro si $a>b$: dans ce cas le saut \JNE ne sera pas effectué.
Puis \INS{FSTP ST(1)} suit~---cette instruction copie la valeur de \ST{0} dans l'opérande
et supprime une valeur de la pile du FPU.
En d'autres mots, l'instruction copie \ST{0} (où la valeur de \GTT{\_a} se trouve)
dans \ST{1}.
Après cela, deux copies de {\_a} sont sur le sommet de la pile.
Puis, une valeur est supprimée.
Après cela, \ST{0} contient {\_a} et la fonction se termine.

Le saut conditionnel \JNE est effectué dans deux cas: si $b>a$ ou $a=b$.
\ST{0} est copié dans \ST{0}, c'est comme une opération sans effet (\ac{NOP}), puis
une valeur est supprimée de la pile et le sommet de la pile (\ST{0}) contient la
valeur qui était avant dans \ST{1} (qui est {\_b}).
Puis la fonction se termine.
La raison pour laquelle cette instruction est utilisée ici est sans doute que le
\ac{FPU} n'a pas d'autre instruction pour prendre une valeur sur la pile et la
supprimer.

\input{patterns/12_FPU/3_comparison/x86/MSVC_Ox/olly_FR.tex}
}
\JA{\myparagraph{\Optimizing MSVC 2010}

\lstinputlisting[caption=\Optimizing MSVC 2010,style=customasmx86]{patterns/12_FPU/3_comparison/x86/MSVC_Ox/MSVC_JA.asm}

\myindex{x86!\Instructions!FCOM}

\FCOM は、単に値を比較し、FPUスタックを変更しないという点で、 \FCOMP とは異なります。
前の例とは異なり、ここではオペランドは逆順になっています。
そのため、 \CThreeBits の比較結果は異なります。

\begin{itemize}
\item この例で $a>b$ の場合、 \CThreeBits ビットは0,0,0として設定されます。
\item $b>a$ の場合、ビットは0,0,1です。
\item $a=b$ の場合、ビットは1,0,0です。
\end{itemize}
% TODO: table?

\INS{test ah, 65}命令は、2ビットの \Cthree と \Czero だけを残します。
$a>b$ の場合は両方ともゼロになります。その場合、 \JNE ジャンプは実行されません。
次に、\INS{FSTP ST(1)}が続きます。この命令は、\ST{0}の値をオペランドにコピーし、
FPUスタックから1つの値をポップします。言い換えれば、命令は\ST{0}(ここでは\GTT{\_a}の値)
が\ST{1}にコピーされます。その後、{\_a}の2つのコピーがスタックの一番上にあります。
次に、1つの値がポップされます。その後、ST(0)には{\_a}が含まれ、機能は終了します。

条件ジャンプ \JNE は、$b>a$ または $a=b$ の2つの場合に実行されます。 
\ST{0}は\ST{0}にコピーされ、アイドル(\ac{NOP})操作と同様に、1つの値がスタックからポップされ、
スタックの先頭(\ST{0})には\ST{1}前(つまり{\_b})です。
その後、関数は終了します。
この命令がここで使用される理由は、\ac{FPU}にスタックから値をポップして破棄するための他の命令がないためです。

\input{patterns/12_FPU/3_comparison/x86/MSVC_Ox/olly_JA.tex}
}

\EN{\myparagraph{GCC 4.4.1}

\lstinputlisting[caption=GCC 4.4.1,style=customasmx86]{patterns/12_FPU/3_comparison/x86/GCC_EN.asm}

\myindex{x86!\Instructions!FUCOMPP}

\FUCOMPP{} is almost like \FCOM, but pops both values from the stack and handles
\q{not-a-numbers} differently.

\myindex{Non-a-numbers (NaNs)}
A bit about \emph{not-a-numbers}.

\newcommand{\NANFN}{\footnote{\href{http://go.yurichev.com/17130}{wikipedia.org/wiki/NaN}}}

The FPU is able to deal with special values which are \emph{not-a-numbers} or \gls{NaN}s\NANFN. 
These are infinity, result of division by 0, etc.
Not-a-numbers can be \q{quiet} and \q{signaling}. It is possible to continue to work with \q{quiet} NaNs, 
but if one tries to do any operation with \q{signaling} NaNs, an exception is to be raised.

\myindex{x86!\Instructions!FCOM}
\myindex{x86!\Instructions!FUCOM}

\FCOM raising an exception if any operand is \gls{NaN}. 
\FUCOM raising an exception only if any operand is a signaling \gls{NaN} (SNaN).

\myindex{x86!\Instructions!SAHF}
\label{SAHF}

The next instruction is \SAHF (\emph{Store AH into Flags})~---this is a rare 
instruction in code not related to the FPU. 
8 bits from AH are moved into the lower 8 bits of the CPU flags in the following order:

\input{SAHF_LAHF}

\myindex{x86!\Instructions!FNSTSW}

Let's recall that \FNSTSW moves the bits that interest us (\CThreeBits) into \AH 
and they are in positions 6, 2, 0 of the \AH register:

\input{C3_in_AH}

In other words, the \INS{fnstsw  ax / sahf} instruction pair moves \CThreeBits into \ZF, \PF and \CF.

Now let's also recall the values of \CThreeBits in different conditions:

\begin{itemize}
\item If $a$ is greater than $b$ in our example, then \CThreeBits are to be set to: 0, 0, 0.
\item if $a$ is less than $b$, then the bits are to be set to: 0, 0, 1.
\item If $a=b$, then: 1, 0, 0.
\end{itemize}
% TODO: table?

In other words, these states of the CPU flags are possible
after three \\
\FUCOMPP/\FNSTSW/\SAHF instructions:

\begin{itemize}
\item If $a>b$, the CPU flags are to be set as: \GTT{ZF=0, PF=0, CF=0}.
\item If $a<b$, then the flags are to be set as: \GTT{ZF=0, PF=0, CF=1}.
\item And if $a=b$, then: \GTT{ZF=1, PF=0, CF=0}.
\end{itemize}
% TODO: table?

\myindex{x86!\Instructions!SETcc}
\myindex{x86!\Instructions!JNBE}

Depending on the CPU flags and conditions, \SETNBE stores 1 or 0 to AL. 
It is almost the counterpart of \JNBE, with the exception that \SETcc 
\footnote{\emph{cc} is \emph{condition code}} stores 1 or 0 in \AL, 
but \Jcc does actually jump or not. 
\SETNBE stores 1 only if \GTT{CF=0} and \GTT{ZF=0}. 
If it is not true, 0 is to be stored into \AL.

Only in one case both \CF and \ZF are 0: if $a>b$.

Then 1 is to be stored to \AL, the subsequent \JZ is not to be triggered and the function will return {\_a}. 
In all other cases, {\_b} is to be returned.

}
\RU{\myparagraph{GCC 4.4.1}

\lstinputlisting[caption=GCC 4.4.1,style=customasmx86]{patterns/12_FPU/3_comparison/x86/GCC_RU.asm}

\myindex{x86!\Instructions!FUCOMPP}
\FUCOMPP~--- это почти то же что и \FCOM, только выкидывает из стека оба значения после сравнения, 
а также несколько иначе реагирует на \q{не-числа}.

\myindex{Не-числа (NaNs)}
Немного о \emph{не-числах}.

\newcommand{\NANFN}{\footnote{\href{http://go.yurichev.com/17129}{ru.wikipedia.org/wiki/NaN}}}

FPU умеет работать со специальными переменными, которые числами не являются и называются \q{не числа} или 
\gls{NaN}\NANFN. 
Это бесконечность, результат деления на ноль, и так далее. Нечисла бывают \q{тихие} и \q{сигнализирующие}. 
С первыми можно продолжать работать и далее, а вот если вы попытаетесь совершить какую-то операцию 
с сигнализирующим нечислом, то сработает исключение.

\myindex{x86!\Instructions!FCOM}
\myindex{x86!\Instructions!FUCOM}
Так вот, \FCOM вызовет исключение если любой из операндов какое-либо нечисло.
\FUCOM же вызовет исключение только если один из операндов именно \q{сигнализирующее нечисло}.

\myindex{x86!\Instructions!SAHF}
\label{SAHF}
Далее мы видим \SAHF (\emph{Store AH into Flags})~--- это довольно редкая инструкция в коде, не использующим FPU. 
8 бит из \AH перекладываются в младшие 8 бит регистра статуса процессора в таком порядке:

\input{SAHF_LAHF}

\myindex{x86!\Instructions!FNSTSW}
Вспомним, что \FNSTSW перегружает интересующие нас биты \CThreeBits в \AH, 
и соответственно они будут в позициях 6, 2, 0 в регистре \AH:

\input{C3_in_AH}

Иными словами, пара инструкций \INS{fnstsw  ax / sahf} перекладывает биты \CThreeBits в флаги \ZF, \PF, \CF.

Теперь снова вспомним, какие значения бит \CThreeBits будут при каких результатах сравнения:

\begin{itemize}
\item Если $a$ больше $b$ в нашем случае, то биты \CThreeBits должны быть выставлены так: 0, 0, 0.
\item Если $a$ меньше $b$, то биты будут выставлены так: 0, 0, 1.
\item Если $a=b$, то так: 1, 0, 0.
\end{itemize}
% TODO: table?

Иными словами, после трех инструкций \FUCOMPP/\FNSTSW/\SAHF возможны такие состояния флагов:

\begin{itemize}
\item Если $a>b$ в нашем случае, то флаги будут выставлены так: \GTT{ZF=0, PF=0, CF=0}.
\item Если $a<b$, то флаги будут выставлены так: \GTT{ZF=0, PF=0, CF=1}.
\item Если $a=b$, то так: \GTT{ZF=1, PF=0, CF=0}.
\end{itemize}
% TODO: table?

\myindex{x86!\Instructions!SETcc}
\myindex{x86!\Instructions!JNBE}
Инструкция \SETNBE выставит в \AL единицу или ноль в зависимости от флагов и условий. 
Это почти аналог \JNBE, за тем лишь исключением, что \SETcc
\footnote{\emph{cc} это \emph{condition code}}
выставляет 1 или 0 в \AL, а \Jcc делает переход или нет. 
\SETNBE запишет 1 только если \GTT{CF=0} и \GTT{ZF=0}. Если это не так, то запишет 0 в \AL.

\CF будет 0 и \ZF будет 0 одновременно только в одном случае: если $a>b$.

Тогда в \AL будет записана 1, последующий условный переход \JZ выполнен не будет 
и функция вернет~\GTT{\_a}. 
В остальных случаях, функция вернет~\GTT{\_b}.
}
\DE{\myparagraph{GCC 4.4.1}

\lstinputlisting[caption=GCC
4.4.1,style=customasmx86]{patterns/12_FPU/3_comparison/x86/GCC_DE.asm}

\myindex{x86!\Instructions!FUCOMPP}
\FUCOMPP{} ist fast wie like \FCOM, nimmt aber beide Werte vom Stand und
behandelt \q{undefinierte Zahlenwerte} anders.


\myindex{Non-a-numbers (NaNs)}
Ein wenig über \emph{undefinierte Zahlenwerte}.

\newcommand{\NANFN}{\footnote{\href{http://go.yurichev.com/17130}{wikipedia.org/wiki/NaN}}}
Die FPU ist in der Lage mit speziellen undefinieten Werten, den sogenannten
\emph{not-a-number}(kurz \gls{NaN})\NANFN umzugehen. Beispiele sind etwa der Wert
unendlich, das Ergebnis einer Division durch 0, etc. Undefinierte Werte können
entwder \q{quiet} oder \q{signaling} sein. Es ist möglich mit \q{quiet} NaNs zu
arbeiten, aber beim Versuch einen Befehl auf \q{signaling} NaNs auszuführen,
wird eine Exception geworfen. 

\myindex{x86!\Instructions!FCOM}
\myindex{x86!\Instructions!FUCOM}
\FCOM erzeugt eine Exception, falls irgendein Operand ein \gls{NaN} ist.
\FUCOM erzeugt eine Exception nur dann, wenn ein Operand eine \q{signaling}
\gls{NaN} (SNaN) ist.

\myindex{x86!\Instructions!SAHF}
\label{SAHF}
Der nächste Befehl ist \SAHF (\emph{Store AH into Flags})~---es handelt sich
hierbei um einen seltenen Befehl, der nicht mit der FPU zusammenhängt.
8 Bits aus AH werden in die niederen 8 Bit der CPU Flags in der folgenden
Reihenfolge verschoben:

\input{SAHF_LAHF}

\myindex{x86!\Instructions!FNSTSW}
Erinnern wir uns, dass \FNSTSW die für uns interessanten Bits (\CThreeBits) auf
den Stellen 6,2,0 im AH Register setzt:

\input{C3_in_AH}
Mit anderen Worten: der Befehl \INS{fnstsw ax / sahf} verschiebt \CThreeBits
nach \ZF, \PF und \CF. 

Überlegen wir uns auch die Werte der \CThreeBits in unterschiedlichen Szenarien:

\begin{itemize} 
  \item Falls in unserem Beispiel $a$ größer als $b$ ist, dann werden die
  \CThreeBits auf 0,0,0 gesetzt.
  \item Falls $a$ kleiner als $b$ ist, werden die Bits auf 0,0,1 gesetzt.
  \item Falls $a=b$, dann werden die Bits auf 1,0,0 gesetzt.
\end{itemize}
% TODO: table?
Mit anderen Worten, die folgenden Zustände der CPU Flags sind nach drei
\FUCOMPP/\FNSTSW/\SAHF Befehlen möglich:

\begin{itemize}
\item Falls $a>b$, werden die CPU Flags wie folgt gesetzt \GTT{ZF=0, PF=0,
CF=0}.
\item Falls $a<b$, werden die CPU Flags wie folgt gesetzt: \GTT{ZF=0, PF=0,
CF=1}.
\item Und falls $a=b$, dann gilt: \GTT{ZF=1, PF=0, CF=0}.
\end{itemize}
% TODO: table?

\myindex{x86!\Instructions!SETcc}
\myindex{x86!\Instructions!JNBE}
Abhängig von den CPU Flags und Bedingungen, speichert \SETNBE entweder 1 oder 0
in AL.
Es ist also quasi das Gegenstück von \JNBE mit dem Unterschied, dass \SETcc

Depending on the CPU flags and conditions, \SETNBE stores 1 or 0 to AL. 
It is almost the counterpart of \JNBE, with the exception that \SETcc
\footnote{\emph{cc} is \emph{condition code}} eine 1 oder 0 in \AL speichert, aber
\Jcc tatsächlich auch springt.
\SETNBE speicher 1 nur, falls \GTT{CF=0} und \GTT{ZF=0}.
Wenn dies nicht der Fall ist, dann wird 0 in \AL gespeichert.

Nur in einem Fall sind \CF und \ZF beide 0: falls $a>b$.

In diesem Fall wird 1 in \AL gespeichert, der nachfolgende \JZ Sprung wird nicht
ausgeführt und die Funktion liefert {\_a} zurück. In allen anderen Fällen wird
{\_b} zurückgegeben.
}
\FR{\myparagraph{GCC 4.4.1}

\lstinputlisting[caption=GCC 4.4.1,style=customasmx86]{patterns/12_FPU/3_comparison/x86/GCC_FR.asm}

\myindex{x86!\Instructions!FUCOMPP}

\FUCOMPP{} est presque comme \FCOM, mais dépile deux valeurs de la pile et traite
les \q{non-nombres} différemment.

\myindex{Non-a-numbers (NaNs)}
Quelques informations à propos des \emph{not-a-numbers} (non-nombres).

\newcommand{\NANFN}{\footnote{\href{http://go.yurichev.com/17130}{wikipedia.org/wiki/NaN}}}

Le FPU est capable de traiter les valeurs spéciales que sont les \emph{not-a-numbers}
(non-nombres) ou \gls{NaN}s\NANFN.
Ce sont les infinis, les résultat de division par 0, etc.
Les non-nombres peuvent être \q{quiet} et \q{signaling}. Il est possible de continuer
à travailler avec les \q{quiet} NaNs, mais si l'on essaye de faire une opération avec
un \q{signaling} NaNs, une exception est levée.

\myindex{x86!\Instructions!FCOM}
\myindex{x86!\Instructions!FUCOM}

\FCOM lève une exception si un des opérandes est \gls{NaN}.
\FUCOM lève une exception seulement si un des opérandes est un signaling \gls{NaN}
(SNaN).

\myindex{x86!\Instructions!SAHF}
\label{SAHF}

L'instruction suivante est \SAHF (\emph{Store AH into Flags} stocker AH dans les Flags)~---est
une instruction rare dans le code non relatif au FPU.
8 bits de AH sont copiés dans les 8-bits bas dans les flags du CPU dans l'ordre suivant:

\input{SAHF_LAHF}

\myindex{x86!\Instructions!FNSTSW}

Rappelons que \FNSTSW déplace des bits qui nous intéressent (\CThreeBits) dans \AH
et qu'ils sont aux positions 6, 2, 0 du registre \AH.

\input{C3_in_AH}

En d'autres mots, la paire d'instructions \INS{fnstsw  ax / sahf} déplace \CThreeBits
dans \ZF, \PF et \CF.

Maintenant, rappelons les valeurs de \CThreeBits sous différentes conditions:

\begin{itemize}
\item Si $a$ est plus grand que $b$ dans notre exemple, alors les \CThreeBits sont
mis à: 0, 0, 0.
\item Si $a$ est plus petit que $b$, alors les bits sont mis à: 0, 0, 1.
\item Si $a=b$, alors: 1, 0, 0.
\end{itemize}
% TODO: table?

En d'autres mots, ces états des flags du CPU sont possible après les
trois instructions \FUCOMPP/\FNSTSW/\SAHF:

\begin{itemize}
\item Si $a>b$, les flags du CPU sont mis à: \GTT{ZF=0, PF=0, CF=0}.
\item Si $a<b$, alors les flags sont mis à: \GTT{ZF=0, PF=0, CF=1}.
\item Et si $a=b$, alors: \GTT{ZF=1, PF=0, CF=0}.
\end{itemize}
% TODO: table?

\myindex{x86!\Instructions!SETcc}
\myindex{x86!\Instructions!JNBE}

Suivant les flags du CPU et les conditions, \SETNBE met 1 ou 0 dans AL.
C'est presque la contrepartie de \JNBE, avec l'exception que \SETcc\footnote{\emph{cc}
est un \emph{condition code}} met 1 ou 0 dans \AL, mais \Jcc effectue un saut ou non.
\SETNBE met 1 seulement si \GTT{CF=0} et \GTT{ZF=0}.
Si ce n'est pas vrai, 0 est mis dans \AL.

Il y a un seul cas où \CF et \ZF sont à 0: si $a>b$.

Alors 1 est mis dans \AL, le \JZ subséquent n'est pas pris et la fonction va renvoyer
{\_a}.
Dans tous les autres cas, {\_b} est renvoyé.

}
\JA{\myparagraph{GCC 4.4.1}

\lstinputlisting[caption=GCC 4.4.1,style=customasmx86]{patterns/12_FPU/3_comparison/x86/GCC_JA.asm}

\myindex{x86!\Instructions!FUCOMPP}

\FUCOMPP{} は \FCOM に似ていますが、スタックから両方の値をポップし、
\q{非数値} を異なる方法で処理します。

\myindex{Non-a-numbers (NaNs)}
\q{非数値} について少々。

\newcommand{\NANFN}{\footnote{\href{http://go.yurichev.com/17130}{wikipedia.org/wiki/NaN}}}

FPUは、数字でない特別な\emph{非数値} や \gls{NaN}\NANFN を扱うことができます。 
これらは無限大で、0で除算した結果です。
非数値は\q{寡黙}であることも\q{シグナルを発する}こともできます。\q{寡黙な} NaNで
作業を続行することは可能ですが、\q{シグナルを発する} NaNで何らかの操作を試みる場合は例外が発生します。

\myindex{x86!\Instructions!FCOM}
\myindex{x86!\Instructions!FUCOM}

オペランドが\gls{NaN}の場合、 \FCOM は例外を送出します。 
\FUCOM は、オペランドがシグナルを発する \gls{NaN}(SNaN)である場合にのみ例外を送出します。

\myindex{x86!\Instructions!SAHF}
\label{SAHF}

次の命令は \SAHF (\emph{AHをフラグにストア})です。これはFPUに
関連しないコードではまれな命令です。
AHからの8ビットは、次の順序でCPUフラグの下位8ビットに移動します。

\input{SAHF_LAHF}

\myindex{x86!\Instructions!FNSTSW}

\FNSTSW が関心のあるビット(\CThreeBits)を \AH に移動し、
\AH レジスタの位置6,2,0にあることを思い出してください。

\input{C3_in_AH}

言い換えると、\INS{fnstsw  ax / sahf}命令ペアは、 \CThreeBits を \ZF 、 \PF 、および \CF に移動します。

異なる条件で \CThreeBits の値を思い出してみましょう。

\begin{itemize}
\item この例で $a$ が $b$ より大きい場合、 \CThreeBits は0,0,0に設定されます。
\item $a$ が $b$ より小さければ、ビットは0,0,1に設定されます。
\item $a=b$ の場合は、1,0,0に設定されます。
\end{itemize}
% TODO: table?

言い換えれば、これらのCPUフラグの状態は、
3つの \FUCOMPP/\FNSTSW/\SAHF 命令の後に可能になります。

\begin{itemize}
\item $a>b$ の場合、CPUフラグは、\GTT{ZF=0, PF=0, CF=0}として設定されます。
\item $a<b$ の場合、フラグは、\GTT{ZF=0, PF=0, CF=1}として設定されます。
\item そして、 $a=b$ ならば、\GTT{ZF=1, PF=0, CF=0}になります。
\end{itemize}
% TODO: table?

\myindex{x86!\Instructions!SETcc}
\myindex{x86!\Instructions!JNBE}

CPUのフラグと条件に応じて、 \SETNBE はALに1または0を格納します。 
これはほぼ \JNBE のものですが、 \SETcc
\footnote{\emph{cc} は \emph{条件コード} です}
はALに1または0を格納しますが、 \Jcc は実際にジャンプするかどうかは異なります。
\SETNBE は、\GTT{CF=0}および\GTT{ZF=0}の場合にのみ1を格納します。 
真でない場合は、0が \AL に格納されます。

$a>b$ の場合でのみ、\CF と \ZF の両方が0になります。

その後、1が \AL に格納され、後続の \JZ は実行されず、関数は{\_a}を返します。 
それ以外の場合は{\_b}が返されます。
}

\EN{\myparagraph{\Optimizing GCC 4.4.1}

\lstinputlisting[caption=\Optimizing GCC 4.4.1,style=customasmx86]{patterns/12_FPU/3_comparison/x86/GCC_O3_EN.asm}

\myindex{x86!\Instructions!JA}

It is almost the same except that \JA is used after \SAHF. 
Actually, conditional jump instructions that check \q{larger}, \q{lesser} or \q{equal} for unsigned number comparison 
(these are \INS{JA}, \JAE, \JB, \JBE, \JE/\JZ, \JNA, \JNAE, \JNB, \JNBE, \JNE/\JNZ) check only flags \CF and \ZF.\\
\\
Let's recall where bits \CThreeBits are located in the \GTT{AH} register after the execution of \INS{FSTSW}/\FNSTSW:

\input{C3_in_AH}

Let's also recall, how the bits from \GTT{AH} are stored into the CPU flags after the execution of \SAHF:

\input{SAHF_LAHF}

After the comparison, the \Cthree and \Czero bits are moved into \ZF and \CF, so the conditional jumps are able work after. \JA is triggering if both \CF are \ZF zero.

Thereby, the conditional jumps instructions listed here can be used after a \FNSTSW/\SAHF instruction pair.

Apparently, the FPU \CThreeBits status bits were placed there intentionally, to easily map them to base CPU flags without additional permutations?

}
\RU{\myparagraph{\Optimizing GCC 4.4.1}

\lstinputlisting[caption=\Optimizing GCC 4.4.1,style=customasmx86]{patterns/12_FPU/3_comparison/x86/GCC_O3_RU.asm}

\myindex{x86!\Instructions!JA}

Почти всё что здесь есть, уже описано мною, кроме одного: использование \JA после \SAHF. 
Действительно, инструкции условных переходов \q{больше}, \q{меньше} и \q{равно} для сравнения беззнаковых чисел 
(а это \INS{JA}, \JAE, \JB, \JBE, \JE/\JZ, \JNA, \JNAE, \JNB, \JNBE, \JNE/\JNZ) проверяют только флаги \CF и \ZF.\\
\\
Вспомним, как биты \CThreeBits располагаются в регистре \GTT{AH} после исполнения \INS{FSTSW}/\FNSTSW:

\input{C3_in_AH}

Вспомним также, как располагаются биты из \GTT{AH} во флагах CPU после исполнения \SAHF:

\input{SAHF_LAHF}

Биты \Cthree и \Czero после сравнения перекладываются в флаги \ZF и \CF так, что перечисленные инструкции переходов могут работать. \JA сработает, если \CF и \ZF обнулены.

Таким образом, перечисленные инструкции условного перехода можно использовать после инструкций \FNSTSW/\SAHF.

Может быть, биты статуса FPU \CThreeBits преднамеренно были размещены таким образом, чтобы переноситься на базовые флаги процессора без перестановок?

}
\DE{\myparagraph{\Optimizing GCC 4.4.1}

\lstinputlisting[caption=\Optimizing GCC
4.4.1,style=customasmx86]{patterns/12_FPU/3_comparison/x86/GCC_O3_DE.asm}

\myindex{x86!\Instructions!JA}
Dies ist fast das gleiche, außer dass \JA nach \SAHF verwendet wird.
Tatsächlich prüfen die bedingte Sprungbefehle, die vorzeichenlose Zahlen auf
\q{größer}, \q{kleiner} oder \q{gleich} prüfen (das sind \INS{JA}, \JAE, \JB, \JBE,
\JE/\JZ, \JNA, \JNAE, \JNB, \JNBE, \JNE/\JNZ) lediglich die Flags \CF und
\ZF.\\\\
Erinnern wir uns, an welchen Stellen die \CThreeBits sich im \GTT{AH} Register
befinden, nachdem der Befehl \INS{FSTSW}/\FNSTSW ausgeführt wurde:

\input{C3_in_AH}
Halten wir uns auch vor Augen wie die Bits aus \GTT{AH} in den CPU Flags nach
der Ausführung von \SAHF abgelegt werden:

\input{SAHF_LAHF}
Nach dem Vergleich werden die \Cthree und \Czero Bits nach \ZF und \CF
verschoben, sodass der bedingte Sprung danach funktionieren kann. \JA wird
ausgeüführt, falls sowohl \CF als auch \ZF gleich 0 sind.

Hierbei können alle hier aufgelisteten Sprungbefehle nach einem \FNSTSW/\SAHF
Befehlspaar verwendet werden. 

Offenbar wurden die \CThreeBits Status Bits der CPU dort bewusst platziert,
sodass diese leicht auf die CPU Flags übertragen werden können, ohne dass
zusätzliche Vertauschungen notwendig sind.
}
\FR{\myparagraph{GCC 4.4.1 \Optimizing}

\lstinputlisting[caption=GCC 4.4.1 \Optimizing,style=customasmx86]{patterns/12_FPU/3_comparison/x86/GCC_O3_FR.asm}

\myindex{x86!\Instructions!JA}

C'est presque le même, à l'exception que \JA est utilisé après \SAHF.
En fait, les instructions de sauts conditionnels qui vérifient \q{plus}, \q{moins} ou \q{égal} pour
les comparaisons de nombres non signés (ce sont \INS{JA}, \JAE, \JB, \JBE, \JE/\JZ, \JNA,
\JNAE, \JNB, \JNBE, \JNE/\JNZ) vérifient seulement les flags \CF et \ZF.\\
\\
Rappelons comment les bits \CThreeBits sont situés dans le registre \GTT{AH} après
l'exécution de \INS{FSTSW}/\FNSTSW:

\input{C3_in_AH}

Rappelons également, comment les bits de \GTT{AH} sont stockés dans les flags du
CPU après l'exécution de \SAHF:

\input{SAHF_LAHF}

Après la comparaison, les bits \Cthree et \Czero sont copiés dans \ZF et \CF, donc
les sauts conditionnels peuvent fonctionner après. \JA est déclenché si \CF et \ZF
sont tout les deux à zéro.

Ainsi, les instructions de saut conditionnel listées ici peuvent être utilisées après
une paire d'instructions \FNSTSW/\SAHF.

Apparemment, les bits d'état du FPU \CThreeBits ont été mis ici intentionnellement,
pour facilement les relier aux flags du CPU de base sans permutations supplémentaires?

}
\JA{\myparagraph{\Optimizing GCC 4.4.1}

\lstinputlisting[caption=\Optimizing GCC 4.4.1,style=customasmx86]{patterns/12_FPU/3_comparison/x86/GCC_O3_JA.asm}

\myindex{x86!\Instructions!JA}

\JA が \SAHF の後に使用されることを除いて、ほとんど同じです。 
実際には、符号なしの番号比較(これらは \INS{JA} , \JAE , \JB , \JBE , \JE/\JZ , \JNA , \JNAE , \JNB , \JNBE , \JNE/\JNZ )のチェックに
\q{大なり}、\q{小なり}、\q{等しい}をチェックする条件ジャンプ命令は \CF および \ZF フラグが立っているときだけチェックします。

\INS{FSTSW}/\FNSTSW の実行後に、 \CThreeBits が\GTT{AH}レジスタのどこにあるかを思い出してみましょう。

\input{C3_in_AH}

\SAHF の実行後に、\GTT{AH}からのビットがCPUフラグの中にどのようにして保存されるかを思いだしてみましょう。

\input{SAHF_LAHF}

比較の後、\Cthree および \Czero ビットは \ZF および \CF に移動するので、条件付きジャンプはその後に働きます。 \CF と \ZF がともにゼロである場合、 \JA は実行します。

したがって、ここにリストされている条件付きジャンプ命令は、 \FNSTSW/\SAHF 命令ペアの後に使用できます。

どうやらFPU \CThreeBits ステータスビットは、追加の順列を付けずにCPUの基本フラグに簡単にマッピングできるよう、意図的に配置されています。
}

\EN{\myparagraph{GCC 4.8.1 with \Othree optimization turned on}
\label{gcc481_o3}

Some new FPU instructions were added in the P6 Intel family\footnote{Starting at Pentium Pro, Pentium-II, etc.}.
\myindex{x86!\Instructions!FUCOMI}
These are \INS{FUCOMI} (compare operands and set flags of the main CPU) and 
\myindex{x86!\Instructions!FCMOVcc}
\INS{FCMOVcc} (works like \INS{CMOVcc}, but on FPU registers).

Apparently, the maintainers of GCC decided to drop support of pre-P6 Intel CPUs (early Pentiums, 80486, etc.).

And also, the FPU is no longer separate unit in P6 Intel family, so now it is possible to modify/check flags of the main CPU from the FPU.

So what we get is:

\lstinputlisting[caption=\Optimizing GCC 4.8.1,style=customasmx86]{patterns/12_FPU/3_comparison/x86/GCC481_O3_EN.s}

Hard to guess why \INS{FXCH} (swap operands) is here.

It's possible to get rid of it easily by swapping the first two \FLD instructions or by replacing 
\INS{FCMOVBE} (\emph{below or equal}) by \INS{FCMOVA} (\emph{above}).
Probably it's a compiler inaccuracy.

So \INS{FUCOMI} compares \ST{0} ($a$) and \ST{1} ($b$) 
and then sets some flags in the main CPU.
\INS{FCMOVBE} checks the flags and copies \ST{1} 
($b$ here at the moment) to 
\ST{0} ($a$ here) if $ST0 (a) <= ST1 (b)$.
Otherwise ($a>b$), it leaves $a$ in \ST{0}.

The last \FSTP leaves \ST{0} on top of the stack, discarding the contents of \ST{1}.

Let's trace this function in GDB:

\lstinputlisting[caption=\Optimizing GCC 4.8.1 and GDB,numbers=left]{patterns/12_FPU/3_comparison/x86/gdb.txt}

Using \q{ni}, 
let's execute the first two \FLD instructions.

Let's examine the FPU registers (line 33).

As it was mentioned before, the FPU registers set is a circular buffer rather than a stack (\myref{FPU_is_rather_circular_buffer}).
And GDB doesn't show \GTT{STx} registers, but internal the FPU registers (\GTT{Rx}). 
The arrow (at line 35) points to the current top of the stack.

You can also see the \GTT{TOP} register contents in \emph{Status Word} (line 44)---it is 6 now, 
so the stack top is now pointing to internal register 6.

The values of $a$ and $b$ are swapped after \INS{FXCH} is executed (line 54).

\INS{FUCOMI} is executed (line 83). 
Let's see the flags: \CF is set (line 95).

\INS{FCMOVBE} has copied the value of $b$ (see line 104).

\FSTP leaves one value at the top of stack (line 136). 
The value of \GTT{TOP} is now 7, so the FPU stack top is pointing to internal register 7.

}
\RU{\myparagraph{GCC 4.8.1 с оптимизацией \Othree}
\label{gcc481_o3}

В линейке процессоров P6 от Intel 
появились новые FPU-инструкции\footnote{Начиная с Pentium Pro, Pentium-II, итд.}.
\myindex{x86!\Instructions!FUCOMI}
Это \INS{FUCOMI} (сравнить операнды и выставить флаги основного CPU) и
\myindex{x86!\Instructions!FCMOVcc}
\INS{FCMOVcc} (работает как \INS{CMOVcc}, но на регистрах FPU).
Очевидно, разработчики GCC решили отказаться от поддержки процессоров до линейки P6 (ранние Pentium, 80486, итд.).

И кстати, FPU уже давно не отдельная часть процессора в линейке P6, так что флаги основного CPU можно модифицировать из FPU.

Вот что имеем:

\lstinputlisting[caption=\Optimizing GCC 4.8.1,style=customasmx86]{patterns/12_FPU/3_comparison/x86/GCC481_O3_RU.s}

Не совсем понимаю, зачем здесь \INS{FXCH} (поменять местами операнды).

От нее легко избавиться поменяв местами инструкции \FLD либо заменив 
\INS{FCMOVBE} (\emph{below or equal}~--- меньше или равно) на 
\INS{FCMOVA} (\emph{above}~--- больше).

Должно быть, неаккуратность компилятора.

Так что \INS{FUCOMI} сравнивает \ST{0} ($a$) и \ST{1} ($b$) 
и затем устанавливает флаги основного CPU.
\INS{FCMOVBE} проверяет флаги и копирует \ST{1} 
(в тот момент там находится $b$) в 
\ST{0} (там $a$) если $ST0 (a) <= ST1 (b)$.
В противном случае ($a>b$), она оставляет $a$ в \ST{0}.

Последняя \FSTP оставляет содержимое \ST{0} на вершине стека, выбрасывая содержимое \ST{1}.

Попробуем оттрассировать функцию в GDB:

\lstinputlisting[caption=\Optimizing GCC 4.8.1 and GDB,numbers=left]{patterns/12_FPU/3_comparison/x86/gdb.txt}

Используя \q{ni}, дадим первым двум инструкциям \FLD исполниться.

Посмотрим регистры FPU (строка 33).

Как уже было указано ранее, регистры FPU это скорее кольцевой буфер, нежели стек (\myref{FPU_is_rather_circular_buffer}).
И GDB показывает не регистры \GTT{STx}, а внутренние регистры FPU (\GTT{Rx}). 
Стрелка (на строке 35) указывает на текущую вершину стека.

Вы можете также увидеть содержимое регистра \GTT{TOP} в \q{Status Word} (строка 44). Там сейчас 6, так что
вершина стека сейчас указывает на внутренний регистр 6.

Значения $a$ и $b$ меняются местами после исполнения \INS{FXCH} (строка 54).

\INS{FUCOMI} исполнилась (строка 83).
Посмотрим флаги: \CF выставлен (строка 95).

\INS{FCMOVBE} действительно скопировал значение $b$ (см. строку 104).

\FSTP оставляет одно значение на вершине стека (строка 136). 
Значение \GTT{TOP} теперь 7, так что вершина FPU-стека указывает на внутренний регистр 7.
}
\DE{\myparagraph{GCC 4.8.1 mit aktivierter \Othree Optimierung}
\label{gcc481_o3}
Mit der P6 Intel Familie\footnote{Beginnend mit Pentium Pro, Pentium-II, etc.}
wurden einige neue FPU Befehle hinzugefügt. 
\myindex{x86!\Instructions!FUCOMI}
Diese sind \INS{FUCOMI} (vergleiche Operanden und setze Flags der CPU) und 
\myindex{x86!\Instructions!FCMOVcc}
\INS{FCMOVcc} (arbeitet wie \INS{CMOVcc}, aber auf FPU Registern).
Offenbar haben sich die Verwalter von GCC dazu entschieden, den Support von
vor-P6 Intel CPUs (frühe Pentiums, 80486, etc.) einzustellen.

Außerdem ist die FPU nicht länger eine separate Einheit in der P6 Intel Familie,
sodass es nun auch möglich ist, die Flags der CPU von der FPU aus zu prüfen oder
zu verändern.

Wir erhalten also das Folgende:

\lstinputlisting[caption=\Optimizing GCC
4.8.1,style=customasmx86]{patterns/12_FPU/3_comparison/x86/GCC481_O3_DE.s}

Schwer zu sagen, warum \INS{FXCH} (vertausche Operanden) hier verwendet wird.

Es ist möglich, diesen Befehl loszuwerden, indem man die ersten beiden \FLD
Befehle vertauscht oder \INS{FCMOVBE} (\emph{below or equal}) durch \INS{FCMOVA}
(\emph{above}) ersetzt.
Wahrscheinlich handelt es sich hierbei um eine Ungenauigkeit im Compiler.

\INS{FUCOMI} vergleicht also \ST{0} ($a$) und \ST{1} ($b$) und setzt einige
Flags in der CPU. 
\INS{FCMOVBE} prüft die Flags und kopiert \ST{1} (in diesem Moment also $b$)
nach \ST{0} (hier: $a$), falls $ST0 (a) <= ST1 (b)$.
Andernfalls ($a>b$) wird $a$ in \ST{0} belassen.

Der letzte \FSTP Befehl belässt \ST{0} oben auf dem Stack und verwirft den
Inhalt von \ST{1}. 

Verfolgen wir den Funktionsverlauf in GDB:

\lstinputlisting[caption=\Optimizing GCC 4.8.1 and GDB,numbers=left]{patterns/12_FPU/3_comparison/x86/gdb.txt}

Unter Verwendung von \q{ni} führen wir die ersten beiden \FLD Befehle aus.

Sehen wir uns die FPU Register (Zeile 33) an.

Wie bereits erwähnt, bildet der FPU Registersatz einen Ringpuffer und keinen
Stack (\myref{FPU_is_rather_circular_buffer}).
Außerdem zeigt GDB nicht die \GTT{STx} Register, sondern die internen FPU
Register (\GTT{Rx}). 
Der Pfeil (in Zeile 35) zeigt auf das aktuell obere Ende des Stacks.

Wir sehen auch den Inhalt des \GTT{TOP} Registers in \emph{Status Word} (Zeile
44)--hier ist dieser 6, sodass das oberste Element im Stack also aktuell auf das
interne Register 6 zeigt.

Die Werte von $a$ und $b$ werden nach Ausführung von \INS{FXCH} (Zeile 54)
vertauscht.

\INS{FUCOMI} wird ausgeführt (Zeile 83).
Betrachten wir die Flags: \CF ist gesetzt (Zeile 95).

\INS{FCMOVBE} hat den Wert von $b$ kopiert (siehe Zeile 104).

\FSTP lässt einen Wert oben auf dem Stack (Zeile 136).
Der Wert von \GTT{TOP} beträgt jetzt 7, was bedeutet, dass das obere Ende des
FPU Stacks jetzt auf das interne Register 7 zeigt.
}
\FR{\myparagraph{GCC 4.8.1 avec l'option d'optimisation \Othree}
\label{gcc481_o3}

De nouvelles instructions FPU ont été ajoutées avec la famille Intel P6\footnote{À partir du Pentium Pro, Pentium-II, etc.}.
\myindex{x86!\Instructions!FUCOMI}
Ce sont \INS{FUCOMI} (comparer les opérandes et positionner les flags du CPU principal)
et \myindex{x86!\Instructions!FCMOVcc}
\INS{FCMOVcc} (fonctionne comme \INS{CMOVcc}, mais avec les registres du FPU).

Apparemment, les mainteneurs de GCC ont décidé de supprimer le support des CPUs Intel
pré-P6 (premier Pentium, 80486, etc.).

Et donc, le FPU n'est plus une unité séparée dans la famille Intel P6, ainsi il est
possible de modifier/vérifier un flag du CPU principal depuis le FPU.

Voici ce que nous obtenons:

\lstinputlisting[caption=GCC 4.8.1 \Optimizing,style=customasmx86]{patterns/12_FPU/3_comparison/x86/GCC481_O3_FR.s}

Difficile de deviner pourquoi \INS{FXCH} (échange les opérandes) est ici.

Il est possible de s'en débarrasser facilement en échangeant les deux premières instructions
\FLD ou en remplaçant \INS{FCMOVBE} (\emph{below or equal} inférieur ou égal) par
\INS{FCMOVA} (\emph{above}).
Il s'agit probablement d'une imprécision du compilateur.

Donc \INS{FUCOMI} compare \ST{0} ($a$) et \ST{1} ($b$) et met certains flags
dans le CPU principal.
\INS{FCMOVBE} vérifie les flags et copie \ST{1} ($b$ ici à ce moment) dans \ST{0}
($a$ ici) si $ST0 (a) <= ST1 (b)$.
Autrement ($a>b$), $a$ est laissé dans \ST{0}.

Le dernier \FSTP laisse \ST{0} sur le sommet de la pile, supprimant le contenu de \ST{1}.

Exécutons pas à pas cette fonction dans GDB:

\lstinputlisting[caption=GCC 4.8.1 \Optimizing and GDB,numbers=left]{patterns/12_FPU/3_comparison/x86/gdb.txt}

En utilisant \q{ni}, exécutons les deux premières instructions \FLD.

Examinons les registres du FPU (ligne 33).

Comme cela a déjà été mentionné, l'ensemble des registres FPU est un buffeur
circulaire plutôt qu'une pile (\myref{FPU_is_rather_circular_buffer}).
Et GDB ne montre pas les registres \GTT{STx}, mais les registre internes du FPU (\GTT{Rx}).
La flèche (à la ligne 35) pointe sur le haut courant de la pile.

Vous pouvez voir le contenu du registre \GTT{TOP} dans le \emph{Status Word} (ligne 44)---c'est
6 maintenant, donc le haut de la pile pointe maintenant sur le registre interne 6.

Les valeurs de $a$ et $b$ sont échangées après l'exécution de \INS{FXCH} (ligne 54).

\INS{FUCOMI} est exécuté (ilgne 83).
Regardons les flags: \CF est mis (ligne 95).

\INS{FCMOVBE} a copié la valeur de $b$ (voir ligne 104).

\FSTP dépose une valeur au sommet de la pile (ligne 136).
La valeur de \GTT{TOP} est maintenant 7, donc le sommet de la pile du FPU pointe
sur le registre interne 7.

}
\JA{\myparagraph{GCC 4.8.1 with \Othree optimization turned on}
\label{gcc481_o3}

いくつかの新しいFPU命令がP6インテルファミリ\footnote{Pentium Pro、Pentium-IIなどに始まる}に追加されました。 
\myindex{x86!\Instructions!FUCOMI}
これらは \INS{FUCOMI} (メインCPUのオペランドとフラグの比較)と
\myindex{x86!\Instructions!FCMOVcc}
\INS{FCMOVcc} (FPOレジスタ上の\INS{CMOVcc}のように機能します)です。

どうやら、GCCのメンテナは、P6以前のインテルCPU(初期のPentium、80486など)のサポートを中止することに決めました。

また、FPUはP6インテルファミリではもはや別個のユニットではなくなったので、FPUからメインCPUのフラグを変更/チェックすることが可能になりました。

つまり私たちが得るものは次のとおりです。

\lstinputlisting[caption=\Optimizing GCC 4.8.1,style=customasmx86]{patterns/12_FPU/3_comparison/x86/GCC481_O3_JA.s}

\INS{FXCH} (スワップオペランド)がどうしてここにあるのかを推測するのは難しいです。

最初の2つの \FLD 命令を交換するか、 \INS{FCMOVBE} (\emph{below or equal})を \INS{FCMOVA} (\emph{above})に
置き換えることで、簡単に取り除くことができます。 
おそらく、それはコンパイラが不正確なためです。

そのため、 \INS{FUCOMI} は \ST{0} ( $a$ )と \ST{1}( $b$ )を比較し、
メインCPUにいくつかのフラグを設定します。 
\INS{FCMOVBE}はフラグをチェックし、 $ST0 (a) <= ST1 (b)$ なら
\ST{1}(ここでは $b$ )を\ST{0}(ここでは $a$ )にコピーします。
そうでなければ( $a>b$ )、\ST{0}に $a$ を残します。

最後の \FSTP は、\ST{1}の内容を破棄してスタックの上に\ST{0}を残します。

GDBでこの関数をトレースしましょう:

\lstinputlisting[caption=\Optimizing GCC 4.8.1 and GDB,numbers=left]{patterns/12_FPU/3_comparison/x86/gdb.txt}

\q{ni}を使って、
最初の \FLD 命令を2つ実行してみましょう。

FPUレジスタを確認してみましょう。(33行目)

以前書いたように、FPUレジスタのセットはスタックではなく循環バッファです。(\myref{FPU_is_rather_circular_buffer})
そしてGDBは\GTT{STx}レジスタを表示しませんが、FPUレジスタの内部を表示します。(\GTT{Rx})
(35行目の)矢印は現在のスタックのトップを示しています。

\emph{Status Word}(44行目)に\GTT{TOP}レジスタの内容を見ることができます。今は6で、
スタックのトップは内部レジスタ6を示しています。

$a$ および $b$ の値は\INS{FXCH}が実行されると交換されます。(54行目)

\INS{FUCOMI}は実行されます。(83行目)
フラグを見てみましょう: \CF がセットされます。(95行目)

\INS{FCMOVBE} は $b$ の値をコピーします。(104行目)

\FSTP はスタックのトップの値を1つ残します。(136行目)
\GTT{TOP}の値は7で、FPUスタックのトップは内部レジスタ7を示しています。
}


\EN{\subsubsection{ARM}

\myparagraph{\OptimizingXcodeIV (\ARMMode)}

\lstinputlisting[caption=\OptimizingXcodeIV (\ARMMode),style=customasmARM]{patterns/12_FPU/3_comparison/ARM/Xcode_ARM_EN.lst}

\myindex{ARM!\Registers!APSR}
\myindex{ARM!\Registers!FPSCR}
A very simple case.
The input values are placed into the \GTT{D17} and \GTT{D16} registers and then compared using the \INS{VCMPE} instruction.

Just like in the x86 coprocessor, the ARM coprocessor has its own status and flags register (\ac{FPSCR}),
since there is a necessity to store coprocessor-specific flags.
% TODO -> расписать регистр по битам
\myindex{ARM!\Instructions!VMRS}
And just like in x86, there are no conditional jump instruction in ARM, 
that can check bits in the status register of the coprocessor. 
So there is \INS{VMRS}, which copies 4 bits (N, Z, C, V) from the coprocessor status word into bits of the \emph{general} status register (\ac{APSR}).

\myindex{ARM!\Instructions!VMOVGT}
\INS{VMOVGT} is the analog of the \INS{MOVGT}, 
instruction for D-registers, it executes if one operand is greater than the other while comparing (\emph{GT---Greater Than}). 

If it gets executed, the value of $a$ is to be written into \GTT{D16} (that is currently stored in \GTT{D17}).
Otherwise the value of $b$ stays in the \GTT{D16} register.

\myindex{ARM!\Instructions!VMOV}

The penultimate instruction \INS{VMOV} prepares the value in the \GTT{D16} register for returning it via the \Reg{0} and \Reg{1}
register pair.

\myparagraph{\OptimizingXcodeIV (\ThumbTwoMode)}

\begin{lstlisting}[caption=\OptimizingXcodeIV (\ThumbTwoMode),style=customasmARM]
VMOV            D16, R2, R3 ; b
VMOV            D17, R0, R1 ; a
VCMPE.F64       D17, D16
VMRS            APSR_nzcv, FPSCR
IT GT 
VMOVGT.F64      D16, D17
VMOV            R0, R1, D16
BX              LR
\end{lstlisting}

Almost the same as in the previous example, however slightly different.
As we already know, many instructions in ARM mode can be supplemented by condition predicate.
But there is no such thing in Thumb mode. 
There is no space in the 16-bit instructions for 4 more bits in which conditions can be encoded.

\myindex{ARM!\ThumbTwoMode}

However, Thumb-2 was extended to make it possible to specify predicates to old Thumb instructions.
Here, in the \IDA-generated listing, we see the \INS{VMOVGT} instruction, as in previous example.

In fact, the usual \INS{VMOV} is encoded there, but \IDA adds the \GTT{-GT} suffix to it, 
since there is a \INS{IT GT} instruction placed right before it.

\label{ARM_Thumb_IT}
\myindex{ARM!\Instructions!IT}
\myindex{ARM!if-then block}
The \INS{IT} instruction defines a so-called \emph{if-then block}. 

After the instruction it is possible to place up to 4 instructions, 
each of them has a predicate suffix.
In our example, \INS{IT GT} implies that the next instruction is to be executed, if the \emph{GT} (\emph{Greater Than}) condition is true.

\myindex{Angry Birds}
Here is a more complex code fragment, by the way, from Angry Birds (for iOS):

\begin{lstlisting}[caption=Angry Birds Classic,style=customasmARM]
...
ITE NE
VMOVNE          R2, R3, D16
VMOVEQ          R2, R3, D17
BLX             _objc_msgSend ; not suffixed
...
\end{lstlisting}

\INS{ITE} stands for \emph{if-then-else} 

and it encodes suffixes for the next two instructions.

The first instruction executes if the condition encoded in \INS{ITE} (\emph{NE, not equal}) is true at, and the second---if the condition is not true.
(The inverse condition of \GTT{NE} is \GTT{EQ} (\emph{equal})).

The instruction followed after the second \INS{VMOV} (or \INS{VMOVEQ}) is a normal one, not suffixed (\INS{BLX}).

\myindex{Angry Birds}
One more that's slightly harder, which is also from Angry Birds:

\begin{lstlisting}[caption=Angry Birds Classic,style=customasmARM]
...
ITTTT EQ
MOVEQ           R0, R4
ADDEQ           SP, SP, #0x20
POPEQ.W         {R8,R10}
POPEQ           {R4-R7,PC}
BLX             ___stack_chk_fail ; not suffixed
...
\end{lstlisting}

Four \q{T} symbols in the instruction mnemonic mean that the four subsequent instructions are to be executed if the condition is true.

That's why \IDA adds the \GTT{-EQ} suffix to each one of them. 

And if there was, for example, \INS{ITEEE EQ} (\emph{if-then-else-else-else}), 
then the suffixes would have been set as follows:

\begin{lstlisting}
-EQ
-NE
-NE
-NE
\end{lstlisting}

\myindex{Angry Birds}
Another fragment from Angry Birds:

\begin{lstlisting}[caption=Angry Birds Classic,style=customasmARM]
...
CMP.W           R0, #0xFFFFFFFF
ITTE LE
SUBLE.W         R10, R0, #1
NEGLE           R0, R0
MOVGT           R10, R0
MOVS            R6, #0         ; not suffixed
CBZ             R0, loc_1E7E32 ; not suffixed
...
\end{lstlisting}

\INS{ITTE} (\emph{if-then-then-else}) 

implies that the 1st and 2nd instructions are to be executed if the \GTT{LE} (\emph{Less or Equal})
condition is true, and the 3rd---if the inverse condition (\GTT{GT}---\emph{Greater Than}) 
is true.

Compilers usually don't generate all possible combinations.
\myindex{Angry Birds}

For example, in the mentioned Angry Birds game (\emph{classic} version for iOS)
only these variants of the \INS{IT} instruction are used: 
\INS{IT}, \INS{ITE}, \INS{ITT}, \INS{ITTE}, \INS{ITTT}, \INS{ITTTT}.
\myindex{\GrepUsage}
How to learn this?
In \IDA, it is possible to produce listing files, so it was created with an option to show 4 bytes for each opcode.
Then, knowing the high part of the 16-bit opcode (\INS{IT} is \GTT{0xBF}), we do the following using \GTT{grep}:

\begin{lstlisting}
cat AngryBirdsClassic.lst | grep " BF" | grep "IT" > results.lst
\end{lstlisting}

\myindex{ARM!\ThumbTwoMode}

By the way, if you program in ARM assembly language manually for Thumb-2 mode, 
and you add conditional suffixes,
the assembler will add the \INS{IT} instructions automatically with the required flags where it is necessary.

\myparagraph{\NonOptimizingXcodeIV (\ARMMode)}

\begin{lstlisting}[caption=\NonOptimizingXcodeIV (\ARMMode),style=customasmARM]
b               = -0x20
a               = -0x18
val_to_return   = -0x10
saved_R7        = -4

                STR             R7, [SP,#saved_R7]!
                MOV             R7, SP
                SUB             SP, SP, #0x1C
                BIC             SP, SP, #7
                VMOV            D16, R2, R3
                VMOV            D17, R0, R1
                VSTR            D17, [SP,#0x20+a]
                VSTR            D16, [SP,#0x20+b]
                VLDR            D16, [SP,#0x20+a]
                VLDR            D17, [SP,#0x20+b]
                VCMPE.F64       D16, D17
                VMRS            APSR_nzcv, FPSCR
                BLE             loc_2E08
                VLDR            D16, [SP,#0x20+a]
                VSTR            D16, [SP,#0x20+val_to_return]
                B               loc_2E10

loc_2E08
                VLDR            D16, [SP,#0x20+b]
                VSTR            D16, [SP,#0x20+val_to_return]

loc_2E10
                VLDR            D16, [SP,#0x20+val_to_return]
                VMOV            R0, R1, D16
                MOV             SP, R7
                LDR             R7, [SP+0x20+b],#4
                BX              LR
\end{lstlisting}

Almost the same as we already saw, 
but there is too much redundant code because the $a$ and $b$ variables are stored in the local stack, as well
as the return value.

\myparagraph{\OptimizingKeilVI (\ThumbMode)}

\begin{lstlisting}[caption=\OptimizingKeilVI (\ThumbMode),style=customasmARM]
                PUSH    {R3-R7,LR}
                MOVS    R4, R2
                MOVS    R5, R3
                MOVS    R6, R0
                MOVS    R7, R1
                BL      __aeabi_cdrcmple
                BCS     loc_1C0
                MOVS    R0, R6
                MOVS    R1, R7
                POP     {R3-R7,PC}

loc_1C0
                MOVS    R0, R4
                MOVS    R1, R5
                POP     {R3-R7,PC}
\end{lstlisting}


Keil doesn't generate FPU-instructions since it cannot rely on them being
supported on the target CPU, and it cannot be done by straightforward bitwise comparing.
%TODO1: why?
So it calls an external library function to do the comparison: \GTT{\_\_aeabi\_cdrcmple}. 
\myindex{ARM!\Instructions!BCS}

N.B. The result of the comparison is to be left in the flags by this function, so the following
\INS{BCS} (\emph{Carry set---Greater than or equal})
instruction can work without any additional code.

}
\RU{\subsubsection{ARM}

\myparagraph{\OptimizingXcodeIV (\ARMMode)}

\lstinputlisting[caption=\OptimizingXcodeIV (\ARMMode),style=customasmARM]{patterns/12_FPU/3_comparison/ARM/Xcode_ARM_RU.lst}

\myindex{ARM!\Registers!APSR}
\myindex{ARM!\Registers!FPSCR}
Очень простой случай.
Входные величины помещаются в \GTT{D17} и \GTT{D16} и сравниваются при помощи инструкции \INS{VCMPE}.
Как и в сопроцессорах x86, сопроцессор в ARM имеет свой собственный регистр статуса и флагов (\ac{FPSCR}),
потому что есть необходимость хранить специфичные для его работы флаги.

% TODO -> расписать регистр по битам
\myindex{ARM!\Instructions!VMRS}
И так же, как и в x86, 
в ARM нет инструкций условного перехода, проверяющих биты в регистре статуса сопроцессора. 
Поэтому имеется инструкция \INS{VMRS}, копирующая 4 бита (N, Z, C, V) 
из статуса сопроцессора в биты \emph{общего} статуса (регистр \ac{APSR}).

\myindex{ARM!\Instructions!VMOVGT}
\INS{VMOVGT} это аналог \INS{MOVGT}, инструкция для D-регистров, срабатывающая, если при сравнении один операнд был больше чем второй
(\emph{GT --- Greater Than}). 

Если она сработает, 
в \GTT{D16} запишется значение $a$, лежащее в тот момент в \GTT{D17}.
В обратном случае в \GTT{D16} остается значение $b$.


\myindex{ARM!\Instructions!VMOV}
Предпоследняя инструкция \INS{VMOV} готовит то, что было в \GTT{D16}, для возврата через 
пару регистров \Reg{0} и \Reg{1}.

\myparagraph{\OptimizingXcodeIV (\ThumbTwoMode)}

\begin{lstlisting}[caption=\OptimizingXcodeIV (\ThumbTwoMode),style=customasmARM]
VMOV            D16, R2, R3 ; b
VMOV            D17, R0, R1 ; a
VCMPE.F64       D17, D16
VMRS            APSR_nzcv, FPSCR
IT GT 
VMOVGT.F64      D16, D17
VMOV            R0, R1, D16
BX              LR
\end{lstlisting}

Почти то же самое, что и в предыдущем примере, за парой отличий.
Как мы уже знаем, многие инструкции в режиме ARM можно дополнять условием.
Но в режиме Thumb такого нет.
В 16-битных инструкций просто нет места для лишних 4 битов, при помощи
которых можно было бы закодировать условие выполнения.

\myindex{ARM!\ThumbTwoMode}
Поэтому в Thumb-2 добавили возможность дополнять \\
Thumb-инструкции условиями.
В листинге, сгенерированном при помощи \IDA, мы видим инструкцию \INS{VMOVGT}, 
такую же как и в предыдущем примере.

В реальности там закодирована обычная инструкция \INS{VMOV}, просто \IDA добавила суффикс \GTT{-GT} к ней, 
потому что перед этой инструкцией стоит \INS{IT GT}.

\label{ARM_Thumb_IT}
\myindex{ARM!\Instructions!IT}
\myindex{ARM!if-then block}
Инструкция \INS{IT} определяет так называемый \emph{if-then block}. 
После этой инструкции можно указывать до четырех инструкций, 
к каждой из которых будет добавлен суффикс условия.

В нашем примере \INS{IT GT} означает,
что следующая за ней инструкция будет исполнена, если условие
\emph{GT} (\emph{Greater Than}) справедливо.

\myindex{Angry Birds}
Теперь более сложный пример. Кстати, из 
Angry Birds (для iOS):

\begin{lstlisting}[caption=Angry Birds Classic,style=customasmARM]
...
ITE NE
VMOVNE          R2, R3, D16
VMOVEQ          R2, R3, D17
BLX             _objc_msgSend ; без суффикса
...
\end{lstlisting}

\INS{ITE} означает \emph{if-then-else} 
и кодирует суффиксы для двух следующих за ней инструкций.

Первая из них исполнится, если условие, закодированное в \INS{ITE} (\emph{NE, not equal}) будет в тот момент справедливо,
а вторая~--- если это условие не сработает.
(Обратное условие от \GTT{NE} это \GTT{EQ} (\emph{equal})).

Инструкция следующая за второй \INS{VMOV} (или VMOEQ) нормальная, без суффикса (\INS{BLX}).

\myindex{Angry Birds}
Ещё чуть сложнее, и снова этот фрагмент из Angry Birds:

\begin{lstlisting}[caption=Angry Birds Classic,style=customasmARM]
...
ITTTT EQ
MOVEQ           R0, R4
ADDEQ           SP, SP, #0x20
POPEQ.W         {R8,R10}
POPEQ           {R4-R7,PC}
BLX             ___stack_chk_fail ; без суффикса
...
\end{lstlisting}

Четыре символа \q{T} в инструкции означают, что четыре последующие инструкции будут исполнены если условие соблюдается.
Поэтому \IDA добавила ко всем четырем инструкциям суффикс \GTT{-EQ}. 
А если бы здесь было, например,
\INS{ITEEE EQ} (\emph{if-then-else-else-else}), 
тогда суффиксы для следующих четырех инструкций были бы расставлены так:

\begin{lstlisting}
-EQ
-NE
-NE
-NE
\end{lstlisting}

\myindex{Angry Birds}
Ещё фрагмент из Angry Birds:

\begin{lstlisting}[caption=Angry Birds Classic,style=customasmARM]
...
CMP.W           R0, #0xFFFFFFFF
ITTE LE
SUBLE.W         R10, R0, #1
NEGLE           R0, R0
MOVGT           R10, R0
MOVS            R6, #0         ; без суффикса
CBZ             R0, loc_1E7E32 ; без суффикса
...
\end{lstlisting}

\INS{ITTE} (\emph{if-then-then-else}) 
означает, что первая и вторая инструкции исполнятся, если условие \GTT{LE} (\emph{Less or Equal})
справедливо, а третья~--- если справедливо обратное условие (\GTT{GT} --- \emph{Greater Than}).

Компиляторы способны генерировать далеко не все варианты.

\myindex{Angry Birds}
Например, в вышеупомянутой игре Angry Birds (версия \emph{classic} для iOS)

встречаются только такие варианты инструкции \INS{IT}: 
\INS{IT}, \INS{ITE}, \INS{ITT}, \INS{ITTE}, \INS{ITTT}, \INS{ITTTT}.
\myindex{\GrepUsage}
Как это узнать?
В \IDA можно сгенерировать листинг (что и было сделано), только в опциях был установлен показ 4 байтов для каждого опкода.

Затем, зная что старшая часть 16-битного опкода (\INS{IT} это \GTT{0xBF}), сделаем при помощи \GTT{grep} это:

\begin{lstlisting}
cat AngryBirdsClassic.lst | grep " BF" | grep "IT" > results.lst
\end{lstlisting}

\myindex{ARM!\ThumbTwoMode}
Кстати, если писать на ассемблере для режима Thumb-2 вручную, и дополнять инструкции суффиксами
условия, то ассемблер автоматически будет добавлять инструкцию \INS{IT} с соответствующими флагами там,
где надо.

\myparagraph{\NonOptimizingXcodeIV (\ARMMode)}

\begin{lstlisting}[caption=\NonOptimizingXcodeIV (\ARMMode),style=customasmARM]
b               = -0x20
a               = -0x18
val_to_return   = -0x10
saved_R7        = -4

                STR             R7, [SP,#saved_R7]!
                MOV             R7, SP
                SUB             SP, SP, #0x1C
                BIC             SP, SP, #7
                VMOV            D16, R2, R3
                VMOV            D17, R0, R1
                VSTR            D17, [SP,#0x20+a]
                VSTR            D16, [SP,#0x20+b]
                VLDR            D16, [SP,#0x20+a]
                VLDR            D17, [SP,#0x20+b]
                VCMPE.F64       D16, D17
                VMRS            APSR_nzcv, FPSCR
                BLE             loc_2E08
                VLDR            D16, [SP,#0x20+a]
                VSTR            D16, [SP,#0x20+val_to_return]
                B               loc_2E10

loc_2E08
                VLDR            D16, [SP,#0x20+b]
                VSTR            D16, [SP,#0x20+val_to_return]

loc_2E10
                VLDR            D16, [SP,#0x20+val_to_return]
                VMOV            R0, R1, D16
                MOV             SP, R7
                LDR             R7, [SP+0x20+b],#4
                BX              LR
\end{lstlisting}

Почти то же самое, что мы уже видели, 
но много избыточного кода из-за хранения $a$ и $b$, 
а также выходного значения, в локальном стеке.


\myparagraph{\OptimizingKeilVI (\ThumbMode)}

\begin{lstlisting}[caption=\OptimizingKeilVI (\ThumbMode),style=customasmARM]
                PUSH    {R3-R7,LR}
                MOVS    R4, R2
                MOVS    R5, R3
                MOVS    R6, R0
                MOVS    R7, R1
                BL      __aeabi_cdrcmple
                BCS     loc_1C0
                MOVS    R0, R6
                MOVS    R1, R7
                POP     {R3-R7,PC}

loc_1C0
                MOVS    R0, R4
                MOVS    R1, R5
                POP     {R3-R7,PC}
\end{lstlisting}

Keil не генерирует FPU-инструкции, потому что не 
рассчитывает на то, что они будет поддерживаться, а простым сравнением побитово здесь не обойтись.

%TODO1: why?
Для сравнения вызывается библиотечная функция \GTT{\_\_aeabi\_cdrcmple}. 
\myindex{ARM!\Instructions!BCS}

N.B. Результат сравнения эта функция оставляет в флагах, чтобы следующая за вызовом инструкция
\INS{BCS} (\emph{Carry set~--- Greater than or equal})
могла работать без дополнительного кода.

}
\DE{\subsubsection{ARM}

\myparagraph{\OptimizingXcodeIV (\ARMMode)}

\lstinputlisting[caption=\OptimizingXcodeIV
(\ARMMode),style=customasmARM]{patterns/12_FPU/3_comparison/ARM/Xcode_ARM_DE.lst}

\myindex{ARM!\Registers!APSR}
\myindex{ARM!\Registers!FPSCR}
Ein recht einfacher Fall.
Die Eingabewerte werden in die Register \GTT{D17} und \GTT{D16} geladen und dann mit dem Befehl \INS{VCMPE} verglichen. 

Genau wie der x86-Koprozessor besitzt auch der ARM-Koprozessor seine eigenen Status und Flag Register (\ac{FPSCR}), denn
es gibt auch hier die Notwendigkeit die spezifischen Flags des Koprozessors zu speichern.

% TODO -> расписать регистр по битам
\myindex{ARM!\Instructions!VMRS}
Und genau wie beim x86 gibt es auch in ARM keine Befehle für bedingte Sprünge, die die Bits im Statusregister des
Koprozessors abfragen können. So gibt es den Befehl \INS{VMRS}, um 4 Bits (N, Z, C, V) vom Statusregister des
Koprozessors in das \emph{allgemeine} Statusregister (\ac{APSR}) zu kopieren.

\myindex{ARM!\Instructions!VMOVGT}
\INS{VMOVGT} ist das Analogon zum \INS{MOVGT} Befehl für D-Register: er wird ausgeführt, wenn ein Operand bezüglich
eines \emph{GT---Greater Than (dt. größer als)} Vergleichs größer ist als der andere.

Wenn er ausgeführt wird, wird der Wert von $a$ nach \GTT{D16} geschrieben (dieser wird aktuell in \GTT{D17}
gespeichert). Andernfalls bleibt der Wert von $b$ im \GTT{D16} Register.

\myindex{ARM!\Instructions!VMOV}
Der vorletzte Befehl \INS{VMOV} bereitet den Wert im \GTT{D16} Register für die Rückgabe über das Registerpaar \Reg{0}
und \Reg{1} vor.

\myparagraph{\OptimizingXcodeIV (\ThumbTwoMode)}

\begin{lstlisting}[caption=\OptimizingXcodeIV (\ThumbTwoMode),style=customasmARM]
VMOV            D16, R2, R3 ; b
VMOV            D17, R0, R1 ; a
VCMPE.F64       D17, D16
VMRS            APSR_nzcv, FPSCR
IT GT 
VMOVGT.F64      D16, D17
VMOV            R0, R1, D16
BX              LR
\end{lstlisting}

Fast das gleiche wie im vorherigen Beispiel, aber in gewisser Weise dennoch unterschiedlich.
Wie wir bereits wissen, können viele Befehl im ARM mode durch bedingte Prädikate unterstützt werden.
Im Thumb mode dagegen gibt es nichts Vergleichbares.
Es gibt keinen Platz in den 16-Bit-Befehlen für 4 weitere Bits, in denen Bedingungen kodiert werden könnten.

\myindex{ARM!\ThumbTwoMode}
Thumb-2 wurde erweitert, um zu ermöglichen alten Thumb-Befehlen nachträglich Prädikate zuzuweisen. Hier, im von \IDA
erzeugten Listing finden wir den \INS{VMOVGT} Befehl aus dem vorherigen Beispiel wieder.

Tatsächlich ist hier das gewöhnliche \INS{VMOV} kodiert, aber \IDA ergänzt den Suffix \GTT{-GT}, da sich direkt davor
eine \INS{IT GT} Befehl befindet.

\label{ARM_Thumb_IT}
\myindex{ARM!\Instructions!IT}
\myindex{ARM!if-then block}
Der \INS{IT} Befehl definiert einen sogenannten \emph{If-Then-Block}.

Nach dem Befehl können bis zu 4 weitere Befehle, jeder mit einem beschreibenden Suffix, verwendet werden.
In unserem Beispiel impliziert \INS{IT GT}, dass der Folgebefehl genau dann ausgeführt werden soll, wenn die IT{GT}
(\emph{Greater Than}) Bedingung wahr ist.

\myindex{Angry Birds}
Hier ist ein komplexeres Codefragment, welches aus Angry Birds (für iOS) stammt:

\begin{lstlisting}[caption=Angry Birds Classic,style=customasmARM]
...
ITE NE
VMOVNE          R2, R3, D16
VMOVEQ          R2, R3, D17
BLX             _objc_msgSend ; ohne Suffix
...
\end{lstlisting}

\INS{ITE} steht für \emph{if-then-else} und kodiert Suffixe für die beiden folgenden Befehle.

Der erste Befehl wird ausgeführt, wenn die durch \INS{ITE} (\emph{NE, not ewual}, dt. ungleich) kodierte Bedingung wahr
ist und der zweite wenn die Bedingung falsch ist (die inverse Bedingung zu \GTT{NE} ist \GTT{EQ} (\emph{equal}, dt.
gleich)).

Der dem zweiten Befehl folgende \INS{VMOV} (oder \INS{VMOVEQ}) ist ein gewöhnlicher Befehl ohne Suffix (\INS{BLX}).

\myindex{Angry Birds}
Ein weiteres etwas schwieriger verständliches Codefragment, ebenfalls aus Angry Birds:

\begin{lstlisting}[caption=Angry Birds Classic,style=customasmARM]
...
ITTTT EQ
MOVEQ           R0, R4
ADDEQ           SP, SP, #0x20
POPEQ.W         {R8,R10}
POPEQ           {R4-R7,PC}
BLX             ___stack_chk_fail ; ohne Suffix
...
\end{lstlisting}
Vier \q{T} Symbole in der Mnemonik des Befehls bedeuten, dass die vier folgenden Befehle alle ausgeführt werden, wenn
die Bedingung wahr ist. 

Aus diesem Grund fügt \IDA den \GTT{-EQ} Suffix an jeden der vier Befehle an.

Gäbe es beispielsweise \INS{ITEEE EQ} (\emph{if-then-else-else-else}), dann würden wie folgt Suffixe angehängt werden:

\begin{lstlisting}
-EQ
-NE
-NE
-NE
\end{lstlisting}

\myindex{Angry Birds}
Ein weiteres Fragment aus Angry Birds:

\begin{lstlisting}[caption=Angry Birds Classic,style=customasmARM]
...
CMP.W           R0, #0xFFFFFFFF
ITTE LE
SUBLE.W         R10, R0, #1
NEGLE           R0, R0
MOVGT           R10, R0
MOVS            R6, #0         ; ohne Suffix
CBZ             R0, loc_1E7E32 ; ohne Suffix
...
\end{lstlisting}

\INS{ITTE} (\emph{if-then-then-else}) impliziert, dass der erste und zweite Befehl ausgeführt werden, wenn die \GTT{LE}
(\emph{Less or Equal}, dt. mindestens) Bedingung wahr ist und der dritte, wenn die inverse Bedingung
(\GTT{GT}---\emph{Greater Than}, dt. mehr als) wahr ist.

Für gewöhnlich erzeugen Compiler nicht alle denkbaren Kombinationen.
\myindex{Angry Birds}
Im betrachteten Spiel Angry Birds beispielsweise (\emph{classic} Version für iOS) werden nur die folgenden Variationen des
\INS{IT} Befehls verwendet:
\INS{IT}, \INS{ITE}, \INS{ITT}, \INS{ITTE}, \INS{ITTT}, \INS{ITTTT}.
\myindex{\GrepUsage}
Bleibt die Frage, wie man dies lernen kann. 
In \IDA ist es mögliche Listing-Dateien zu erzeugen mit der Option 4 Bytes für jeden Opcode anzuzeigen. 
Dadurch können wir bei Kenntnis des höherwertigen Teils des 16-Bit-Opcodes (\INS{IT} entspricht \GTT{0xBF}) unter
Zuhilfenahme von \GTT{grep} wie folgt vorgehen:

\begin{lstlisting}
cat AngryBirdsClassic.lst | grep " BF" | grep "IT" > results.lst
\end{lstlisting}

\myindex{ARM!\ThumbTwoMode}
Übrigens, wenn man von Hand Assemblerprogramme für ARM in Thumb-2 mode schreibt und man die Suffixe für die Bedingungen
selbst anhängt, wird der Assemblierer die entsprechenden \INS{IT} Befehle inklusive der benötigten Flags automatisch an
den benötigten Stellen hinzufügen.

\myparagraph{\NonOptimizingXcodeIV (\ARMMode)}

\begin{lstlisting}[caption=\NonOptimizingXcodeIV (\ARMMode),style=customasmARM]
b               = -0x20
a               = -0x18
val_to_return   = -0x10
saved_R7        = -4

                STR             R7, [SP,#saved_R7]!
                MOV             R7, SP
                SUB             SP, SP, #0x1C
                BIC             SP, SP, #7
                VMOV            D16, R2, R3
                VMOV            D17, R0, R1
                VSTR            D17, [SP,#0x20+a]
                VSTR            D16, [SP,#0x20+b]
                VLDR            D16, [SP,#0x20+a]
                VLDR            D17, [SP,#0x20+b]
                VCMPE.F64       D16, D17
                VMRS            APSR_nzcv, FPSCR
                BLE             loc_2E08
                VLDR            D16, [SP,#0x20+a]
                VSTR            D16, [SP,#0x20+val_to_return]
                B               loc_2E10

loc_2E08
                VLDR            D16, [SP,#0x20+b]
                VSTR            D16, [SP,#0x20+val_to_return]

loc_2E10
                VLDR            D16, [SP,#0x20+val_to_return]
                VMOV            R0, R1, D16
                MOV             SP, R7
                LDR             R7, [SP+0x20+b],#4
                BX              LR
\end{lstlisting}
Fast identisch mit dem, was wir schon gesehen haben, aber hier gibt es zu viel redundanten Code, weil die Variablen $a$
und $b$ im lokalen Stack und zusätzlich als Rückgabewerte gespeichert werden.

\myparagraph{\OptimizingKeilVI (\ThumbMode)}

\begin{lstlisting}[caption=\OptimizingKeilVI (\ThumbMode),style=customasmARM]
                PUSH    {R3-R7,LR}
                MOVS    R4, R2
                MOVS    R5, R3
                MOVS    R6, R0
                MOVS    R7, R1
                BL      __aeabi_cdrcmple
                BCS     loc_1C0
                MOVS    R0, R6
                MOVS    R1, R7
                POP     {R3-R7,PC}

loc_1C0
                MOVS    R0, R4
                MOVS    R1, R5
                POP     {R3-R7,PC}
\end{lstlisting}

Keil erzeugt keine FPU-Befehle, da er sich sich darauf verlassen kann, dass diese von der Ziel-CPU unterstützt werden
und dies nicht durch einfache bitweisen Vergleich erledigt werden kann.

%TODO1: why?
Keil ruft also eine Funktion einer externen Programmbibliothek (\GTT{\_\_aeabi\_cdrcmple}) auf, um den Vergleich
durchzuführen.
\myindex{ARM!\Instructions!BCS}
Das Ergebnis des Vergleich wird von der Funktion in den Flags belassen, sodass der folgende \INS{BCS} (\emph{Carry
set---Greater than or equal}) Befehl ohne zusätzlichen Code funktioniert.

}
\FR{\subsubsection{ARM}

\myparagraph{\OptimizingXcodeIV (\ARMMode)}

\lstinputlisting[caption=\OptimizingXcodeIV (\ARMMode),style=customasmARM]{patterns/12_FPU/3_comparison/ARM/Xcode_ARM_FR.lst}

\myindex{ARM!\Registers!APSR}
\myindex{ARM!\Registers!FPSCR}
Un cas très simple.
Les valeurs en entrée sont placées dans les registres \GTT{D17} et \GTT{D16} puis
comparées en utilisant l'instruction \INS{VCMPE}.

Tout comme dans le coprocesseur x86, le coprocesseur ARM a son propre registre de
flags (\ac{FPSCR}), puisqu'il est nécessaire de stocker des flags spécifique au coprocesseur.
% TODO -> расписать регистр по битам
\myindex{ARM!\Instructions!VMRS}
Et tout comme en x86, il n'y a pas d'instruction de saut conditionnel qui teste des
bits dans le registre de status du coprocesseur.
Donc il y a \INS{VMRS}, qui copie 4 bits (N, Z, C, V) du mot d'état du coprocesseur
dans les bits du registre de status \emph{général} (\ac{APSR}).

\myindex{ARM!\Instructions!VMOVGT}
\INS{VMOVGT} est l'analogue de l'instruction \INS{MOVGT} pour D-registres, elle s'exécute
si un opérande est plus grand que l'autre lors de la comparaison (\emph{GT---Greater Than}).

Si elle est exécutée, la valeur de $a$ sera écrite dans \GTT{D16} (ce qui est écrit
en ce moment dans \GTT{D17}).
Sinon, la valeur de $b$ reste dans le registre \GTT{D16}.

\myindex{ARM!\Instructions!VMOV}

La pénultième instruction \INS{VMOV} prépare la valeur dans la registre \GTT{D16}
afin de la renvoyer dans la paire de registres \Reg{0} et \Reg{1}.

\myparagraph{\OptimizingXcodeIV (\ThumbTwoMode)}

\begin{lstlisting}[caption=\OptimizingXcodeIV (\ThumbTwoMode),style=customasmARM]
VMOV            D16, R2, R3 ; b
VMOV            D17, R0, R1 ; a
VCMPE.F64       D17, D16
VMRS            APSR_nzcv, FPSCR
IT GT 
VMOVGT.F64      D16, D17
VMOV            R0, R1, D16
BX              LR
\end{lstlisting}

Presque comme dans l'exemple précédent, toutefois légèrement différent.
Comme nous le savons déjà, en mode ARM, beaucoup d'instructions peuvent avoir un
prédicat de condition.
Mais il n'y a rien de tel en mode Thumb.
Il n'y a pas d'espace dans les instructions sur 16-bit pour 4 bits dans lesquels
serait encodée la condition.

\myindex{ARM!\ThumbTwoMode}

Toutefois, cela à été étendu en un mode Thumb-2 pour rendre possible de spécifier
un prédicat aux instructions de l'ancien mode Thumb.
Ici, dans le listing généré par \IDA, nous voyons l'instruction \INS{VMOVGT}, comme
dans l'exemple précédent.

En fait, le \INS{VMOV} usuel est encodé ici, mais \IDA lui ajoute le suffixe \GTT{-GT},
puisque que l'instruction \INS{IT GT} se trouve juste avant.

\label{ARM_Thumb_IT}
\myindex{ARM!\Instructions!IT}
\myindex{ARM!if-then block}
L'instruction \INS{IT} défini ce que l'on appelle un \emph{bloc if-then}.

Après cette instruction, il est possible de mettre jusqu'à 4 instructions, chacune
d'entre elles ayant un suffixe de prédicat.
Dans notre exemple, \INS{IT GT} implique que l'instruction suivante ne sera exécutée
que si la condition \emph{GT} (\emph{Greater Than} plus grand que) est vraie.

\myindex{Angry Birds}
Voici un exemple de code plus complexe, à propos, d'Angry Birds (pour iOS):

\begin{lstlisting}[caption=Angry Birds Classic,style=customasmARM]
...
ITE NE
VMOVNE          R2, R3, D16
VMOVEQ          R2, R3, D17
BLX             _objc_msgSend ; not suffixed
...
\end{lstlisting}

\INS{ITE} est l'acronyme de \emph{if-then-else}
et elle encode un suffixe pour les deux prochaines instructions.

La première instruction est exécutée si la condition encodée dans \INS{ITE} (\emph{NE, not equal})
est vraie, et la seconde---si la condition n'est pas vraie (l'inverse de la condition
\GTT{NE} est \GTT{EQ} (\emph{equal})).

L'instruction qui suit le second \INS{VMOV} (ou \INS{VMOVEQ}) est normale, non suffixée
(\INS{BLX}).

\myindex{Angry Birds}
Un autre exemple qui est légèrement plus difficile, qui est aussi d'Angry Birds:

\begin{lstlisting}[caption=Angry Birds Classic,style=customasmARM]
...
ITTTT EQ
MOVEQ           R0, R4
ADDEQ           SP, SP, #0x20
POPEQ.W         {R8,R10}
POPEQ           {R4-R7,PC}
BLX             ___stack_chk_fail ; not suffixed
...
\end{lstlisting}

Les quatre symboles \q{T} dans le mnémonique de l'instruction signifient que les quatre
instructions suivantes seront exécutées si la condition est vraie.

C'est pourquoi \IDA ajoute le suffixe \GTT{-EQ} à chacune d'entre elles.

Et si il y avait, par exemple, \INS{ITEEE EQ} (\emph{if-then-else-else-else}),
alors les suffixes seraient mis comme suit:

\begin{lstlisting}
-EQ
-NE
-NE
-NE
\end{lstlisting}

\myindex{Angry Birds}
Un autre morceau de code d'Angry Birds:

\begin{lstlisting}[caption=Angry Birds Classic,style=customasmARM]
...
CMP.W           R0, #0xFFFFFFFF
ITTE LE
SUBLE.W         R10, R0, #1
NEGLE           R0, R0
MOVGT           R10, R0
MOVS            R6, #0         ; not suffixed
CBZ             R0, loc_1E7E32 ; not suffixed
...
\end{lstlisting}

\INS{ITTE} (\emph{if-then-then-else})

implique que les 1ère et 2ème instructions seront exécutées si la condition \GTT{LE}
(\emph{Less or Equal} moins ou égal) est vraie, et que la 3ème---si la condition inverse
(\GTT{GT}---\emph{Greater Than} plus grand que) est vraie.

En général, les compilateurs ne génèrent pas toutes les combinaisons possible.
\myindex{Angry Birds}

Par exemple, dans le jeu Angry Birds mentionné ((\emph{classic} version pour iOS)
seules les les variantes suivantes de l'instruction \INS{IT} sont utilisées:
\INS{IT}, \INS{ITE}, \INS{ITT}, \INS{ITTE}, \INS{ITTT}, \INS{ITTTT}.
\myindex{\GrepUsage}
Comment savoir cela?
Dans \IDA, il est possible de produire un listing dans un fichier, ce qui a été utilisé
pour en créer un avec l'option d'afficher 4 octets pour chaque opcode.
Ensuite, en connaissant la partie haute de l'opcode de 16-bit (\GTT{0xBF} pour \INS{IT}),
nous utilisons \GTT{grep} ainsi:

\begin{lstlisting}
cat AngryBirdsClassic.lst | grep " BF" | grep "IT" > results.lst
\end{lstlisting}

\myindex{ARM!\ThumbTwoMode}

À propos, si vous programmez en langage d'assemblage ARM pour le mode Thumb-2, et
que vous ajoutez des suffixes conditionnels, l'assembleur ajoutera automatiquement
l'instruction \INS{IT} avec les flags là où ils sont nécessaires.

\myparagraph{\NonOptimizingXcodeIV (\ARMMode)}

\begin{lstlisting}[caption=\NonOptimizingXcodeIV (\ARMMode),style=customasmARM]
b               = -0x20
a               = -0x18
val_to_return   = -0x10
saved_R7        = -4

                STR             R7, [SP,#saved_R7]!
                MOV             R7, SP
                SUB             SP, SP, #0x1C
                BIC             SP, SP, #7
                VMOV            D16, R2, R3
                VMOV            D17, R0, R1
                VSTR            D17, [SP,#0x20+a]
                VSTR            D16, [SP,#0x20+b]
                VLDR            D16, [SP,#0x20+a]
                VLDR            D17, [SP,#0x20+b]
                VCMPE.F64       D16, D17
                VMRS            APSR_nzcv, FPSCR
                BLE             loc_2E08
                VLDR            D16, [SP,#0x20+a]
                VSTR            D16, [SP,#0x20+val_to_return]
                B               loc_2E10

loc_2E08
                VLDR            D16, [SP,#0x20+b]
                VSTR            D16, [SP,#0x20+val_to_return]

loc_2E10
                VLDR            D16, [SP,#0x20+val_to_return]
                VMOV            R0, R1, D16
                MOV             SP, R7
                LDR             R7, [SP+0x20+b],#4
                BX              LR
\end{lstlisting}

Presque la même chose que nous avons déjà vu, mais ici il y a beaucoup de code redondant
car les variables $a$ et $b$ sont stockées sur la pile locale, tout comme la valeur
de retour.

\myparagraph{\OptimizingKeilVI (\ThumbMode)}

\begin{lstlisting}[caption=\OptimizingKeilVI (\ThumbMode),style=customasmARM]
                PUSH    {R3-R7,LR}
                MOVS    R4, R2
                MOVS    R5, R3
                MOVS    R6, R0
                MOVS    R7, R1
                BL      __aeabi_cdrcmple
                BCS     loc_1C0
                MOVS    R0, R6
                MOVS    R1, R7
                POP     {R3-R7,PC}

loc_1C0
                MOVS    R0, R4
                MOVS    R1, R5
                POP     {R3-R7,PC}
\end{lstlisting}


Keil ne génère pas les instructions pour le FPU car il ne peut pas être sûr qu'elles
sont supportées sur le CPU cible, et cela ne peut pas être fait directement en comparant
les bits.
%TODO1: why?
Donc il appelle une fonction d'une bibliothèque externe pour effectuer la comparaison:
\GTT{\_\_aeabi\_cdrcmple}.
\myindex{ARM!\Instructions!BCS}

N.B. Le résultat de la comparaison est laissé dans les flags par cette fonction,
donc l'instruction \INS{BCS} (\emph{Carry set---Greater than or equal} plus grand ou
égal) fonctionne sans code additionnel.
}
\JA{\subsubsection{ARM}

\myparagraph{\OptimizingXcodeIV (\ARMMode)}

\lstinputlisting[caption=\OptimizingXcodeIV (\ARMMode),style=customasmARM]{patterns/12_FPU/3_comparison/ARM/Xcode_ARM_JA.lst}

\myindex{ARM!\Registers!APSR}
\myindex{ARM!\Registers!FPSCR}
非常に単純なケースです。
入力値は\GTT{D17}および\GTT{D16}レジスタに格納され、次に\INS{VCMPE}命令を使用して比較されます。

コプロセッサ固有のフラグを格納する必要があるため、x86コプロセッサと同様に、
ARMコプロセッサには独自のステータスおよびフラグレジスタ(\ac{FPSCR})があります。 
% TODO -> расписать регистр по битам
\myindex{ARM!\Instructions!VMRS}
また、x86と同様に、ARMでは条件付きジャンプ命令がなく、
コプロセッサのステータスレジスタ内のビットをチェックできます。 
したがって、コプロセッサステータスワードからの4ビット(N, Z, C, V)を汎用ステータスレジスタ(\ac{APSR})のビットにコピーするVMRSがあります。

\myindex{ARM!\Instructions!VMOVGT}
\INS{VMOVGT}はDレジスタ用の\INS{MOVGT}命令に類似のもので、
比較中に一方のオペランドが他方のものより大きい場合に実行されます。(\emph{GT---Greater Than})

実行されると、 (現在\GTT{D17}に格納されている) $a$ の値は\GTT{D16}に書き込まれます。 
それ以外の場合は、 $b$ の値は\GTT{D16}レジスタにとどまります。

\myindex{ARM!\Instructions!VMOV}

最後から2番目の\INS{VMOV}命令は、D0レジスタ内の値を\Reg{0}および\Reg{1}レジスタ対を介して戻すための値を準備します。

\myparagraph{\OptimizingXcodeIV (\ThumbTwoMode)}

\begin{lstlisting}[caption=\OptimizingXcodeIV (\ThumbTwoMode),style=customasmARM]
VMOV            D16, R2, R3 ; b
VMOV            D17, R0, R1 ; a
VCMPE.F64       D17, D16
VMRS            APSR_nzcv, FPSCR
IT GT 
VMOVGT.F64      D16, D17
VMOV            R0, R1, D16
BX              LR
\end{lstlisting}

前の例とほとんど同じですが、少し異なります。 
すでにわかっているように、ARMモードの多くの命令は条件述語で補うことができます。 
しかしThumbモードではこのようなことはありません。 
条件を符号化できる4ビット以上の16ビット命令にはスペースがありません。

\myindex{ARM!\ThumbTwoMode}

ただし、Thumb-2は、古いThumb命令に対する述部を指定できるように拡張されました。
ここでは、 \IDA で生成されたリストでは、前の例のように\INS{VMOVGT}命令が表示されます。

実際、通常の\INS{VMOV}はそこにエンコードされますが、その直前に\INS{IT GT}命令が置かれているため、
\IDA は\GTT{-GT}接尾辞を追加します。

\label{ARM_Thumb_IT}
\myindex{ARM!\Instructions!IT}
\myindex{ARM!if-then block}
\INS{IT}命令は、いわゆる\emph{if-then ブロック}を定義します。

命令の後に最大4つの命令を配置することができ、
それぞれに述語接尾辞があります。
この例では、\emph{GT}(\emph{Greater Than})条件が真である場合、\INS{IT GT}は次の命令が実行されることを意味します。

\myindex{Angry Birds}
ちなみにAngry Birds(iOS向け)のより複雑なコードの断片は次のとおりです。

\begin{lstlisting}[caption=Angry Birds Classic,style=customasmARM]
...
ITE NE
VMOVNE          R2, R3, D16
VMOVEQ          R2, R3, D17
BLX             _objc_msgSend ; not suffixed
...
\end{lstlisting}

\INS{ITE}は\emph{if-then-else}を表し、

次の2つの命令の接尾辞をエンコードします。

最初の命令は、\INS{ITE}でエンコードされた条件(\emph{NE, not equal})が真である場合に実行され、2番目の場合は条件が真でない場合に実行されます。
(\GTT{NE}の逆条件は\GTT{EQ}(\emph{等しい})です。

2番目の\INS{VMOV}(または\INS{VMOVEQ})が通常のもので、接尾辞(\INS{BLX})ではない命令の後に続きます。

\myindex{Angry Birds}
もう1つはやや難しく、これもAngry Birdsからです。

\begin{lstlisting}[caption=Angry Birds Classic,style=customasmARM]
...
ITTTT EQ
MOVEQ           R0, R4
ADDEQ           SP, SP, #0x20
POPEQ.W         {R8,R10}
POPEQ           {R4-R7,PC}
BLX             ___stack_chk_fail ; not suffixed
...
\end{lstlisting}

命令ニーモニック内の4つの\q{T}記号は、条件が真である場合に4つの後続の命令が実行されることを意味します。

これが \IDA がそれぞれに\GTT{-EQ}サフィックスを追加する理由です。

たとえば、\INS{ITEEE EQ}(\emph{if-then-else-else-else})があった場合、
接尾辞は次のように設定されます。

\begin{lstlisting}
-EQ
-NE
-NE
-NE
\end{lstlisting}

\myindex{Angry Birds}
Angry Birdsから別の断片です。

\begin{lstlisting}[caption=Angry Birds Classic,style=customasmARM]
...
CMP.W           R0, #0xFFFFFFFF
ITTE LE
SUBLE.W         R10, R0, #1
NEGLE           R0, R0
MOVGT           R10, R0
MOVS            R6, #0         ; not suffixed
CBZ             R0, loc_1E7E32 ; not suffixed
...
\end{lstlisting}

\INS{ITTE}(\emph{if-then-then-else})は、\GTT{LE}(\emph{Less or Equal})
条件が真であれば第1および第2の命令が実行されることを意味し、逆条件(\GTT{GT}---\emph{Greater Than})
が真であれば第3の命令を実行することを意味します。

コンパイラは通常、可能な組み合わせのすべてを生成しません。
\myindex{Angry Birds}

たとえば、前述のAngry Birdsゲーム(iOSの\emph{クラシック}バージョン)では、
\INS{IT}命令の変種である、\INS{ITE}、\INS{ITT}、\INS{ITTE}、\INS{ITTT}、\INS{ITTTT}のみが使用されます。 
\myindex{\GrepUsage}
これらをを学ぶのはどうしたらいいでしょうか?
\IDA ではリストファイルを作成することができるため、各オペコードに4バイトを表示するオプションで作成されました。 
次に、16ビットのオペコードの大部分(\INS{IT} は \GTT{0xBF})を知っているので、\GTT{grep}を使って次のことを行います。

\begin{lstlisting}
cat AngryBirdsClassic.lst | grep " BF" | grep "IT" > results.lst
\end{lstlisting}

\myindex{ARM!\ThumbTwoMode}

ところで、ARMアセンブリ言語でThumb-2モードを手動でプログラムし、
条件付きサフィックスを追加すると、
アセンブラは必要な場所に必要なフラグを自動的に\INS{IT}命令に追加します。

\myparagraph{\NonOptimizingXcodeIV (\ARMMode)}

\begin{lstlisting}[caption=\NonOptimizingXcodeIV (\ARMMode),style=customasmARM]
b               = -0x20
a               = -0x18
val_to_return   = -0x10
saved_R7        = -4

                STR             R7, [SP,#saved_R7]!
                MOV             R7, SP
                SUB             SP, SP, #0x1C
                BIC             SP, SP, #7
                VMOV            D16, R2, R3
                VMOV            D17, R0, R1
                VSTR            D17, [SP,#0x20+a]
                VSTR            D16, [SP,#0x20+b]
                VLDR            D16, [SP,#0x20+a]
                VLDR            D17, [SP,#0x20+b]
                VCMPE.F64       D16, D17
                VMRS            APSR_nzcv, FPSCR
                BLE             loc_2E08
                VLDR            D16, [SP,#0x20+a]
                VSTR            D16, [SP,#0x20+val_to_return]
                B               loc_2E10

loc_2E08
                VLDR            D16, [SP,#0x20+b]
                VSTR            D16, [SP,#0x20+val_to_return]

loc_2E10
                VLDR            D16, [SP,#0x20+val_to_return]
                VMOV            R0, R1, D16
                MOV             SP, R7
                LDR             R7, [SP+0x20+b],#4
                BX              LR
\end{lstlisting}

既に見たのとほぼ同じですが、
$a$ と $b$ の変数がローカルスタックに格納され、戻り値も格納されるため、冗長コードがとても多くなっています。

\myparagraph{\OptimizingKeilVI (\ThumbMode)}

\begin{lstlisting}[caption=\OptimizingKeilVI (\ThumbMode),style=customasmARM]
                PUSH    {R3-R7,LR}
                MOVS    R4, R2
                MOVS    R5, R3
                MOVS    R6, R0
                MOVS    R7, R1
                BL      __aeabi_cdrcmple
                BCS     loc_1C0
                MOVS    R0, R6
                MOVS    R1, R7
                POP     {R3-R7,PC}

loc_1C0
                MOVS    R0, R4
                MOVS    R1, R5
                POP     {R3-R7,PC}
\end{lstlisting}

ターゲットCPUでサポートされるかに依存できず、また、簡単なビット単位の比較では実行できないため、
KeilはFPU命令を生成しません。
%TODO1: why?
したがって、外部ライブラリ関数を呼び出して比較を行います:\GTT{\_\_aeabi\_cdrcmple}
\myindex{ARM!\Instructions!BCS}

注意:比較の結果はこの関数によってフラグに残されます。したがって、
次の\INS{BCS}(\emph{Carry set---Greater than or equal})命令は
追加コードなしで動作します。
}

\EN{\subsubsection{ARM64}

\myparagraph{\Optimizing GCC (Linaro) 4.9}

\lstinputlisting[style=customasmARM]{patterns/12_FPU/3_comparison/ARM/ARM64_GCC_O3_EN.lst}

The ARM64 \ac{ISA} has FPU-instructions 
which set \ac{APSR} the CPU flags instead of \ac{FPSCR} for convenience.
The\ac{FPU} is not a separate device here anymore (at least, logically).
\myindex{ARM!\Instructions!FCMPE}
Here we see \INS{FCMPE}. It compares the two values passed in \RegD{0} and \RegD{1} (which are the first and second arguments of the function)
and sets \ac{APSR} flags (N, Z, C, V).

\myindex{ARM!\Instructions!FCSEL}
\INS{FCSEL} (\emph{Floating Conditional Select}) copies the value of \RegD{0} or \RegD{1} into \RegD{0} depending on the condition (\GTT{GT}---Greater Than),
and again, it uses flags in \ac{APSR} register instead of \ac{FPSCR}.

This is much more convenient, compared to the instruction set in older CPUs.

If the condition is true (\GTT{GT}), then the value of \RegD{0} 
is copied into \RegD{0} (i.e., nothing happens).
If the condition is not true, the value of \RegD{1} 
is copied into \RegD{0}.

\myparagraph{\NonOptimizing GCC (Linaro) 4.9}

\lstinputlisting[style=customasmARM]{patterns/12_FPU/3_comparison/ARM/ARM64_GCC_EN.lst}

Non-optimizing GCC is more verbose.

First, the function saves its input argument values in the local stack (\emph{Register Save Area}).
Then the code reloads these values into registers
\RegX{0}/\RegX{1} and finally copies them to 
\RegD{0}/\RegD{1} to be compared using \INS{FCMPE}. 
A lot of redundant code, 
but that is how non-optimizing compilers work.
\INS{FCMPE} compares the values and sets the \ac{APSR} flags.
At this moment, 
the compiler is not thinking yet about the more convenient \INS{FCSEL} instruction, so it proceed using old methods: 
using the \INS{BLE} instruction (\emph{Branch if Less than or Equal}).
In the first case ($a>b$), the value of $a$ gets loaded 
into \RegX{0}.
In the other case ($a<=b$), the value of $b$ gets loaded into 
\RegX{0}.
Finally, the value from \RegX{0} gets copied into \RegD{0}, 
because the return value needs to be in this 
register.

\mysubparagraph{\Exercise}

As an exercise, you can try optimizing this piece of code 
manually by removing redundant instructions and not introducing new ones (including \INS{FCSEL}).

\myparagraph{\Optimizing GCC (Linaro) 4.9---float}

Let's also rewrite this example to use \Tfloat instead of \Tdouble.

\begin{lstlisting}[style=customc]
float f_max (float a, float b)
{
	if (a>b)
		return a;

	return b;
};
\end{lstlisting}

\lstinputlisting[style=customasmARM]{patterns/12_FPU/3_comparison/ARM/ARM64_GCC_O3_float_EN.lst}

It is the same code, but the S-registers are used instead of D- ones.
It's because numbers of type \Tfloat are passed in 32-bit S-registers (which are in fact the lower parts of the 64-bit D-registers).

}
\RU{\subsubsection{ARM64}

\myparagraph{\Optimizing GCC (Linaro) 4.9}

\lstinputlisting[style=customasmARM]{patterns/12_FPU/3_comparison/ARM/ARM64_GCC_O3_RU.lst}

В ARM64 \ac{ISA} теперь есть FPU-инструкции, устанавливающие флаги CPU \ac{APSR} вместо \ac{FPSCR} для удобства.
\ac{FPU} больше не отдельное устройство (по крайней мере логически).
\myindex{ARM!\Instructions!FCMPE}
Это \INS{FCMPE}. Она сравнивает два значения, переданных в \RegD{0} и \RegD{1} 
(а это первый и второй аргументы функции) и выставляет флаги в \ac{APSR} (N, Z, C, V).

\myindex{ARM!\Instructions!FCSEL}
\INS{FCSEL} (\emph{Floating Conditional Select}) копирует значение \RegD{0} или
\RegD{1} в \RegD{0} в зависимости от условия 
(\GTT{GT} --- Greater Than --- больше чем),
и снова, она использует флаги в регистре \ac{APSR} вместо \ac{FPSCR}.
Это куда удобнее, если сравнивать с тем набором инструкций, что был в процессорах раньше.

Если условие верно (\GTT{GT}), тогда значение из \RegD{0} копируется в \RegD{0} (т.е. ничего не происходит).
Если условие не верно, то значение \RegD{1} копируется в \RegD{0}.

\myparagraph{\NonOptimizing GCC (Linaro) 4.9}

\lstinputlisting[style=customasmARM]{patterns/12_FPU/3_comparison/ARM/ARM64_GCC_RU.lst}

Неоптимизирующий GCC более многословен.
В начале функция сохраняет значения входных аргументов в локальном стеке (\emph{Register Save Area}).
Затем код перезагружает значения в регистры
\RegX{0}/\RegX{1} и наконец копирует их в 
\RegD{0}/\RegD{1} для сравнения инструкцией \INS{FCMPE}. 
Много избыточного кода, но так работают неоптимизирующие компиляторы.
\INS{FCMPE} сравнивает значения и устанавливает флаги в \ac{APSR}.
В этот момент компилятор ещё не думает о более удобной инструкции \INS{FCSEL}, так что он работает старым 
методом: 
использует инструкцию \INS{BLE} (\emph{Branch if Less than or Equal} (переход если меньше или равно)).
В одном случае ($a>b$) значение $a$ перезагружается в \RegX{0}.
В другом случае ($a<=b$) значение $b$ загружается в \RegX{0}.
Наконец, значение из \RegX{0} копируется в \RegD{0}, 
потому что возвращаемое значение оставляется в этом регистре.

\mysubparagraph{\Exercise}

Для упражнения вы можете попробовать оптимизировать этот фрагмент кода вручную, удалив избыточные инструкции,
но не добавляя новых (включая \INS{FCSEL}).

\myparagraph{\Optimizing GCC (Linaro) 4.9: float}

Перепишем пример. Теперь здесь \Tfloat вместо \Tdouble.

\begin{lstlisting}[style=customc]
float f_max (float a, float b)
{
	if (a>b)
		return a;

	return b;
};
\end{lstlisting}

\lstinputlisting[style=customasmARM]{patterns/12_FPU/3_comparison/ARM/ARM64_GCC_O3_float_RU.lst}

Всё то же самое, только используются S-регистры вместо D-.
Так что числа типа \Tfloat передаются в 32-битных S-регистрах (а это младшие части 64-битных D-регистров).

}
\DE{\subsubsection{ARM64}

\myparagraph{\Optimizing GCC (Linaro) 4.9}

\lstinputlisting[style=customasmARM]{patterns/12_FPU/3_comparison/ARM/ARM64_GCC_O3_DE.lst}
Der ARM64 \ac{ISA} verfügt über FPU-Befehle, die der Einfachheit halber die Flags der CPU \ac{APSR} anstelle von
\ac{FPSCR} setzen.
Die \ac{FPU} ist hier kein separates Gerät mehr (zumindest logisch).

\myindex{ARM!\Instructions!FCMPE}
Wir finden hier \INS{FCMPE}. Er vergleicht die beiden über \RegD{0} und \RegD{1} übergebenen Werte (dabei handelt es
sich um das erste und zweite Argument der Funktion) und setzt \ac{APSR} die Flags (N, Z, C, V).

\myindex{ARM!\Instructions!FCSEL}
\INS{FCSEL} (\emph{Floating Conditional Select}) kopiert den Wert von \RegD{0} oder \RegD{1} nach \RegD{0}, abhängig von
der Bedingung (\GTT{GT}---Greater Than), und verwendet wiederum Flags im \ac{APSR} Register anstatt derer von
\ac{FPSCR}.

Dies ist im Vergleich zum Befehlssatz alter CPUs deutlich praktischer.

Falls die Bedingung wahr ist (\GTT{GT}), dann wird der Wert von \RegD{0} nach \RegD{0} kopiert (d.h. es geschieht
nichts).
Falls die Bedingung falsch ist, wird der Wert von \RegD{1} nach \RegD{0} kopiert.

\myparagraph{\NonOptimizing GCC (Linaro) 4.9}

\lstinputlisting[style=customasmARM]{patterns/12_FPU/3_comparison/ARM/ARM64_GCC_DE.lst}
Der nicht optimierende GCC ist weniger kompakt.

Zunächst speichert die Funktion ihre Eingabewerte auf dem lokalen Stack (\emph{Register Save Area}), danach lädt der Code
die Werte erneut in die Register \RegX{0}/\RegX{1} und kopiert sie schließlich nach \RegD{0}/\RegD{1}, um sie mittels
\INS{FCMPE} zu vergleichen.
Eine Menge redundanter Code, aber so arbeitet ein nicht optimierender Compiler nun einmal.
\INS{FCMPE} vergleich die Werte und setzt die \ac{APSR} Flags.
Zu diesem Zeitpunkt entscheidet sich der Compiler noch nicht für den praktischeren \INS{FCSEL} Befehl und arbeitet
stattdessen mit herkömmlichen Methoden:
er verwendet den \INS{BLE} Befehl (\emph{Branch if Less than or Equal}).
Im ersten Fall ($a>b$) wird der Wert von $a$ nach \RegX{0} geladen. 
Im anderen Fall ($a<=b$) wird der Wert von $b$ nach \RegX{0} geladen.
Schließlich wird der Wert aus \RegX{0} nach \RegD{0} kopiert, denn der Rückgabewert muss sich in diesem Register
befinden.


\mysubparagraph{\Exercise}
Dem Leser bleibt als Übung, den vorstehenden Code zu optimieren, indem manuell die redundanten Instruktionen entfernt
werden ohne dabei neue einzuführen (wie etwa \INS{FCSEL}).

\myparagraph{\Optimizing GCC (Linaro) 4.9---float}
Wir wollen nun dieses Beispiel umschreiben, indem wir \Tfloat anstelle von \Tdouble verwenden.

\begin{lstlisting}[style=customc]
float f_max (float a, float b)
{
	if (a>b)
		return a;

	return b;
};
\end{lstlisting}

\lstinputlisting[style=customasmARM]{patterns/12_FPU/3_comparison/ARM/ARM64_GCC_O3_float_DE.lst}
Es ist der gleiche Code, aber hier werden die S-Register anstelle der D-Register verwendet.
Das ist darauf zurückzuführen, dass der \Tfloat Typ in 32-Bit-S-Registern übergeben wird (welche in Wirklichkeit nichts
anderes als die niederen Teile der 64-Bit-D-Register sind).
}
\FR{\subsubsection{ARM64}

\myparagraph{GCC (Linaro) 4.9 \Optimizing}

\lstinputlisting[style=customasmARM]{patterns/12_FPU/3_comparison/ARM/ARM64_GCC_O3_FR.lst}

L'ARM64 \ac{ISA} possède des instructions FPU qui mettent les flags CPU \ac{APSR}
au lieu de \ac{FPSCR}, par commodité.
Le \ac{FPU} n'est plus un device séparé (au moins, logiquement).
\myindex{ARM!\Instructions!FCMPE}
Ici, nous voyons \INS{FCMPE}. Ceci compare les deux valeurs passées dans \RegD{0}
et \RegD{1} (qui sont le premier et le second argument de la fonction) et met les
flags \ac{APSR} (N, Z, C, V).

\myindex{ARM!\Instructions!FCSEL}
\INS{FCSEL} (\emph{Floating Conditional Select} (sélection de flottant conditionnelle)
copie la valeur de \RegD{0} ou \RegD{1} dans \RegD{0} suivant le résultat de la comparaison
(\GTT{GT}---Greater Than), et de nouveau, il utilise les flags dans le registre \ac{APSR}
au lieu de \ac{FPSCR}.

Ceci est bien plus pratique, comparé au jeu d'instructions des anciens CPUs.

Si la condition est vraie (\GTT{GT}), alors la valeur de \RegD{0} est copiée dans
\RegD{0} (i.e., il ne se passe rien).
Si la condition n'est pas vraie, la valeur de \RegD{1} est copiée dans \RegD{0}.

\myparagraph{GCC (Linaro) 4.9 \NonOptimizing}

\lstinputlisting[style=customasmARM]{patterns/12_FPU/3_comparison/ARM/ARM64_GCC_FR.lst}

GCC sans optimisation est plus verbeux.

Tout d'abord, la fonction sauve la valeur de ses arguments en entrée dans la pile
locale (\emph{Register Save Area}, espace de sauvegarde des registres).
Ensuite, le code recharge ces valeurs dans les registres \RegX{0}/\RegX{1} et finalement
les copie dans \RegD{0}/\RegD{1} afin de les comparer en utilisant \INS{FCMPE}.
Beaucoup de code redondant, mais c'est ainsi que fonctionne les compilateurs sans
optimisation.
\INS{FCMPE} compare les valeurs et met les flags du registre \ac{APSR}.
À ce moment, le compilateur ne pense pas encore à l'instruction plus commode \INS{FCSEL},
donc il procède en utilisant de vieilles méthodes:
en utilisant l'instruction \INS{BLE} (\emph{Branch if Less than or Equal} branchement si
inférieur ou égal).
Dans le premier cas ($a>b$), la valeur de $a$ est chargée dans \RegX{0}.
Dans les autres cas ($a<=b$), la valeur de $b$ est chargée dans \RegX{0}.
Enfin, la valeur dans \RegX{0} est copiée dans \RegD{0}, car la valeur de retour
doit être dans ce registre.

\mysubparagraph{\Exercise}

À titre d'exercice, vous pouvez essayer d'optimiser ce morceau de code manuellement
en supprimant les instructions redondantes et sans en introduire de nouvelles (incluant
\INS{FCSEL}).

\myparagraph{GCC (Linaro) 4.9 \Optimizing---float}

Ré-écrivons cet exemple en utilisant des \Tfloat à la place de \Tdouble.

\begin{lstlisting}[style=customc]
float f_max (float a, float b)
{
	if (a>b)
		return a;

	return b;
};
\end{lstlisting}

\lstinputlisting[style=customasmARM]{patterns/12_FPU/3_comparison/ARM/ARM64_GCC_O3_float_FR.lst}

C'est le même code, mais des S-registres sont utilisés à la place de D-registres.
C'est parce que les nombres de type \Tfloat sont passés dans des S-registres de 32-bit
(qui sont en fait la partie basse des D-registres 64-bit).

}
\JA{\subsubsection{ARM64}

\myparagraph{\Optimizing GCC (Linaro) 4.9}

\lstinputlisting[style=customasmARM]{patterns/12_FPU/3_comparison/ARM/ARM64_GCC_O3_JA.lst}

ARM64 \ac{ISA}には、便宜上、
\ac{FPSCR}の代わりにCPUフラグを\ac{APSR}に設定するFPU命令があります。
\ac{FPU}はもはや別個のデバイスではありません。(少なくとも論理的には)
\myindex{ARM!\Instructions!FCMPE}
ここでは\INS{FCMPE}を参照してください。 \RegD{0}と\RegD{1}(関数の第1引数と第2引数)で渡された2つの値を比較し、
\ac{APSR}フラグ(N, Z, C, V)を設定します。

\myindex{ARM!\Instructions!FCSEL}
\INS{FCSEL}(\emph{Floating Conditional Select})は、条件(\GTT{GT}---Greater Than)に応じて\RegD{0}または\RegD{1}の値を\RegD{0}にコピーし、
再び\ac{FPSCR}の代わりに\ac{APSR}レジスタのフラグを使用します。

これは、古いCPUの命令セットに比べてはるかに便利です。

条件が真(\GTT{GT})の場合、\RegD{0}の値が
\RegD{0}にコピーされます。(つまり何も起こりません)
条件が真でない場合、\RegD{1}の値が\RegD{0}にコピーされます。

\myparagraph{\NonOptimizing GCC (Linaro) 4.9}

\lstinputlisting[style=customasmARM]{patterns/12_FPU/3_comparison/ARM/ARM64_GCC_JA.lst}

非最適化GCCはより冗長です。

まず、関数は入力引数の値をローカルスタック(\emph{Register Save Area})に保存します。
次に、これらの値をレジスタ
\RegD{0}/\RegD{1}にリロードし、
最終的に\RegX{0}/\RegX{1}にコピーして\INS{FCMPE}を使用して比較します。
冗長なコードがたくさんありますが、
最適化されていないコンパイラの仕組みです。 
\INS{FCMPE}は値を比較し、\ac{APSR}フラグを設定します。
現時点では、コンパイラは、より便利な\INS{FCSEL}命令についてはまだ考えていないため、古いメソッドを使用して処理を進めます。
つまり、\INS{BLE}命令を使用します(\emph{Branch if Less than or Equal})。
最初のケース($a>b$)では、 $a$ の値が\RegX{0}にロードされます。
それ以外の場合($a<=b$)、 $b$ の値は\RegX{0}にロードされます。
最後に、戻り値がこのレジスタにある必要があるため、
\RegX{0}からの値が\RegD{0}にコピーされます。

\mysubparagraph{\Exercise}

練習として、冗長な命令を削除し、
新しい命令(\INS{FCSEL}を含む)を導入しないで手動でこのコードを最適化することができます。

\myparagraph{\Optimizing GCC (Linaro) 4.9---float}

\Tdouble の代わりに \Tfloat を使うようにこの例を書き直しましょう。

\begin{lstlisting}[style=customc]
float f_max (float a, float b)
{
	if (a>b)
		return a;

	return b;
};
\end{lstlisting}

\lstinputlisting[style=customasmARM]{patterns/12_FPU/3_comparison/ARM/ARM64_GCC_O3_float_JA.lst}

これは同じコードですが、D-レジスタの代わりにS-レジスタが使用されています。 
これは、浮動小数点数が32ビットSレジスタ(実際には64ビットDレジスタの下位部分)に渡されるためです。
}

\EN{\subsubsection{MIPS}
% FIXME better start at non-optimizing version?

The function uses a lot of S- registers which must be preserved, so that's why its 
values are saved in the function prologue and restored in the epilogue.

\lstinputlisting[caption=\Optimizing GCC 4.4.5 (IDA),style=customasmMIPS]{patterns/13_arrays/1_simple/MIPS_O3_IDA_EN.lst}

Something interesting: there are two loops and the first one doesn't need $i$, it needs only 
$i*2$ (increased by 2 at each iteration) and also the address in memory (increased by 4 at each iteration).

So here we see two variables, one (in \$V0) increasing by 2 each time, and another (in \$V1) --- by 4.

The second loop is where \printf is called and it reports the value of $i$ to the user, 
so there is a variable
which is increased by 1 each time (in \$S0) and also a memory address (in \$S1) increased by 4 each time.

That reminds us of loop optimizations: \myref{loop_iterators}.

Their goal is to get rid of multiplications.

}
\RU{\subsubsection{MIPS}
% FIXME better start at non-optimizing version?
Функция использует много S-регистров, которые должны быть сохранены. Вот почему их значения сохраняются
в прологе функции и восстанавливаются в эпилоге.

\lstinputlisting[caption=\Optimizing GCC 4.4.5 (IDA),style=customasmMIPS]{patterns/13_arrays/1_simple/MIPS_O3_IDA_RU.lst}

Интересная вещь: здесь два цикла и в первом не нужна переменная $i$, а нужна только переменная
$i*2$ (скачущая через 2 на каждой итерации) и ещё адрес в памяти (скачущий через 4 на каждой итерации).

Так что мы видим здесь две переменных: одна (в \$V0) увеличивается на 2 каждый раз, и вторая (в \$V1) --- на 4.

Второй цикл содержит вызов \printf. Он должен показывать значение $i$ пользователю,
поэтому здесь есть переменная, увеличивающаяся на 1 каждый раз (в \$S0), а также адрес в памяти (в \$S1) 
увеличивающийся на 4 каждый раз.

Это напоминает нам оптимизацию циклов: \myref{loop_iterators}.
Цель оптимизации в том, чтобы избавиться от операций умножения.

}
\DE{\subsubsection{MIPS}

\myindex{MIPS!\Registers!FCCR}
Der Koprozessor des MIPS Prozessors hat ein Condition Bit, welches in der FPU
gesetzt und in der CPU geprüft werden kann.

Frühere MIPS haben nur ein Condition Bit (genannt FCC0), spätere haben deren 8
(genannt FCC7-FCC0). 

Diese(s) Bit(s) befinden sich im Register FCCR.

\lstinputlisting[caption=\Optimizing GCC 4.4.5 (IDA),style=customasmMIPS]{patterns/12_FPU/3_comparison/MIPS_O3_IDA_DE.lst}

\myindex{MIPS!\Instructions!C.LT.D}
\INS{C.LT.D} vergleicht zwei Werte. 
\GTT{LT} ist die Bedingung \q{Less Than} (weniger als).
\GTT{D} impliziert einen Wert vom Typ \Tdouble.
Abhängig vom Ergebnis des Vergleichs wird das FCC0 Condition Bit entweder
gesetzt oder gelöscht.

\myindex{MIPS!\Instructions!BC1T}
\myindex{MIPS!\Instructions!BC1F}
\INS{BC1T} prüft das FCC0 Bit und sprint, falls das Bit gesetzt ist.
\GTT{T} bedeutet, dass der Sprung ausgeführt wird, wenn das Bit gesetzt
(\q{True}) ist.
Daneben gibt es auch den Befehl \INS{BC1F}, der springt, wenn das Bit gelöscht
(\q{FALSE}) ist.

Abhängig vom Sprung wird einer der Funktionsargument in \$F0 abgelegt.
}
\FR{\subsubsection{MIPS}

\myindex{MIPS!\Registers!FCCR}
Le coprocesseur du processeur MIPS possède un bit de condition qui peut être mis
par le FPU et lu par le CPU.

Les premiers MIPSs avaient seulement un bit de condition (appelé FCC0), les derniers
en ont 8 (appelés FCC7-FCC0).

Ce bit (ou ces bits) sont situés dans un registre appelé FCCR.

\lstinputlisting[caption=\Optimizing GCC 4.4.5 (IDA),style=customasmMIPS]{patterns/12_FPU/3_comparison/MIPS_O3_IDA_FR.lst}

\myindex{MIPS!\Instructions!C.LT.D}
\INS{C.LT.D} compare deux valeurs.
\GTT{LT} est la condition \q{Less Than} (plus petit que).
\GTT{D} implique des valeurs de type \Tdouble.
Suivant le résultat de la comparaison, le bit de condition FCC0 est mis à 1 ou à
0.

\myindex{MIPS!\Instructions!BC1T}
\myindex{MIPS!\Instructions!BC1F}
\INS{BC1T} teste le bit FCC0 et saute si le bit est mis à 1.
\GTT{T} signifie que le saut sera effectué si le bit est mis à 1 (\q{True}).
Il y a aussi une instruction \INS{BC1F} qui saute si le bit n'est pas mis (donc
est à 0) (\q{False}).

Dépendant du saut, un des arguments de la fonction est placé dans \$F0.
}
\JA{\subsubsection{MIPS}

\myindex{MIPS!\Registers!FCCR}
MIPSプロセッサのコプロセッサには条件ビットがあり、これをFPUにセットしてCPUでチェックすることができます。

以前のMIPSには1つの条件ビット(FCC0と呼びます)があり、後のモデルには8つのビット(FCC7-FCC0と呼びます)があります。

このビット(または複数のビット)はFCCRと呼ばれるレジスタに配置されています。

\lstinputlisting[caption=\Optimizing GCC 4.4.5 (IDA),style=customasmMIPS]{patterns/12_FPU/3_comparison/MIPS_O3_IDA_JA.lst}

\myindex{MIPS!\Instructions!C.LT.D}
\INS{C.LT.D}は2つの値を比較します。 
\GTT{LT}は\q{Less Than}の条件です。
\GTT{D}は \Tdouble 型の値を意味します。 
比較の結果に応じて、FCC0条件ビットはセットまたはクリアされます。

\myindex{MIPS!\Instructions!BC1T}
\myindex{MIPS!\Instructions!BC1F}
\INS{BC1T} checks the FCC0 bit and jumps if the bit is set.
\GTT{T} means that the jump is to be taken if the bit is set (\q{True}).
There is also the instruction \INS{BC1F} which jumps if the bit is cleared (\q{False}).

\INS{BC1T}はFCC0ビットをチェックし、ビットがセットされていればジャンプします。 
\GTT{T}は、ビットがセット(\q{True})されている場合にジャンプが行われることを意味します。 
ビットがクリアされるとジャンプする\INS{BC1F}命令もあります。(\q{False})

ジャンプに応じて、関数引数の1つが \$F0 に配置されます。
}


\subsection{Some constants}

It's easy to find representations of some constants in Wikipedia for IEEE 754 encoded numbers.
It's interesting to know that 0.0 in IEEE 754 is represented as 32 zero bits (for single precision) or 64 zero bits
(for double).
So in order to set a floating point variable to 0.0 in register or memory, one can use \MOV or \TT{XOR reg, reg} instruction.
\myindex{\CStandardLibrary!memset()}
This is suitable for structures where many variables present of various data types.
With usual memset() function one can set all integer variables to 0, all boolean variables to \emph{false}, all pointers
to NULL, and all floating point variables (of any precision) to 0.0.

\subsection{Copying}

One may think inertially that \INS{FLD}/\INS{FST} instructions must be used to load and store (and hence, copy) IEEE 754 values.
Nevertheless, same can be achieved easier by usual \INS{MOV} instruction, which, of course, copies values bitwisely.

\subsection{Stack, calculators and reverse Polish notation}

\myindex{Reverse Polish notation}

Now we understand why some old calculators use reverse Polish notation
\footnote{\href{http://go.yurichev.com/17354}{wikipedia.org/wiki/Reverse\_Polish\_notation}}.

For example, for addition of 12 and 34 one has to enter 12, then 34, then press \q{plus} sign.

It's because old calculators were just stack machine implementations, and this was much simpler
than to handle complex parenthesized expressions.

\subsection{80 bits?}

\myindex{Punched card}
Internal numbers representation in FPU --- 80-bit.
Strange number, because the number not in $2^n$ form.
There is a hypothesis that this is probably due to historical reasons---the standard IBM puched card can encode 12 rows of 80 bits.
$80\cdot 25$ text mode resolution was also popular in past.

Wikipedia has another explanation: \url{https://en.wikipedia.org/wiki/Extended_precision}.

If you know better, please a drop email to the author: \EMAIL{}.

\subsection{x64}

On how floating point numbers are processed in x86-64, read more here: \myref{floating_SIMD}.

% sections
\input{patterns/12_FPU/exercises}
