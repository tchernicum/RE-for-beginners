\myparagraph{GCC 4.4.1 \Optimizing}

\lstinputlisting[caption=GCC 4.4.1 \Optimizing,style=customasmx86]{patterns/12_FPU/3_comparison/x86/GCC_O3_FR.asm}

\myindex{x86!\Instructions!JA}

C'est presque le même, à l'exception que \JA est utilisé après \SAHF.
En fait, les instructions de sauts conditionnels qui vérifient \q{plus}, \q{moins} ou \q{égal} pour
les comparaisons de nombres non signés (ce sont \INS{JA}, \JAE, \JB, \JBE, \JE/\JZ, \JNA,
\JNAE, \JNB, \JNBE, \JNE/\JNZ) vérifient seulement les flags \CF et \ZF.\\
\\
Rappelons comment les bits \CThreeBits sont situés dans le registre \GTT{AH} après
l'exécution de \INS{FSTSW}/\FNSTSW:

\input{C3_in_AH}

Rappelons également, comment les bits de \GTT{AH} sont stockés dans les flags du
CPU après l'exécution de \SAHF:

\input{SAHF_LAHF}

Après la comparaison, les bits \Cthree et \Czero sont copiés dans \ZF et \CF, donc
les sauts conditionnels peuvent fonctionner après. \JA est déclenché si \CF et \ZF
sont tout les deux à zéro.

Ainsi, les instructions de saut conditionnel listées ici peuvent être utilisées après
une paire d'instructions \FNSTSW/\SAHF.

Apparemment, les bits d'état du FPU \CThreeBits ont été mis ici intentionnellement,
pour facilement les relier aux flags du CPU de base sans permutations supplémentaires?

