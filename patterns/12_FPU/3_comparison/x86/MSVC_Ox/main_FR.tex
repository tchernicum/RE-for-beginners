\myparagraph{MSVC 2010 \Optimizing}

\lstinputlisting[caption=MSVC 2010 \Optimizing,style=customasmx86]{patterns/12_FPU/3_comparison/x86/MSVC_Ox/MSVC_FR.asm}

\myindex{x86!\Instructions!FCOM}

\FCOM diffère de \FCOMP dans le sens où il compare seulement les deux valeurs, et
ne change pas la pile du FPU.
Contrairement à l'exemple précédent, ici les opérandes sont dans l'ordre inverse,
c'est pourquoi le résultat de la comparaison dans \CThreeBits est différent.

\begin{itemize}
\item si $a>b$ dans notre exemple, alors les bits \CThreeBits sont mis comme suit: 0, 0, 0.
\item si $b>a$, alors les bits sont: 0, 0, 1.
\item si $a=b$, alors les bits sont: 1, 0, 0.
\end{itemize}
% TODO: table?

L'instruction \INS{test ah, 65} laisse seulement deux bits~---\Cthree et \Czero.
Les deux seront à zéro si $a>b$: dans ce cas le saut \JNE ne sera pas effectué.
Puis \INS{FSTP ST(1)} suit~---cette instruction copie la valeur de \ST{0} dans l'opérande
et supprime une valeur de la pile du FPU.
En d'autres mots, l'instruction copie \ST{0} (où la valeur de \GTT{\_a} se trouve)
dans \ST{1}.
Après cela, deux copies de {\_a} sont sur le sommet de la pile.
Puis, une valeur est supprimée.
Après cela, \ST{0} contient {\_a} et la fonction se termine.

Le saut conditionnel \JNE est effectué dans deux cas: si $b>a$ ou $a=b$.
\ST{0} est copié dans \ST{0}, c'est comme une opération sans effet (\ac{NOP}), puis
une valeur est supprimée de la pile et le sommet de la pile (\ST{0}) contient la
valeur qui était avant dans \ST{1} (qui est {\_b}).
Puis la fonction se termine.
La raison pour laquelle cette instruction est utilisée ici est sans doute que le
\ac{FPU} n'a pas d'autre instruction pour prendre une valeur sur la pile et la
supprimer.

\clearpage
\mysubparagraph{Premier exemple sous \olly: a=1.2 et b=3.4}
\myindex{\olly}

Les deux instructions \FLD ont été exécutées:

\begin{figure}[H]
\centering
\myincludegraphics{patterns/12_FPU/3_comparison/x86/MSVC_Ox/olly1_1.png}
\caption{\olly: les deux \FLD exécutées}
\label{fig:FPU_comparison_Ox_case1_olly1}
\end{figure}

\FCOM exécutée:
\olly montre le contenu de \ST{0} et \ST{1} par commodité.

\clearpage
\FCOM a été exécutée:

\begin{figure}[H]
\centering
\myincludegraphics{patterns/12_FPU/3_comparison/x86/MSVC_Ox/olly1_2.png}
\caption{\olly: \FCOM a été exécutée}
\label{fig:FPU_comparison_Ox_case1_olly2}
\end{figure}

\Czero est mis, tous les autres flags de condition sont à zéro.

\clearpage
\FNSTSW a été exécutée, \GTT{AX}=0x3100:

\begin{figure}[H]
\centering
\myincludegraphics{patterns/12_FPU/3_comparison/x86/MSVC_Ox/olly1_3.png}
\caption{\olly: \FNSTSW est exécutée}
\label{fig:FPU_comparison_Ox_case1_olly3}
\end{figure}

\clearpage
\TEST est exécutée:

\begin{figure}[H]
\centering
\myincludegraphics{patterns/12_FPU/3_comparison/x86/MSVC_Ox/olly1_4.png}
\caption{\olly: \TEST est exécutée}
\label{fig:FPU_comparison_Ox_case1_olly4}
\end{figure}

ZF=0, le saut conditionnel va être déclenché maintenant.

\clearpage
\INS{FSTP ST} (ou \FSTP \ST{0}) a été exécuté~---1.2 a été dépilé, et 3.4 laissé
au sommet de la pile:

\begin{figure}[H]
\centering
\myincludegraphics{patterns/12_FPU/3_comparison/x86/MSVC_Ox/olly1_5.png}
\caption{\olly: \FSTP est exécutée}
\label{fig:FPU_comparison_Ox_case1_olly5}
\end{figure}

Nous voyons que l'instruction \INS{FSTP ST}

fonctionne comme dépiler une valeur de la pile du FPU.

\clearpage
\mysubparagraph{Second exemple sous \olly: a=5.6 et b=-4}

Les deux \FLD sont exécutées:

\begin{figure}[H]
\centering
\myincludegraphics{patterns/12_FPU/3_comparison/x86/MSVC_Ox/olly2_1.png}
\caption{\olly: les deux \FLD sont exécutée}
\label{fig:FPU_comparison_Ox_case2_olly1}
\end{figure}

\FCOM est sur le point de s'exécuter.

\clearpage
\FCOM a été exécutée:

\begin{figure}[H]
\centering
\myincludegraphics{patterns/12_FPU/3_comparison/x86/MSVC_Ox/olly2_2.png}
\caption{\olly: \FCOM est terminé}
\label{fig:FPU_comparison_Ox_case2_olly2}
\end{figure}

Tous les flags de conditions sont à zéro.

\clearpage
\FNSTSW fait, \GTT{AX}=0x3000:

\begin{figure}[H]
\centering
\myincludegraphics{patterns/12_FPU/3_comparison/x86/MSVC_Ox/olly2_3.png}
\caption{\olly: \FNSTSW a été exécutée}
\label{fig:FPU_comparison_Ox_case2_olly3}
\end{figure}

\clearpage
\TEST a été exécutée:

\begin{figure}[H]
\centering
\myincludegraphics{patterns/12_FPU/3_comparison/x86/MSVC_Ox/olly2_4.png}
\caption{\olly: \TEST a été exécutée}
\label{fig:FPU_comparison_Ox_case2_olly4}
\end{figure}

ZF=1, le saut ne va pas se produire maintenant.

\clearpage
\FSTP \ST{1} a été exécutée: une valeur de 5.6 est maintenant au sommet de la pile
du FPU.

\begin{figure}[H]
\centering
\myincludegraphics{patterns/12_FPU/3_comparison/x86/MSVC_Ox/olly2_5.png}
\caption{\olly: \FSTP a été exécutée}
\label{fig:FPU_comparison_Ox_case2_olly5}
\end{figure}

Nous voyons maintenant que l'instruction \FSTP \ST{1}
fonctionne comme suit: elle laisse ce qui était au sommet de la pile, mais met \ST{1}
à zéro.


