\subsection{Krótkie wprowadzenie do CPU}

\ac{CPU} (mikroprocesor) jest urządzeniem, które wykonuje bezpośrednio kod maszynowy programu.

\textbf{Terminologia:}

\begin{description}
\item[Instrukcja]: prymitywny rozkaz \ac{CPU}.
Najprostsze przykłady: przenoszenie (kopiowanie) danych między rejestrami, zarządzanie pamięcią, proste operacje arytmetyczne. 

Z reguły, każdy \ac{CPU} ma swój zestaw instrukcji (\ac{ISA}).

\item[Kod maszynowy]: kod wykonywany bezpośrednio przez \ac{CPU}. 
Każda instrukcja kodu maszynowego zwykle jest kodowana za pomocą kilku bajtów.
\item[Język asemblera]: kod maszynowy plus niektóre rozszerzenia (np. makra), stworzone po to, żeby ułatwić pracę programiście.
\item[Rejestr CPU]: Każdy \ac{CPU} ma swój zestaw rejestrów ogólnego przeznaczenia (\ac{GPR}).
$\approx 8$ w x86, $\approx 16$ w x86-64, $\approx 16$ w ARM.
Najłatwiej traktować rejestr jako zmienną bez określonego typu - wyobraź sobie, że pisząc w języku wyższego poziomu masz dostępne tylko 8 zmiennych o szerokości 32 (lub 64) bity. Te 8 zmiennych to właśnie rejestry. Wbrew pozorom można z nimi naprawdę wiele zrobić!
\end{description}

Dlaczego występują języki niższego i wyższego poziomu? Odpowiedź jest prosta: ludzie i procesory różnią się między sobą - dla człowieka jest o wiele łatwiej pisać w wysokopoziomowym języku programowania typu \CCpp, Java, Python, а dla procesora łatwiej jest pracować na niższym poziomie abstrakcji, ponieważ "rozumie" tylko kilkubajtowe instrukcje.
Możnaby zbudować procesor, który wykonywałby kod wysokiego poziomu, ale jego budowa byłaby dużo bardziej skomplikowana niż  budowa procesorów jakie obecnie znamy. 
\myindex{ARM!\ARMMode}%
\myindex{ARM!\ThumbMode}%
\myindex{ARM!\ThumbTwoMode}%

\subsubsection{Kilka słów o różnicy między \ac{ISA}}
x86 od zawsze  zawierał instrukcje o różnej długości, więc kiedy nadeszła era 64-bitowej architektury, rozszerzenia x64 nie wpłynęły znacząco na \ac{ISA}. 
ARM to \ac{RISC}- procesor zaprojektowany tak, żeby zawierał wszystkie instrukcje tej samej długości, co miało sporo zalet w przeszłosci. Na samym początku wszystkie instrukcje ARM były kodowane na czterech bajtach (obecnie \q{tryb ARM})%
\footnote{
Instrukcje o stałym rozmiarze są wygodne, bo dzięki temu można łatwo znaleźć adres następnej (lub poprzedniej) instrukcji. Dokładniejsze wyjaśnienie znajduje się w sekcji o operatorze switch()~(\myref{sec:SwitchARMLot}).
}.
Później się okazało, że jednak nie jest to zbyt wydajne, bo najczęściej używane przez procesor instrukcje\footnote{Są nimi MOV/PUSH/CALL/Jcc} mogą być zakodowane z wykorzystaniem mniejszej ilości informacji. Więc dodano inną \ac{ISA} o nazwie "Thumb", gdzie każda instrukcja jest kodowana za pomocą tylko 2 bajtów. Obecnie ta metoda jest nazywana \q{tryb Thumb}.
Jednak nie wszystkie instrukcje ARM mogą być zakodowane na 2 bajtach, także zestaw instrukcji Thumb jest ograniczony. Kod skompilowany dla trybu ARM i Thumb może współdziałać w jednym programie. Później producenci ARM stwierdzili, że Thumb można poszerzyć: tak się pojawił Thumb-2 (w ARMv7). Thumb-2 to wciąż dwubajtowe instrukcje, ale niektóre nowe instrukcje mają długość 4 bajtów. Szeroko rozpowszechnioną i błędną opinią jest to, że Thumb-2 to mieszanina ARM i Thumb. Tryb Thumb-2 był dopełniony, żeby lepiej wykorzystywać możliwości procesora i jest poważnym konkurentem trybu ARM. Aplikacje dla \idevices są głównie skompilowane dla zestawu instrukcji Thumb-2, ponieważ Xcode jest domyślnie skonfigurowane dla tego zestawu instrukcji. Później pojawił się 64-bitowy ARM. Jest to \ac{ISA} znowu z 4-bajtowymi instrukcjami, bez dodatkowego trybu Thumb. Jednak nowe 64-bitowe wymagania wpłyneły na \ac{ISA} tak, że obecnie mamy 3 zestawy instrukcji ARM: tryb ARM, tryb Thumb/Thumb-2 i ARM64. Te zestawy instrukcji gdzieniegdzie są podobne, ale można powiedzieć, że są to różne zestawy. W tej książce postaramy się zaprezentować fragmenty kodu we wszystkich trzech ARM  \ac{ISA}.
\myindex{PowerPC}%
\myindex{MIPS}%
\myindex{Alpha AXP}%
Istnieje jeszcze wiele \ac{RISC} \ac{ISA} z instrukcjami 32-bitowej długości~--- np. MIPS, PowerPC и Alpha AXP.

