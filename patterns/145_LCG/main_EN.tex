\mysection[Linear congruential generator]{Linear congruential generator as pseudorandom number generator}
\myindex{\CStandardLibrary!rand()}
\label{LCG_simple}

Perhaps, the linear congruential generator is the simplest possible way to generate random numbers.

It's not in favour nowadays\footnote{Mersenne twister is better}, but it's so simple 
(just one multiplication, one addition and AND operation), 
that we can use it as an example.

\lstinputlisting[style=customc]{patterns/145_LCG/rand_EN.c}

There are two functions: the first one is used to initialize the internal state, and the second one is called
to generate pseudorandom numbers.

We see that two constants are used in the algorithm.
They are taken from
[William H. Press and Saul A. Teukolsky and William T. Vetterling and Brian P. Flannery, \emph{Numerical Recipes}, (2007)].

Let's define them using a \TT{\#define} \CCpp statement. It's a macro.

The difference between a \CCpp macro and a constant is that all macros are replaced 
with their value by \CCpp preprocessor,
and they don't take any memory, unlike variables.

In contrast, a constant is a read-only variable.

It's possible to take a pointer (or address) of a constant variable, but impossible to do so with a macro.

The last AND operation is needed because by C-standard \TT{my\_rand()} has to return a value in 
the 0..32767 range.

If you want to get 32-bit pseudorandom values, just omit the last AND operation.

\subsection{x86}

\lstinputlisting[caption=\Optimizing MSVC 2013,style=customasmx86]{patterns/145_LCG/rand_MSVC_2013_x86_Ox.asm}

Here we see it: both constants are embedded into the code.
There is no memory allocated for them.

The \TT{my\_srand()} function just copies its input value into the internal\\
\TT{rand\_state} variable.

\TT{my\_rand()} takes it, calculates the next \TT{rand\_state}, cuts it and leaves it in the EAX register.

The non-optimized version is more verbose:

\lstinputlisting[caption=\NonOptimizing MSVC 2013,style=customasmx86]{patterns/145_LCG/rand_MSVC_2013_x86.asm}

\subsection{x64}

The x64 version is mostly the same and uses 32-bit registers instead of 64-bit ones 
(because we are working with \Tint values here).

But \TT{my\_srand()} takes its input argument from the \ECX register rather than from stack:

\lstinputlisting[caption=\Optimizing MSVC 2013 x64,style=customasmx86]{patterns/145_LCG/rand_MSVC_2013_x64_Ox_EN.asm}

GCC compiler generates mostly the same code.

\subsection{32-bit ARM}

\lstinputlisting[caption=\OptimizingKeilVI (\ARMMode),style=customasmARM]{patterns/145_LCG/rand.s_Keil_ARM_O3_EN.s}

It's not possible to embed 32-bit constants into ARM instructions, so Keil has to place them externally and load them additionally.
One interesting thing is that it's not possible to embed the 0x7FFF constant as well.
So what Keil does is shifting \TT{rand\_state} left by 17 bits and then shifting it right by 17 bits.
This is analogous to the $(rand\_state \ll 17) \gg 17$ statement in \CCpp.
It seems to be useless operation, but what it does is clearing the high 17 bits, leaving the low 15 bits intact, and that's our goal after all. \\
\\
\Optimizing Keil for Thumb mode generates mostly the same code.

\subsubsection{MIPS}
% FIXME better start at non-optimizing version?

The function uses a lot of S- registers which must be preserved, so that's why its 
values are saved in the function prologue and restored in the epilogue.

\lstinputlisting[caption=\Optimizing GCC 4.4.5 (IDA),style=customasmMIPS]{patterns/13_arrays/1_simple/MIPS_O3_IDA_EN.lst}

Something interesting: there are two loops and the first one doesn't need $i$, it needs only 
$i*2$ (increased by 2 at each iteration) and also the address in memory (increased by 4 at each iteration).

So here we see two variables, one (in \$V0) increasing by 2 each time, and another (in \$V1) --- by 4.

The second loop is where \printf is called and it reports the value of $i$ to the user, 
so there is a variable
which is increased by 1 each time (in \$S0) and also a memory address (in \$S1) increased by 4 each time.

That reminds us of loop optimizations: \myref{loop_iterators}.

Their goal is to get rid of multiplications.



\subsection{Thread-safe version of the example}

The thread-safe version of the example is to be demonstrated later: \myref{LCG_TLS}.
