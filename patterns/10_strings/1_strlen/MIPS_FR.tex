\subsubsection{MIPS}

\lstinputlisting[caption=\Optimizing GCC 4.4.5 (IDA),style=customasmMIPS]{patterns/10_strings/1_strlen/MIPS_O3_IDA_FR.lst}

\myindex{MIPS!\Instructions!NOR}
\myindex{MIPS!\Pseudoinstructions!NOT}

Il manque en MIPS une instruction \NOT, mais il y a \NOR qui correspond à l'opération
\TT{OR~+~NOT}.

Cette opération est largement utilisée en électronique digitale\footnote{NOR est
appelé \q{porte universelle}}.
Par exemple, l'Apollo Guidance Computer (ordinateur de guidage Apollo) utilisé dans
le programme Apollo, a été construit en utilisant seulement 5600 portes NOR:
[Jens Eickhoff, \emph{Onboard Computers, Onboard Software and Satellite Operations: An Introduction}, (2011)].
Mais l'élément NOT n'est pas très populaire en programmation informatique.

Donc, l'opération NOT est implémentée ici avec \TT{NOR~DST,~\$ZERO,~SRC}.

D'après le chapitre sur les fondamentaux \myref{sec:signednumbers} nous savons qu'une
inversion des bits d'un nombre signé est la même chose que changer son signe et soustraire
1 du résultat.

Donc ce que \NOT fait ici est de prendre la valeur de $str$ et de la transformer
en $-str-1$.
L'opération d'addition qui suit prépare le résultat.
