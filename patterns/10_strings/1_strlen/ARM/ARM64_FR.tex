\myparagraph{ARM64}

\mysubparagraph{GCC \Optimizing (Linaro) 4.9}

\lstinputlisting[style=customasmARM]{patterns/10_strings/1_strlen/ARM/ARM64_GCC_O3_FR.lst}

L'algorithme est le même que dans \myref{strlen_MSVC_Ox}:
trouver un octet à zéro, calculer la différence antre les pointeurs et décrémenter
le résultat de 1.\TT{size\_t}
Quelques commentaires ont été ajouté par l'auteur de ce livre.

La seule différence notable est que cet exemple est un peu faux: \\
\TT{my\_strlen()} renvoie une valeur \Tint 32-bit, tandis qu'elle devrait renvoyer
un type \TT{size\_t} ou un autre type 64-bit.

La raison est que, théoriquement, \TT{strlen()} peut-être appelée pour un énorme
bloc de mémoire qui dépasse 4GB, donc elle doit être capable de renvoyer une valeur
64-bit sur une plate-forme 64-bit.

À cause de cette erreur, la dernière instruction \SUB opère sur la partie 32-bit
du registre, tandis que la pénultième instruction \SUB travaille sur un registre
64-bit complet (elle calcule la différence entre les pointeurs).

C'est une erreur de l'auteur, il est mieux de la laisser ainsi, comme un exemple
de ce à quoi ressemble le code dans un tel cas.

\mysubparagraph{GCC \NonOptimizing (Linaro) 4.9}

\lstinputlisting[style=customasmARM]{patterns/10_strings/1_strlen/ARM/ARM64_GCC_O0_FR.lst}

C'est plus verbeux.
Les variables sont beaucoup manipulées vers et depuis la mémoire (pile locale).
Il y a la même erreur ici: l'opération de décrémentation se produit sur la partie
32-bit du registre.
