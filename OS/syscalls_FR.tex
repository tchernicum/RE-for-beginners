\mysection{Appels systèmes (syscall-s)}

\label{syscalls}
\myindex{syscall}

\myindex{kernel space}
\myindex{user space}
Comme nous le savons, tous les processus d'un \ac{OS} sont divisés en deux catégories:
ceux qui ont un accès complet au matériel (\q{kernel space} espace noyau)
et ceux qui ne l'ont pas (\q{user space} espace utilisateur).

Le noyau de l'OS et, en général, les drivers sont dans la première catégorie.

Toutes les applications sont d'habitude dans la seconde catégorie.

\myindex{Glibc}
Par exemple, le noyai Linux est dans le \emph{kernel space}, mais la Glibc est dans
le \emph{user space}.

Cette séparation est cruciale pour la sécurité de l'\ac{OS}: il est très important
de ne pas donner la possibilité à n'importe quel processus de casser quelque chose
dans un autre processus ou même dans le noyau de l'\ac{OS}.
\myindex{kernel panic}
\myindex{BSoD}
D'un autre côté, un driver qui plante ou une erreur dans le noyau de l'\ac{OS} se
termine en général par un kernel panic ou un \ac{BSOD}.

La protection en anneau dans les processeurs x86 permet de séparer l'ensemble dans
quatre niveaux de protection (rings), mais tant dans Linux que Windows, seuls deux
d'entre eux sont utilisés: ring0 (\q{kernel space}) et ring3 (\q{user space}).

Les appels systèmes (syscall-s) sont un point où deux espaces sont connectés.

On peut dire que c'est la principale \ac{API} fournie aux applications.

Comme dans \gls{Windows NT}, la table des appels système se trouve dans la \ac{SSDT}.

\myindex{Shellcode}

L'utilisation des appels système est très répandue parmis les auteurs de shellcode
et de virus, car il est difficile de déterminer l'adresse des fonctions nécessaires
dans les bibliothèques système, mais il est simple d'utiliser les appels système.
Toutefois, il est nécessaire d'écrire plus de code à cause du niveau d'abstraction
plus bas de l'\ac{API}.

Il faut aussi noter que les numéros des appels systèmes peuvent être différent dans
différentes versions d'un OS.

\subsection{Linux}
\label{linux_syscall}

\myindex{x86!\Instructions!INT!INT 0x80}
Sous Linux, un appel système s'effectue d'habitude par \TT{int 0x80}.
Le numéro de l'appel est passé dans le registre \EAX, et tout autre paramètre~---dans
les autres registres.

\lstinputlisting[caption=Un exemple simple de l'utilisation de deux appels système,style=customasmx86]{OS/linux_syscall.s}

Compilation:

\begin{lstlisting}
nasm -f elf32 1.s
ld 1.o
\end{lstlisting}

La liste complète des appels systèmes sous Linux: \url{http://go.yurichev.com/17319}.

Pour intercepter les appels système et les suivre sous Linux, strace(\myref{strace})
peut être utilisé.

\subsection{Windows}

\myindex{x86!\Instructions!INT!INT 0x2e}
\myindex{x86!\Instructions!SYSENTER}

Ici, ils sont appelés via \TT{int 0x2e} ou en utilisant l'instruction x86 spéciale
\TT{SYSENTER}.

La liste complète des appels systèmes sous Windows: \url{http://go.yurichev.com/17320}.

Pour aller plus loin:

\q{Windows Syscall Shellcode} par Piotr Bania: \url{http://go.yurichev.com/17321}.

