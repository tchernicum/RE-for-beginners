\subsection{CRT (win32)}
\label{sec:CRT}
\myindex{CRT}

L'exécution du programme débute donc avec la fonction \main{}?
Non, pas du tout.

Ouvrez un exécutable dans \IDA ou HIEW, et vous constaterez que l'\ac{OEP} pointe sur un bloc de 
code qui se situe ailleurs.

Ce code prépare l'environnement avant de passer le contrôle à notre propre code. Ce fragment de code 
initial est dénommé CRT (pour C RunTime). \\
\\
La fonction \main{} accepte en arguments un tableau des paramètres passés en ligne de commande, et 
un autre contenant les variables d'environnement.
En réalité, ce qui est passé au programme est une simple chaîne de caractères. Le code du CRT repère 
les espaces dans la chaîne et découpe celle-ci en plusieurs morceaux qui constitueront le tableau des 
paramètres.
Le CRT est également responsable de la préparation du tableau des variables d'environnement \TT{envp}.

En ce qui concerne les applications win32 dotées d'un interface graphique (\ac{GUI}). La fonction 
principale est nommée \TT{WinMain} et non pas \main{}. Elle possède aussi ses propres arguments:

\begin{lstlisting}[style=customc]
int CALLBACK WinMain(
  _In_  HINSTANCE hInstance,
  _In_  HINSTANCE hPrevInstance,
  _In_  LPSTR lpCmdLine,
  _In_  int nCmdShow
);
\end{lstlisting}

C'est toujours le CRT qui est responsable de leur préparation.

La valeur retournée par la fonction \main{} est également ce qui constitue le code retour du 
programme.

Le CRT peut utiliser cette valeur lors de l'appel de la fonction \TT{ExitProcess()} qui prend le code 
retour du programme en paramètre. \\
\\
En règle générale, le code du CRT est propre à chaque compilateur. \\
\\
Voici celui du MSVC 2008.

\lstinputlisting[numbers=left,style=customasmx86]{OS/win32_CRT/crt_msvc_2008.asm}

Nous y trouvons des appels à \TT{GetCommandLineA()} (line 62),
puis \TT{setargv()} (line 66) et \TT{setenvp()} (line 74),
qui semblent utilisés pour initialiser les variables globales 
\TT{argc}, \TT{argv}, \TT{envp}.

Enfin, \main{} est invoqué avec ces arguments (line 97).

Nous observons également des appels à des fonctions au nom évoquateur telles que \TT{heap\_init()} 
(line 35), \TT{ioinit()} (line 54).

Le tas \glslink{heap}{heap} est lui aussi initialisé par le \ac{CRT}.
Si vous tentez d'utiliser la fonction \TT{malloc()} dans un programme ou le CRT n'a pas été ajouté, 
le programme va s'achever de manière anormale avec l'erreur suivante:

\begin{lstlisting}
runtime error R6030
- CRT not initialized
\end{lstlisting}

L'initialisation des objets globaux en \Cpp est également de la responsabilité du \ac{CRT} avant 
qu'il ne passe la main à \main{}: 
\myref{sec:std_string_as_global_variable}.

La valeur retournée par \main{} est passée soit à la fonction \TT{cexit()}, soit à \TT{\$LN32} qui 
à son tour invoque \TT{doexit()}.

Mais pourrait-on se passe du \ac{CRT}?
Oui, si vous savez ce que vous faites.

L'éditeur de lien \ac{MSVC} accepte une option \TT{/ENTRY} qui permet de définir le point d'entrée 
du programme.

\begin{lstlisting}[style=customc]
#include <windows.h>

int main()
{
	MessageBox (NULL, "hello, world", "caption", MB_OK);
};
\end{lstlisting}

Compilons ce programme avec MSVC 2008.

\begin{lstlisting}
cl no_crt.c user32.lib /link /entry:main
\end{lstlisting}

Nous obtenons un programme exécutable .exe de 2560 octets qui contient en tout et pour tout l'en-tête 
PE, les  instructions pour invoquer \TT{MessageBox} et deux chaînes de caractères dans le segment de 
données:  le nom de la fonction \TT{MessageBox} et celui de sa DLL d'origine \TT{user32.dll}.

Cela fonctionne, mais vous ne pouvez pas déclarer \TT{WinMain} et ses 4 arguments à la place de 
\main{}.

Pour être précis, vous pourriez, mais les arguments ne seraient pas préparés au moment de 
l'exécution du programme.

Cela étant, il est possible de réduire encore la taille de l'exécutable en utilisant une valeur 
pour l'alignement des sections du fichier au format \ac{PE} une valeur inférieur à celle par 
défaut de 4096 octets.

\begin{lstlisting}
cl no_crt.c user32.lib /link /entry:main /align:16
\end{lstlisting}

L'éditeur de lien prévient cependant:

\begin{lstlisting}
LINK : warning LNK4108: /ALIGN specified without /DRIVER; image may not run
\end{lstlisting}

Nous pouvons ainsi obtenir un exécutable de 720 octets.
Son exécution est possible sur Windows 7 x86, mais pas en environnement x64 
(un message d'erreur apparaîtra alors si vous tentez de l'exécuter).

Avec des efforts accrus il est possible de réduire encore la taille, mais avec des problèmes de
compatibilité comme vous le voyez.
