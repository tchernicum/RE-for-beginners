\mysection{lokaler Thread-Speicher}
\label{TLS}
\myindex{TLS}

TLS ist ein Datenbereich der für jeden einzelnen Thread spezifisch ist.
Jeder Thread kann hier alles speichern was er möchte.
Eine bekanntes Beispiel ist die globale C-Standardvariable \emph{errno}.

Mehrere Threads können Funktionen simultan aufrufen, die einen Fehlercode in \emph{errno} zurückgeben.
Eine globale Variable würde hier nicht korrekt für Multithread-Programme funktionieren,
aus diesem Grund muss \emph{errno} im \ac{TLS} gesichert.

\myindex{\Cpp!C++11}
Im C++11-Standard wurde ein neues Schlüsselwort \emph{thread\_local} eingeführt,
um anzuzeigen, dass jeder Thread eine eigene Kopie dieser Variable, die initialisiert werde kann
und sich auf dem \ac{TLS} befindet\footnote{\myindex{C11} C11 hat ebenfalls einen (optionalen) Thread-Support}:

\begin{lstlisting}[caption=C++11]
#include <iostream>
#include <thread>

thread_local int tmp=3;

int main()
{
	std::cout << tmp << std::endl;
};
\end{lstlisting}

Kompiliert mit MinGW GCC 4.8.1 jedoch nicht MSVC 2012.

Wenn es um PE-Dateien geht, also die ausführbaren Dateien, wird die \emph{tmp}-Variable
in der Sektion mit dem Namen \ac{TLS} gesichert.

\subsection{Nochmals Linearer Kongruenzgenerator}
\label{LCG_TLS}

Der Pseudozufallszahlen-Generator der in \myref{LCG_simple} bereits erwähnt wurde hat einen Nachteil:
Er ist nicht Thread-sicher, weil er eine interne Zustandsvariable hat, die von verschiedenen Threads
gleichzeitig gelesen und verändert werden kann.

% subsections
\subsubsection{Win32}

\myparagraph{Uninitialisierte \ac{TLS}-Daten}

Eine mögliche Lösung ist es den \TT{\_\_declspec( thread )}-Modifizierer zu der
globalen Variable hinzuzufügen, so dass diese im \ac{TLS} alloziert wird (Zeile 9):

\lstinputlisting[numbers=left,style=customc]{OS/TLS/win32/rand_uninit.c}

Hiew zeigt, dass eine neue PE-Sektion in der ausführbaren Datei existiert: \TT{.tls}.
% TODO1 hiew screenshot?

\lstinputlisting[caption=\Optimizing MSVC 2013 x86,style=customasmx86]{OS/TLS/win32/rand_x86_uninit.asm}

\TT{rand\_state} befindet sich nun im \ac{TLS}-Segment und jeder Thread hat seine eigene Kopie dieser Variable.

Auf diese Variable wird wie folgt zugegriffen: lade die Adresse von \ac{TIB} von FS:2Ch,
anschließend addiere einen zusätzlichen Index (falls notwendig),
zuletzt berechne die Adresse des \ac{TLS}-Segments.

Nun ist es möglich auf die \TT{rand\_state}-Variable über des ECX-Register zuzugreifen,
welches auf eine spezifische Stelle in jedem Thread zeigt.

\myindex{x86!\Registers!FS}

Der \TT{FS:}-Selektor ist bei jedem Reverse Engineer gut bekannt und zeigt immer auf \ac{TIB},
so dass es schnell möglich ist thread-spezifische Daten zu laden.

\myindex{x86!\Registers!GS}

Der \TT{GS:}-Selektor wird unter Win64 genutzt, die Adresse von \ac{TLS} ist 0x58:

\lstinputlisting[caption=\Optimizing MSVC 2013 x64,style=customasmx86]{OS/TLS/win32/rand_x64_uninit.asm}

\myparagraph{Initialisierte \ac{TLS}-Daten}

Angenommen es soll ein fester Wert für \TT{rand\_state} gesetzt werden und der Programmierer vergisst dies,
wird dies automatisch in Zeile 9 gemacht:

\lstinputlisting[numbers=left,style=customc]{OS/TLS/win32/rand_init.c}

Der Code ist nicht anders als zuvor, aber in IDA sieht man folgendes:

\lstinputlisting[style=customasmx86]{OS/TLS/win32/rand_init_IDA.lst}

Jedes mal wenn ein neuer Thread gestartet wird, wird ein neuer \ac{TLS} alloziert
und alle Daten, inklusive der 1234 dorthin kopiert.

Dies ist eine typische Situation:

\begin{itemize}
\item Thread A wird gestartet und ein \ac{TLS} wird dafür erstellt.
1234 wird in \TT{rand\_state} kopiert.

\item Die Funktion \TT{my\_rand()} wird einige Male in Thread A aufgerufen.\\
\TT{rand\_state} unterscheidet sich 1234.

\item Thread B wird gestartet und ein \ac{TLS} wird dafür erstellt.
1234 wird in \TT{rand\_state} kopiert, während Thread A einen anderen Wert
in der gleichen Variable hat.
\end{itemize}

\myparagraph{\ac{TLS}-Callback-Funktionen}
\myindex{TLS!Callbacks}

Was passiert wenn die Variablen in \ac{TLS} mit Daten gefüllt werden, die in einer
bestimmten Weise verändert werden sollen?

Angenommen es existiert folgende Aufgabe:
Der Programmierer hat vergessen die \TT{my\_srand()}-Funktion aufzurufen um den \ac{PRNG} zu initialisieren,
dieser muss jedoch initialisiert werden um \q{echte} Zufallszahlen anstatt den Wert 1234 zu erzeugen.
In diesem Fall kann die \ac{TLS}-Callback-Funktion genutzt werden.

Der folgende Code ist nicht sehr portabl, nichtsdestotrotz ist die Idee dahinter erkennbar.

Was hier passiert ist eine Funktion zu definieren (\TT{tls\_callback()}) welche \emph{vor}
dem Starten eines Prozesses oder Threads aufgerufen wird.

Die Funktion initialisiert den \ac{PRNG} mit dem Wert der von \TT{GetTickCount()} zurückgegeben wird.

\lstinputlisting[style=customc]{OS/TLS/win32/rand_TLS_callback.c}

Nachfolgend das Ergebnis in IDA:

\lstinputlisting[caption=\Optimizing MSVC 2013,style=customasmx86]{OS/TLS/win32/rand_TLS_callback.lst}

%TODO Not very elegant translation.
TLS-Callback-Funktionen werden manchmal genutzt um Routinen zu entpacken und deren Funktion zu verschleiern.

Manche Menschen sind verwirrt darüber, dass einiger Code genau vor dem \ac{OEP} ausgeführt wird.

\input{OS/TLS/LCG_linux_DE}
