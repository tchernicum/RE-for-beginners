\subsubsection{Windows x64}

\label{SEH_win64}
Как видно, это не самая быстрая штука, устанавливать SEH-структуры в каждом прологе функции.
Еще одна проблема производительности --- это менять переменную 
\emph{previous try level} много раз в течении исполнении функции.
Так что в x64 всё сильно изменилось, теперь все указатели на \TT{try}-блоки, функции фильтров и обработчиков,
теперь записаны в другом PE-сегменте
 \TT{.pdata}, откуда обработчик исключений \ac{OS} берет всю информацию.

Вот два примера из предыдущей секции, скомпилированных для x64:

\lstinputlisting[caption=MSVC 2012,style=customasmx86]{OS/SEH/3/2_x64.asm}

\lstinputlisting[caption=MSVC 2012,style=customasmx86]{OS/SEH/3/3_x64.asm}

Смотрите \IgorSkochinsky для более детального описания.

Помимо информации об исключениях, секция \TT{.pdata} 
также содержит начала и концы почти всех функций, так что эту информацию можно использовать в каких-либо
утилитах, предназначенных для автоматизации анализа.

