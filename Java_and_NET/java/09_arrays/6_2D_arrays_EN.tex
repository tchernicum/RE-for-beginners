% TODO proof-reading
\subsubsection{Two-dimensional arrays}

Two-dimensional arrays in Java are just one-dimensional arrays of \emph{references} to another 
one-dimensional arrays.


Let's create a two-dimensional array:

\begin{lstlisting}[style=customjava]
	public static void main(String[] args)
	{
		int[][] a = new int[5][10];
		a[1][2]=3;
	}
\end{lstlisting}

\begin{lstlisting}
  public static void main(java.lang.String[]);
    flags: ACC_PUBLIC, ACC_STATIC
    Code:
      stack=3, locals=2, args_size=1
         0: iconst_5      
         1: bipush        10
         3: multianewarray #2,  2      // class "[[I"
         7: astore_1      
         8: aload_1       
         9: iconst_1      
        10: aaload        
        11: iconst_2      
        12: iconst_3      
        13: iastore       
        14: return        
\end{lstlisting}

It's created using the \TT{multianewarray} instruction: the object's type and dimensionality are passed
as operands.

The array's size (10*5) is left in stack (using the instructions \TT{iconst\_5} and \TT{bipush}).


A \emph{reference} to row \#1 is loaded at offset 10 (\TT{iconst\_1} and \TT{aaload}).

The column is chosen using \TT{iconst\_2} at offset 11.

The value to be written is set at offset 12.

\TT{iastore} at 13 writes the array's element.


How it is an element accessed?

\begin{lstlisting}[style=customjava]
	public static int get12 (int[][] in)
	{
		return in[1][2];
	}
\end{lstlisting}

\begin{lstlisting}
  public static int get12(int[][]);
    flags: ACC_PUBLIC, ACC_STATIC
    Code:
      stack=2, locals=1, args_size=1
         0: aload_0       
         1: iconst_1      
         2: aaload        
         3: iconst_2      
         4: iaload        
         5: ireturn       
\end{lstlisting}

A \emph{Reference} to the array's row is loaded at offset 2, the column is set at offset 3, 
then \TT{iaload} loads the array's element.

