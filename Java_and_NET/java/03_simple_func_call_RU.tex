% TODO proof-reading
\subsection{Простой вызов функций}


\TT{Math.random()} возвращает псевдослучайное число в пределах [0.0 \dots 1.0), но представим,
по какой-то причине, нам нужна функция, возвращающая число в пределах [0.0 \dots 0.5):

\begin{lstlisting}[style=customjava]
public class HalfRandom
{ 
	public static double f()
	{
		return Math.random()/2;
	}
}
\end{lstlisting}

\begin{lstlisting}[caption=Constant pool]
...
   #2 = Methodref          #18.#19    //  java/lang/Math.random:()D
   #3 = Double             2.0d
...
  #12 = Utf8               ()D
...
  #18 = Class              #22        //  java/lang/Math
  #19 = NameAndType        #23:#12    //  random:()D
  #22 = Utf8               java/lang/Math
  #23 = Utf8               random
\end{lstlisting}

\begin{lstlisting}
  public static double f();
    flags: ACC_PUBLIC, ACC_STATIC
    Code:
      stack=4, locals=0, args_size=0
         0: invokestatic  #2          // Method java/lang/Math.random:()D
         3: ldc2_w        #3          // double 2.0d
         6: ddiv          
         7: dreturn       
\end{lstlisting}


\TT{invokestatic} вызывает функцию \TT{Math.random()} и оставляет результат на \ac{TOS}.

Затем результат делится на 2.0 и возвращается.

Но как закодировано имя функции?

Оно закодировано в пуле констант используя выражение \TT{Methodref}.

Оно определяет имена класса и метода.

Первое поле \TT{Methodref} указывает на выражение \TT{Class}, которое, в свою очередь,
указывает на обычную текстовую строку (\q{java/lang/Math}).

Второе выражение \TT{Methodref} указывает на выражение \TT{NameAndType}, которое также
имеет две ссылки на строки.

Первая строка это \q{random}, это имя метода.

Вторая строка это \q{()D}, которая кодирует тип функции. Это означает, что возвращаемый тип --- \emph{double} (отсюда \emph{D} в строке).

Благодаря этому
1) JVM проверяет корректность типов данных; 
2) Java-декомпиляторы могут восстанавливать типы данных из class-файлов.


Наконец попробуем пример \q{Hello, world!}:

\begin{lstlisting}[style=customjava]
public class HelloWorld
{
	public static void main(String[] args)
	{
		System.out.println("Hello, World");
	}
}
\end{lstlisting}

\begin{lstlisting}[caption=Constant pool]
...
   #2 = Fieldref           #16.#17        //  java/lang/System.out:Ljava/io/PrintStream;
   #3 = String             #18            //  Hello, World
   #4 = Methodref          #19.#20        //  java/io/PrintStream.println:(Ljava/lang/String;)V
...
  #16 = Class              #23            //  java/lang/System
  #17 = NameAndType        #24:#25        //  out:Ljava/io/PrintStream;
  #18 = Utf8               Hello, World
  #19 = Class              #26            //  java/io/PrintStream
  #20 = NameAndType        #27:#28        //  println:(Ljava/lang/String;)V
...
  #23 = Utf8               java/lang/System
  #24 = Utf8               out
  #25 = Utf8               Ljava/io/PrintStream;
  #26 = Utf8               java/io/PrintStream
  #27 = Utf8               println
  #28 = Utf8               (Ljava/lang/String;)V
...
\end{lstlisting}

\begin{lstlisting}
  public static void main(java.lang.String[]);
    flags: ACC_PUBLIC, ACC_STATIC
    Code:
      stack=2, locals=1, args_size=1
         0: getstatic     #2        // Field java/lang/System.out:Ljava/io/PrintStream;
         3: ldc           #3        // String Hello, World
         5: invokevirtual #4        // Method java/io/PrintStream.println:(Ljava/lang/String;)V
         8: return        
\end{lstlisting}


\TT{ldc} по смещению 3 берет указатель (или адрес) на строку \q{Hello, World}
в пуле констант и заталкивает его в стек.

В мире Java это называется \emph{reference}, но это скорее указатель или просто адрес
\footnote{О разнице между указателями и \emph{reference} в С++: \myref{cpp_references}.}.


Уже знакомая нам инструкция \TT{invokevirtual} берет информацию о функции (или методе) \TT{println} 
из пула констант и вызывает её.

Как мы можем знать, есть несколько методов \TT{println}, каждый предназначен для каждого типа
данных.

В нашем случае, используется та версия \TT{println}, которая для типа данных \emph{String}.


Что насчет первой инструкции \TT{getstatic}?

Эта инструкция берет \emph{reference} (или адрес) поля объекта \TT{System.out}
и заталкивает его в стек.

Это значение работает как указатель \emph{this} для метода \TT{println}.

Таким образом, внутри, метод \TT{println} берет на вход два аргумента:
1) \emph{this}, т.е. указатель на объект
\footnote{Или \q{экземпляр класса} в некоторой русскоязычной литературе.};
2) адрес строки \q{Hello, World}.


Действительно, \TT{println()} вызывается как метод в рамках инициализированного объекта 
\TT{System.out}.


Для удобства, утилита \TT{javap} пишет всю эту информацию в комментариях.
