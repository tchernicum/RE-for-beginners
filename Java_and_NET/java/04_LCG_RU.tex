% TODO proof-reading
\subsection{Линейный конгруэнтный \ac{PRNG}}


Попробуем простой генератор псевдослучайных чисел, 
который мы однажды уже рассматривали в этой книге (\myref{LCG_simple}):

\begin{lstlisting}[style=customjava]
public class LCG 
{
	public static int rand_state;

	public void my_srand (int init)
	{
		rand_state=init;
	}

	public static int RNG_a=1664525;
	public static int RNG_c=1013904223;

	public int my_rand ()
	{
		rand_state=rand_state*RNG_a;
		rand_state=rand_state+RNG_c;
		return rand_state & 0x7fff;
	}
}
\end{lstlisting}


Здесь пара полей класса, которые инициализируются в начале.
Но как?

В выводе \TT{javap} мы можем найти конструктор класса:

\begin{lstlisting}
  static {};
    flags: ACC_STATIC
    Code:
      stack=1, locals=0, args_size=0
         0: ldc           #5         // int 1664525
         2: putstatic     #3         // Field RNG_a:I
         5: ldc           #6         // int 1013904223
         7: putstatic     #4         // Field RNG_c:I
        10: return        
\end{lstlisting}


Так инициализируются переменные.

\TT{RNG\_a} занимает третий слот в классе и \TT{RNG\_c} --- четвертый, 
и \TT{putstatic} записывает туда константы.


Функция \TT{my\_srand()} просто записывает входное значение в \\ 
\TT{rand\_state}:

\begin{lstlisting}
  public void my_srand(int);
    flags: ACC_PUBLIC
    Code:
      stack=1, locals=2, args_size=2
         0: iload_1       
         1: putstatic     #2         // Field rand_state:I
         4: return        
\end{lstlisting}


\TT{iload\_1} берет входное значение и заталкивает его в стек. Но почему не \TT{iload\_0}?

Это потому что эта функция может использовать поля класса, а переменная \emph{this} 
также передается в эту функцию как нулевой аргумент.

Поле \TT{rand\_state} занимает второй слот в классе,
так что \TT{putstatic} копирует переменную из \ac{TOS} во второй слот.

Теперь \TT{my\_rand()}:

\begin{lstlisting}
  public int my_rand();
    flags: ACC_PUBLIC
    Code:
      stack=2, locals=1, args_size=1
         0: getstatic     #2         // Field rand_state:I
         3: getstatic     #3         // Field RNG_a:I
         6: imul          
         7: putstatic     #2         // Field rand_state:I
        10: getstatic     #2         // Field rand_state:I
        13: getstatic     #4         // Field RNG_c:I
        16: iadd          
        17: putstatic     #2         // Field rand_state:I
        20: getstatic     #2         // Field rand_state:I
        23: sipush        32767
        26: iand          
        27: ireturn       
\end{lstlisting}


Она просто загружает все переменные из полей объекта, производит с ними операции
и обновляет значение \TT{rand\_state}, используя инструкцию \TT{putstatic}.

По смещению 20, значение \TT{rand\_state} перезагружается снова
(это потому что оно было выброшено из стека перед этим, инструкцией \TT{putstatic}).

Это выглядит как неэффективный код, но можете быть уверенными, \ac{JVM} обычно достаточно
хорош, чтобы хорошо оптимизировать подобные вещи.
